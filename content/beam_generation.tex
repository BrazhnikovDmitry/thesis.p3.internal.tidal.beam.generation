\documentclass[12pt]{article}
\input{/home/dmitry/Work/Research/thesis/FINALE/settings.tex}
%\doublespacing
%\graphicspath{{/home/dmitry/Work/Research/thesis/FINALE/P3_ITS_GENERATION/figures/}}

%\documentclass[PhD_Thesis.tex]{subfiles}

\begin{document}
	\iftoggle{only_Chapter} {
		\title{Dynamics of variable internal tide generation at the Macquarie Ridge}
		\maketitle
	}

\section*{Abstract}
The Macquarie Ridge south of New Zealand is a strong generator of internal tidal waves forming a 
baroclinic mode-1 tidal beam which was a subject of an extensive field program (TTIDE). The beam 
generation takes place in a complex bathymetric region that is reminiscent to resonant 
setting of Luzon Strait. Additionally, local oceanographic conditions are highly variable due 
to presence of the southern subtropical frontal zone. These factors suggest significant variability 
in generation that was studied by numerical simulations initialized with realistic hydrographic 
conditions. The experiments showed mean value of $1.8~GW$ of barotropic $M_2$ energy was converted 
into mode-1 baroclinic field, whilst variability range comprised $\pm 15\%$(1.4 and 2 GW).
The changes were concentrated at a topographic gap in the ridge and driven by remote baroclinic 
waves emitted from the Campbell Plateau. Direct analysis of the wave field was burdened by 
interference. The complexity was then overcame by decomposition with a novel approach based on 
inverse calculation of directional spectra. This allowed to quantify magnitude and arrival time of 
the remote waves emphasizing their in-phase relation with barotropic forcing and thus, enhanced 
generation.
Levels of the reinforcement strongly depend on the three-dimensional structure of the remote waves 
rather than along two-dimensional transect arrival time usually associated with resonant condition. 
Hence, complexity of the problem increases as there is a high degree of mutual influence 
on generation or coupling between the Macquarie ridge and the Campbell Plateau. In the numerical 
experiments destabilization mechanism was associated with variable phase speeds and stratification. 
Even small deviations from the mean were further amplified in the generation levels. Undoubtedly, 
the nonstationarity was then transmitted as variable tidal beam intensity and thus, could cause 
intermittency in any field observations such as TTIDE.

\section{Introduction}
Baroclinic semidiurnal tides originate as a strong barotropic flow along topography forces heaving 
of isopynal surfaces. This process renders scattering of barotropic tidal energy into baroclinic 
motions \citep{hendershott1981long}. The so dispersed energy constitutes a third of global budget 
for lunar semidiurnal constituent \citep{egbert2000significant, munk1997once} and contributes 
significantly to internal wave climate. At most conversion sites because of highly inclined 
slopes, internal waves of tidal period (internal tides) are radiated as low baroclinic mode. Due to 
their large length scales decay rates are subtle. This makes low mode tide efficient in carrying 
baroclinic energy (kinetic?) over distances comparable to size of ocean basins. While generation 
sites were identified \citep{morozov1995semidiurnal, simmons2004internal, arbic2010concurrent, 
zhao2016global}, some 
were studied in detail \citep{rudnick2003tides, klymak2011breaking, althaus2003internal} and 
analytical models have been developed \citep{garrett2007internal}, little is known on how fast and 
where internal tidal energy is dissipated (deposited). A lot of uncertainty arises because of close 
relation between the waves and the dynamical oceanic medium, so the wave field is subject to 
continuous change.\\
Water column stratification directly impacts internal wave dispersion and generation. In fact, 
analytical models 
of generation emphasize a ratio between angle of internal wave characteristics to bathymetric slope 
\citep{garrett2007internal} along with a height of topography as primary quantities in setting 
conversion levels \citep{llewellyn2003tidal, petrelis2006tidal}. For tall, steeply inclined 
submarine ridges the energy transfer approaches an upper theoretical limit 
\citep{petrelis2006tidal, st2003generation} making them to be ``oases" of barotropic tide 
scattering 
and internal tide production \citep{morozov1995semidiurnal, egbert2000significant}. Clearly, the 
energy rates can be modulated by changing buoyancy frequency \citep{holloway1999internal}, 
especially when transformation of water properties happens at same depths as the steepest 
bottom gradients 
\citep{gerkema2004internal}. Nevertheless, in later field studies it was realized that presence of 
external baroclinic tidal signal leads to even larger temporal variability \citep{Kelly2010a, 
zilberman2011incoherent, pickering2015structure}. This might occur as opposite ridge slopes 
affect each other \citep{nash2004internal, zilberman2011incoherent, echeverri2010internal} or due 
to spatially inhomogeneous distribution of production hotspots \citep{osborne2011spatial, 
ponte2013coastal}\footnote{there should a verb here, needs to be restructured}, or as separate 
topographic features mediate each others generation energy levels \citep{xing1998three, 
buijsman2012modeling, buijsman2014three}. This study addresses temporal and spatial variability of 
tide production happening at the Macquarie Ridge, south from New Zealand. Quick recourse to a map 
of 
the 
Tasman Sea \fignmp{C3:fig:geo.map} suggests that location of major sea bottom features leads to 
complex internal tide regime representative both of Kaena Ridge and Luzon Strait.\\

\newpage

\section{Numerical experiments and analysis}
\subsection{Numerical experiments}
To study variability of internal tide generation around New Zealand and its propagation numerical 
simulations were performed with Regional Ocean Modeling System \citep{shchepetkin2005regional}. 
The numerical domain covered the southern Tasman Sea from subantractic waters of $60^{\circ}$ S 
to subtropics in $35^{\circ}$ S. And the zonal extent stretched from $142^{\circ}$ to $172^{\circ}$ 
E. This ensued correct representation of reach regional oceanographic conditions. The horizontal 
grid spacing was taken to be of $1/32^{\circ}$ corresponding on average to discretization of 3 km 
in zonal direction and 2.5 km in meridional. The nonuniformly separated, vertical 50 $s$-levels 
were placed to smoothly follow subsurface terrain.\\
Such discretization of vertical momentum equation tends to induce artificial, horizontal  
along-slope flows \citep{haidvogel1999numerical} due to errors in reproducing of pressure 
gradient force. Especially severe errors are made by steep terrain. The misbehavior is usually 
resolved by artificial smoothing of topography. This procedure additionally increases numerical 
stability, but has an adverse effect on internal tide generation \citep{di2006numerical} since 
primary production sites are collocated with large topographic gradients. To test the numerical 
setup, a sensitivity study was carried out with simulations of 
$1/8^{\circ},~1/16^{\circ},~1/64^{\circ}$ horizontal resolution. The essential for this study 
internal tide behavior manifested at $1/16^{\circ}$ and converged for $1/32^{\circ}$ and 
$1/64^{\circ}$ cases. There no marked differences were observed, except a substantial increase in 
high mode content which is in line with \footnote{previous investigations} 
\citep{di2006numerical}.\\
(bad transition)This work addresses the gravest baroclinic mode dynamics in the deep ocean. Spatial 
extent of 
waves is large compared to associated vertical displacements. This ensures linear regime of 
propagation without dispersive and nonhydrostatic effects taken place such as fission into 
solitons. A hydrostatic solver used in ROMS seems to be a proper choice for the simulations. Such 
simplification in wave dynamics was assumed in previous studies \citep{carter2008energetics, 
merrifield2001generation,  merrifield2002model, kerry2013effects}. In more dynamically accurate   
simulations of \citep{kang2012energetics, zhang2011three} the nonhydrostatic effects are found to 
be important only for internal tides in shallow waters, while for main part generation follows 
linear dynamics with vertical accelerations to have a negligible contribution.\\
The horizontal boundary conditions were imposed to be open for depth-averaged, barotropic flows 
following recommendations of \citep{marchesiello2001open}. The baroclinic fields are 
nudged to zero by linearly increased lateral viscosity and diffusivity over sponge layers. Through 
the same outer boundaries numerical simulations were forced with barotropic tide. The tidal 
currents and sea level are derived from TPXO atlas, version 7.2 \citep{egbert2002efficient} and 
prescibed as linearly interpolated volume transports. It was used only the largest semidiurnal 
constituent $M_2$. Amplitude ratio between the principal lunar and solar components are 4-to-1 
suggestive of slight open-ocean spring-neap modulation. The diurnal species are weak in the region 
except shoals east of New Zealand \citep{walters2001ocean}.\\
To investigate variations of baroclinic tide dynamics several ocean states were prescribed and 
analyzed separately. In the simplest setting lateral gradients in water properties were absent, 
while buoyancy frequency was set to be representative of the Tasman Basin. The second set of 
simulations 
was comprised to investigate interannual and interseasonal variability (Table 1). And the third 
calculation was intended to cover period of TTIDE/TBEAM/Tshelf field programs 
\citep{pinkel2015breaking}, a single experiment once initialized was left to proceed for three 
months.
\begin{table}
	\caption{Carried out numerical experiments}
	\begin{tabular}{ |p{3cm}||p{5cm}|p{5cm}|  }
		\hline
		\multicolumn{3}{|c|}{Numerical experiments used in this study} \\
		\hline
		Experiment abbreviation & Simulation period & Comments (reason?) \\
		\hline
		Uniform & ~ & No mesoscale \\
		2012 &   Jan 1st - Jan 15th, 2012 & Interannual \\
		2013 &   Jan 1st - Jan 15th, 2013 & Interannual \\
		2014 &   Jan 1st - Jan 15th, 2014 & Interannual \\
		2013\_Oct &   Oct 1st - Oct 15th, 2013 & Interseaonal \\
		2015\_Mar &   Mat 1st - Mar 15th, 2015 & Interseaonal \\
		2015\_TTIDE$^{\ast}$ &   Jan 1st - Mar 1st, 2015 & Field period \\
		\hline
		\multicolumn{3}{|l|}{\footnotesize$^{\ast}$ the results are named as respective day of 
		year over which post-analysis was performed, e.g. $d20-25$ }\\
		\hline
	\end{tabular}
	\label{ch2:table_exp}
\end{table}
The simulations with variable conditions were at first initialized with HYCOM hindcasts 
\footnote{(NAVGEM;	downloaded from hycom.org)} for respective start dates. Then during 
integration, 
along with barotropic tidal flow, time-variable, subtidal two dimensional fields 
\footnote{(vertical coordinate and along boundary coordinate)} of horizontal currents, temperature 
and salinity were imposed onto the numerical ocean on the boundaries. Conditions of air-sea 
interface obtained from MERRA-reanalysis \citep{rienecker2011merra} were as well utilized to 
prescribe realistic insolation, air temperature, EP rates and most importantly, wind stresses.

\subsection{Internal tide analysis}
As seen in \tblnm{ch2:table_exp} the simulations were carried out for 15 days or longer. In all 
cases the first 10 days were regarded as spin up of baroclinic tide generation and 
propagation. Particularly it accommodated a single passage of mode-1 from New Zealand 
to Tasmania that takes roughly 7 days. After the spin-up period three 
dimensional fields of velocity, temperature and salinity were sampled hourly. These 
were subject to high pass filtering with Butterworth filter of order $6$ with cut off time of $36$ 
hours to remove subtidal motions. Left-out signal was further fit in a least square sense to the 
principle semidiurnal harmonic. Then the three dimensional fields 
underwent a separation into barotropic and baroclinic signals \citep{cummins1997simulation, 
kunze2002internal, carter2008energetics}. A depth-averaged current is thought to represent a pure 
barotropic signal and any vertical deviation is attributed to a baroclinic wave,
\begin{equation}
\label{ch2:bt_bc_vel}
\vec{u}_{bt}(x,y) = \frac{1}{H} \int_{-H}^{0} \vec{u}(x,y,z)  dz,~\vec{u}_{bc}(x,y,z) =  
\vec{u}(x,y,z) - \vec{u}_{bt}(x,y)
\end{equation}
To obtain distribution of pressure, at first, the linear equation of state and respective 
\textit{TS}-fields are used to determine density perturbations. At second, via the hydrostatic 
approximation consequent vertical integration leads to total pressure field. This 
is then subject to baroclinicity condition, stating that baroclinic pressure anomaly associated 
with internal tides is taken to be a deviation from the depth-averaged,
\begin{equation}
\label{ch2:bt_bc_pres}
p(x,y,z) = \int_{-z}^{0} \rho(x,y,z) dz,~p_{bc}(x,y) = p(x,y,z) - \frac{1}{H} \int_{-H(x,y)}^{0} 
\rho(x,y,z) dz
\end{equation}
In the both expressions rigid-lid approximation is used. This is a valid assumption unless 
vertical accelerations are greater than acceleration due to gravity \citep{kelly2010}.\\
Vertical distribution of each dynamical variable was then decomposed into vertical modes. The 
structure functions were 
obtained from local Brunt-Vaisala frequency profiles found from time-averaged density fields. 
These were used in Sturm-Liouville problem under the hydrostatic approximation,
\begin{equation}
\frac{d}{dz}\Big( \frac{1}{N^2}  \frac{d \psi(z)}{dz}\Big) + c^{-2}_n \psi(z) 
= 0
\end{equation}
where $c_n$ is the mode phase speed in non-rotating ocean. The first 3 eigenmode structural 
function obtained per each numerical grid point were used to describe respective vertical 
distribution in the dynamical fields. And further presented results focus only on mode-1 energetics 
as of its primary importance.\\
Here the low mode is characterized by rates of conversion from barotropic tide to baroclinic 
\citep{simmons2004internal, kurapov2003m}, depth-averaged energy flux, horizontal kinetic energy 
(HKE) and available potential energy (APE) that were found as
\begin{align}
\label{C3.eq:conv}
C_{bt\to 1} = -\frac{1}{2}(\cj{\vec{u}_{bt}} \cdot \nabla H) p_{1}|_{z = -H(x,y)}\\
\vec{F} = \frac{1}{2} \frac{1}{H} \cj{\vec{u}_1} p_1 \int_{-H}^{0} \psi_1(z) \psi_1(z) dz\\
HKE = \\
APE = 
\end{align}
The fractional coefficient appears because complex amplitudes are used in the expressions.\\

Additionally, the mode-1 internal tide field was subject to directional analysis in order to 
extract signals associated with the tidal beam. The details are presented in appendix 
\ref{C3.app:A1} and summarized as follows. Dynamical fields of pressure and currents sampled over 
finite number of grid points can be described by directional spectra. Its circular Fourier 
transform is then obtained as solution by damped least squares of an ill-posed inverse problem. The 
spectral, directional analysis is different from the approach of \cite{zhao2010long} for two main 
reasons. The underlying inverse model equations \eqref{C1:p.eq} produce simultaneous fit to all 
compass directions, rather than a finite number of plane waves. This makes a difference in regions 
where wave diffraction prevails. For instance, wave field near sites of internal tide generation or 
scattering will be comprised by angular distribution of energy confined to diffractive lobes 
\citep[e.g.,][]{munroe2005topographic, johnston2003internal} rather than a single plane wave. On 
the other hand, because of finite-window sampling artificial lobes are introduced as in classical 
Fourier analysis. The analysis artifacts lead to unambiguous interpretation, so judgment should be 
used when wave fields are back-synthesized over sampled angular sectors. Secondary, 
velocity measurements are utilized along with pressure observations \eqref{C1:uv.eq}. The 
formulation provides additional constrain on the model. Indeed, the velocity equations tend to 
lessen Gibbs phenomena. More importantly, fitting of directional spectra to current and pressure 
observations can be rewritten for point-wise measurements such as a stationary mooring.\\

All of the above calculations produced sets of dynamical variables for barotropic, baroclinic 
fields and directional representations of the lowest mode for each experiment. Per experiment 
values hereafter will be referred as a realization. For instance, the longest 
experiment, 2015\_TTIDE had 10 realizations and in total it was 16 realizations. To study 
variability in the tidal beam, mean energy characteristics were defined by simple arithmetic mean 
(ensemble-mean) and range of variability - by standard deviation. Such definitions have a caveat. 
While ensemble-mean allows a straightforward interpretation of mode-1 dynamics, it will incorporate 
a portion of variable (non-stationary) signal since any energy characteristic is a nonlinear 
quantity \citep{zaron2014time}. On contrary, for instance, a flux found from ensemble-mean pressure 
and currents will exclusively provide a stationary part. Yet if a mode-decomposed signal is 
considered, finding the means will entail averaging of vertical basis functions. As a consequence, 
their orthogonality will not be preserved leading to uncertainty in dynamics. Performed 
comparison (not shown) between an ensemble-mean of flux and a flux of ensemble-mean did not reveal 
significant differences in spatial structure, though energy of ensemble-mean signals was smaller. 

\section{Results}
\subsection{Internal tide production}
The semidiurnal barotropic tide propagates into the Southern Tasman Sea from the North 
\fignmp{C3.fig:BT}. Its phase increases in counterclockwise manner with maximum amplitude of sea 
level located along New Zealand's coast, behavior typical of Kelvin wave \citep{walters2001ocean}. 
The strong tidal phenomena is also observed in shallow Bass Strait, but relatively weak anywhere 
else in the basin. The simulated sea surface tidal oscillation closely follows TPXO7.2 with gross 
features well captured. But the sea level magnitude and cotidal (phase) lines exhibit wave-like, 
small scale perturbations. This is a manifestation of low-mode baroclinic waves 
\fignmp{C3.fig:beam}. The baroclinic tidal field appears as a complex pattern produced by 
multiple generation sites defined by steep topography. The primary sites are just south of 
New Zealand. Here the barotropic Kelvin wave faces the Macquarie Ridge stretched for 2000 km. As 
the barotropic current decays away from the coastline baroclinic tide production decreases as well. 
But low-mode beams are emitted from many locations. Henceforth, analysis concerns the 
most energetic beam emitted from a hotspot at $49.5^{\circ}$ which hereinafter referred to as the 
Macquarie Ridge.\\

The major beam is produced by conversion of the barotropic tidal currents impinging on 
supercritical bathymetry \fignmp{C3.fig:gen}. The depth of the strongest conversion lies between 
1000-3000 m and spatially confined to two seamounts that are separated with a gap. This valley 
topography has more gradual slopes and should play a secondary role in production of the tidal 
beam. But from distribution of conversion rates \fignmlp[C3.fig:gen]{b} and their spatial averages 
\fignmlp[C3.fig:gen]{c} a different perspective emerges. Generation at the gap makes up a 
significant portion of the converted energy and displays high variability between the 
experiments. Overall, the Macquarie ridge converts $1.8 \pm 0.2~GW$ of the surface tide into mode-1 
baroclinic tide. This is a half of production at Kaena Ridge, Hawaii \citep{carter2008energetics} 
and tenfold smaller than Luzon Strait \citep{kerry2014impact}. But similar to Luzon Strait the 
Macquarie ridge features complex internal tide dynamics that is manifested as regions of negative 
conversion (green shading, \fignml[C3.fig:gen]{b}). It is known \citep{Kelly2010a} that such 
reverse energy transfer and destruction of internal tide is forced by coupling with remote 
baroclinic waves. In fact, oppositely located slopes of the Campbell Plateau (Auckland Escarpment) 
provides a source of such signal. Further interference creates a standing wave 
\fignmp{C3.fig:stand_wave}. Its presence is evident in the distribution of HKE marked by a node in 
the middle of Solander trough. Additionally, around the node energy flux forms a circular pattern  
\fignmp{C3.fig:stand_wave} with counterclockwise progression since Coriolis parameter is negative 
in the Southern Hemisphere.\\

By the directional method (appendix \ref{C3.app:A1}) the standing wave is separated 
into elemental east-west oriented components \fignmlp[C3.fig:stand_wave]{b, c}. On the leeward side 
of the Macquarie Ridge generation takes place at the same seamounts but there is a region of 
destruction that coincides with incidence of waves emitted from the escarpment. On the trough's 
opposite side generation has richer structure both due to more complicated topography but also due 
to effect of the Macquarie ridge waves on the local generation. Respective energy transfer across 
the trough was quantified and compared with spatially integrated conversion rates 
\fignmlp[C3.fig:stand_wave]{d}. On average, energy of each elemental wave field equally divided 
amongst waves reflected from the supercritical slopes and wave energy derived from the barotropic 
field. Though the system exhibits large departures from the average state due to variable 
conditions of the coupling on the slopes.\\

The described internal tide dynamics is similar to Luzon strait where resonance conditions exist 
between two almost parallel ridges \citep{buijsman2012double, buijsman2014three} separated by a 
distance of 0.6 mode-1 wavelength. In the case of the Macquarie Ridge and the Campbell Plateau 
resonance should be weaker since the former has an orientation of $15^{\circ}$ relative to the 
latter. As well the separation distance at $49.5^{\circ}$ is 3/4 of mode-1 wavelength. These 
parameters suggest relevance of wave's travel-time to the observed variability in baroclinic tide 
production levels and its spatial pattern. To examine this the internal tide dynamics is further 
studied along a two-dimensional transect crossing the gap (horizontal line on 
\fignml[C3.fig:gen]{b}) in simulations with diametrically different generation regimes (\sqm{2014} 
and \sqm{d10-15, 2015}, \fignml[C3.fig:gen]{c}, \fignmlp[C3.fig:stand_wave]{d}). Consequent 
results are similar to findings of \citep{echeverri2010internal, buijsman2012double} with 
exception that here the decomposition method provides a complete description of variations in 
mode-1 dynamics.\\

\subsection{Increase in conversion rate}
Vertically integrated conversion rate (\ref{C3.eq:conv}) quantifies how much work was done by the 
baroptropic tide in displacing isopycnal interfaces against buoyant forces throughout a water 
column\footnote{Following this concept a derivation from the first principles is provided in 
appendix \ref{C3.app:B} \citep[see also]{kurapov2003m, kelly2010topo}.}. It is recognized 
that the barotropic forcing might be oppositely directed to vertical motions if somewhere in the 
water column other processes take place. For case of \sqm{2014} simulation 
\fignmlp[C3.fig:gen_2d]{a, b} during ebb tide on the tideward side (solid black line, 
$164.2^{\circ}$) there is vertically net energy conversion into the baroclinic field 
\fignmlp[C3.fig:gen_2d]{e}. But note along the bottom an internal wave ray is developing as 
isopycnals lift following preceding flooding stage. Hence, at this location and depths the locally 
generated internal wave does work against the downward barotropic flow and the contribution is 
increasing with time. At about hour 7 on \fignml[C3.fig:gen_2d]{e} the net conversion becomes 
negative, but then flips sign as the tidal current reverses direction \fignmlp[C3.fig:gen_2d]{b}. 
Averaged over tidal period the transfer is positive, i.e. the surface tide losses energy, but weak 
compare to \sqm{d10-15} simulation \fignmlp[C3.fig:gen_2d]{c, d}. This experiment demonstrates 
enhancement of production on the gap's slope due to presence of the remotely generated internal 
tide. Its propagation from Solander Trough ($165^{\circ}$) and across the gap is the major 
difference between the two. In \sqm{d10-15} along slope formation of an internal wave is still 
found, but now due to the shoaling mode-1, conversion is positive throughout the full tidal cycle. 
In summary, the remote signal supports positive energy transfer during periods of tidal current 
reversals, so that excess of energy is drawn from the surface tide into the baroclinic field 
\fignmlp[C3.fig:gen_2d]{e}.\\

Clearly, phase difference between localized barotropic forcing and remote, incident signal is a 
crucial parameter in describing the system dynamics which is illustrated by the leeward side (black 
dashed line, $164.5^{\circ}E$). Here the vertical barotropic flow is opposite to the previous 
situation. For \sqm{d10-15} the traveling mode-1 internal tide is losing energy as it displaces 
isopycnals oppositely to the forcing. In actuality, mode-1 reflection takes place as an internal 
wave ray is emitted towards the trough. The associated baroclinic currents are easterly downward, 
same direction as group speed \fignmlp[C3.fig:gen_2d]{c-d}. In \sqm{2014} the along slope current 
is opposite, coinciding with the isopycnal displacements and the flow of barotropic tide. Thus, in 
both cases the energy propagates downward and away from the slope, but the corresponding phases are 
$180^{\circ}$-shifted denoting either reflection of the internal tide or unobstructed generation 
respectively \fignml[C3.fig:gen_2d]{e}. The period-averaged conversion rates further highlight the 
differences \fignmlp[C3.fig:gen_2d]{f}. Overall, the simulations demonstrate three distinct cases 
of energy transfer between barotropic and baroclinic modes on a sloping bottom and pathway taken 
is driven by the remote wave. Its influence is further demonstrated by exploring phase distribution 
along the section \fignmlp[C3.fig:gen_2d]{g} in mode-1 pressure field.\\

The total phase shows a stagnant region eastward from of $165^{\circ}E$ that corresponds to the 
node in HKE (\fignml[C3.fig:stand_wave]{a}, dashed line marks the considered transect). As before 
the decomposition reveals it as a sum of the ridge-generated waves (eastward propagation) and the 
escarpment-originated waves (westward). West of $165^{\circ}E$ there is almost a free propagation 
in \sqm{d10-15} of the ``escarpment" waves as the total signal is similar, though some eastward 
contribution is present as a result of the reflection. At the previously considered locations 
mode-1 phase is above $-90^{\circ}$ (black line) which results respectively in enhancement or 
destruction on the gap's sides. Note that the phases are referenced relative to the barotropic 
flow on the tideward side. In \sqm{'2014'} on the leeward side the total signal is mostly described 
by the eastward waves and the phase tends to $-180^{\circ}$, hence, the wave is mainly related to 
the local generation. On the tideward side both signals are in the same quadrature as the local 
barotropic flow, but in \sqm{'d10-15'} there is a twenty-fold increase in production levels 
associated not only with higher bottom pressure, but also because of small phase lag with the 
barotropic forcing.

\section{Discussion on Variability}
\subsubsection{Three dimensional structure of wave fields}
Generation at the gap strongly depends on magnitude and phase of the incident, westward wave. 
Thus, the variability in conversion is linked to three dimensional structure of the wave field. 
This is illustrated in \fignm{C3.fig:wv_fld_dist} where decomposition results for three 
experiments are shown. In case of \sqm{2014} the westward wave is weak due to reduced generation at 
the Campbell Plateau (see also \fignml[C3.fig:gen]{c}). More importantly, lines of constant phase 
are slanted relative to the Macquarie Ridge leading to insignificant energy flux across the gap. In 
contrast, for \sqm{d10-15} there is substantially more cross-ridge energy flux. Note that the 
dynamical state is not permanent as adjustments in the incident wave field bring new changes such 
as seen in the results a month later, \sqm{d50-55} \fignmlp[C3.fig:wv_fld_dist]{c}. The incidence 
has changed and the wave arrives earlier driving more intense generation at the gap as well as the 
ridge.\\

The westward wave field is defined by dynamics occurring on the slopes of the Campbell Plateau 
where reflection couples with local generation. In these experiments it is characterized by highly 
variable spatial pattern of generation/destruction. For \sqm{2014} there is a strong destruction 
evident near $(166.5E, 48.75S)$, \fignmlp[C3.fig:wv_fld_dist]{a}, while in the other cases 
magnitude 
of energy loss is smaller. The location corresponds to impingement of an eastward traveling wave 
produced at the Macquarie 
Ridge \fignmlp[C3.fig:wv_fld_dist]{bottom row}. And its phase lag relative to local barotropic tide 
is a primary factor. That is, earlier arrival such as in \sqm{'d10-15'} and 
\sqm{'d50-55'} corresponds to weak energy losses and overall higher production rates (see 
\fignml[C3.fig:gen]{c} for actual values). This is especially true in \sqm{d10-15} as large 
portions of the Auckland Escarpment demonstrate substantial increase in the conversion.\\

It is straightforward to establish a relation between the phase of the eastward waves and mode-1 
eigenspeed under extreme conditions. For example, \sqm{2014}-experiment exhibits $15\%$ slower 
progression, i.e. a $\sim 20^{\circ}$ or $40$ minute later arrival. The same holds for 
\sqm{uniform}-experiment where higher phase speed leads to much earlier arrival and very strong 
enhancement (compare \fignml[C3.fig:gen]{b} with \fignml[C3.fig:wv_fld_dist]{upper 
row}). But in the cases of modest anomaly, such interpretation becomes complicated with initial 
phase of the eastward waves. The field emitted by the northern seamount of the 
Macquarie Ridge initially have phase of about $0^{\circ}$ in \sqm{d10-15}, while in \sqm{d50-55} - 
$\sim 50^{\circ}$. In turn, the difference arises from variable incidence and arrival of the 
westward waves. Their characteristic strongly depends on complicate pattern of multiple site 
generation at the Campbell Plateau rather than anomaly in phase speed. This is contrary to the 
eastward waves that are produced by much simpler generation pattern. And since the phase lines 
closely repeat the emitting seamount's topography, a plane wave description is fairly appropriate 
and hence, facilitates linear phase dependence on inverse of the phase speed.\\

\subsubsection{Conversion rates}
The presented evidences state importance of mutual influence in the generation/destruction 
pattern on the slopes of Solanders trough in controlling magnitude of the internal tidal beam. It 
also seems that stationary regime can be hardly observed. Changes in stratification and phase 
speeds control lags in the arrival times to affect coupling with the barotropic tide and strength 
of the westward waves.\\

This conclusion is qualitatively illustrated by \fignm{C3.fig:gen_sill}. Variability of 
generation at the gap explains $75\%$ of variability in the overall production at the Macquarie 
ridge. While the generation by the gap is mainly related to strength of the incident, westward 
wave. In this case its amplitude is quantified as integral of energy flux across grey line on 
\fignml[C3.fig:wv_fld_dist]{f} and given by black line on \fignml[C3.fig:gen_sill]{a}. Correlation 
coefficient between the generation and the flux is $R^2 = 50\%$, but if \sqm{uniform} experiment is 
excluded, $R^2$ rises to $80\%$. The \sqm{uniform} experiment is an outlier as uniformly prescribed 
stratification typical of the Tasman basin led to unrealistically high phase speeds in the 
Macquarie ridge region. The westward wave arrived in-phase with the local barotropic tide on the 
lee side of the ridge and out-phase on the western side \fignmlp[C3.fig:gen]{b}. This regime is 
clearly opposite to other experiments \fignmlp[C3.fig:gen]{c}.\\

Effect of local changes in stratification has also to be investigated since the region is located 
in highly dynamical ocean environment.  To account for this, an analytical model 
\citep{st2003generation} is applied. The ridge is thought to be a two-dimensional knife-edge 
topography whose height is found from the topography's mean depth in WKB-scaled vertical coordinate 
\citep{althaus2003internal} relative to the abyssal depth. Under this description in strongly 
stratified ocean the knife-edge ridge is taller and sheds more baroclinic energy. Surprisingly, the 
calculation \fignmlp[C3.fig:gen_sill]{b} identifies converse dependence. The inconsistency is 
resolved by considering effect of stratification on energy transmission across the Macquarie ridge. 
To estimate the transmission coefficient the ridge's height ratio is now obtained relative to 
Solander Trough. Than using closed-form solutions of \citep{larsen1969internal} the coefficient is 
calculated and used to scale the overall incident flux (integrated across dashed line on 
\fignml[C3.fig:stand_wave]{c}). \fignmlp[C3.fig:gen_sill]{c} suggests the changes in stratification 
affect ridge's transmission and taller ridge allows less across-ridge transport. The result does 
not exclude the implied role of stratification, but rather emphasizes the primary mechanism of 
enhanced generation due to the remote wave.\\

Nevertheless, actual amount of the remote wave crossing the ridge cannot be accurately established. 
As presence of complex bathymetry and two distinctly different wave fields on the opposite sides of 
the ridge introduce large errors in the decomposition method. It is also apparent that the 
calculation based on the knife-edge theory seems to be an underestimate. For instance, 
using the equations of \citep{larsen1969internal} transmission across the gap alone is about 
$15\%$, but such amount contradicts largely subcritical slope, while amplitude of the transversing 
mode-1 appears to change only slightly \fignmlp[C3.fig:gen_2d]{c-d}. But from the previously 
described mechanism (Section 3.2) another estimate can be made. The incident mode-1 energy loss is 
mainly takes form of work against the local barotropic tide, so the expression for conversion rate 
\eqref{C3.eq:conv} can be applied, i.e.
\begin{equation}
F_{sill} - F_{trough} = \nabla \vec{F}_W = C_W = -\frac{1}{2}(\cj{\vec{u}_{bt}} \cdot \nabla H) 
p_{1,~W}|_{z = -H(x,y)}
\end{equation}
where $p_{1,~W}|_{z = -H(x,y)}$ is mode-1 baroclinic pressure associated with the westward wave 
only. It must be emphasized that the expression is not correct as reflection is not explicitly 
accounted for and high mode scattering is omitted. Regardless of the caveat, the spatially averaged 
losses to barotropic tide from the incident, westward wave \fignmlp[C3.fig:gen_sill]{a} draw more 
realistic estimate of the transmission coefficient. The gap alone transmits $47 \pm 15~\%$. And 
average for the Macquarie ridge is $40 \pm 10~\%$ which is clearly an overestimate because of the 
assumptions. The variability is quite notable. Following previous section it is dictated for 
the gap mainly by the westward wave arrival. The transmission across the seamounts is controlled 
along with the westward wave phase by its incidence and additionally, by ocean stratification.\\

\section{Conclusion}

\section*{Appendices}

\renewcommand{\thesubsection}{\Alph{subsection}}
\setcounter{subsection}{0}
\subsection{Application of directional spectra to internal tidal fields}
\subsubsection{Antenna analysis}
\label{C3.app:A1}
Similar approaches were used previously in internal tide field programs 
\citep{hendry1977observations, lozovatsky2003spatial} \footnote{that were based on array 
beamforming method and stationarity of the field} or satellite altimetry \citep{dushaw2002mapping} 
or in surface wave studies \citep{longuet1961observations, munk1963directional, long1986inverse}. 
Let pressure in complicated seas to be described by an angular spectrum
\begin{equation}
\label{C1:eq.spectrum}
p(\vec{r}, t) = \int_0^{2\pi}  d\theta_k S(\theta_k) e^{i \vec{k}(\theta_k) \cdot \vec{r} + 
	\phi(\theta_k) - i \omega t}
\end{equation}
Here each monochromatic wave of wavenumber $k$ travels in direction $\theta$ with 
energy $S(\theta)^2 d\theta$ and phase of $\phi(\theta)$. Now time dependence is dropped and all 
physical quantities are given by corresponding complex amplitudes. The statement can be 
reformulated in terms of Fourier transform \citep{munk1963directional}. Recall Jacobi-Anger 
expansion,
\begin{equation}
p(r, \theta) = e^{i \vec{k}(\theta) \cdot \vec{r}} = \sum_{m = -\infty}^{m = \infty} i^{m} J_{m}(k 
r) e^{im(\theta - \theta_k)}
\end{equation}
that shows a field at point $(r, \theta)$ produced by plane wave can be expanded in series of 
Bessel functions and circular functions. Then substitution produces
\begin{equation}
\label{C1:p.eq}
p(r, \theta) = \sum_{m=-\infty}^{m=\infty} J_m(kr) e^{im(\theta + \pi/2)} \Big\{ \int_0^{2\pi}  
d\theta_k 
S(\theta_k) 
e^{i\phi(\theta_k)} e^{-im\theta_k} \Big\}
\end{equation}
Term in brackets represent convolution integrals defining circular Fourier coefficients of 
order $m,~A_m - i B_m$. Series \eqref{C1:p.eq} states an inverse problem that seeks the 
unknown coefficients from the known, measured pressure field that is sampled at a set of points 
$(r_i, \theta_i)$ and if infinite series is truncated at some order $N$. Real and imaginary parts 
will constitute two separate problems.\\
The same steps are undertaken for velocities as plane wave polarization relations 
\citep[e.g.,][]{muller2000scattering} are utilized. The following is found
\begin{align}
\label{C1:uv.eq}
\begin{Bmatrix}
u \\ v
\end{Bmatrix}
= \frac{1}{2} \sum_{m = -N}^{m = N} J_{m} (kr) e^{im(\theta + \pi/2)}
\begin{Bmatrix}
(\omega - f) A_{m + 1} + (\omega + f) A_{m - 1} - i [(\omega - f) B_{m + 1} + (\omega + f) B_{m - 
	1}] \\ 
(\omega - f) B_{m + 1} - (\omega + f) B_{m - 1} + i [ (\omega - f) A_{m + 1} - (\omega + f) A_{m - 
	1}]
\end{Bmatrix}
\end{align}
The dependence of currents on wave bearing causes splitting of Fourier coefficients. To describe 
velocity field higher circular harmonics have to be used. Or physically, velocities have higher 
spatial wavenumber. As well \eqref{C1:uv.eq} emphasize an asymmetry between clockwise 
and counterclockwise motions due to Coriolis effect. Now formally the inverse problem can be stated 
\begin{equation}
y = K x
\end{equation}
where $y$ is comprised of measured dynamical variables, $K$ combines bracketed terms in 
\eqref{C1:p.eq} and \eqref{C1:uv.eq} and $x$ are the unknown Fourier coefficients. Generally, the 
problem is over-determined (i.e., number of unknowns is smaller than number of 
measurements) and unstable to small errors in data, so the model yields physically inconsistent 
results. This is circumvented by seeking a damped least square solution \citep{munk2009ocean} where 
a minimization function is 
\begin{equation}
\label{C1:Tikh_prob}
J = ||K x - y||^2_2 + \alpha ||x||^2_2
\end{equation}
The unknown regularization parameter $\alpha$ acts as a high-pass filter in a singular value 
decomposition of $K$ \citep{bennett1992inverse}. In field studies this is usually set by a 
signal-to-noise ratio \citep{munk2009ocean}, since the parameter scales noise variance (residue) 
to actual signal's strength. To obtain $\alpha$ in data-driven way a straightforward approach is 
adapted that based on 
trade-off curve method \citep{hansen1993use}. In \eqref{C1:Tikh_prob} amount of allowed error 
is competing with solution's variance. An optimal parameter should balance these factors. This is 
seen as a rapid change in behavior of curve associating residue with model's norm as regularization 
varies. In most cases the curve has a sharp corner connecting aforementioned limits, hence, the 
method's name is a L-curve \citep{hansen1999curve}. And the corner is thought to occur for an 
optimal regularization parameter.\\
The equations \eqref{C1:p.eq} and \eqref{C1:uv.eq} are sampled at locations along concentric 
arrays placed at $\lambda,~0.5\lambda,~0.25\lambda$ where $\lambda$ is local mode-1 wavelength. 
At each location $u,~v,~p$ are used as data and for a region embraced by array Fourier coefficients 
are found. And these then are used in reconstructions.
\subsubsection{Point-wise analysis}
\label{C3.app:A2}

\subsection{Derivation of conversion from the first principles}
\label{C3.app:B}
While \eqref{C3.eq:conv} provides a convenient way to quantify energy transformations, it does 
not have much room for physical interpretation. Such as in complicated situation when along with 
local baroclinic tide production a remote signal is present. The resultant perturbation of bottom 
pressure might lead to rather ambiguous result of internal tide destruction. It is said that such 
regime is to occur when a phase difference between $w_{bt}$ and $p_{1}$ is in range of 
$(\pi/2,~3\pi/2)$. This is understood as an internal tide performing work against barotropic 
forcing. Unfortunately, this statement does not directly follow from \eqref{C3.eq:conv}. Hence, to 
provide cleaner physical picture, let derive the expression for conversion rate from the first 
principles. Body force of \citep{baines1982internal} is performing work by displacing isopycnal 
surfaces 
throughout a water column,
\begin{equation}
\label{C3:eq.convd1}
C(z) = \frac{dW(z)}{dt} = F_{B} w_{bc} = \frac{N^2 (-\vec{u} \cdot \nabla h) z}{i \omega h} 
\frac{d 
\xi}{dt} = \frac{w_{bt}}{i \omega h} \frac{z d(-b)}{dt} = -\frac{1}{h} w_{bt} zb
\end{equation}
where isopycnal displacements $\xi$ were changed to buoyancy, $b = -N^2 \xi$ and temporal 
variation 
was assumed to be harmonic, $\sim e^{i \omega t}$. On a final step, integration by parts can be 
employed as,
\begin{equation}
\label{C3:eq.convd2}
\int_{-h}^{0} z b dz = \int_{-h}^{0} z d \big( \int^0_{z} b dz \big) = \big( z \int^0_{z} b dz 
\big)\big|_{-h}^0 - \int_{-h}^{0} dz \int_{z}^{0} b dz^{\prime} = h (\int_{-h}^{0}b dz - 
\frac{1}{h} \int_{-h}^{0} dz \int_{z}^{0} b dz^{\prime})
\end{equation}
The last expression is a bottom pressure perturbation. Noteworthy, a baroclinicity condition was 
not employed. Combining \eqref{C3:eq.convd1} and \eqref{C3:eq.convd2}, familiar result 
for conversion rate is obtained. Note that a similar approach to \eqref{C3:eq.convd1} was used 
by \citep{nash2006structure} to estimate an upper limit on emitted energy.\\

\newpage
\section{Figures}

\begin{figure}
	\centering
	\includegraphics[scale = 
	0.35]{/home/dmitry/Work/Research/thesis/FINALE/P3_ITS_GENERATION/figures/fig_1_map.png}
	\caption{Domain of numerical simulations with geographical locations used throughout the text.}
	\label{C3:fig:geo.map}
\end{figure}

\begin{figure}
	\centering
	\includegraphics[scale = 
	0.5]{/home/dmitry/Work/Research/thesis/FINALE/P3_ITS_GENERATION/figures/fig_1_BT_tide.png}
	\caption{Comparison of $M_2$ sea level oscillations simulated by ROMS (left panel) with 
	TPXO-model (right panel).}
	\label{C3.fig:BT}
\end{figure}

\begin{figure}
	\centering
	\includegraphics[scale = 
	0.5]{/home/dmitry/Work/Research/thesis/FINALE/P3_ITS_GENERATION/figures/fig_2_uni_flux.png}
	\caption{Beam of Tasman Sea with major internal tide production sites identified by superposed 
	heat map.}
	\label{C3.fig:beam}
\end{figure}

\begin{figure}
	\centering
	\includegraphics[scale = 
	0.5]{/home/dmitry/Work/Research/thesis/FINALE/P3_ITS_GENERATION/figures/fig_3_gen.png}
	\caption{Generation of internal tides at Macquarie Ridge. (a) The heatmap illustrates 
	criticality in the region of major generation. And ellipses are representative of barotropic 
	current. (b) The diagnosed conversion rates for 'uniform' experiment. The boxes outline regions 
	used in further analysis. (c) Variability of conversion rates in the three regions identified 
	on the previous panel.}
	\label{C3.fig:gen}
\end{figure}

\begin{figure}
	\centering
	\includegraphics[scale = 
0.5]{/home/dmitry/Work/Research/thesis/FINALE/P3_ITS_GENERATION/figures/fig_4_stand_wave_2by2.png}
	\caption{Standing wave between Macquarie Ridge and Auckland Escarpments. (a) Distribution of 
	horizonal kinetic energy in 'uniform' simulation with overlaid energy flux of total mode-1 
	field. (b) Energy flux map of the westward component and the eastward component (c). Variation 
	in energy characteristics of mode-1 field between Macquarie Ridge and Aucklands Escarpment. The 
	integrated flux was obtained for a 	dashed transect in (c) and is given by arrows. Dots show 
	conversion rates spatially integrated over boxes given on (\fignm{C3.fig:gen}, b).}
	\label{C3.fig:stand_wave}
\end{figure}

\begin{figure}
	\centering
	\includegraphics[scale = 
	0.5]{/home/dmitry/Work/Research/thesis/FINALE/P3_ITS_GENERATION/figures/fig_5_2d_section_sill.png}
	\caption{Distribution of baroclinic fields during ebb and slack tide along transect crossing 
	Macquarie Ridge on \fignm{C3.fig:gen}, (b). (a-d) Baroclinic pressure anomaly with superposed 
	isopycnal displacements given by contour lines. Positive color is assigned to lifted interfaces 
	and negative - for the opposite. Stick lines show currents. And white line is distribution 
	of barotropic velocity. The tideward and leeward sides of Macquarie Ridge are identified by 
	vertical solid and dashed lines. On (e) it is shown time progression of 
	baroptropic-to-baroclinic conversion for both experiments and ridge sides. While (f) gives 
	distribution of period averaged conversion rate along the transect. The same is for (g) where 
	progression of mode-1 baroclinic pressure is shown by phases of total field and elemental 
	components. The phases are referenced to flood current across the sill.}
	\label{C3.fig:gen_2d}
\end{figure}

\begin{figure}
	\centering
	\includegraphics[scale = 
	0.5]{/home/dmitry/Work/Research/thesis/FINALE/P3_ITS_GENERATION/figures/fig_6_flux_examp.png}
	\caption{Energy and phase distribution in the decomposed wave field. The upper row shows the 
	westward component and bottom row - eastward component. Respective energy fluxes are shown by 
	black arrows. And distribution of phase is given by blue lines separated by 
	$50^{\circ},~1.75~h$. To emphasize differences isochrons for $90^{\circ}$ and $180^{\circ}$ are 
	designated by red lines. Note that the components are referenced relative to phase of the 
	barotropic tide at the arrival topography, i.e. the westward component  - relative to the 
	Macquarie Ridge and the eastward component - to the Auckland Escarpment. On the upper row 
	background shading shows distribution of conversion rates and on the lower row - mode-1 
	eigenspeed anomaly.}
	\label{C3.fig:wv_fld_dist}
\end{figure}

\begin{figure}
	\centering
	\includegraphics[scale = 
	0.5]{/home/dmitry/Work/Research/thesis/FINALE/P3_ITS_GENERATION/figures/fig_7_sill_generation.png}
	\caption{(a) Variation of generation at the Macquarie Ridge, gap. As well energy incident flux 
	from Solanders Trough. Estimate of energy loss from the incident and transmitted. (b) Relation 
	of the overall production and generation by the knife-edge barrier. (c) Relation of the 
	cross-ridge energy transport to the overall production.}
	\label{C3.fig:gen_sill}
\end{figure}

%\begin{figure}
%\centering
%\includegraphics[scale = 
%0.75]{/home/dmitry/Work/Research/thesis/FINALE/P3_ITS_GENERATION/figures/fig_11_example_beam_mesoscale}
%\caption{Example of refraction. }
%\label{C3.fig:fig_11_beam_mesoscale}
%\end{figure}
%
%\begin{figure}
%\centering
%\includegraphics[scale=0.75]{/home/dmitry/Work/Research/thesis/FINALE/P3_ITS_GENERATION/figures/fig_12_beam_var_cp}
%\caption{}
%\label{C3.fig:fig_12_beam_var_cp}
%\end{figure}

%\begin{figure}
%	\centering
%	\includegraphics[scale = 
%	
%0.5]{/home/dmitry/Work/Research/thesis/FINALE/P3_ITS_GENERATION/figures/fig_11_along_beam_inv.png}
%	\caption{Variation of beam parameters.}
%	\label{C3.fig:beam_alng_var}
%\end{figure}

%Overall, combined discussion of the numerical experiments and field observations suggests that the 
%far field signal is affected both by generation levels at the Macquarie Ridge and mesoscale 
%conditions. The former process mainly factors magnitude of the incident beam, while mesoscale can 
%largely refract beam. The numerical experiments given here were capable to predict gross, 
%time-averaged characteristics of the beam and degree of temporal variability. 
%
%Nevertheless,  This interaction was mainly taken place in February of 
%2015 which corresponds to reorientation 
%Even though the given here result is 
%probably not correct, but from comparison of the simulated wave field 
%
%Another reason for variation is change in the reflected wave. So that increase 
%leads to additional 
%modulation. Even though results are not that trustful in the reflected part, they simple 
%retranslate change in the ratio. So that the first period before day 40 when strong reflected 
%signal was leading to developing strong interference pattern and large deviation from the plane 
%wave value. At the end of that period, decrease in reflected wave was producing more plane wave 
%value. Such variations could be related to changes in beam incidence and reflection at East Tasman 
%Plateau and Tasmania. There is a similar pattern that follows from changes. So more when incidence 
%happens strongly in westward direction, amount of reflected wave is less. And for northward there 
%is more relfective component. The total energy flux orientation in the numerical model does not 
%stand close to the observed.
%
%%provide ideas on the Tasman tidal beam behavior. The former dataset comprises highly averaged 
%%both 
%%in time and space baroclinic tide dynamics. \fignm
%
%WE intend to compare with two datasets available. Satelite altimetry emphasizes highly temporal 
%and 
%spatially smoothed observations. But field observations highly hetergoneous point-wise 
%observations.\\
%At first, satelite altimetry shows more northward located beam compared to the numerical 
%experiments. 
%Further, the phase speed deviations were superposed with the position of the beam (Fig). 
%The previous result on \fignm{C3.fig:beam_dcmp_cm} can now be discussed further. Now it is shown 
%distribution of semi major axis of variance ellipse \fignm{C3.fig:fig_12_beam_var_cp}. Here it is 
%clearly seen that primary variations are concentrated in regions downstream that are affected by 
%mesoscale. Additional region is next to the Macquarie Ridge due to generation. But further it is 
%impossible to propagation quite rapidly becomes affected by variable ocean conditions. There are 
%three regions one next to the ridge, than in the basin. And also by the ETP there is strong 
%interaction causing beam to deflect.\\
%
%
%
%This rein
%In the current 
%These processes are the most important. And 
%refraction seems to be the major player. Refraction is a well-known phenomena in surface gravity 
%waves. In respective theoretical works the directional spectrum plays a pivotal role. Here these 
%approaches are used to diagnose the spectrum. This is carried out by computing two spectral 
%moments,
%\begin{align}
%\label{C3:eq.m.spectra}
%\bar{\theta} = \arctan \frac{\int_{\pi/2}^{\pi} d \theta S \sin \theta}{\int_{\pi/2}^{\pi} 
%d \theta S \cos \theta }\\
%\sigma_\theta = 2\bigg[ 1 - \sqrt{\Big( \frac{\int_{\pi/2}^{\pi} d \theta S \sin 
%\theta}{\int_{\pi/2}^{\pi} d 
%\theta S} \Big)^2 + \Big( \frac{\int_{\pi/2}^{\pi} d \theta S \cos \theta 
%}{\int_{\pi/2}^{\pi} d \theta S}\Big)^2}  \bigg]
%\end{align}
%where $S=S(\theta)$ - directional spectrum at each grid point obtained previously. These moments 
%simply show mean direction in the spectrum and the spreading of waves around the mean. Clearly, if 
%there is any refraction than the mean direction will undergo changes. But the spectrum spreading 
%will also change as close-to-each wavenumbers will experience slightly different adjustments 
%leading either to more spreading as wave group propagates travels into a region with 
%higher phase speeds than surrounding or focusing in the opposite case.\\
%
%Over the field period simulation there was a continuous change in the beam as mesoscale field was 
%evolving. Over the first period, Januaries 10-15, 2015 (Fig), the beam underwent the most striking 
%adjustment. Clearly, some of the waves were refracted equatorward, this portion then interfered 
%with the rest producing a horizontal reorientation of the fluxes. Similar phenomena was observed 
%in 
%numerical experiments of \citep{dunphy2014focusing}. The region is characterized by northward 
%orientation of spectra and high spreading. The waves cross a complex mesoscale field consisting of 
%two closely located eddies. They have signature in the increased eigenspeed and strong currents. 
%Vorticity effects were smaller. Overall, the eigenspeed was locally increased by $~10\%$. This 
%diverted further north already more northward oriented beam.\\
%The further development are shown on Fig. where the fields are given for mid-February, 10th - 
%15th. 
%The previous discussed eddies have merged producing a feature at (155, -46.5). 
%This time it has no effect on the tidal beam. On part because of weak signature in the eigenspeed, 
%but also since the strongest currents are southward and hence, leading to advection in poleward 
%which is seen in the spectrum orientation. Nevertheless, there are two regions that cause some 
%effects. One eddy is located at (158, 48) where again northward refraction happens. Since this a 
%dipole of two oppositely oriented rotating eddies, there are strong gradients in eigenspeed. This 
%produces high spreading in the spectrum. Fluxes again show a localized horizontal reorientataion. 
%Nevertheless, due to small lengthscales this does not have a substantial effect on the beam. 
%Oppositely to the region by ETP (152, -44) here a region with large cross-beam currents, strong 
%imprint in eigenspeed. This leading to diverting portion of the beam northward.\\
%Thence, two major effects were found one when currents are in cross-beam leading to diverting the 
%beam, scattering by mesoscale eddies and refraction. 
%half-month in  
%later
%But later developments 
%lead to further 
%Over the next we 
%Combined two effects lead 
%Here it is presented two cases that shows how complex such interactions can be (Fig.). In the 
%first 
%case due to a region of abnormal 
%
%The spatial-temporal characteristics complicates direct interpretation of diagnosed energy 
%quantities in describing the tidal beam. Its properties are obscured due to 
%interference and hence, needs additional inferences. This is related to variability. It is unclear 
%what can produce beam's energy levels and its orientation/position. Two reasons could be named as 
%accumulating interaction with mesoscale field and generation producing.\\
%Now comparison of the decomposed planar beam can be made 
%with factual observations made by \citep{waterhouse2018observations, zhao2018satellite}. \\

%\newpage
%\iftoggle{only_Chapter} {
%	\appendix
%}
%
%\nottoggle{only_Chapter} {
%	\addcontentsline{toc}{section}{Appendices}
%}
%

\bibliographystyle{apa}
\bibliography{/home/dmitry/Bibtex_lib/my_first_lib}

\end{document}