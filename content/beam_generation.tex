\documentclass[12pt]{article}
\input{/home/dmitry/Work/Research/thesis/FINALE/settings.tex}
\newcommand{\SCALEO}{2}
\newcommand{\SCALET}{1.25}
%\doublespacing
%\graphicspath{{/home/dmitry/Work/Research/thesis/FINALE/P3_ITS_GENERATION/figures/}}

%\documentclass[PhD_Thesis.tex]{subfiles}

\begin{document}
	\iftoggle{only_Chapter} {
		\title{Aspects of variable internal-tide generation at a double-ridge system of the 
		Macquarie Ridge.}
		\maketitle
	}

\section*{Abstract}
Variability in the conversion of barotropic tidal energy into internal tidal waves at the Macquarie 
Ridge is addressed by carrying out numerical experiments with variable hydrographic conditions. The 
main goal of the simulations is to compliment a field program, the Tasman TIdal Dissipation 
Experiment (TTIDE), which was designed to study the fate of radiated internal tides as 
they impinge on the Tasman continental slope located 1000 km to the west. The Macquarie Ridge 
and the nearby Campbell Plateau form a double-ridge system found here to be conducive to strong 
interference and complicated wave mechanics. The waves, emitted from the plateau couple 
with internal wave generation occurring at the ridge, and consequently drive amplification or 
reduction in the local conversion rates. 
The numerical experiments show that at the Macquarie Ridge energy transfer to mode-1 changes by 
$\pm 
10\%$ in magnitude and exhibits spatial variability. These modulations are inherently linked to the 
coupling with the nearby emitted waves. The remote wave properties are comprehensively examined by 
a 
novel technique for directional decomposition of the interfering wave fields. 
The 
result emphasizes non-plane wave propagation and the importance of multiple reflections on the 
wave amplitudes. It is shown that temporal variability stems from water mass redistribution 
attributed to dynamics of the local front. Additionally, 
local stratification modulates the reflectivity/transmission and hence, the degree of 
the amplification in the double-ridge system. Variable strength of the conversion leads to 
basin-scale changes to the wave 
field that can produce irregular signals in far-field observations and variable amounts 
of energy arriving at the Tasmanian slope.\\

\newpage
\section{Introduction}
\label{C3.sec:intro}
Baroclinic tides originate from heaving of isopycnal surfaces driven by a strong 
barotropic flow across topographic relief. This process converts barotropic tidal 
energy into baroclinic motions \citep{hendershott1981long}, constitutes a 
third of the global budget for the lunar semidiurnal constituent \citep{egbert2000significant, 
munk1997once}, contributes significantly to internal wave climate \citep{wunsch1975deep} and 
interior ocean mixing \citep{wunsch2004vertical}. At most conversion sites, because of highly 
inclined slopes, the radiated 
internal waves of tidal period (internal tides) have vertical wavenumber spectrum dominated by 
highly stable low modes \citep{st2002role} able to transmit tidal energy over ocean basin-scale 
distances \citep{zhao2016global} with as yet unknown fate. While generation sites have been 
identified 
\citep{morozov1995semidiurnal, simmons2004internal, arbic2010concurrent} and some have been 
studied in detail, e.g. Hawaii Islands \citep{rudnick2003tides}, Luzon Strait 
\citep{alford2015formation}, Mendocino Escarpment \citep{althaus2003internal}, 
the Bay of Biscal 
\citep{gerkema2004internal}, and analytical models have been developed \citep{garrett2007internal}, 
there is still large uncertainty about variability in the conversion and its dynamics.\\

The analytical models of internal wave generation emphasize a ratio of the angle of internal wave 
characteristics to the bathymetric slope, known as slope's criticality, 
\citep{sutherland2010internal, 
garrett2007internal} along with a height of topography as primary quantities in setting conversion 
levels \citep{llewellyn2003tidal, petrelis2006tidal}. For tall, steeply inclined submarine ridges 
the energy transfer approaches an upper theoretical limit \citep{petrelis2006tidal, 
st2003generation} making them ``hotspots" of barotropic tide scattering and internal tide 
production \citep{morozov1995semidiurnal, egbert2000significant}. The amount of the converted 
energy is 
quantified as a correlation between the resultant baroclinic bottom pressure and the 
vertical component of the barotropic tidal current \citep{kurapov2003m, simmons2004internal}. 
Therefore, conversion  
represents work done by the 
barotropic tidal current in displacing the isopycnal interfaces (see Appendix \ref{C3.app:conv} for 
derivation). Clearly, stratification  
directly 
impacts the energy transfers \citep{holloway1999internal} with variability especially 
pronounced when water properties change near supercritical bottom gradients 
\citep{gerkema2004internal}.\\

Besides hydrographic conditions, temporal variability arises when nonlocal, baroclinic tidal signal 
couples  
with local generation processes \citep{Kelly2010a, zilberman2011incoherent, 
pickering2015structure}. The interaction affects not only the baroclinic pressure term, but also 
the 
timing 
of the baroclinic motions relative to the surface tide forcing. Moreover, the phase lag determines 
the sign of the energy transfer. For example, the conversion rate is negative if the downward 
barotropic current coincides with isopycnal heaving, and hence, the baroclinic wave 
loses 
energy by performing work against the external force imposed by the barotropic tide. In the 
opposite case of the in-phase coupling, an amplified energy transfer from the surface tide takes 
place. Examples of the coupling can be found in cases when opposite slopes of a single ridge 
affect each other \citep{zilberman2011incoherent, echeverri2010internal}, or there is spatially 
inhomogeneous distribution of internal tide generation \citep{osborne2011spatial, 
ponte2013coastal}, or when neighboring topographic features drastically alter the internal tide 
dynamics \citep{xing1998three, buijsman2012modeling, klymak2013parameterizing, buijsman2014three}. 
The present investigation considers similar processes at the Macquarie Ridge, south of New Zealand 
\fignmp{C3.fig:geo.map}, with emphasis on temporal and spatial variability of internal 
tide generation.\\

The Macquarie Ridge scatters barotropic energy into internal tidal waves; in the Tasman basin the 
waves constructively interfere \citep{rainville2010interference} and form a baroclinic tidal beam 
\citep{zhao2016global}. Its 
fate is a subject of a recent field program, TTIDE \citep{pinkel2015breaking}, and one of the goals 
of this study is to compliment the observations with numerical simulations. The beam generation is 
complicated because of the neighboring Campbell Plateau. It is possible, that similar to the 
double ridge geometry of Luzon Strait \citep{alford2011energy}, in-phase coupling can substantially 
amplify 
conversion of barotropic tide \citep{buijsman2012double, echeverri2010internal}. In turn, 
this can affect the magnitude of the beam and far-field observations.\\

Oceanographic conditions present additional complexity as the Macquarie Ridge is located in region 
of interaction 
between 
subtropical (STW) and subpolar (SAW) water masses \citep{chiswell2015physical}. 
The zone is bounded by two fronts delimited by surface isohalines of $35.1$ and $34.7$ psu
\citep{belkin1996southern, hamilton2006structure} \fignmp{C3.fig:geo.map}. The Southern 
Substropical Front 
(S-STF) defines the poleward extent of STW and has strong signature in the upper 100-200 m. The 
associated geostrophic current is weak because of strong density compensation 
\citep{graham2013dynamical}, but topographic 
steering by the ridge produces meanders of S-STF that can result in eddy shedding 
\citep{smith2013interaction}. Mesoscale variability is also induced by atmospheric cyclones and 
local winds that drive front filamentation \citep{james2002summer}. On synoptic scales, there is a 
possible 
migration of up to $1^{\circ}$ meridionally \citep{smith2017variability} in response to a seasonal 
cycle of 
insolation and strength of the westerlies.\\

The primary goal of this study is to describe the effect of the complex internal wave field on the 
generation of the tidal beam and how the dynamics evolves in response to the changing 
oceanographic conditions. These problems were investigated by means of regional numerical 
simulations that 
combine baroclinic tide dynamics and variable, realistic hydrographic conditions (Section 
\ref{C3.sec:model}). General 
characteristics of the generation are described in Section \ref{C3.sec:main_res}. The primary 
mechanism of 
amplification is identified, and its connection to the remote internal waves is established 
(Section \ref{C3.sec:amp_mech}) via the application of a novel decomposition technique (Appendix 
\ref{C3.app:ddec}). 
This allows for 
complete description of how generation is governed by a three-dimensional wave field, dependent 
upon 
the ocean state (Section \ref{C3.sec:3d_var}). To conclude, the effect of the discussed dynamical 
factors are related to the conversion levels and intensity of the tidal beam 
(Section \ref{C3.sec:disc}).\\
 
\newpage

\section{Numerical experiments and analysis}
\subsection{Numerical experiments}
\label{C3.sec:model}
Numerical simulations were performed with Regional Ocean Modeling System 
\citep{shchepetkin2005regional} to study dynamics of internal tide generation around New Zealand, 
propagation in Tasman Sea and reflection by Tasmania. The domain covered the southern Tasman Sea 
from $60^{\circ}$ S to $35^{\circ}$ S, the zonal extent stretched from $142^{\circ}$ to 
$172^{\circ}$ E. The horizontal grid spacing was $1/32^{\circ}$ on average corresponding to 
discretization of 3 km 
in zonal direction and 2.5 km in meridional. The nonuniformly separated, vertical 50 $s$-levels 
were placed to smoothly follow subsurface terrain.\\
Such discretization of vertical momentum equation can induce artificial, horizontal  
along-slope flows \citep{haidvogel1999numerical} due to errors in reproducing of pressure 
gradient force. Especially severe errors are made by steep terrain. The misbehavior is usually 
resolved by artificial smoothing of topography. This procedure additionally increases numerical 
stability, but has an adverse effect on dynamics of internal tide \citep{di2006numerical} since 
its generation sites are collocated with large topographic gradients. To test the numerical 
setup, a sensitivity study was carried out by performing additional experiments with grid size of   
$1/8^{\circ},~1/16^{\circ},~1/64^{\circ}$. Essential for this study 
internal tide behavior manifested at $1/16^{\circ}$ and converged for $1/32^{\circ}$ and 
$1/64^{\circ}$ cases with no marked differences observed.\\

This work addresses the gravest baroclinic mode dynamics in the deep ocean. Spatial 
extent of 
waves is large compared to associated vertical displacements. This ensures linear regime of 
propagation without dispersive and nonhydrostatic effects taken place such as fission into 
solitons. A hydrostatic solver used in ROMS seems to be a proper choice for the simulations. Such 
simplification in wave dynamics was also assumed in previous studies \citep{carter2008energetics, 
merrifield2001generation,  merrifield2002model, kerry2013effects}. In more dynamically 
accurate   
simulations of \citep{kang2012energetics, zhang2011three} the nonhydrostatic effects were found to 
be important only for internal tide propagation in shallow water, while the gravest mode generation 
mostly followed linear dynamics with vertical accelerations to have negligible contribution.\\

\textbf{The horizontal boundary conditions were imposed to be open for depth-averaged, barotropic 
flows 
(modified Flather condition) and for surface elevation (ref Chapman). The three dimensional fields 
of currents and tracers were subject to mixed radiative-nudging conditions of 
\cite{marchesiello2001open} (with time scale of 3 days).} The baroclinic fields were  
nudged to zero by linearly increasing lateral viscosity and diffusivity over sponge layers. Through 
the same outer boundaries numerical simulations were forced with barotropic tide. The $M_2$ tidal 
currents and sea level were derived from TPXO atlas, version 7.2 \citep{egbert2002efficient} and 
prescibed as volume transports linearly interpolated onto the model's grid. This study considered 
only $M_2$ constituent since amplitude ratio with $S_2$ is 4-to-1 suggestive of 
slight open-ocean 
spring-neap modulation. The diurnal species are weak in the region except shoals east of New 
Zealand \citep{walters2001ocean}.\\

Several ocean states were prescribed and analyzed separately to investigate variations of 
baroclinic tide dynamics. In the \sqm{uniform}-case setting lateral gradients in water properties 
were 
absent, 
while buoyancy frequency was set to be representative of the Tasman Basin. The second type of 
simulations was performed to investigate interannual and interseasonal variability (Table 1). The 
third calculation was intended to cover whole period of TTIDE/TBEAM/Tshelf field programs 
\citep{pinkel2015breaking}.
\begin{table}
	\caption{Carried out numerical experiments}
	\begin{tabular}{ |p{3cm}||p{5cm}|p{5cm}|  }
		\hline
		\multicolumn{3}{|c|}{Numerical experiments used in this study} \\
		\hline
		Experiment abbreviation & Simulation period & Comments \\
		\hline
		Uniform & ~ & No mesoscale \\
		2012 &   Jan 1st - Jan 15th, 2012 & Interannual \\
		2013 &   Jan 1st - Jan 15th, 2013 & Interannual \\
		2014 &   Jan 1st - Jan 15th, 2014 & Interannual \\
		2013\_Oct &   Oct 1st - Oct 15th, 2013 & Interseaonal \\
		2015\_Mar &   Mat 1st - Mar 15th, 2015 & Interseaonal \\
		2015\_TTIDE$^{\ast}$ &   Jan 1st - Mar 1st, 2015 & Field period \\
		\hline
		\multicolumn{3}{|l|}{\footnotesize$^{\ast}$ the results are named as respective day of 
		year over which post-analysis was performed, e.g. $d20-25$ }\\
		\hline
	\end{tabular}
	\label{ch2:table_exp}
\end{table}
The simulations with variable conditions were at first initialized with HYCOM hindcasts
\footnote{(NAVGEM;	downloaded from hycom.org)} for respective start dates. Then during 
integration, 
along with barotropic tidal flow, time-variable, subtidal fields 
of horizontal currents, temperature and salinity were imposed along the 
boundaries. Air-sea interaction was simulated by utilizing MERRA-reanalysis 
\citep{rienecker2011merra} to prescribe realistic fluxes of heat, moisture and momentum.

\subsection{Internal tide analysis}
\label{C3.sec:anlsys}
Most of the simulations proceeded for 15 days \tblnm{ch2:table_exp}. In all 
cases the first 10 days were regarded as spin up period that allowed a single 
passage of mode-1 internal tide from New Zealand to Tasmania that takes roughly 7 days. After the 
spin-up period three 
dimensional fields of velocity, temperature and salinity were sampled hourly over 5 days. And in 
the TTIDE-experiment the quantities were subsampled with 5 day window, hence, naming convention 
\tblnm{ch2:table_exp}. Then the time series high pass filtered with Butterworth filter 
of 
order $6$ with cut-off frequency of $1/36~cph$ to remove subtidal motions. The filtered signal was 
further fit in a least square sense to $M2$ harmonic. Then the three dimensional field of 
horizontal currents underwent a separation into barotropic and baroclinic signals using following 
decomposition \citep{cummins1997simulation, kunze2002internal, carter2008energetics}
\begin{equation}
\label{ch2:bt_bc_vel}
\vec{u}_{bt}(x,y) = \frac{1}{H} \int_{-H}^{0} \vec{u}(x,y,z)  dz,~\vec{u}_{bc}(x,y,z) =  
\vec{u}(x,y,z) - \vec{u}_{bt}(x,y)
\end{equation}
Pressure associated with internal tidal motions was found from density field via the  
hydrostatic approximation and the same decomposition \citep{kunze2002internal, kelly2010topo},
\begin{equation}
\label{ch2:bt_bc_pres}
p(x,y,z) = \int_{-z}^{0} \rho(x,y,z) g dz,~p_{bc}(x,y) = p(x,y,z) - \frac{1}{H} \int_{-H(x,y)}^{0} 
p(x,y,z) dz
\end{equation}
In the both expressions rigid-lid approximation is used. This is a valid assumption unless 
vertical accelerations are greater than acceleration due to gravity \citep{kelly2010topo}.\\
Each variable was then decomposed into vertical normal modes. The eigenfunctions were obtained by 
solving Sturm-Liouville problem under the hydrostatic approximation,
\begin{equation}
\label{C3.eq:sturm}
\frac{d}{dz}\Big( \frac{1}{N^2(z)}  \frac{d \psi(z)}{dz}\Big) + c^{-2}_n \psi(z) 
= 0
\end{equation}
where $c_n$ is the mode phase speed in non-rotating ocean, $N^2$ - local Brunt-Vaisala 
frequency profiles found from time-averaged density fields.\\
This work focuses on energy characteristics of the low mode internal tidal wave field. 
Period-averaged rates of conversion from barotropic tide to 
baroclinic \citep{simmons2004internal, kurapov2003m}, depth-averaged energy flux, horizontal 
kinetic energy (HKE) and available potential energy (APE) are found as
\begin{align}
\label{C3.eq:conv}
C_{bt\to 1} = -\frac{1}{2}(\cj{\vec{u}_{bt}} \cdot \nabla H) p_{1}|_{z = -H(x,y)} = |w_{bt}| \cdot 
|p_{1,bot}| \cos (\phi_{w_{bt}} - \phi_{p_{1,bot}})\\
\vec{F} = \frac{1}{2} \frac{1}{H} \cj{\vec{u}_1} p_1 \int_{-H}^{0} \psi_1(z) \psi_1(z) dz\\
HKE = \frac{1}{4} \rho_0 \cj{\vec{u}_1} \vec{u}_1 \int_{-H}^{0} \psi_1(z) \psi_1(z) dz\\
APE = \frac{1}{4c_1^2} \rho_0 \cj{p}_1 {p}_1 \int_{-H}^{0} \psi_1(z) \psi_1(z) dz
\end{align}
Here the subscript $1$ identifies the first eigenmode, all variables are given in complex number 
notation with the complex conjugate defined by $\cj{}$. The coefficient $\frac{1}{2}$ appears due 
to 
time 
averaging.\\

The mode-1 internal tide fields were additionally subject to directional analysis in order to 
extract primary wave components usually hindered by interference. The details are presented in 
appendix 
\ref{C3.app:ddec} and summarized as follows. Dynamical fields of pressure and currents sampled over 
area-limited footprint can be described by directional 
spectra. Its circular Fourier 
transform was then obtained as solution by damped least squares of an ill-posed inverse problem. 
The 
spectral, directional analysis differs from the approach of \cite{zhao2010long} for two main 
reasons. The used here technique seeks for continuous distribution of 
energy over all compass directions, rather than a finite number of plane waves. This makes a 
difference in regions 
where wave diffraction prevails. For instance, wave field near sites of internal tide generation or 
scattering will be comprised by angular distribution of energy confined to diffractive lobes 
\citep[e.g.,][]{munroe2005topographic, johnston2003internal}. On 
the other hand, because of finite-window sampling artificial lobes are introduced as in classical 
Fourier analysis. The analysis artifacts lead to unambiguous interpretation, so judgment should be 
used when wave fields are back-synthesized over sampling angular sectors. Secondary, 
velocity measurements are utilized along with pressure observations \eqref{C1:uv.eq}. The 
formulation provides additional constrain on the inverse model. Indeed, the velocity equations tend 
to lessen Gibbs phenomena. Furthermore, the directional spectra technique can be extended 
point-wise observations where currents and pressure are measured simultaneously  \footnote{such as 
a stationary mooring}.\\

The above analysis produced sets of dynamical variables for barotropic, baroclinic 
fields and directional representation of the gravest mode for each experiment. To characterize 
temporal changes in the generation, mean energy characteristics were defined as simple arithmetic 
mean and variability - by standard deviations.

\newpage
\section{Results}
\subsection{Internal tide generation at the Macquarie Ridge}
\label{C3.sec:main_res}
The semidiurnal barotropic tide propagates into the Southern Tasman Sea from the North 
\fignmp{C3.fig:BT}. Its phase increases in a counterclockwise manner with maximum amplitude of sea 
level located along New Zealand's coast, consistent with Kelvin wave dynamics in the Southern 
Hemisphere \citep{walters2001ocean}. The simulated sea surface tidal oscillation closely followed 
TPXO7.2 with 
gross 
features well captured, but the sea level amplitude and cotidal (phase) lines exhibited wave-like, 
small scale perturbations. This is a typical manifestation of low-mode baroclinic waves 
\citep{niwa2004three}. The baroclinic tidal fluxes \fignmp{C3.fig:beam} revealed complicated wave 
field structure resulted from 
multiple generation sites determined by steep topographic relief. The primary sites were located  
just south of New Zealand as flow of the barotropic Kelvin wave interacted with relief of the 
Macquarie Ridge. As a result, several low-mode beams \citep{rainville2010interference} were emitted 
into the Tasman 
sea. Henceforth, analysis focuses on the most energetic beam that originated from a site centered 
at 
$49.5^{\circ}S$ (inset on \fignm{C3.fig:geo.map}), hereafter referred to as the Macquarie 
Ridge.\\

The major beam was mainly generated on highly supercritical slopes where kinetic energy of the 
barotropic tidal currents was converted into available potential energy of the internal tidal 
waves \fignmlp[C3.fig:gen]{A}. The energy transfer into mode-1 baroclinic tide was found to be $1.7 
\pm 0.2~GW$. 
This is half the amount found at Kaena Ridge, Hawaii \citep{carter2008energetics} and one tenth of 
that at Luzon Strait \citep{alford2011energy, kerry2014impact}. Further, relief of 
the Macquarie Ridge suggests that the strongest generation should be at the southern portion of the 
ridge where flanks of a seamount has steep slopes and average criticality of 2.25. The rest of the 
ridge consists of a gap and another seamount \fignmlp[C3.fig:geo.map]{inset}, both 
having more gradual slopes with criticality of 1.8. Contrary to the expectations, 
the conversion rates \fignmlp[C3.fig:gen]{B, C} revealed the gap contribution was significant and 
at times the largest. The temporal variability was unusually high with the conversion hot spot 
migrating from south to north and vice versa.\\

The Macquarie ridge features complex internal tide mechanics that were manifested as 
regions of negative conversion (green shading, \fignml[C3.fig:gen]{B}). Such reverse energy 
transfer and destruction of internal tide is driven by coupling with non-local baroclinic waves 
\citep{Kelly2010a}. The adjacent slopes of the Campbell Plateau (Auckland Escarpment) provided 
a source for such remote signal. Further superposition with the 
ridge-emitted waves created a standing wave \fignmlp[C3.fig:stand_wave]{A}. The wave presence was 
evident in the distribution of HKE marked by a node in the middle of Solander Trough. Additionally, 
around the node, energy flux formed a circular pattern \fignmp{C3.fig:stand_wave} with 
counterclockwise progression since the Coriolis parameter is negative in the Southern Hemisphere. 
A similar interference pattern was observed and modeled by \cite{buijsman2014three} in Luzon 
Strait where topography forms a double-ridge system with amplified levels of conversion 
\citep{buijsman2012double, klymak2013parameterizing}.\\

Further insights were obtained from the directional decomposition that separated the 
standing wave into east- and westward traveling components \fignmlp[C3.fig:stand_wave]{B, C}. On 
the eastern side of the Macquarie Ridge generation took 
place at the same seamounts. On slopes of the escarpment the conversion had complicated  
distribution of positive and negative sign manifesting complicated structure of energy transfers. 
This is partly result of more complicated sea bottom relief and the barotropic forcing 
\fignmlp[C3.fig:BT]{A, B}. Dynamically, as the ridge-emitted waves were 
back-reflected from the plateau, they interfered with the local generation 
resulting in the 
negative conversion.\\

Respective energy transfers across the trough were quantified and compared with spatially 
integrated conversion rates \fignmlp[C3.fig:stand_wave]{D}. On average, each directional component 
was almost equally divided amongst waves reflected from the supercritical slopes and wave energy 
converted from the barotropic field. At times the system exhibited large departures from the 
average state due to variable conditions of the coupling on the trough's slopes.\\

It is apparent that the coupling between tidal forcing at the ridge and reflected 
from the plateau internal waves represents a key mechanism driving variability of the 
conversion \fignmlp[C3.fig:gen]{C}. The associated energy transfers are examined along a 
two-dimensional transect (horizontal line on \fignml[C3.fig:gen]{B}) for 
experiment \sqm{2014} when conversion in the gap was the weakest. \sqm{2014} is contrasted with 
experiment \sqm{d10-15} to characterize what appears to be a normal situation at the Macquarie 
Ridge.

\newpage
\subsection{Spatial and temporal variability in conversion}
\subsubsection{Coupling of local generation with remote waves}
\label{C3.sec:amp_mech}
The experiments, \sqm{2014} and \sqm{d10-d15} serve to illustrate a number of pathways that the 
energy transfer can take. Panels A-D on \fignm{C3.fig:gen_2d} depict wave evolution throughout ebb 
stage of the $M_2$ tidal cycle; E-F shows respective time 
series of the conversion at two designated locations on opposite slopes of the ridge. At 
ebb, on the ridge's western side \fignmlp[C3.fig:gen_2d]{A, C}, the downward barotropic current 
forces near the bottom isopycnals to dip. At hour 6 of the tidal cycle, in both experiments the 
energy is transferred from the barotropic tide into the baroclinic waves, and the conversion is 
positive \fignmlp[C3.fig:gen_2d]{E}. However, in \sqm{2014}, at about hour 7, the 
transfer becomes negative as, contrary to the forcing, isopycnals begin to rise. The ensued  
positive displacements are clearly seen at the moment of slack current \fignmlp[C3.fig:gen_2d]{B}. 
As the 
barotropic current reverses direction, the conversion again changes sign. Not surprisingly, the 
period-averaged energy transfer is weak \fignmlp[C3.fig:gen_2d]{G}. Experiment \sqm{d10-15} 
reveals another regime. Throughout the ebb flow the conversion rate is much larger and positive 
because the isopycnals are continuously depressed.\\

A striking difference originated from dynamics of the remote internal waves. During the ebb flow 
in \sqm{d10-15} a mode-1 wave emitted from the Campbell Plateau progresses across 
the ridge, evident by a characteristic vertical dipole pattern in the baroclinic pressure. The wave 
travel forces downward isopycnal motion along both slopes of the ridge \fignmlp[C3.fig:gen_2d]{C}. 
On the western side, the 
interaction is in-phase with the barotropic current, driving the strongly positive energy 
transfer. On the eastern side, there is upward isopycnal motion, so the transfer is negative 
\fignmlp[C3.fig:gen_2d]{F}, corresponding to energy loss from the remote wave. As a consequence, 
the 
conversion time series on opposite sides of the gap \fignml[C3.fig:gen_2d]{E, F} are almost 
reversed one to another, and the small shift between the two curves accounts for the remote wave 
transition.
In contrast, the 
mode-1 wave is not observed on the ridge slopes in \sqm{2014}. Here the eastern side exhibits 
positive 
conversion that leads to internal wave radiation from the ridge crest \fignmlp[C3.fig:gen_2d]{A} 
seen 
as positive displacements obliquely propagating away from the topography. 
Probably, the rising of the isopycnals on the western side  
in the middle of the ebb flow in \sqm{2014} results from propagation of the 
radiated internal wave. It destructively interferes with the generation on the western side 
leading to the weak conversion \fignmlp[C3.fig:gen_2d]{G}.\\

It should be noted that the remote mode-1 wave was present in Solanders Trough 
($165^{\circ} - 165.5^{\circ}$) in both experiments, but its propagation was different since in 
\sqm{2014} the wave did not cross the gap. This result is further illustrated by the directional 
decomposition and deduced wave phase distributions \fignmp{C3.fig:2d_phase}. It is seen that  
the trough was occupied 
by a standing wave, i.e. almost constant phase and hence, no propagation of the total 
signal. 
Near the Campbell Plateau ($165.5^{\circ}$) the decomposition revealed almost equal contribution 
of the east and west-traveling waves. However, at the ridge in \sqm{d10-15} the total signal 
propagated across the ridge with the westward component to be dominant. In \sqm{2014} 
the eastward wave was stronger and the phase tended to $-180^{\circ}$ (relative to phase of the  
surface tide on the western side) implying in-phase relation with the local barotropic tide and 
local generation. These results are consistent with the previous discussion. The two 
dimensional section can not explain the difference in the reported conversion, because 
the westward wave near the ridge had similar phase in the experiments and the phase lags can not 
result in the different sign of the energy transfer on the eastern side of the gap.\\

\subsubsection{Variability of the wave fields in Solanders Trough}
\label{C3.sec:3d_var}
In this section more detailed analysis of the directional decomposition was 
undertaken to address characteristic of the waves in Solanders Trough. These results are summarized 
on 
\fignm{C3.fig:wv_fld_dist} for the previously considered 
experiments and one added. The experiment \sqm{d35-40} manifested the nonstationary 
character of the conversion process at the Macquarie Ridge as the energy transfer increased from 
1.6 GW to 1.9 GW in a three week period \fignmlp[C3.fig:gen]{D}. The panels on 
\fignm{C3.fig:wv_fld_dist} 
provide energy flux vectors, and lines of constant phase are plotted to show travel times and  
wavefronts. For example, on the upper panels isoline of $-90^{\circ}$ is marked by red color. If 
the 
waves arrived to the gap earlier, then they would be in-phase with the surface tide on the eastern 
slope. But in all three experiments the arrival 
happened later in the tidal cycle, and hence, there were regions of the negative conversion except 
\sqm{2014}. Here, small flux vectors emphasized almost negligible transmission 
of the waves across the gap. This presentation reaffirms the previous section. But an 
additional comment can be made. Comparison of the $-90^{\circ}$-isoline between \sqm{d10-15} and 
\sqm{d35-40} revealed a phase difference of $15^{\circ}$ or $\sim30~min$ near the gap. The westward 
wave in the latter experiment arrived slightly earlier to the gap, and on the western slope of the 
ridge the consequent interaction with the local barotropic current was more in-phase resulting in 
the stronger barotropic energy conversion.\\

Energy flux direction along with wavefronts of the westward wave on 
\fignml[C3.fig:wv_fld_dist]{A-C} illustrate differences in structure of the wave field. This is 
primarily seen in the northern part of Solanders Trough (equatorward of $-49^{\circ}$ latitude). 
Experiment 
\sqm{2014} showed mainly zonal wave propagation, whereas in the other experiments the wave field  
had also southward component. Consequently, the wave progression was affected by topographic 
refraction on slopes of the northern seamount and the waves were steered towards the gap. The 
amplitude 
of the southward component was dependent on the interactions at the Campbell Plateau. 
Thus, the predominantly zonal propagation in \sqm{2014} was caused by internal 
tide 
destruction at $(166.5^{\circ}E, 48.75^{\circ}S)$, \fignmlp[C3.fig:wv_fld_dist]{A, blue box} where 
slopes of the plateau have oblique orientation. Within the region the eastward waves lost 
0.14, 0.07 and 0.13 GW during reflection. Along 
with the weak incident wave \fignmlp[C3.fig:wv_fld_dist]{D} in \sqm{2014}, the strong internal tide 
destruction suppressed the southward component of the westward waves.\\

Reason for the different energy losses in the experiments stemmed from variable propagation speed 
which is displayed by 
shading on \fignml[C3.fig:wv_fld_dist]{D-F}. Here the phases are now referenced to the barotropic 
tide inside the blue box, and the red line marks $90^{\circ}$-isoline. In the experiments 
the eastward waves were out of phase with the local surface tide at the reference point, and the 
conversion was negative. Additionally, in \sqm{2014} the wave propagation was slower as the 
oceanographic conditions promoted small phase speeds. The resultant phase lag was about 
$130^{\circ}$. For comparison, in \sqm{d10-15} 
the eastward waves had lag of $115^{\circ}$ and \sqm{d35-40} - $125^{\circ}$. This 
is consistent with the reported levels of the energy loss. Furthermore, the wave speed anomaly 
explained weak generation of the eastward waves on the eastern side of the northern seamount  
\fignmlp[C3.fig:wv_fld_dist]{A}, since arrival of the westward waves 
was delayed in \sqm{2014} \fignmlp[C3.fig:wv_fld_dist]{A-C, position of $-150^{\circ}$-isoline}. 
In general, the westward waves were less dependent on the phase speeds. This point is 
explained next.\\

Phase of a plane wave grows inversely to its phase speed, and for internal tides the speed is 
determined by local water column properties and stratification (see eq. \eqref{C3.eq:sturm}). 
However, the phase increase can deviate from $\sim 1/c_p$ in complex wave fields consisting of 
multiple components. For instance, in extreme case of a standing wave phase is constant. 
Furthermore, classical 
work of \cite{longuet1957statistical} demonstrated that phase speed depends on moments of 
directional spectra 
such as skewness and spread, and the phase speed approaches a plane wave value for narrow spectra. 
In the setting of Solanders Trough the westward wave was 
composed of 
multiple components since it originated from many sites on corrugated slopes of the plateau. The 
non plane propagation was signified by change in curvature of the phase 
lines. For example, consider progression from $-250^{\circ}$ to $-200^{\circ}$ on 
\fignml[C3.fig:wv_fld_dist]{A}. Contrary, phase lines of the eastward waves mirrored the emitting 
topography of the seamounts and were approximately constant in their shape, thence, they 
represented plane wave propagation. As a result, these two cases of the wave fields illustrated 
different dependence of phase growth on ocean conditions. On \fignm{C3.fig:2d_phase} the eastward 
waves clearly marked different phase speeds, \sqm{2014} has larger slope which is consistent with 
the slower speed. But this was not true for the westward waves that had similar growth.\\

Regardless of propagation, it was shown that variation of the phase speed cause variability of the 
wave field structure. Since, for example, the strongly stratified waters result in the faster wave 
travel, it is worth exploring dependence on hydrographic conditions \fignmp{C3.fig:ocean_cond}. 
For the region of 
the Macquarie 
Ridge sea surface salinity serves to distinct warmer subtropical waters (STW is seen by brighter 
colors on the figure) and fresher subantarctic waters (SAW - colder color), and the surface 
expression is well representative of the upper water column density structure and strength of 
stratification. As consequence, there was a strong correspondence between water mass surface 
signature and phase speed anomalies (compare \fignm{C3.fig:wv_fld_dist} and 
\fignmp{C3.fig:ocean_cond}). For 
example, the experiments \sqm{d10-15} and \sqm{d35-40} showed development of a frontal filament 
after passage of an atmospheric cyclone \fignmlp[C3.fig:ocean_cond]{B, C}. The advected STW forced 
stronger pycnocline in the upper 200 m \fignmlp[C3.fig:ocean_cond]{lower panels, blue and green 
	lines} and led to the positive phase speed anomaly in \sqm{d35-40}. These examples suggest that 
the presence of STW produces positive anomalies, while opposite is true for SAW. 
However, it is also worth considering conditions of \sqm{2014} when the water properties were 
characterized as a mixture of SAW and STW. The stratification was strongly surface intensified, but 
weaker deeper in the column.\\

\subsection{Interpretation of variability in the conversion and the standing wave}
The previous analysis examined the complicated and variable internal tide dynamics occurring at the 
Macquarie Ridge, in Solanders Trough and on the slopes of the Campbell Plateau 
\fignmp{C3.fig:geo.map}. 
The 
numerical 
simulations revealed how hydrographic conditions affect the plateau-emitted waves and how 
the variable remote to the Macquarie Ridge waves change the generation of the Tasman Sea 
internal tidal beam. In the next section these findings are used to achieve more concrete 
understanding of internal tide variability at the ridge based on all of the experiments.

\subsubsection{Spatial and temporal variability in the conversion}
\label{C3.sec:spat_conv}
The strength of the coupling and amplification of the internal tide generation on the western side 
of 
the 
ridge  
depends on amount of the remote, plateau-emitted wave energy to reach the western side. Clearly, 
this amount is defined by the transmission and amplitude of the westward wave in the 
trough. The transmission is addressed by considering spatial distribution of the eastern slope 
reflectance 
\fignmp{C3.fig:crit}. It follows that the remote energy was mainly fluxed through the 
subcritical gap, 
whereas the seamounts were supercritical and reflective, and hence, transfer of the 
westward wave energy from the trough was suppressed. Thus, the gap's contribution to the overall 
conversion is directly a result of the ridge's relief.\\

Because of the inhomogeneous transmission, the remote waves have spatially variable contribution to 
the variation of the conversion. This is particularly captured by how much variance terms in 
eq. 
\eqref{C3.eq:conv} induce separately. In order to do so, the equation is 
differentiated with respect to each term, and the differentials are substituted with variances 
\citep{kerry2014impact}. Results are summarized in \fignm{C3.fig:var_sp}. Notably, the total 
variance found as sum of the individual contributions is equal to 0.9 GW, which is much larger 
0.2 GW reported in Section \ref{C3.sec:main_res}. The disparity is attributed to non-zero 
covariances that result from dynamical relationships between the terms. Regardless, the 
spatial distribution of term-by-term contribution on the western side of the ridge  
shows a pattern that was dictated by the inhomogeneous transmission. Variability due to 
the 
phase lag was dominant near the gap, but it was smaller at the 
seamounts. The bottom baroclinic pressure, on other hand, contributed more at the 
seamounts and less at the gap. Physically, on the western slopes of the gap the remote waves were 
larger than anywhere 
else 
on the ridge, and they strongly impacted the local generation affecting both amplitude and phase of 
the 
bottom pressure. On the steeper flanks of the seamounts the local generation was strong, 
the remote waves were weak, so the coupling mainly happened by adjusting the pressure. For the 
Campbell Plateau most of variability arose from phase of the incident 
eastward waves as indicated by \fignm{C3.fig:wv_fld_dist}.\\

\fignm{C3.fig:crit} also illustrates variation in the reflectivity of the eastern slope between 
the experiments. It seems that migration of the hot spot in the conversion and spatial variability 
\fignmp{C3.fig:gen} can be explained by changes in the criticality and the transmission 
of the westward waves. For example, between \sqm{d10-15} and \sqm{d35-40} the northern portion of 
the ridge became more transparent for the westward wave. As a result, the latitude of the 
conversion-weighted mean moved northward as well \fignmlp[C3.fig:gen]{D}. This suggests another 
effect of stratification on the double-ridge system which is addressed in Section 
\ref{C3.sec:strat_eff}.\\

Temporal evolution of the conversion magnitude is shown on \fignm{C3.fig:ener_var} 
where panel B is computed for the gap alone. Here it becomes apparent that processes in the gap are 
a primary contributor to changes in the overall Macquarie Ridge conversion since the time series of 
conversion for the gap and the ridge are strongly correlated with $R^2 = 0.75$. On the other hand, 
a primary contributor to this variability was variable amplitude of the westward waves in 
Solanders Trough that were incident on the gap. The magnitude of the conversion on the western side 
was correlated by $R^2 = 0.5$ to strength of the remote waves on the eastern side. The coefficient 
was substantially higher and rose to $0.8$ if \sqm{uniform}-experiment was omitted.\\

Experiment \sqm{uniform} is an outlier similar to \sqm{2014} as it represents a dynamically 
opposite regime to other experiments. Contrary to slower propagation in \sqm{2014}, the prescribed 
uniform stratification led to extremely high wave speeds for the region. The stratification was 
representative 
of the central Tasman Basin where water masses are comprised by subtropical waters. The resultant 
phase speeds 
caused much earlier arrival of the westward wave to the ridge \fignmlp[C3.fig:ener_var]{red inset, 
panel B}. The energy transfers at the gap and on the western side of the ridge had different 
character. For example, the internal tidal field was losing energy at the northern seamount 
\fignmp{C3.fig:gen} since the remote waves were out of phase with the local forcing. In general, 
unlike the outlier experiments of \sqm{uniform} and \sqm{2014}, in other experiments the arrival of 
the westward wave seems to have a secondary effect that only complicates the relation between the 
conversion at the gap and the remote waves.

\subsubsection{Effect of variable stratification}
\label{C3.sec:strat_eff}
Purple filled dots on \fignm{C3.fig:ener_var} represent transmission coefficients 
obtained from the directional decomposition. Not surprisingly, the criticality of the eastern 
slopes determined spatial character of the directional analysis estimates: the gap alone 
transmitted $47 
\pm 8~\%$, for the seamounts 
the coefficient fell to $27 \pm 4\%$, and the overall Macquarie ridge showed $35 \pm 4~\%$. In 
addition, these results were compared to a closed-form solution of 
\cite{larsen1969internal} for transmission by a single knife-edge in uniformly stratified ocean. 
The Macquarie Ridge was thought 
as a barrier whose transmission varies with its height. The height was defined as the 
topography's mean depth relative to Solanders Trough in the ocean with WKB-scaled vertical 
coordinate 
\citep{althaus2003internal}. Since the WKB-scaling depends on local stratification profile, such 
approach provides a perspective on effect of variable conditions in the ocean. It is worth 
emphasizing that application of the WKB approximation usually results in underprediction, 
and the 
knife-edge representation is applicable to topographies with criticality above 3 
\citep{mathur2016internal}, which fails 
for the gap. For the supercritical seamounts at the Macquarie Ridge following 
\cite{mathur2016internal} the error is estimated to be of $30\%$.\\

The scaled analytical results are given by empty purple diamonds on \fignm{C3.fig:ener_var}. 
Both calculations of the transmission coefficient reveal similar character of temporal change. 
Therefore, the local stratification not only affects kinematic characteristic of the internal 
tides 
in 
Solanders Trough (Section \ref{C3.sec:3d_var}), but also it influences dynamical properties of the 
double-ridge system. In particular, the oceanographic conditions modulate how much of the 
plateau-emitted energy will reach the conversion sites on the western slopes. Besides the temporal 
change, scaling coefficients used on \fignm{C3.fig:ener_var} highlight phenomena of 
increased transmission by a knife-edge when interaction with the barotropic tide occur such as in 
case of a double-ridge system. This 
phenomena was previously identified by \cite{klymak2013parameterizing} in relation to 
Luzon Strait. It was shown that the increase depends on ridge separation distance or phase of the 
impinging waves. Here it is additionally observed that the increase in transmission is stronger for 
more reflective or taller topography. Following the theory for a single knife-edge ratio of the gap 
transmission to one of the seamounts is 3, while the ratio obtained from the simulation is 1.5. 
Note 
that the previously outlined caveats of the analytical calculations do not undermine the 
conclusions and even reinforce the one on the transmission ratio.\\

It is as well expected that topography in strongly stratified ocean produce more internal tides 
\citep{holloway1999internal}. This dependence is investigated by another application of the 
knife-edge approach \citep{st2003generation}. This time the barrier's height is calculated 
relative to the abyssal plain in WKB-scaled ocean, and in order to obtain total conversion 
instead 
of rates per 
distance, the barrier is set to have  
lateral extent of $240~km$. The forcing barotropic current is taken to be uniform and equal to 
$0.03~m/s$. In any case, the ridge length and the current amplitude are just linear scale factors. 
The knife-edge generation is plotted against the results from the numerical experiments on 
\fignml[C3.fig:regr]{A}. It was expected that the strongly stratified ocean leading to taller 
knife edge will ensue higher conversion. Contrary, relation is found to be negative as the 
simulations with lower knife edges conversed more barotropic energy. However, the result is 
consistent with the transmission analysis. One expects smaller barrier to be more transparent for 
the westward wave incident from the trough. This is confirmed by \fignml[C3.fig:regr]{B} where on 
horizontal axis are now given the diagnosed westward wave magnitude multiplied by the analytically 
obtained transmission coefficients, i.e. on \fignml[C3.fig:ener_var]{A} the black dots were scaled 
by the 
empty 
purple diamonds. Here the dependence is positive. Higher amount of the remote energy was reaching 
the western side, the stronger coupling and amplification were getting. The result does not exclude 
the direct role of stratification in process of the internal tide generation, but rather emphasizes 
the primary 
mechanism 
for the amplified generation.\\

\subsubsection{Strength of internal tides in Solanders Trough}
\label{C3.sec:disc_sw}
It is seen that the magnitude of the westward wave changes the strength of the 
conversion at the Macquarie Ridge. Variability of the wave is associated with reflection and 
conversion on the slopes of the trough. These parameters can be linked by invoking 
analogy of a 
double-ridge system to a Fabry-Perot interferometer as follows\footnote{see another application of 
such approach to internal wave problem by \cite{mathur2010internal}}.\\

Let conversion rates on the slopes be $C_M$ and $C_C$ and reflection coefficients - $r_M,~r_C$ with 
indexes denoting 
respectively the eastern side of the Macquarie Ridge and the slopes of the Campbell Plateau. 
Initially, the strength of the westward wave is just $C_C$. Then it crosses the trough, reflects 
from the ridge and the plateau. After the first crossing its magnitude becomes $r_C r_M C_C$, and 
after $n$-th crossing - $(r_C r_M)^n C_C$. Besides the plateau portion, the westward wave 
also includes contribution of the eastward wave. After it is emitted from the ridge, it reflects 
from the plateau having now west orientation and amplitude $r_C C_M$. The $n$-th term is given by 
$(r_C)^n (r_M)^{n-1} C_M$. In overall, the total flux is $F_W = \sum_{i = 1}^N (r_C r_M)^i C_C 
+ (r_C)^i (r_M)^{i-1} C_M$. Assuming equilibrium, $N \rightarrow \infty$, transition to 
geometrical series is made, hence,
\begin{equation}
\label{C3.eq:fl_fabry}
F_W = \frac{1}{1 - r_C r_M} (C_C + r_C C_M)
\end{equation}
In order to test applicability of such representation, the equation is solved for the reflection 
coefficient. The averaged result 
from the numerical experiments \fignmlp[C3.fig:stand_wave]{D} was $C_M \simeq C_C = 0.45~GW$ and 
$F_W = 1.2~GW$, under the assumption of $r_C = r_M$, it is found that the reflection is $63\%$ 
which is in agreement with the transmission estimates obtained previously.\\

The Fabry-Perot description emphasizes amplification of the westward wave due to multiple 
reflection that is expressed as a fraction in eq. \eqref{C3.eq:fl_fabry}. The amplification factor 
reveals the effect of both slopes reflectivity on the wave field strength. Specifically, if 
reflection coefficients change, the consequent effect is multiplicative and hence, nonlinear. The 
reflectance 
variability has as well asymmetrical response in the eastward and westward components. For example, 
the 
results shown in \fignml[C3.fig:stand_wave]{D} indicated more consistency of the eastward component 
in 
the 
TTIDE-simulation. This implies less variation of the ridge reflection than at the plateau that was 
also apparent from \fignm{C3.fig:wv_fld_dist}.\\

The expression \eqref{C3.eq:fl_fabry} represents a simplified one-dimensional description of the 
multiple reflection in a double-ridge system. Energy terms, i.e. conversions and reflection 
coefficients, are diagnosed separately, and as it was shown in Section \ref{C3.sec:3d_var},  
these characteristics are dependent on properties of the three dimensional wave fields 
that vary in response to oceanographic conditions. Thus, the standing wave and the involved 
wave components has a high degree of variability stemming from the environmentally controlled 
parameters. The dependence is multifaceted as by means of the described mechanisms change in 
stratification triggers adjustment both in propagation of the waves and their energetics.

\newpage

\section{Discussion \& Conclusion}
\subsection{Summary}
The numerical experiments were conducted to investigate dynamics of the low-mode internal tidal 
beam in Tasman Sea under realistic and variable oceanographic conditions. This chapter was concerned with 
origination of the beam. The generation takes place along the western side of the Macquarie Ridge, south of New Zealand, 
where the semidiurnal barotropic Kelvin wave sloshes across the steep relief. The process of energy 
transfer of the surface tide energy into the internal tides is amplified because the strong remote 
signal constructively couples with the barotropic forcing. The remote signal is incident on the 
opposite side and is transmitted across the ridge. Thus, details of the wave transmission shapes 
the 
amplified generation. The inhomogeneous ridge topography lead to concentration of the conversion in 
the subcritical gap separating two supercritical seamounts. Nevertheless, the spatial pattern of 
the conversion is subject to change under temporally variable stratification. The numerical 
experiments indicated spatial variability such as migration of the conversion hot spot as a 
result of varying transmission.\\

In general, internal tide dynamics around the Macquarie Ridge represents a case of double-ridge 
system which is analogous to Luzon Strait. In such settings there is strong interaction in internal 
tide generation on opposing slopes of ridges forming the system. At the same time, the emitted fields are oppositely directed, and as they interfere, a standing wave is created. The standing wave pattern was observed in Solanders 
Trough that separates the Macquarie Ridge from the nearby Campbell Plateau. In order to describe 
the involved energy budgets directional spectra analysis was carried out. Its results revealed 
several important features.\\

At first, the westward traveling waves that are incident on the 
Macquarie Ridge and cause the amplification originate from multiple sites on the plateau slopes. 
However, strength of the generation strongly depends on phase of the arriving ridge-emitted 
eastward waves.\\

At second, both components of the wave in the trough depend 
on water mass distribution and variable stratification that change mode-1 eigenspeed. However, the 
westward wave, because of its non plane propagation, demonstrate less relation to the phase speed. 
In contrary, the eastward wave travel is purely plane.\\

At third, the directional analysis made possible estimates of transmission of the westward waves 
across the Macquarie Ridge. There is a clear expression of temporal evolution. Analytical 
calculation based on WKB-scaling and transmission for a single knife-edge ridge produced similar 
dependence, but the values were 5 and 10 times smaller for the gap and the overall ridge 
respectively. This illustrates previously identified phenomena of increased transmission in the 
double-ridge system when separation distance is near half wavelength of mode-1. The numerical 
estimates of the transmission also made possible to successfully test application of the 
Fabry-Perot theory. The obtained relation highlights role of the 
multiple reflection in amplifying the westward wave whose amplitude is not simply defined by the 
conversion at the Campbell Plateau. At the same time, the amplification of the westward and 
eastward waves is asymmetrical relative to the involved reflection coefficients. This explains 
non-equal energy fluxes of the eastward and westward components found in the numerical 
experiments.\\

Further, it was shown that the variability of the westward wave translates into the tidal beam 
generation. However, relation between the generation and the beam intensity is less certain as 
illustrated by \fignml[C3.fig:ener_var]{A} and by rather low correlation of $R^2 = 
0.43$\footnote{If only NW quadrant portion of the directional spectra is considered, the 
coefficient increases to 0.6. This is shown in the next chapter.}. Nature of this disparity was not 
investigated, but likely due to three-dimensionality of the problem and not accounted dissipative 
processes. On average, the beam acquired $87 \pm 7\%$ of the barotropic energy converted into the 
baroclinic mode-1. Such efficiency is higher than reported $\sim 70\%$ for the beam radiated from 
the Hawaiian islands \citep{carter2008energetics}. But note the loss for the Hawaiian case 
also included the unstable high modes. In this study the high mode dynamics was not accounted 
for. Hence, the obtained here loss is hypothesized to be primarily because of numerical dissipation 
\citep{di2006numerical}.\\

Discussion of the dissipation and its effects was also not considered in any depth because the 
coarse resolution 
does not permit direct simulation of wave breaking and no subgrid parametrization was utilized. As 
consequence, the analysis was not intended to provide robust estimates and accounting of all 
energy losses. Furthermore, nonlinear process of lee wave generation was neglected for the same 
reason of the resolution. This is partly warranted by scaling analysis. The topographic Froude 
number $Ft_t = 
U_0/(N_{ridge}h_{ridge}) \sim 0.001$ is much less than $1$, as well as the tidal excursion 
number $\delta = U_0/(L \omega_{M_2}) = 0.015 \ll 1$. These values ensure largely 
linear regime of internal tide generation \citep{garrett2007internal}. Nevertheless, similar range 
was obtained for Luzon Strait by \citep{buijsman2012double} who addressed strong dissipation due to 
the breaking high-mode lee waves. Thus, such dissipative mechanism possibly does take place in 
reality 
at the Macquarie Ridge, but its strength and spatial distribution can not be envisioned from the 
current numerical results. At the same time, such knowledge will be of great importance for the 
local oceanography considering that \citep{smith2013interaction} observed the frontal zone to be 
positioned between the seamounts, at the gap, where the strongest amplification and coupling takes 
place following this study. Such intriguing dynamical setting invites an investigation centered on 
the internal tide dissipation and its effects on the frontal dynamics.

\subsection{Comparison with Luzon Strait}
\label{C3.sec:disc_luzon}
Throughout this investigation several remarks were made regarding similarity of the Macquarie Ridge 
to Luzon Strait. Luzon Strait is a well studied site of strong internal tide generation that is 
amplified by two near parallel ridges. The amplification process was addressed in work of 
\citep{buijsman2012double, klymak2013parameterizing}\footnote{add ramp, mathew?} who either 
numerically integrated or solved mode-matching equations in the two dimensions for internal wave 
field enclosed by ridges. These  
investigations referred to the generation in the strait as resonant because of the ridge separation 
is close to half of the mode-1 wavelength. In the later realistic simulations 
of \citep{buijsman2014three} it was further found that three dimensionality brought about even 
higher amplification due to more in-phase coupling and different surface tide flow in comparison to 
2D investigations. Contrary, 
the Macquarie Ridge is oriented obliquely to the slopes of the Campbell Plateau, the separation 
distance varies from $1/3$ in the north to $1$ in the south of the mode-1 wavelength. Furthermore, 
the directional analysis (\ref{C3.sec:3d_var}) demonstrated that two-dimensional section 
perspective is not applicable when the wave fields are not plane. Thus, in this study term 
``resonant generation" was deliberately not used, instead the numerical results were investigated 
through more general perspective of the coupling between local generation and remote wave initially 
suggested by \citep{Kelly2010a}.\\

The oblique orientation of the topographies as well leads to lower amplification of the 
generation in comparison to Luzon Strait. In this study experiments to obtain conversion rates for 
isolated Macquarie Ridge were not performed unlike to \citep{buijsman2014three}. Instead, the 
amplification coefficient is estimated either from eq. \eqref{C3.eq:fl_fabry} giving $2.6$ or as 
ratio of the numerical simulation results to the knife-edge prediction \fignmp[C3.fig:regr]{A} 
suggesting an increase in $2$ times. For Luzon Strait the simulations of \citep{buijsman2014three} 
revealed spatially variable amplification of the conversion with maximum value of $4.5$, but 
average of about $3$.\\

Variability of internal tides in Luzon Strait was studied by \cite{kerry2014impact} with numerical 
simulations. Their result showed $\pm 10-15~\%$ change in the conversion rates which is similar 
to the reported here level of variability. Unlike the Macquarie Ridge, primary mechanism was 
attributed to arrival of far-field remote waves radiated from the Mariana arc. The background ocean 
conditions seems to had secondary importance on the generation in this double-ridge system, which 
is a bit surprising considering stronger amplification. Probable explanation can be found in 
height disparity of the ridges located in Luzon Strait, their respective WKB-heights are 0.2 and 
0.6, and the higher eastern ridge ``casts a shadow" over the western ridge 
\citep{klymak2013parameterizing}. It is speculated that there is less feedback between the ridges 
as the multiple reflection has lesser role in the amplification and its variability. Field 
observations of 
\cite{pickering2015structure} can serve as an illustration. They documented the  
conversion at the taller eastern ridge was on $90~\%$ stationary. The presented here numerical 
results showed highly variable conversion on the eastern side of the Macquarie Ridge that it can 
deviate by a third, $\pm 0.15~GW$ from the mean $0.45~GW$ \fignmlp[C3.fig:stand_wave]{D}. However, 
the variability decreases with time series length, for \sqm{TTIDE}-simulation the deviations 
comprised only $15~\%$.\\

The ridges' height difference at Luzon Strait also leads to highly asymmetrical energy fluxes and 
formation of a progressive-standing wave (Nekrasov), i.e. energy flux vectors show progression of a 
stronger wave. Experiments of \citep{buijsman2014three} clearly exhibited such pattern where the 
waves radiated from the eastern ridge were dominant. Nevertheless, it is expected that similar to 
the Macquarie Ridge any changes to the interfering components will cause spatial rearrangement of 
the progressive-standing wave.

\subsection{Conclusion}
Generation of internal tides at the Macquarie Ridge demonstrated spatial and temporal variability 
driven by the ocean conditions that trigger 
adjustments in the double-ridge system. The ocean state characterized by distribution 
of water masses is induced by dynamics of the frontal zone that has several time scales of 
variability. 
Some numerical experiments were carried out to test hypothesis on seasonal and interannual 
variability in the internal tide generation in response to the frontal zone position. No evidences 
were found of such dependence, but rather it was seen that variability is dictated by wind 
impelling water mass advection and frontal filamentation. On the other hand, it is not clear if 
oceanic eddies produced as a result of the front interaction with the Macquarie Ridge has any 
impact. Therefore, the generation of the internal tidal beam varies on time scales dominant by 
wind variability, i.e. several inertial periods. Under such scenario the transfer of the 
semidiurnal barotropic energy into the baroclinic tides varies constantly. As it was shown some 
of this nonstationarity is manifested in intensity of the internal tidal beam and hence, in Tasman 
Sea internal tidal field with possible consequences to far field.\\

This study has detailed aspects of the internal tide generation in the double-ridge setting when 
remote waves have somewhat extreme contribution to the dynamics. Nevertheless, for less severe 
interaction it is still expected that in presence of the remote signals the production of internal 
tides is highly variable and has multifaceted dependence both on intrinsic internal-wave parameters 
and the ocean properties. This further poses questions pertaining predictive description of the 
internal tides and stationarity of their characteristics with no clear answer in sight.

\section*{Appendices}

\renewcommand{\thesubsection}{\Alph{subsection}}
\setcounter{subsection}{0}
\subsection{Derivation of conversion from the first principles}
\label{C3.app:conv}
While \eqref{C3.eq:conv} provides a convenient way to quantify energy transformations, it does 
not have much room for physical interpretation. Such as in complicated situation when along with 
local baroclinic tide production a remote signal is present. The resultant perturbation of bottom 
pressure might lead to rather ambiguous result of internal tide destruction. It is said that such 
regime is to occur when a phase difference between $w_{bt}$ and $p_{1}$ is in range of 
$(\pi/2,~3\pi/2)$. This is understood as an internal tide performing work against barotropic 
forcing. Unfortunately, this statement does not directly follow from \eqref{C3.eq:conv}. Hence, to 
provide cleaner physical picture, let derive the expression for conversion rate from the first 
principles. Body force of \citep{baines1982internal}, $F_B$ is performing work by displacing 
isopycnal 
surfaces throughout a water column,
\begin{equation}
\label{C3:eq.convd1}
C(z) = \frac{dW(z)}{dt} = F_{B} w_{bc} = \frac{N^2 (-\vec{u} \cdot \nabla h) z}{i \omega h} 
\frac{d 
	\xi}{dt} = \frac{w_{bt}}{i \omega h} \frac{z d(-b)}{dt} = -\frac{1}{h} w_{bt} zb
\end{equation}
where isopycnal displacements $\xi$ were changed to buoyancy, $b = -N^2 \xi$ and temporal 
variation 
was assumed to be harmonic, $\sim e^{i \omega t}$. At the final step, integration by parts can be 
employed as,
\begin{equation}
\label{C3:eq.convd2}
\int_{-h}^{0} z b dz = \int_{-h}^{0} z d \big( \int^0_{z} b dz \big) = \big( z \int^0_{z} b dz 
\big)\big|_{-h}^0 - \int_{-h}^{0} dz \int_{z}^{0} b dz^{\prime} = h (\int_{-h}^{0}b dz - 
\frac{1}{h} \int_{-h}^{0} dz \int_{z}^{0} b dz^{\prime})
\end{equation}
The last expression is a bottom pressure perturbation. Noteworthy, the baroclinicity condition was 
not employed. Combining \eqref{C3:eq.convd1} and \eqref{C3:eq.convd2}, familiar result 
for conversion rate is obtained. Note that similar approach to \eqref{C3:eq.convd1} was used 
by \citep{nash2006structure} to estimate an upper limit on emitted energy.\\

\subsection{Application of directional spectra to internal tidal fields}
\label{C3.app:ddec}
Similar approaches were used previously in internal tide field programs by 
\cite{hendry1977observations, lozovatsky2003spatial}\footnote{They were based on array 
beamforming method and stationarity of the field} or satellite altimetry by \cite{dushaw2002mapping}, 
or in surface wave studies \citep{longuet1961observations, munk1963directional, long1986inverse}. 
Let pressure of complicated wave field to be described by an angular spectrum
\begin{equation}
\label{C1:eq.spectrum}
p(\vec{r}, t) = \int_0^{2\pi}  d\theta_k S(\theta_k) e^{i \vec{k}(\theta_k) \cdot \vec{r} + 
	\phi(\theta_k) - i \omega t}
\end{equation}
Here each monochromatic wave of wavenumber $k$ travels in direction $\theta$ with 
energy $S(\theta)^2 d\theta$ and phase $\phi(\theta)$. Now time dependence is dropped and all 
physical quantities are given by corresponding complex amplitudes. The statement can be 
reformulated in terms of Fourier transform \citep{munk1963directional}. Recall Jacobi-Anger 
expansion,
\begin{equation}
p(r, \theta) = e^{i \vec{k}(\theta) \cdot \vec{r}} = \sum_{m = -\infty}^{m = \infty} i^{m} J_{m}(k 
r) e^{im(\theta - \theta_k)}
\end{equation}
that shows a field at point $(r, \theta)$ produced by plane wave can be expanded in series of 
Bessel functions and circular functions. Then substitution produces
\begin{equation}
\label{C1:p.eq}
p(r, \theta) = \sum_{m=-\infty}^{m=\infty} J_m(kr) e^{im(\theta + \pi/2)} \Big\{ \int_0^{2\pi}  
d\theta_k 
S(\theta_k) 
e^{i\phi(\theta_k)} e^{-im\theta_k} \Big\}
\end{equation}
Term in brackets represent convolution integrals defining circular Fourier coefficients of 
order $m,~A_m - i B_m$. Series \eqref{C1:p.eq} state an inverse problem that seeks the 
unknown coefficients from the known, measured pressure field that is sampled at a set of points 
$(r_i, \theta_i)$ and if infinite series is truncated at some order $N$. Real and imaginary parts 
constitute two separate problems.\\
The same steps are undertaken for velocities as plane wave polarization relations 
\citep[e.g.,][]{muller2000scattering} are utilized. The following is found
\begin{align}
\label{C1:uv.eq}
\begin{Bmatrix}
u \\ v
\end{Bmatrix}
= \frac{1}{2} \sum_{m = -N}^{m = N} J_{m} (kr) e^{im(\theta + \pi/2)}
\begin{Bmatrix}
(\omega - f) A_{m + 1} + (\omega + f) A_{m - 1} - i [(\omega - f) B_{m + 1} + (\omega + f) B_{m - 
	1}] \\ 
(\omega - f) B_{m + 1} - (\omega + f) B_{m - 1} + i [ (\omega - f) A_{m + 1} - (\omega + f) A_{m - 
	1}]
\end{Bmatrix}
\end{align}
The dependence of currents on wave bearing causes splitting of Fourier coefficients. To describe 
velocity field higher circular harmonics have to be used. Or physically, currents have higher 
spatial wavenumber. As well \eqref{C1:uv.eq} emphasizes an asymmetry between clockwise 
and counterclockwise motions due to the Coriolis effect. Now formally the inverse problem can be stated 
\begin{equation}
y = K x
\end{equation}
where $y$ is comprised of measured dynamical variables, $K$ combines bracketed terms in 
\eqref{C1:p.eq} and \eqref{C1:uv.eq} and $x$ are the unknown Fourier coefficients. Generally, the 
problem is over-determined (i.e., number of unknowns is smaller than number of 
measurements) and unstable to small errors in data, so the model yields physically inconsistent 
results. This is circumvented by seeking a damped least square solution \citep{munk2009ocean} where 
a minimization function is 
\begin{equation}
\label{C1:Tikh_prob}
J = ||K x - y||^2_2 + \alpha ||x||^2_2
\end{equation}
The unknown regularization parameter $\alpha$ acts as a high-pass filter in a singular value 
decomposition of $K$ \citep{bennett1992inverse}. In field studies this is usually set by a 
signal-to-noise ratio \citep{munk2009ocean}, since the parameter scales noise variance (residue) 
to actual signal's strength. To obtain $\alpha$ in data-driven way a straightforward approach is 
adapted that based on 
trade-off curve method \citep{hansen1993use}. In \eqref{C1:Tikh_prob} amount of allowed error 
competes with solution's variance. An optimal parameter is said to balance these factors. This is 
seen as a rapid change in behavior of a curve that relate residue with model's norm as amount of regularization varies. In most cases the curve has a sharp corner connecting aforementioned limits, hence, the 
method's name is a L-curve \citep{hansen1999curve}. And the corner is thought to occur for an 
optimal regularization parameter.\\
In the presented here analysis the equations \eqref{C1:p.eq} and \eqref{C1:uv.eq} are sampled at locations along concentric 
arrays placed at $\lambda,~0.5\lambda,~0.25\lambda$ where $\lambda$ is mode-1 wavelength found in 
the central point. This clearly assumes uniformity of the wave field over sampling window. The 
footprint could be changed to accommodate some spatial variability, but only in expanse of 
resolving power. Further, at each location $u,~v,~p$ from the numerical simulations are used as 
input data and for the central point of the array Fourier coefficients are found. The results are 
then used in reconstructions over an angular sector to obtain elemental components in the 
respective grid point.

\newpage
\section{Figures}

\begin{figure}
	\centering
	\includegraphics[scale = 
	\SCALET]{/home/dmitry/Work/Research/thesis/FINALE/P3_ITS_GENERATION/figures/fig_1_bathy_geo_fin.png}
	\caption{Domain of numerical simulations with geographical locations used throughout the text. 
	The pink line marks mean position of two fronts in HYCOM simulations. The inset provides 
	three dimensional relief of the study region.}
	\label{C3.fig:geo.map}
\end{figure}

\begin{figure}
	\centering
	\includegraphics[scale = 
	\SCALET]{/home/dmitry/Work/Research/thesis/FINALE/P3_ITS_GENERATION/figures/fig_2_BT_tide.png}
	\caption{Comparison of $M_2$ sea level amplitude simulated by ROMS (B) with 
	TPXO-model (A). Shading shows distribution of the tidal amplitude. Progression of the wave is 
	illustrated by black lines of constant phase (in degrees).}
	\label{C3.fig:BT}
\end{figure}

\begin{figure}
	\centering
	\includegraphics[scale = 
	\SCALET]{/home/dmitry/Work/Research/thesis/FINALE/P3_ITS_GENERATION/figures/fig_3_uni_flux.png}
	\caption{The internal tidal beam in the Tasman Sea is showing by mode-1 depth-integrated energy 
	fluxes. Major 
	generation sites identified by underlying shading.}
	\label{C3.fig:beam}
\end{figure}

\begin{figure}
	\centering
	\includegraphics[scale = 
	\SCALEO]{/home/dmitry/Work/Research/thesis/FINALE/P3_ITS_GENERATION/figures/fig_4_gen_fin.png}
	\caption{Generation of internal tides at the Macquarie Ridge. (A) Slope's criticality and 
	barotropic forcing given by current ellipses. (B) The diagnosed conversion rates in the 
	'uniform' experiment. The boxes outline regions used in (C) and further calculations. (C) 
	Variability of conversion rates in the three regions. Geographical change (abscissa) is given 
	by weighted-mean latitude of the conversion in corresponding averaging box. Results for 
	experiments discussed in the text are connected by lines. Black crosses show 1 standard 
	deviation in magnitude of the conversion and position within respective box.}
	\label{C3.fig:gen}
\end{figure}

\begin{figure}
	\centering
	\includegraphics[scale = 
\SCALEO]{/home/dmitry/Work/Research/thesis/FINALE/P3_ITS_GENERATION/figures/fig_5_stand_wave_2by2.png}
	\caption{Standing wave between the Macquarie Ridge and the Campbell Plateau. (A) Distribution 
	of HKE in 'uniform' simulation with superposed energy flux in the total mode-1 field. (B-C) 
	Energy flux map of the westward component and the eastward component, respectively. (D) 
	Variation in energy characteristics of the mode-1 field. Energy fluxes were integrated across 
	the blue transect shown in (C) and result is given by arrows. Dots are for spatially integrated 
	conversion 
	rates 
	over boxes given on \fignml[C3.fig:gen]{B}. The upward direction is used for the westward 
	integrated flux and Auckland Escarpment and downward is used for the eastward component and the 
	conversion on the eastern slope of the Macquarie Ridge. The continuous line connects results 
	for the experiment that covered the TTIDE field period.}
	\label{C3.fig:stand_wave}
\end{figure}

\begin{figure}
	\centering
	\includegraphics[scale = 
	\SCALET]{/home/dmitry/Work/Research/thesis/FINALE/P3_ITS_GENERATION/figures/fig_6_2d_section_sill.png}
	\caption{(A-D) Distribution of baroclinic pressure anomaly with superposed isopycnal 
	displacements given by contour lines during ebb and slack tidal stages in the two experiments. 
	Positive (red) color is assigned to lifted interfaces and negative (blue) - for the opposite. 
	Insets between (A) and (C), (B) and (D) provide vertical component of the barotropic current 
	at the considered time moments (blue - \sqm{2014} and red - \sqm{d10-15} throughout the 
	figure). The western and eastern slopes of the Macquarie Ridge are 
	designated by vertical solid lines. For the respective locations panels (E, F) depict evolution 
	of the conversion in both experiments. (G) Along-section distribution of the period-averaged 
	conversion rate along.}
	\label{C3.fig:gen_2d}
\end{figure}

\begin{figure}
	\centering
	\includegraphics[scale = 
	1.5]{/home/dmitry/Work/Research/thesis/FINALE/P3_ITS_GENERATION/figures/fig_7_phase_distr.png}
	\caption{Phase of the mode-1 baroclinic pressure at ocean bottom along the section 
	previously 
	considered on \fignm{C3.fig:gen_2d}. Solid line gives the total wave signal. Dashed and dotted 
	lines provide the result of the directional decomposition, i.e. the westward and eastward 
	traveling waves	respectively. All phases are referenced relative to the barotropic current on 
	the western side of the Macquarie Ridge. The eastern and western locations of the Macquarie 
	Ridge are identified by vertical solid lines and the same as on \fignm{C3.fig:gen_2d}.}
\label{C3.fig:2d_phase}
\end{figure}

\begin{figure}
	\centering
	\includegraphics[scale = 
	\SCALET]{/home/dmitry/Work/Research/thesis/FINALE/P3_ITS_GENERATION/figures/fig_8_flux_examp.png}
	\caption{Energy and phase distribution in the decomposed wave field. The upper row shows the 
	westward component and bottom row - eastward component. Respective energy fluxes are shown by 
	black arrows. And distribution of phase is given by blue lines separated by 
	$50^{\circ},~1.75~h$. To emphasize differences isochrons for $-90^{\circ}$ and $-180^{\circ}$ 
	are 
	designated by red lines. Note that the components are referenced relative to phase of the 
	barotropic tide at arrival topography, i.e. the westward component  - relative to the 
	Macquarie Ridge and the eastward component - to the Auckland Escarpment. On the upper row 
	background shading shows distribution of conversion rates and on the lower row - mode-1 
	eigenspeed anomaly.}
	\label{C3.fig:wv_fld_dist}
\end{figure}

\begin{figure}
	\centering
	\includegraphics[scale = 
	\SCALEO]{/home/dmitry/Work/Research/thesis/FINALE/P3_ITS_GENERATION/figures/fig_9_ocean_conds.png}
	\caption{Sea surface salinity in the considered experiments (A-C). Lower panels show 
	vertical water column Brunt-Vaisala frequency, temperature and salinity at locations 1 and 2 
	marked on (A-C). The corresponding eigenspeeds of mode-1 are given.}
	\label{C3.fig:ocean_cond}
\end{figure}

\begin{figure}
	\centering
	\includegraphics[scale = 
	\SCALEO]{/home/dmitry/Work/Research/thesis/FINALE/P3_ITS_GENERATION/figures/fig_10_crit.png}
	\caption{Averaged slopes criticality of the eastern side of the Macquarie Ridge. The spatial 
	averaging considered only slopes where criticality was at least 0.7. The color lines show 
	the average for the previously considered experiments (see legend). The grey color is used for 
	other experiments.}
	\label{C3.fig:crit}
\end{figure}

\begin{figure}
	\centering
	\includegraphics[scale = 
	\SCALEO]{/home/dmitry/Work/Research/thesis/FINALE/P3_ITS_GENERATION/figures/fig_11_var_gen.png}
	\caption{Deviation in the conversion due to variations in barotroptic vertical velocity (A), 
		baroclinic mode-1 pressure (B) and phase lag term (C). Spatially integrated values for  
		the gap alone and all ridge are given by respective numbers.}
	\label{C3.fig:var_sp}
\end{figure}

\begin{figure}
	\centering
	\includegraphics[scale = 
	\SCALET]{/home/dmitry/Work/Research/thesis/FINALE/P3_ITS_GENERATION/figures/fig_12_ener_relations.png}
	\caption{Panels show time series in the conversion (filled green), the beam 
	intensity (empty green diamonds), the westward incident flux (black dots) for the Macquarie 
	Ridge (panel A) and the gap alone (panel B). Variability of transmission coefficient is given 
	within purple shaded insets. The coefficients were obtained from the analytical theory (empty 
	diamonds) and from the decomposition analysis (filled dots). On the panel B in the red shaded 
	inset it is also plotted change of the westward wave phase. Results for the TTIDE-simulation 
	are connected by solid lines and the other experiments - by dashed lines.}
	\label{C3.fig:ener_var}
\end{figure}

\begin{figure}
	\centering
	\includegraphics[scale = 
	\SCALEO]{/home/dmitry/Work/Research/thesis/FINALE/P3_ITS_GENERATION/figures/fig_13_regressions.png}
	\caption{Figure examines dependence of the ROMS calculated conversions for the 
	Macquarie Ridge (y-axis) on either the semi-analytical calculation for the knife-edge barrier 
	or the estimate for energy transmitted across the ridge (x-axis). On both panels colors 
	show change in WKB-height and respective values are given by legends.}
	\label{C3.fig:regr}
\end{figure}

\bibliographystyle{apa}
\bibliography{/home/dmitry/Bibtex_lib/my_first_lib}

\end{document}