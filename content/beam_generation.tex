\documentclass[12pt]{article}
\input{/home/dmitry/Work/Research/thesis/FINALE/settings.tex}
%\doublespacing
%\graphicspath{{/home/dmitry/Work/Research/thesis/FINALE/P3_ITS_GENERATION/figures/}}

%\documentclass[PhD_Thesis.tex]{subfiles}

\begin{document}
	\iftoggle{only_Chapter} {
		\title{Generation of internal tidal beam in Tasman Sea}
		\maketitle
	}

\section*{Abstract}
Macquarie Ridge south of New Zealand is a moderate generator of internal tidal waves (ITs) forming 
a tidal beam. The beam was subject to study of TTIDE/TBEAM field program. Here, beam's generation 
and characteristics are investigated by means of numerical experiments with prescribed different 
mesoscale conditions. At the strongest site, conversion of barotropic tidal produce on average 
1.6 GW of baroclinic mode-1 and with a range of $30~\%$. This variation is mainly associated with 
amplitude of remote ITs originating on slopes of Campbell Plateau. The two almost parallel 
conversion sites form a system similar to a semi-enclosed resonator. Its efficiency of energy 
extraction is shown to depend on local stratification by (a) changing WKB-scaled topography and (b) 
phase lag between the sites. The produced baroclinic waves are radiated into deep ocean as a 
spatially tight beam. Obtained here far-field energy characteristics ($152^{\circ}$ and 4 kW/m ) 
are in a good agreement with previously reported altimeter and in-situ observations. Though, the 
numerical experiments points to spatial variation. These modulations arise from interference with 
intricate wave field setting in due to reflection from Tasmania and multiple generators found 
throughout Tasman Basin. Among experiments variability arises both due to spatial changes of 
production hot spots along Macquarie Ridge and interaction with mesoscale eddies found near 
Tasmania. Effect of the former process is estimated by a semi-analytical model. Besides a direct 
dependence of baroclinic tide amplitude on conversion magnitude, its spatial distribution can 
modulate beam heading by $\pm 3^{\circ}$ which results in far-field position shifts. (No ending, no 
strong statement so far).
Macquarie Ridge, Tasman Basin where the, where not.\\
%This might lead to large variation of beam's magnitude. In the far-field beam's can also be 
%refractive by eddy field. All of these processes are discussed in terms of the field observations. 

\section{Introduction}
Baroclinic semidiurnal tides originate as a strong barotropic flow along topography forces heaving 
of isopynal surfaces. This process renders scattering of barotropic tidal energy into baroclinic 
motions \citep{hendershott1981long}. The so dispersed energy constitutes a third of global budget 
for lunar semidiurnal constituent \citep{egbert2000significant, munk1997once} and contributes 
significantly to internal wave climate (ref?). At most conversion sites because of highly inclined 
slopes, internal waves of tidal period (internal tides) are radiated as low baroclinic mode. Due to 
their large length scales decay rates are subtle. This makes low mode tide efficient in carrying 
baroclinic energy (kinetic?) over distances comparable to size of ocean basins. While generation 
sites were identified \citep{morozov1995semidiurnal, simmons2004internal, arbic2010concurrent, 
zhao2016global}, some 
were studied in detail \citep{rudnick2003tides, klymak2011breaking, althaus2003internal} and 
analytical models have been developed \citep{garrett2007internal}, little is known on how fast and 
where internal tidal energy is dissipated (deposited). A lot of uncertainty arises because of close 
relation between the waves and the dynamical oceanic medium, so the wave field is subject to 
continuous change.\\
Water column stratification directly impacts internal wave dispersion. In fact, analytical models 
of generation emphasize a ratio between angle of internal wave characteristics to bathymetric slope 
\citep{garrett2007internal} along with a height of topography as primary quantities in setting 
conversion levels \citep{llewellyn2003tidal, petrelis2006tidal}. For tall, steeply inclined 
submarine ridges the energy transfer approaches an upper theoretical limit 
\citep{petrelis2006tidal, st2003generation} making them to be ``oases" of barotropic tide 
scattering 
and internal tide production \citep{morozov1995semidiurnal, egbert2000significant}. Clearly, the 
energy rates can be modulated by changing buoyancy frequency \citep{holloway1999internal}, 
especially when seasonal transformation of water properties happens at same depths as the steepest 
bottom gradients 
\citep{gerkema2004internal}. Nevertheless, in later field studies it was realized that presence of 
external baroclinic tidal signal leads to even larger temporal variability \citep{Kelly2010a, 
zilberman2011incoherent, pickering2015structure}. This might occur as opposite ridge slopes 
affect each other \citep{nash2004internal, zilberman2011incoherent, echeverri2010internal} or due 
to spatially inhomogeneous distribution of production hotspots \citep{osborne2011spatial, 
ponte2013coastal}\footnote{there should a verb here, needs to be restructured}, or as separate 
topographic features mediate each others generation energy levels \citep{xing1998three, 
buijsman2012modeling, buijsman2014three}. This study addresses temporal and spatial variability of 
tide production happening at Macquarie Ridge, south from New Zealand. Quick recourse to a map of 
Tasman Sea (\fignm{C3:fig:geo.map}) suggests that location of major sea bottom features leads to 
complex internal tide regime representative both of Kaena Ridge and Luzon Strait.\\
Macquarie Ridge emits energy forming a spatially confined beam \citep{simmons2004internal, 
zhao2016global}. This is an ubiquitous characteristic of low mode internal tide propagation in the 
deep ocean that is thought to be a result of multiple source interference 
\citep{rainville2010interference}. The Tasman beam carries away most of the 
conversed energy and partly deposits it on Tasmanian continental slope found $\sim 1000~km$ from 
the ridge. To detail contributing (concurrent) physical processes several field experiments 
(TBEAM/TTIDE/Tshelf) were conducted \citep{pinkel2015breaking} along with an investigation of 
satellite altimetry observations \citep{zhao2018satellite}. The latter results were favorably 
compared to averaged in-situ measurements \citep{waterhouse2018observations} corroborating 
(existence of?) northwesterly propagating low mode beam of small decay rate. Nevertheless, the 
observed temporal variability of the beam's heading and amplitude needs an interpretation to 
restrain boundary conditions for a problem of shoaling (scattering and reflection?) internal tide 
on three dimensional topography and consequent energy dissipation \citep{klymak2016reflection}.\\
The non-stationary behavior in propagation of the baroclinic tidal waves results from interaction 
with varying oceanographic conditions \citep[e.g.,][]{mooers1975several}. Depending on involved 
length scales and magnitudes, different regimes can be realized \citep{buhler2014waves}\footnote{I 
didn't read Buhler, 2014 book, just he discusses 
in great detail the topic}. On the first order, when the (wave-flow) scales are largely detached as 
in  
geometric optics limit, the oceanic conditions simply change mode-1 phase speed and cause wave 
front refraction \citep{rainville2006propagation, zaron2014time, kelly2016internal}. The phenomena 
is augmented in presence of considerable mean flows with vertical structure 
\citep{park2006internal, buijsman2017eq}. This can further produce non negligible Doppler 
shifting 
\citep{chavanne2010surface} and shifts of apparent wave frequency in strong vortical flows 
\citep{kunze1985near}. In the higher orders, nonlinear interactions lead to scattering into 
high modes \citep{dunphy2014focusing}, directional spreading \citep{wagner2017asymptotic, 
dunphy2017low} and nonintuitive energy transfers via resonant triad interactions with geostrophic 
turbulence \citep{ward2010scattering}. Still in typical oceanographic setting the first order 
mechanisms are the most widespread \citep{kelly2016internal, zaron2014time}. The latter 
work as well investigated role of generation in the producing time variable far-field. This was  
 hypothesized by \citep{wunsch1975internal} who suggested that ``energetic beams will be moved 
comparatively large distances by small changes in angle and may be missed by isolated instruments" 
\footnote{this is a direct quote, marks are ok?}.\\
In the setting of Tasman Sea both phenomena are plausible reasons to produce the documented 
variation 
in incidence of the low mode tidal beam on Tasman continental slope. To identify cause-and-effect 
relationship numerical experiments are carried out with different conditions of the oceanic medium 
(Section 2). Variable levels of internal tide production are examined in Section 3a and 
resultant beam's characteristics are quantified in its traverse of Tasman Basin (Section 3b). 
These results are brought together to be studied in terms of a semi-analytical generation 
model and action of mesoscale (Section 4a). This helps to provide context for the field 
observations (Section 4b). This follows by conclusions. And in the Appendices mathematical nuances 
are described in greater detail.

\newpage

\section{Numerical experiments and analysis}
\subsection{Numerical experiments}
To study variability of internal tide generation around New Zealand and its propagation numerical 
simulations were performed with Regional Ocean Modeling System \citep{shchepetkin2005regional}. 
The numerical domain covered southern Tasman Sea from subantractic waters of $60^{\circ}$ S 
to subtropics in $35^{\circ}$ S. And the zonal extent stretched from $142^{\circ}$ to $172^{\circ}$ 
E. This ensued correct representation of reach regional oceanographic conditions. The horizontal 
grid spacing was taken to be of $1/32^{\circ}$ corresponding on average to discretization of 3 km 
in zonal direction and 2.5 km in meridional. The nonuniformly separated, vertical 50 $s$-levels 
were placed to smoothly follow subsurface terrain.\\
\begin{figure}
	\centering
	\includegraphics[scale = 0.35]{../figures/fig_1.png}
	\caption{Domain of numerical simulations with geographical locations used in the text}
	\label{C3:fig:geo.map}
\end{figure}
Such discretization of vertical momentum equation tends to induce artificial, horizontal  
along-slope flows \citep{haidvogel1999numerical} due to errors in reproducing of pressure 
gradient force. Especially severe errors are made by steep terrain. The misbehavior is usually 
solved by aritificial smoothing of topography. This procedure additionally increases numerical 
stability, but has an adverse effect on internal tide generation \citep{di2006numerical} since 
primary production sites are collocated with large topographic gradients. To test the numerical 
setup, a sensitivity study was carried out with simulations of 
$1/8^{\circ},~1/16^{\circ},~1/64^{\circ}$ horizontal resolution. The essential for this study 
internal tide behavior manifested at $1/16^{\circ}$ and converged for $1/32^{\circ}$ and 
$1/64^{\circ}$ cases. There no marked differences were observed, except a substantial increase in 
high mode content which is in line with \footnote{previous investigations} 
\citep{di2006numerical}.\\
This work addresses the gravest baroclinic mode dynamics in the deep ocean. Spatial extent of 
waves is large compared to associated vertical displacements. This ensures linear regime of 
propagation without dispersive and nonhydrostatic effects taken place such as fission into 
solitons. A hydrostatic solver used in ROMS seems to be a proper choice for the simulations. Such 
simplification in wave dynamics was assumed in previous studies (\citep{carter2008energetics, 
merrifield2001generation,  merrifield2002model, kerry2013effects}). In more dynamically accurate   
simulations of \citep{kang2012energetics, zhang2011three} the nonhydrostatic effects are found to 
be important only for internal tides in shallow waters, while for main part generation follows 
linear dynamics with vertical accelerations to have a negligible contribution.\\
The horizontal boundary conditions were imposed to be open for depth-averaged, barotropic flows 
following recommendations proposed by \citep{marchesiello2001open}. The baroclinic fields are 
nudged to zero by linear increased lateral viscosity and diffusivity over sponge layers. Through 
the same outer boundaries numerical simulations were forced with barotropic tide. The tidal 
currents and sea level are derived from TPXO atlas, version 7.2 \citep{egbert2002efficient} and 
prescibed as linearly interpolated volume transports. It was used only the largest semidiurnal 
constituent $M_2$. Amplitude ratio between the principal lunar and solar components are 4-to-1 
suggestive of slight open-ocean spring-neap modulation. The diurnal species are weak in the region 
except shoals east of New Zealand \citep{walters2001ocean}.\\
To investigate variations of baroclinic tide dynamics several ocean states were prescribed and 
analyzed separately. In the simplest setting lateral gradients in water properties were absent, 
while buoyancy frequency was set to representative of Tasman Basin. The second set of simulations 
was comprised to investigate interannual and interseasonal variability (Table 1). And the third 
calculation was intended to cover period of TTIDE/TBEAM/Tshelf field programs 
\citep{pinkel2015breaking}, a single experiment once initialized was left to proceed for three 
numerical months.
\begin{table}
	\caption{Carried out numerical experiments}
	\begin{tabular}{ |p{3cm}||p{5cm}|p{5cm}|  }
		\hline
		\multicolumn{3}{|c|}{Numerical experiments used in this study} \\
		\hline
		Experiment abbreviation & Simulation period & Comments (reason?) \\
		\hline
		Uniform & ~ & No mesoscale \\
		2012 &   Jan 1st - Jan 15th, 2012 & Interannual \\
		2013 &   Jan 1st - Jan 15th, 2013 & Interannual \\
		2014 &   Jan 1st - Jan 15th, 2014 & Interannual \\
		2013\_Oct &   Oct 1st - Oct 15th, 2013 & Interseaonal \\
		2015\_Mar &   Mat 1st - Mar 15th, 2015 & Interseaonal \\
		2015\_TTIDE$^{\ast}$ &   Jan 1st - Mar 1st, 2015 & Field period \\
		\hline
		\multicolumn{3}{|l|}{\footnotesize$^{\ast}$ the results are named as respective day of 
		year over which post-analysis was performed, e.g. $d20-25$ }\\
		\hline
	\end{tabular}
	\label{ch2:table_exp}
\end{table}
The simulations with variable conditions were at first initialized by HYCOM hindcasts 
\footnote{(NAVGEM;	downloaded from hycom.org)} for respective start date. Then during integration, 
along with barotropic tidal flow, time-variable, subtidal two dimensional fields 
\footnote{(vertical coordinate and along boundary coordinate)} of horizontal currents, temperature 
and salinity were imposed onto the numerical ocean. The air-sea interaction obtained from 
MERRA-reanalysis \citep{rienecker2011merra} was also given by insolation, air  temperature, EP 
rates and most importantly, wind stresses.

\subsection{Internal tide analysis}
As it is seen in table 1, the simulations were carried out for 15 days or longer. The first 10 days 
were left for spin up of baroclinic tide generation and propagation. Roughly, it takes about 7 
days for the mode-1 signal to traverse Tasman Sea from New Zealanda to Tasmania. After that 
period, three dimensional fields of velocity, temperature and salinity were sampled hourly. These 
were 
later subject to high pass filtering with Butterworth filter of order $6$ with cut off time of $36$ 
hours. This removed subtidal motions and left out signal was further fit in a least 
square sense to the principle semidiurnal harmonic. Then the three dimensional fields 
underwent a separation into barotropic and baroclinic signals \citep{cummins1997simulation, 
kunze2002internal, carter2008energetics}. A depth-averaged current is thought to represent a pure 
barotropic signal and any vertical deviation is attributed to a baroclinic wave,
\begin{equation}
\label{ch2:bt_bc_vel}
\vec{u}_{bt}(x,y) = \frac{1}{H} \int_{-H}^{0} \vec{u}(x,y,z)  dz,~\vec{u}_{bc}(x,y,z) =  
\vec{u}(x,y,z) - \vec{u}_{bt}(x,y)
\end{equation}
To describe distribution of pressure, at first, from a linear equation of state and respective 
\textit{TS}-fields density perturbation from the reference is found. Then the hydrostatic 
approximation is employed and after vertical integration the total pressure field is found. This 
is then subject to baroclinicity condition, so that baroclinic pressure anomaly is taken to be a 
deviation 
from the depth-averaged,
\begin{equation}
\label{ch2:bt_bc_pres}
p(x,y,z) = \int_{-z}^{0} \rho(x,y,z) dz,~p_{bc}(x,y) = p(x,y,z) - \frac{1}{H} \int_{-H(x,y)}^{0} 
\rho(x,y,z) dz
\end{equation}
In the both expressions rigid-lid approximation is used. This is a valid statement unless vertical 
accelerations are smaller than acceleration due to gravity which is true except shallow depths 
\citep{kelly2010}.\\
Each dynamical variable was then decomposed into vertical modes. The structure functions were 
obtained from local Brunt-Vaisala frequency profiles found from time-averaged density fields. 
These were used in Sturm-Liouville problem for the hydrostatic approximation,
\begin{equation}
\frac{d}{dz}((\frac{\omega^2 - f^2}{N^2} ) \frac{d \psi(z)}{dz}) + c^2_n \psi(z) = 0
\end{equation}
where $c_n$ is the mode phase speed in nonrotating ocean. The first 3 vertical modes were 
fit into three-dimensional fields. And only mode-1 was used in the following results.\\
Now energy diagnostics could be obtained. First, depth-averaged mode-1 energy flux is
\begin{equation}
\vec{F} = \frac{1}{2} \frac{1}{H} \cj{\vec{u}} p \int_{-H}^{0} \psi_1(z) \psi_1(z) dz
\end{equation}
At second, rates of conversion from barotropic to baroclinic \citep{simmons2004internal, 
kurapov2003m} were calculated as
\begin{equation}
\label{C3:eq.conv}
C_{bt\to 1} = -\frac{1}{2}(\cj{\vec{u}_{bt}} \cdot \nabla H) p_{1,~bot}
\end{equation}
The fraction $\frac{1}{2}$ in front of the energy characteristics appear because harmonic, complex 
amplitudes are used in the expressions.\\
These calculations had produced a set of dynamical variables of barotropic and baroclinic 
fields in each experiment. The obtained values per experimenter hereafter will be referred as a 
realization. For instance, the longest experiment, 2015\_TTIDE had 10 realizations. To study 
variability of the system, mean values were defined as arithmetic mean,
\begin{equation}
<\bullet> = \frac{1}{N} \sum_i \bullet_i
\end{equation}
where $\bullet$ a field being averaged and $N$ is a number of experiments used. The variation 
between realizations is studied by mean deviation,
\begin{equation}
\Delta \bullet = \frac{1}{N} \sum_i (\bullet_i - <\bullet>)
\end{equation}

\subsection{Discrete Fourier Decomposition by inverse modeling} \footnote{I will move it to 
Appendix and will leave just a paragraph or two}
In addition to the above characteristics the mode-1 internal tide field was subject to directional 
analysis in order to remove interference modulations. Similar methods 
were used previously in internal tide field programs \citep{hendry1977observations, 
lozovatsky2003spatial} \footnote{that were based on array beamforming method and stationarity of 
the field} or 
satellite altimetry \citep{dushaw2002mapping} or in surface wave studies 
\citep{longuet1961observations, munk1963directional, long1986inverse}. Let 
mode-1 pressure in complicated seas to be described by an angular spectrum
\begin{equation}
\label{C1:eq.spectrum}
p(\vec{r}, t) = \int_0^{2\pi}  d\theta_k S(\theta_k) e^{i \vec{k}(\theta_k) \cdot \vec{r} + 
\phi(\theta_k) - i \omega t}
\end{equation}
Here each elementary (monochromatic) sine wave of wavenumber $k$ travels in direction $\theta$ with 
energy $S(\theta)^2 d\theta$ and temporal (spatial) lag of $\phi(\theta)$. The statement can be 
reformulated in terms of Fourier coefficients \citep{munk1963directional} by application of 
Jacobi-Anger expansion,
\begin{equation}
p(r, \theta) = e^{i \vec{k}(\theta) \cdot \vec{r}} = \sum_{m = -\infty}^{m = \infty} i^{m} J_{m}(k 
r) e^{im(\theta - \theta_k)}
\end{equation}
shows that a field at point $(r, \theta)$ produced by plane wave can be expanded in series of 
Bessel functions and circular functions. Then its substitution into \eqref{C1:eq.spectrum} and 
reorganization lead to
\begin{equation}
\label{C1:p.eq}
p(r, \theta) = \sum_{m=-\infty}^{m=\infty} \big[ \int_0^{2\pi}  d\theta_k S(\theta_k) 
e^{i\phi(\theta_k)} e^{-im\theta_k} \big] i^m J_m(kr) e^{im\theta}
\end{equation}
Term in brackets (square brackets) represent convolution integrals defining Fourier coefficients of 
order $m,~A_m - i B_m$. Thence, series \eqref{C1:eq.series} state a model equation to find the 
unknown coefficients from the known, measured pressure field that were sampled at a set of points 
$(r_i, \theta_i)$ and if infinite series is truncated at some order $N$. Real and imaginary parts 
will constitute two separate problems allowing deterministic definition of the spectrum.\\
The same steps are repeated but with current velocities instead. Plane wave  
polarization relations \citep[e.g.,][]{muller2000scattering} are inserted into 
\eqref{C1:eq.spectrum} and the following equations are found,
\begin{align}
\label{C1:uv.eq}
\begin{Bmatrix}
u_i \\ v_i
\end{Bmatrix}
= \frac{1}{2} \sum_{m = -N}^{m = N} J_{m} (kr_i) e^{im(\theta + \pi/2)}
\begin{Bmatrix}
(\omega - f) A_{m + 1} + (\omega + f) A_{m - 1} - i [(\omega - f) B_{m + 1} + (\omega + f) B_{m - 
1}] \\ 
(\omega - f) B_{m + 1} - (\omega + f) B_{m - 1} + i [ (\omega - f) A_{m + 1} - (\omega + f) A_{m - 
1}]
\end{Bmatrix}
\end{align}
The dependence of currents on wave bearing causes splitting of Fourier coefficients and 
asymmetry via Coriolis effect. This results points out that to describe velocity field 
higher circular harmonics have to be used. Physically, velocity field has higher spatial 
wavenumber. But in \eqref{C1:uv.eq} additionally, the asymmetry is observed 
for clockwise and counterclockwise components.\\
The inverse model combines dynamical relations of \eqref{C1:p.eq} and \eqref{C1:uv.eq} into a 
matrix equation
\begin{equation}
y = K x
\end{equation}
Generally, it is unstable to small errors in data and produce physically inconsistent results. This 
can circumvented by seeking a damped least square solution \citep{munk2009ocean} where a 
minimization function is given by
\begin{equation}
\label{C1:Tikh_prob}
J = ||K x - y||^2_2 + \alpha ||x||^2_2
\end{equation}
The unknown regularization parameters $\alpha$ acts as a high-pass filter in a singular value 
decomposition of $K$ \citep{bennett1992inverse}. In field studies this is usually set by a 
signal-to-noise ratio \citep{munk2009ocean}, since the parameter scales noise variance (residue) 
to actual signal's strength. To obtain $\alpha$ in data-driven way a straightforward approach is 
adapted that based on 
trade-off curve method \citep{hansen1993use}. In \eqref{C1:Tikh_prob} amount of allowed error 
is competing with solution's variance. An optimal parameter should balance these factors. This is 
seen as a rapid change in behavior of curve associating residue with model's norm as regularization 
varies. In most cases the curve has a sharp corner connecting aforementioned limits, hence, the 
method's name is a L-curve \citep{hansen1999curve}. And the corner is to occur for an optimal 
regularization parameter.\\
The equations \eqref{C1:p.eq} and \eqref{C1:uv.eq} are sampled at locations in a concentric 
arrays placed at $\lambda,~0.5\lambda,~0.25\lambda$ where $\lambda$ is a local mode-1 wavelength. 
At each location $u,~v,~p$ are used as data and for a region embraced by array Fourier coefficients 
are found. And these then are used in reconstructions.\\
The method used here is different from \citep{zhao2010long} for two main reasons. The model 
equations produce simultaneous fit of all the components, rather than a finite number of a single 
directed plane waves. This can make a difference in regions where diffraction is important such as 
near internal tide generation or scattering regions. And at second, velocity field is utilized 
which provides an additional constrain. Moreover, in synthetic experiments with 
\eqref{C1:Tikh_prob} where instead of $L2$-norm regularization it was used $L1$-norm, the results 
were approaching one of plane wave technique of \citep{zhao2010long}. Additionally, the proposed 
method can be utilized for a single mooring where half-space separation is necessary.

\section{Results}
\subsection{Generation of internal tidal beam}
Surface tide arrives to Southern Tasman Sea from North (\fignm{C3.fig:BT}). Its 
advancement happens in counterclockwise manner with maximum amplitude of sea level located along 
New Zealand's coast. This is a typical Kelvin wave behavior \citep{walters2001ocean}. Also the 
barotropic tide produces strong currents in shallow Bass Strait, but relatively weak anywhere else 
in the basin. The simulated sea surface tidal oscillation closely follows TPXO atlas with gross 
features well captured. Presence of baroclinic field manifests in perturbation of sea level 
magnitude and cotidal lines. In the basin this have a striking wavy character. This corresponds to 
propagation of low mode tidal wave (\fignm{C3.fig:beam}).\\
The baroclinic tidal field represents a complex pattern produced by multiple generation sites 
defined by steep topography. Primary production sites are just south of New Zealand. Here 
barotropic Kelvin wave faces prominent Macquarie Ridge stretched for 2000 km. As barotropic current 
decays away from the coastline conversion lessens as well. Nevertheless, low-mode beams are emitted 
from many locations. Major conversion happens at $49.5^{\circ}$ shedding away the strongest beam. 
This and two nearby beams were identified in altimetric observations \citep{zhao2018satellite}. 
Henceforth, analysis is concentrated on the most energetic, central beam.\\
The central beam is produced by tidal currents impinging on supercritical bathymetry 
(\fignm{C3.fig:gen}). Depth of the highest conversion is between 1000-3000 m and spatially confined 
to two seamounts that are separated with a sill. It has less inclined slopes and plays 
lesser role in production of the tidal beam. On average, this region of Macquarie Ridge converts 
1.6 GW of surface tide. This is half of production of Kaena Ridge, Hawaii 
\citep{carter2008energetics} and much less than Luzon Strait \citep{}. Pattern of conversion 
(\fignm{C3.fig:gen}) also exhibits regions of internal tide destruction produced by complex 
dynamics caused by superposition. In fact, an oppositely located Aucklands Escarpment presents an 
important 
source of 
baroclinic energy. \fignm{C3.fig:beam} clearly illustrates existence of a standing wave in 
Solanders Trough. The total field is characterized 
by a node in horizontal kinetic energy with fluxes revolving in counterclockwise direction 
\fignm{C3.fig:stand_wave} because of Southern Hemisphere. By the method proposed in 
ref-to methods, the standing wave is separated into elemental east-west directed components 
(\fignm{C3.fig:stand_wave}). On leeward side of Macquarie Ridge generation occurs at the same 
seamounts but there is a region of destruction that coincides with incidence of waves emitted by 
the escarpment. There generation has reacher structure both due to more complicated 
topography that is crisscrossed by canyons but also because of Macquarie ridge produced waves. 
Quantifying energy transfer across the trough and comparing with spatially integrated conversion 
rates points out to fact that wave energy is being recirculated by slope's supercritical 
reflection and only partly fed by barotropic field. In overall, such system is similar to Luzon 
strait where resonance conditions exist between two parallel ridges \citep{buijsman2014three}. In 
case of Macquarie Ridge and Campbell Plateau resonance is only partial since the former has a slant 
orientation of $15^{\circ}$. Though at $49.5^{\circ}$ the distance corresponds to 3/4 of mode-1 
wavelength. Such spacing is sensible to phase lags and can either lead to intensification or 
destruction of generation. This is illustrated by comparison of two simulations that presents 
dynamically different regimes of generation.\\
%While \eqref{C3:eq.conv} provides a convenient way to quantify energy transformation, it does 
%not have much room for physical interpretation. Such as in complicated situation when along with 
%local baroclinic tide production a remote signal is present, resultant perturbation of bottom 
%pressure might lead to rather ambiguous result of internal tide destruction. It is said that such 
%regime is to occur when a phase difference between $w_{bt}$ and $p_{1}$ is in range of 
%$(\pi/2,~3\pi/2)$. This is understood as an internal tide performing work against barotropic 
%forcing. Unfortunately, this statement does not directly follow from \eqref{C3:eq.conv}. Hence, to 
%provide cleaner physical picture, let derive expression for conversion rate from the first 
%principles. Body force of \citep{baines1982} is performing work by displacing isopycnal surfaces 
%throughout a water column,
%\begin{equation}
%\label{C3:eq.convd1}
%C(z) = \frac{dW(z)}{dt} = F_{B} w_{bc} = \frac{N^2 (-\vec{u} \cdot \nabla h) z}{i \omega h} 
%\frac{d 
%\xi}{dt} = \frac{w_{bt}}{i \omega h} \frac{z d(-b)}{dt} = -\frac{1}{h} w_{bt} zb
%\end{equation}
%where isopycnal displacements $\xi$ were changed to buoyancy, $b = -N^2 \xi$ and temporal 
%variation 
%was assumed to be harmonic, $\sim e^{i \omega t}$. On a final step, integration by parts can be 
%employed as,
%\begin{equation}
%\label{C3:eq.convd2}
%\int_{-h}^{0} z b dz = \int_{-h}^{0} z d \big( \int^0_{z} b dz \big) = \big( z \int^0_{z} b dz 
%\big)\big|_{-h}^0 - \int_{-h}^{0} dz \int_{z}^{0} b dz^{\prime} = h (\int_{-h}^{0}b dz - 
%\frac{1}{h} \int_{-h}^{0} dz \int_{z}^{0} b dz^{\prime})
%\end{equation}
%The last expression is a bottom pressure perturbation. Noteworthy, a baroclinicity condition was 
%not employed. Combining \eqref{C3:eq.convd1} and \eqref{C3:eq.convd2}, familiar result 
%for conversion rate is obtained. Note that a similar approach to \eqref{C3:eq.convd1} was used 
%by \citep{nash2006structure} to estimate an upper limit on emitted energy.\\
Conversion rate shows how much work was done by baroptropic tide to displace isopycnal interfaces. 
This is understood as work against buoyant forces. But it can happen that barotropic forcing will 
be oppositely directed if somewhere in the water column other forces are present. For case of 
'2014' 
simulation (\fignm{C3.fig:gen_2d}, a-b) during ebb tide on tideward side there is net energy 
conversion to baroclinic field even that along bottom an internal wave ray is developing by upward 
displaced interfaces as a result of previous tidal phase. Hence, at these location a newly 
generated internal wave does work against downward barotropic flow. And this produces negative 
conversion (\fignm{C3.fig:gen_2d}, e) at some moment. As tide turns conversion changes sign 
again. Overall, period averaged transfer is positive, i.e. surface tide losses energy. In 
the other experiment presented (d10-15, \fignm{C3.fig:gen_2d}, c-d), there is an intensification 
due 
to advancement of a mode-1 wave. In actuality, its propagation from Solanders Trough 
($165^{\circ}$) 
and over the sill is the major difference between two simulations. So in '2015'-setting similar 
along slope advancement of an internal wave ray is observed, but now due to shoaling mode-1 
conversion is positive throughout tidal cycle.\\
The contrary situation is found on leeward side ($164.5^{\circ}$) where surface tide current has the
opposite direction. For '2015' it appears that propagating mode-1 is losing energy since it does 
work against barotropical forcing by dipping isopycnals. Though in transition some energy is lost 
from surface tide. In '2014' there is a reflection of an internal wave ray as phase is advancing 
onto the sill. This coincides with upward barotropic flow that in total leads to intensification. 
In total, period averaged conversions have different signs on leeward side. Comparison of mode-1 
wave 
position in the trough at ebb tide (\fignm{C3.fig:gen_2d}, a,c) suggests difference in timing of 
the 
remote wave arrival or its advancement. This is explored on (\fignm{C3.fig:gen_2d}, g). The total 
signal shows a region of low phase change eastward of $165^{\circ}$ that corresponds to 
concentration of kinetic energy (\fignm{C.3:stand_wave}, a). Decomposition of the signal presents 
it roughly as a sum of the ridge generated waves (eastward propagation) and the escarpment 
originated (westward) waves. The actual relation will depend on relative magnitudes 
\citep{martini2007diagnosing}. 
But westward of $165^{\circ}$ there is almost a free propagation in '2015' of the ``escarpment" 
waves as the total signal closely corresponds to them. But in '2014' due to sills generation and  
strong reflection, phase difference with local baroptropic tide falls below 
$90^{\circ}$, so that there is an intensification of generation.\\
The described situation provides a several approaches to narrate variability in conversion rates. 
At first, it is clear that amount of remote energy crossing Macquarie Ridge through the sill will 
shape the overall conversion. Hence, energy incident from leeward side can provide such 
estimate. It is quantified from the numerical simulations by line-integration of energy fluxes 
through leeward side (\fignm{C3.fig:gen}). Additionally, this amount is modulated by reflection of 
the 
sill. Here it is thought as a knife-edge barrier for which reflectivity was analytically found by  
\citep{larsen1969internal} with similar investigations of \citep{klymak2013parameterizing}.  
Sill's depth is obtained as a depth jump from the trough to the sill that were WKB-scaled. The 
resultant calculation is given by (\fignm{C3.fig:gen_regr}, a) where total conversion of tideward 
side 
of Macquarie Ridge is plotted against amount of transmitted mode-1 energy. At second, efficiency 
of generation will depend on phase difference between the remote waves and local forcing. This is 
estimated from mode-1 eigenspeed for Solanders Trough. Taken distance separating Macquarie Ridge 
and Aucklands Escarpment to be about $115~km$ (average mode-1 wavelength $155~km$), the time to 
cross the trough can be found (\fignm{C3.fig:gen_regr}, b). And additional factor is environmental 
changes that are associated with overall efficiency of generation by the ridge. Again by applying 
WKB-scaling variation of ridge's depth relative to the surrounding deep ocean is found and then 
scaled to converted energy by application of theory by \citep{st2003generation}. The 
mean barotropic current of $0.03~m/s$ generation was considered to produce 
(\fignm{C3.fig:gen_regr}, 
c).\\
The linear regression for the first parameters has correlation coefficient $R^2$ slightly higher 
than 0.5, and for linear multiple variable regression it increases to $0.7$. The least dependence 
is found for stratification variability on the tideward side. Though if all three parameters 
combined $R^2$ becomes $80\%$. Hence, most variability is associated with leeward dynamics and not 
local stratification. Additionally, all three environmental parameters do not always change in 
similar fashion suggestive for different mesoscale dynamics occurring in the deep sea, over 
Macquarie Ridge and in Solanders Trough. This is not surprising since the region is affected by 
frontal zone and reach in subtidal dynamics \citep{smith2013interaction}. Further, inclusion of 
transmission coefficient increases correlation, but most variability comes from amount of energy 
traveling from Aucklands Escarpment 
(\fignm{C3.fig:stand_wave}, c). This is much harder problem to estimate since generation in that 
region has much more spotty character (\fignm{C3.fig:gen}) because of complex topography and large 
influence of eastward traveling waves which can also act either to intensify local generation or 
destroy baroclinic tides.
%So in total, generation at Macquarie Ridge is occurring in 
%complex dynamical environment that largely modulated by ocean conditions. 


%Generation of baroclinic tide primarily happnes south of New Zealand. 
%in circular fashion 
%Southern Tasman Sea represents a case of complicated of 
%internal tidal field. Even baroptropic 
%currents are 
%small and topography is not steep, there are a lot of regions surrounding the Southern Tasman sea 
%to produce internal tides (Figure 1). To name a few Southern Tasman Rise, Lord Howe Rise, Bass 
%Strait. Though the primary production site is found at Macquarie Ridge that stretches poleward 
%from 
%New Zealand for almost 2000 km. There several internal tidal beams are emitted along different 
%sections. These were previously identified in satelite altimetry \citep{zhao2016global}. The major 
%beam is produced where most of energy transfer from baroptropic tide to baroclinic field happens 
%at 
%$45^\circ$S.\\
%The steep topography and strong semidiurnal currents produce convergence of barotropic flow. This 
%primarily occurs at two seamounts and along 3000 m isobath. Here the topography is super critical 
%internal waves to be radiated in the low mode signal. The other generation is located by sill 
%connecting two seamounts. This pattern and its spatial distribution is subject to variability. As 
%it is observed there could occur a shift in position.\\
%As it is seen on Figure 1 this region has a complex internal wave field. The changes in generation 
%attributed to interaction of remote tidal waves and locally generated. This is illustrated by two 
%opposite experiments, 2014 and 2015. It is quite obvious to see marked difference in Solander 
%trough. On tidal flood stage barotropic tide is climbing up flanks of Macquarie Ridge causing 
%development of vertical velocity. At the same time, in 2015 there is a propagation of mode-1 
%internal wave which then slams generation. On the opposite, generation at 2014 shows beam pattern, 
%so there is a direct generation of internal wave. This is then emphasized by period-averaged 
%conversion. It shows different signs of conversion.\\
%To study these difference we perform directional decomposition (Figure). The total field is 
%comprised of two waves oppositely directed: one generated on lee side of Macquarie Ridge and the 
%other at Aucklands Escarpment they result in a superposed field. Rather than employing mechanism 
%of 
%resonance it is more apparent that the total field and consequent energy transfer happens 
%depending 
%on amplitudes of the mentioned waves and due to different slanting angles. In the simplest case of 
%equal ratio there is no energy transport and portion of it is directed towards.\\

\newpage
\subsection{Characteristics of the Tasman tidal beam}
Widespread energy conversion at the Macquarie Ridge produce a clearly defined internal tidal beam 
(\fignm{C3.fig:beam}). Here its energy characteristics are investigated in terms of spatial 
averages and ensemble means. The spatial averaging takes place in across beam direction 
illustrated on \fignm{C3.fig:beam}, the ensemble mean is defined as an arithmetic mean of energy 
quantities among all the experiments. The results are given on \fignm{C3.fig:beam_prms} for 
spatially-averaged orientation of the energy fluxes to represent beam's heading. Then beam's 
strength is expressed in terms of the across-beam integrated flux. And to diagnose underlying 
dynamics the final panel shows ratio of spatially-averaged horizontal kinetic energy (HKE) to 
available potential energy (APE). All 
these quantities exhibit high degree of spatial and temporal variability as regions of 
highs and lows interchange throughout the beam propagation. At first, this complexity is left 
behind by fitting a second-degree polynomial to the ensemble-means (\fignm{C3.fig:beam_prms}, 
dashed lines). The fits now emphasize large-scale, gross characteristics of the beam propagation. 
Such as by the Macquarie Ridge the beam is mainly oriented towards west, but then markedly turns by 
$25^\circ$ towards north. This transition occurs over the first $200~km$. And further, there is 
only a slight tendency to the equator. Away from the generation site, this open-ocean behavior is 
in accord with previous studies \citep{cummins2001north, rainville2006propagation, 
zhao2018satellite}. They recognized meridional decrease of Coriolis parameter as a cause for 
decline in phase speed and equatorward refraction of the internal tides. Thereby, near Tasmania the 
beam is directed northwesterly by angle of $150^\circ$.\\

The other important feature is a decay in the beam's strength. Initially, it carries 1.4 GW 
(\fignm{C3.fig:beam_prms}, (b)), but then the energy flux vectors are slowly diminishing. This 
decline is induced by two main factors. At first, geometrical spreading lessens wave energy 
crossing the constant integration transects. At second, dynamical processes such as friction, 
energy transfers to higher modes and interaction with ocean circulation introduce irreversible 
losses into the wave field. To account only for the physical mechanisms, the geometrical effects 
are removed by calculating energy flux divergence. This produced a mean estimate of $35~\%$ energy 
loss along the basin transverse and the beam brings $0.9~GW$ of mode-1 energy to Tasmania.\\

On scales comparable to mode-1 wavelength ($\sim 180~km$) the depicted gross characteristics are 
perturbed by  
notable deviations. For instance, energy flux vectors divert by $\pm 30^\circ$ following an 
undulating pattern (\fignm{C3.fig:beam_prms}, (a)). Similar spatial variability is observed in the 
other characteristics. In fact, there is a mutual relation between the alterations. Each turning of 
the 
flux vectors aligns with either growth or decline in magnitude and hence, in the integrated flux. 
At these local extrema the mechanical energy is partitioned similarly to a theoretical plane wave 
relation (\fignm{C3.fig:beam_prms}, (c)). On contrary, where horizontal kinetic energy overtakes 
available potential energy, the mean flux orientation aligns with the large-scale propagation 
pattern and the flux magnitudes are representative of the gross beam's intensity. This situation as 
well 
holds when $HKE < APE$. The regions where one form of the mechanical 
energy prevails correspond to nodes and antinodes. \footnote{In this study such terminology is used 
in more 
general sense since no pure standing waves were found.} Respectively, there baroclinic 
currents are at 
its strongest or pressure amplitudes are at maximum. Therefore, energy travels in a winding 
fashion going round a system of nodes and antinodes. This aspect of propagation is indicative of 
interference \citep{martini2007diagnosing, zhao2010long} between the internal tidal beam and 
distant baroclinic tides.\\

For further investigation the superposed wave field is decomposed by the same method of 
directional spectra (Appendix A) as before. However, this time multi-directional components are 
\footnote{back-}synthesized via integration over 4 circular quadrants. The so-obtained wave fields 
are then subject to the same procedure of spatial and temporal averaging. The results are shown on 
\fignm{C3.fig:beam_dcmp}. Here, not surprisingly, the northwestern quadrant 
\fignmlp[C3.fig:beam_dcmp]{a} contains the largest wave-field components and highlights plane-wave 
advancement of the tidal beam. Further on, this extracted component will be referred to as a 
planar. This major property is emphasized in a ratio between HKE and APE that closely follows the 
theoretical prediction. \footnote{Notable that the gross characteristics given on 
\fignm{C3.fig:beam_prms} are quite representative of the planar beam.} As well 
in-situ\footnote{apparent} group speed defined as a ratio of flux magnitude to total mechanical 
energy is in close agreement with the mode-1 dispersion relation\footnote{Minor oscillations are 
artifacts of analysis owing to finite size of fitting windows. }. Its relevance is additionally 
augmented by slight meridional increase in the group speed for the same reason as the meridional 
internal tide refraction. On average, it takes 6.8 days with the speed of 1.7 m/s to transmit 
energy 
from the 
Macquarie Ridge to Tasmania. This estimate is close to one obtained during the TBEAM study when 
``age" of the tide, i.e. lag period between observations of spring-neap modulation in barotropic 
and baroclinic fields, was found to be 6.7 days \citep{waterhouse2018observations}. Nevertheless, 
the field observations have shown spatial variations of the in-situ group speed and large 
deviations from the plane wave value. This point is discussed further as the planar beam becomes 
obscured by interference.\\

Distant baroclinic waves are radiated from multiple bathymetric features scattered around the 
Tasman Sea. The Lord Howe Rise and Gilbert Seamount \fignmp{C3:fig:geo.map} produce southwesterly 
wave 
components shown by the flux vectors on \fignml[C3.fig:beam_dcmp]{b}. They intersect the planar 
beam in the middle of the Tasman Basin. The resultant superposition is noticeable in APE field 
\fignmlp[C3.fig:beam_dcmp]{b} as undulating spatial pattern emerge. Length scale of the modulations 
will be given as wave-vector difference of the two approximately perpendicular fields, $2 
\pi/\Delta = \frac{1}{2} (\vec{k}_{beam} - \vec{k}_{SW}) = \frac{1}{2} \frac{2 \pi}{180~km} \cos 
\frac{\pi}{4} \rightarrow  \Delta = 255~km$. This is the distance between two consequent nodes or 
antinodes. On the other hand, the HKE-to-APE ratio is set by distance between a node and an 
anti-node and hence, oscillates twice faster, i.e. $\sim 130~km$. This roughly corresponds to the 
observed spatial variation of $180~km$, the estimate can be improved by considering more realistic 
angle between the two interfering wave fields. Further, as the southeastern quadrant 
\fignmlp[C3.fig:beam_dcmp]{c} is taken into account, the modulations show $\sim 90~km$ length scale 
of variation. This is a result of interference with oppositely traveling waves produced as the 
planar beam reflects from Tasmania. The partly standing wave is especially noticeable near the East 
Tasman Plateau where regions of high APE are alternating with low-level bands. The pattern sharpens 
further as baroclinic tides emitted at the South Tasman Rise are added 
\fignmlp[C3.fig:beam_dcmp]{d}. 
Now the overall field has only weak resemblance to the planar beam as small-scale spatial structure 
prevails. It is evident in antiphase relation between the given energy characteristics. Notable 
that the ratio is oscillating around the theoretical value, while the in-situ group speed becomes 
smaller its planar value as the complexity arises. In the total field information travels with 
$1.4~m/s$ resulting in 1.5 day later arrival of the spring-neap cycle to the far-field. This result 
contradicts the TBEAM observations of the age of the tide.\\

\footnote{This paragraph will undergo large changes}Overall, it was established that in the complex 
field interference has not affected 
actual wave group propagation, but on contrary the in-situ group speed can largely deviate from the 
plane-wave dispersion relation. But principles of wave mechanics \citep[e.g.,][]{Lighthill2001} or 
\citep[e.g.,][]{leblond1978preface} equate the in-situ group speed with the energy propagation 
velocity, hence, there is an inconsistency that deserves explanation. Contrary to the in-situ group 
speed, definition of the group speed by $c_g = \pder[\omega]{k}$ is based on dispersion in wave 
packet consisted of closely 
spaced harmonics such as the principle lunar and solar tidal species. Their respective wave fields 
can be largely distinct so that wave group propagation takes absolutely different pathway. This can 
be realized in multiple source wave field where not all internal tide production sites have 
pronounced spring-neap cycle. In the setting of the Tasman Sea $S_2$ barotropic wave has strong 
enough currents at the Macquarie Ridge to create far-flung internal waves. At other locations 
cross-isobath currents are weaker and so internal tide generation is less expected. Unfortunately 
satellite observations of the principle solar mode-1 tide \citep{zhao2017global} do not provide any 
conclusive arguments. So it is speculated that $S_2$ internal tide is not impacted by interference 
and the beat signal from the Macquarie Ridge is transmitted with the planar value even that 
baroclinic $M_2$ field has complicated spatial structure. Hence, the apparent group speed is 
reasonable to estimate arrival of the spring-neap beat if and only if the tidal constituents 
experience similar spatial modulation due to superposition \citep[e.g.,][]{holloway2003spring}.\\

The previous description of the ensemble-mean wave fields suggests possibility of interference to 
control some of temporal variability. This is investigated by calculation of standard deviations in 
the spatial averages. These results are presented on 
\fignm{C3.fig:beam_prms} by red (standard deviation) and thin gray lines (each realization). 
Clearly, previously outlined characteristics are preserved and stationary. Such that large 
differences in the energy partitioning between the experiments are collocated with position of 
nodes and antinodes. And as the flux vectors turn around these points, vector orientation can 
substantially vary. Nevertheless, the integrated flux alters less and differences are more uniform 
through the course of the beam propagation. These facts suggest that variance in magnitude of the 
interfering waves produce the shifts. To detail such processes the temporal variability is 
investigated in the previously obtained reconstructions \fignmp{C3.fig:beam_dcmp_cm}. Here the 
along-beam distribution of standard deviations and energy flux variance ellipses are documented as 
complexity arises by adding the components in counterclockwise manner. 
\fignml[C3.fig:beam_dcmp_cm]{a} depicts planar beam variation. It is mainly seen in magnitude 
of the wave field as a result of changes in conversion levels at the Macquarie Ridge. As the beam 
travels away variability also becomes apparent in the energy flux direction and hence, heading of 
the beam. This is observed as variance ellipses are obliquely oriented to the mean vectors. 
Overall, there is a high degree of spatial uniformity in the planar beam variability, so that 
standard deviations in the spatial averages are relatively constant 
\fignmlp[C3.fig:beam_dcmp_cm]{a, inset}. Nevertheless, deviation in the beam heading increases as 
the beam travels across the basin. These features allude to changes in generation or large-scale 
refraction by ocean media as main reasons and are discussed further in the text.\\
The planar beam and its simple temporal variability start to disappear as the distant waves are 
superposed. Inclusion of the internal tidal swell (Southwesterly\footnote{ or Northeasterly} 
waves, \fignml[C3.fig:beam_dcmp_cm]{b}) leads to appearance of nodes-antinodes and consequently,  
concentration of the energy density. This is translated into temporal variability. As an example, 
deviations in HKE are not monotonic anymore and experience jumps in magnitude over 180 km distance. 
The same holds for the variance of the energy fluxes. Tilt of the ellipses from the mean 
follows an alternating pattern of increase and decrease. This coincides with changes in the 
ensemble-mean flux magnitude. Additionally, the interference increases variation of the spatially 
averaged flux orientation. Further, the wave field and its changes 
become more elaborate as the reflected waves are added \fignmlp[C3.fig:beam_dcmp_cm]{c}. While 
previously the ellipses were still rectilinear and partly aligned with the planar beam heading, 
they now take oval shape and hence, represent more isotropic distribution. This is understood 
from emergence of the partial standing wave that reorients the flux vectors transversely to 
interfering components. Such alignment itself and any changes result in isotropic 
variability. And similar to the mean field, the temporal changes tend to concentrate 
following a strict spatial pattern highlighted by half mode-1 wavelength modulations in the 
spatially averaged 
characteristics. This property is preserved in the total field when Northeasterly swell from the 
South 
Tasman Rise is included. There is a slight increase in variability next to the East Tasman Plateau 
where the swell intersects with the main wave field \fignmlp[C3.fig:beam_dcmp]{d}. As well 
orientation of the flux vectors and the variance ellipses tend to slightly turn clockwise.\\

\footnote{The following part will also go through rewrite}Overall, 
there is a large spatial-temporal variability produced both by changes in the planar beam and in 
the remote waves. Nevertheless, the total field is characterized by the energy fluxes persistently 
directed in NW direction (except maybe the southern beam edge) and have magnitude representative of 
the planar beam. The spatial variations can lead to some ambiguity, but do not seem to obscure the 
gross characteristics of the tidal beam. On the other hand, in the given results it is difficult to 
identify reasons for temporal variability if point-wise measurements of the total field are taken. 
In the following discussion the found planar-beam variations are scrutinized in terms of generation 
and mesoscale effects and obtained ideas are then applied to the TBEAM mooring observations. 

%\footnote{It is worth noting that while 
%ensemble-mean allows a straightforward interpretation of mode-1 dynamics, it will incorporate a
%portion of variable (non-stationary) signal since any energy characteristic is a nonlinear 
%quantity 
%\citep{zaron2014time}. On contrary, for instance, a flux found from ensemble-mean pressure and 
%currents will exclusively provide a stationary part. Yet if a mode-decomposed signal is 
%considered, 
%finding the means will entail averaging of vertical basis functions. As a consequence, their 
%orthogonality will not be preserved leading to ambiguity in dynamics. Henceforth, an ordinary mean 
%over realizations of beam's energetics is considered. Performed comparison (not shown) between an 
%ensemble-mean of flux and a flux of ensemble-mean did not reveal significant differences in 
%spatial 
%structure, though magnitude of latter was roughly half smaller.}

\newpage
\section{Discussion}
The given results detailed variability in the internal tide production at the Macquarie ridge 
and in the deep-ocean characteristics of the gravest mode tidal beam. The latter inevitably 
affects internal tide reflection from the Tasmania continental slope. Hence, a special attention 
should be given for causes of such baroclinic tide variability. Additionally, this will provide a 
valuable setting for the satelite altimetry and the field observations. So this discussion is 
focused on the field period experiment. At first, the tidal beam properties are related to 
magnitude and heterogeneity of its generation. Then, importance of oceanographic conditions in the 
Tasman Sea is discriminated. And at final, the numerical results are compared with the observations.

\subsection{Analytical model for generation at a knife-edge ridge}
The generation is assessed through an analytical model (appendix A) which is an extension of 
previous work by \citep{rainville2010interference} but with no ad-hoc parameters necessary. The 
model provides a wave field radiated into rotating ocean of mode-1 equivalent depth by 
a piston making oscillatory motion \citep{morse1946methods}. To assess its applicability a set of 
complimentary numerical experiments in MITgcm were carried out. There a knife edge topography was 
placed against a semidiurnal barotropic tide (Table). The emitted baroclinic field was then 
diagnosed by the same methods as before. The 
comparison between the numerical experiments and analytical solution are presented on 
\fignm{C3.fig:anlt_sol}. Overall, the analytical model is well representative of spatial pattern of 
the radiated waves. Its primary feature is a wavefront curvature. The model of 
\citep{rainville2010interference} represents it by convex shape that is reminiscent of underlying 
single pole generation. It is shown here that in the near field wavefronts at first perturbed by 
diffraction and are convex, then they flatten and over only several wavelengths become akin to 
the single pole model. Similar effect was produced by calculations of \citep{zhang2011three} who 
summed wave fields emitted by point sources. Nevertheless of the near field, as extent of 
generating topography becomes indiscernible, the solutions are similar in the far field.\\
Now in presence of Coriolis force along-ridge flow as well becomes subject to no-flow boundary 
condition. This spatially redistributes the emitted energy leading to asymmetry. Such that the 
tideward and leeward wave fields are shifted one to another. The produced flow across the ridge 
will be non-uniform and in order to be satisfied the generated higher modes will inevitably also 
spatially modulated. It suggests further that breaking will also have some along-ridge 
irregularity \citep{buhler2007instability}. This has not been explored in detail as the analytical 
solution does not incorporate inter-modal coupling and left for future investigations. The other 
effect is that initially energy flux vectors have orientation that is slanted to the ridge. This 
phenomena is seen in the Tasman tidal beam that shows marked turning in the near-field 
\fignmp{C3.fig:beam_prms} and was diagnosed by the angular spectrum decomposition as Southwestern 
waves \fignmp{C3.fig:beam_dcmp}. Such wave propagation then develops phase lags and organizes into  
spirally emitting waves \citep{baines2007internal}. All of these effects highly dependable on 
number of wavelengths aligned along extent of the ridge or as a non-dimensional parameter $k L/2$. 
\footnote{As well the given analytical solution is valid near the critical latitude (e.g., $f = 
0.99 
\omega_{tide}$) where spatial structure was observed as well, but the waves are trapped near the 
ridge.}\\
The Macquarie Ridge has multiple sources distributed along the ridge which constructively 
interfere to produce the Tasman tidal beam \citep{rainville2010interference, 
klymak2016reflection}. This principle will be applied here in reverse, i.e. from the distribution 
of energy in the beam let obtain magnitude of single sources represented by the shallow-water 
pistons. They are described by their orientation that is found from localized curvature of $3000$ m 
isobath. And their size is set constant to be $30~km$. Respective strengths are found by minimizing 
mismatch in flux vectors between composed analytical solution and ROMS experiments. This is a 
nonlinear problem that is solved iteratively (computational details are given in Appendix A). 
For 'uniform' experiment the result is presented on \fignm{C3.fig:beam_inv} and can be compared 
with \fignm{C3.fig:beam}. The synthesized far-field structure agrees well with the planar field. 
And the inversely found source magnitude and location has strong correlation with the actual 
internal tide production sites. To study effect of variable generation only the major part of the 
Macquarie Ridge is examined \fignmlp[C3.fig:gen]{b}. There pistons adjusted to be representative of 
the previously considered primary production sites \fignmlp[C3.fig:gen]{c}. The averaged conversion 
rates are directly substituted as $magnitude \sim Conversion^{1/2}$ into the piston model. An 
example is given on \fignml[C3.fig:beam_inv]{b}. The beam's spatial finiteness is lost and the 
emitted field is quite similar to a point source generation, but lead to more feasible analysis of 
generation effects on the beam.\\
During the field period generation intensified in the northern part of the ridge 
\fignmlp[C3.fig:gen_var_beam]{a}. Its effect on beam position is diagnosed by identifying beam's 
''center of mass",
\begin{equation}
L_{C} = \frac{\int_0^{L} dr |F(r)|r}{\int_0^{L} dr |F(r)|}
\end{equation}
where $L_{C}$ is a position of the beam's center of mass in the cross beam direction, $L$ - length 
of the cross-beam transect, $|F(r)|$ - magnitude of energy flux along the transect. This procedure 
is carried out for the Northwest quadrant (planar beam) and the analytical solutions. The 
comparison of results confirm that the beam has moved its position by $20~km$ 
northward \fignmlp[C3.fig:gen_var_beam]{b} following generation pattern. Two outliers, 'uniform' 
and 
'2014' \fignml[C3.fig:gen]{c} are included to further emphasize a dominant role  in the near-field 
of inhomogeneous generation. Its effects can be discerned in the analytical solutions over 1-2 
wavelengths after which the model becomes similar to uniform point-wise generation. While the 
numerical experiments become irrespective to generation much faster as effects of interaction with 
mesoscale are building up.\\
Another consequence of the variable generation is alterations in the beam magnitude. Over the 
considered period mode-1 production increased by $30\%$ \fignmlp[C3.fig:gen_var_beam]{c}. Unlike to 
previous this characteristic seems to be consistent throughout the whole Tasman basin. This is 
shown by considering temporal variability in the integrated flux in the near-field, mid-field and 
far-field. Note that the far-field record was shifted in color to account for time lag necessary to 
energy propagation ($\sim 6.7~days$). All the time series show similar pattern of modulation that 
follows the total conversion. But character declines with distance, but at rates slower than 
as $1/r$ predicted by the analytical models \citep{zaron2014time}[e.g.] or given here. To account 
for the unknown spreading law, the analytical results are scaled by mean magnitude of the numerical 
experiments in the far field and given by diamonds on \fignmlp[C3.fig:gen_var_beam]{c}. The 
correlation is surprisingly good.\\
In conclusion, the given before changes in beam's heading and consequently, beam's position are 
related to generation in the near-field and it quite rapidly becomes obscured by effect of variable 
oceanographic conditions on the beam. These interactions seem to proceed without energy loss in the 
tidal beam as its magnitude correlates with conversion rates at the ridge quite good.

%strength of the beam  
%where $L_{C}$ is a position of the beam's center of mass in the cross beam direction, $L$ - length 
%of the cross-beam transect, $|F(r)|$ - magnitude of the energy flux. This procedure is carried out 
%for the Northwest quadrant (planar beam) and the analytical solutions. To emphasize the shift the 
%results were averaged over equal two periods: 10 January to 10 February and 10 February - March. 
%The results clearly show that there was a shift in the position by 5 km, this is well captured by 
%the analytical model. Hence, suggesting that the inhomogeneous generation lead to shift in the 
%beam's initial characteristics. Nevertheless, they are relatively weak as it seems, 
%though outliers deviated from the field-period mean by 20 km. From Fig it is apparent that soon 
%after leaving near-ridge region these differences become less apparent and further superposed with 
%interaction with variable oceanographic media. At second, unsurprisingly, the magnitude of the 
%beam 
%also has increased following larger conversion rates (Fig). But this signal only weakly reflected 
%in the far-field by Tasmania. These results apparently suggest that generation has effect on the 
%beam location only near ridge, but as soon as the beam leaves Fresnel zone and as effects of 
%mesoscale accumulates the beam is becoming less and less linked with generation history. But this 
%is not true for the magnitude which seems to be kept. This is probably due to interactions with 
%mesoscale without pronounced energy transfers.\\

\footnote{ Two 
	regions quite 
	clearly 
	can separated. First is near the generation site where generation has strong directional 
	characteristic. In the far-field orientation of the fluxes is more uniform, nevertheless 
	spatially 
	energy is still has some structure and not spread everywhere. These two regions are similar to 
	Fresnel and Fraunhofer zones from optics. In the former the wave field shows presence of 
	diffraction, while the latter is characterized by plane wave.}\\

%So now we are in position to investigate effects of the inhomogeneous generation on the tidal 
%beam. 
%The conversion rates from the field period experiments were set into the analytical model. The 
%obtained wave field were then subject to the same analysis as the numerical experiments. Figure 
%... 
%shows comparison of the mean energy flux direction and integrated flux in case of each 
%realization. 
%The first conclusion comes right away. The marked turning of the beam is a direct result of the 
%generation or more precisely Coriolis effect on the generation. This is the before observed skewed 
%pattern of the emitted field. Secondary, obvious result of the increased generation leading to 
%more energy carried by the beam. And at the third, the mean flux orientation can be described by 
%generation model in the first $~300~km$ of the propagation. It should be noted even that 
%the changes are small in orientation of the flux, in the far-field this produces large spatial 
%shifts in position of the beam. For example, under small angle approximation change in $5^{\circ}$ 
%by the Macquarie Ridge lead to $100~km$ shift in position of the beam's center by Tasmania. 
%Nevertheless, there are apparent differences between the analytical model and the numerical 
%results 
%in the basin. Hence, additional mechanisms such as mesoscale are in hand. This is discussed in 
%the following section.

\subsection{Variability of the Tasman Sea beam}
The Tasman beam is inevitably affected by mesoscale circulation in the region. 
\fignml[C3.fig:meso_exampe]{a} gives an example of oceanograhpic conditions. The 
deep ocean of the Tasman Sea is dissected by southern subtropical frontal zone. Its meandering 
produce vortices. These features are relatively weak and do not have significant kinetic energy in 
comparison to Tasmania with strong currents and eddies associated with EAC 
\citep{zhao2018satellite}. Nevertheless, the basin vortices have Rossby number of $Ro \sim O(0.1)$ 
and scales close to baroclinic Rossby number, i.e. $\sim 40~km$. Scaling performed by 
\citep{dunphy2017low} showed that in such regime tide-mesoscale interaction terms will scale as Ro 
and major phenomena will restrain to refraction with negligibly small energy scattering and 
transfers. This is further diagnosed by application of the dispersion relation in presence of 
background circulation \citep{zaron2014time},
\begin{align}
\omega_{obs} = f_e^2 + \frac{\omega_{obs}}{\omega} gD |k|^2\\
c_{obs}^2 = \big( \frac{\omega_{obs}}{|k|} \big)^2 = \frac{\omega_{obs}}{\omega} \frac{gD}{1 - 
\frac{f_e^2}{\omega^2}}
\end{align}
where $\omega_{obs} = \omega + \vec{k}\cdot \vec{U}$ - observed, Eulerian frequency which is 
Doppler shifted intrinsic frequency, $f_e = f(1 + \frac{1}{2} \frac{\zeta}{f})$ - effective 
Coriolis frequency adjusted by rotation in vortical flow of $\zeta$, $c_0^2 = gD$ - phase speed of 
internal tide in non-rotating ocean with effective depth of $D$. The latter term will encompass 
variations in stratification. The dispersion relation was then used to occur respective phase 
speeds. To do so, three dimensional currents from the numerical experiments were temporally and 
vertically averaged. And since Doppler shift depends on alignment between wave propagation and 
currents, a bulk beam orientation of $150^\circ$ was used in calculations.\\
Results \fignmlp[C3.fig:meso_exampe]{b} show that major regions of variations are confined to the 
meanders. The deviations due to stratification on average describe 50\% of anomaly, due to 
background currents 30\% and due to vorticity - 20\%. The phase speed diverts by $0.2~m/s$ or 
$\sim5\%$ from the mean. Even though these changes appear to be small they can substantially divert 
beam's propagation. This can be estimated using shallow-water gravity theory of refraction 
(citation) as
\begin{align}
\frac{\Delta \theta}{\Delta s} = \frac{1}{c} \frac{\delta c}{\Delta n}
\end{align}
where $\Delta s$ and $\Delta n$ are distances in along-beam and cross-beam directions. Setting them 
equal and $\delta c/c = 10\%$ (gradient between positive and negative anomalies) produce 
reorientation of the beam by $\Delta \theta 5^{\circ}$.\footnote{ These calculations were supported 
by ray 
tracing that is not given.} This corresponds to the results presented previously (Section). Now as 
waves refracted, they interfere with non-interacted portion of the beam leading to 
obscure patterns as illustrated on \fignml[C3.fig:meso_examp]{b}. The beam seems to be separated in 
two parts with a portion actually leaving the Tidal beam strongly northward.\\
The phase speed deviations drive variations in the beam propagation. These were studied by variance 
ellipses \fignmlp[C3.fig:beam_dcmp_cm]{a}. Since the corresponding semi-minor axis are small, it is 
worth investigating only semi-major axis. Such characteristic is mainly associated with directional 
variance. The first region is located by the Macquarie Ridge in accordance with the previous 
section on generation. Further regions of high directional variability appear to come in slanted 
manner representative of self-interference. It originates after internal tide passage over regions 
of high phase speed variance or active mesoscale circulation \fignmlp[C3.fig:beam_dcmp_cm]{b}. 
Clearly that direction variability leads to shift in position of the beam. And as the beam arrives 
to Tasmania this could lead different location of the beam. During the field period simulation the 
beam was constantly southward within $30~km$ from 'uniform'-experiment. This was mainly due to 
located near ETP warm-core eddy \fignmlp[C3.fig:beam_dcmp_cm]{b}. Over all experiments the 
deviations could be much larger that position shifted in meridional direction by $\pm 30~km$.\\
The numerical experiments clearly presented large scale variability in the beam incidence on Tasman 
continental slope as a result of interaction with mesoscale circulation. It seems that changes can 
appear on different time scales. Analysis has not shown any seasonal trends, but rather turbulent 
behavior resulted as refractive effects accumulate as the beam transverse the Tasman basin. These 
interactions seem not to redistribute energy and only divert it since mesoscale currents are 
relatively weak and driven mainly by small frontal zone instabilities. This is probably different 
near Tasmania where strong East Australian Current can shed eddies of high vertical shear and 
significant spatial extent.

\subsection{Comparison with altimetry observations and TBEAM observations}
Results of the presented investigation are compared here with satelite altimetry observations 
\citep{zhao2018satellite} and TBEAM observations \citep{waterhouse2018observations}. The aim of 
such comparison is to provide insight into actual beam parameters and variability.\\
Overall comparison between 20-year average of satelite observations \citep{zhao2018satellite} 
favorably compares with the obtained results here. The comparison of plan-view of the internal 
tidal beam is provided by \fignm{C3.fig:cmp_sat_nexp}. Over the first half of the Tasman Basin both 
beams are similar in position, travel direction and magnitude. By the inverse approach energy 
sources were deduced and then compared to averaged distribution. The same part of the ridge was 
identified as emittor of the Tasman sea beam. Though it appears that role of the seamounts 
generation is greater compare to the numerical experiments. This could be due to smoothing of 
bathymetry in ROMS. It produced 200-300 m deeper seamounts compare to actual bathymetry. So it 
could be true that ROMS simulations underestimated generation at the seamounts. On the other hand, 
the mechanism described in Section 2.1 seems to presented as Campbell Plateau emitted waves are 
also found in satelite altrimetry observations.\\
The major differences are found near Tasmania and as previously discussed interaction of the 
internal tidal beam with variable ocean conditions lead to refraction and shift in position. And 
due to a small number of realizations there might be a bias in averaging. These suggestions were 
tested by calculating kinetic energy of background currents in the Tasman Basin. This is thought as 
a measure of eddy presence and their strength, so that amount of refraction. Along with used here 
experiments, the same metric was found for HYCOM. Both are presented on \fignmp{C3.fig:KE_beam}. 
HYCOM results show a pronounced increase in Kinetic energy happened in February 2013. Since ROMS 
was initialized from HYCOM, this change was well captured. Actual reason for this increase is not 
known. But it is noted that there was a change in position of Southern Subtropic Frontal 
zone defined by surface salinity of 34.4. Over the jump the surface isohaline shifted 
0.5-1$^\circ$ northward. This does occur in the region, but it is unknown if such changes lead to 
more unstable frontal zone and hence, stronger mesoscale circulation.\\
\fignm{C3.fig:KE_beam} also gives how the beam was shifting its position relative to satelite 
altimetry. The first period with weaker currents is found to be closer to the altimetry result. 
While later where eddies are more persistent in the region, there is stronger refraction southward, 
away from the altimetry. It follows that in the considered numerical experiments there is bias 
towards souhtward due to inadequate series coming from long-period variations in oceanographic 
conditions.\\
This concludes to that souhtward position was prevailing during TBEAM field period. This is result 
now is compared to factual observation made during TBEAM observations. These were made at several 
LADCP stations and A1 mooring. The LADCP observations have large uncertainty in currents due 
to impossibility of separating semidiurnal baroclinic tides from near-inertial internal waves 
\citep{waterhouse2018observations}. These data now are compared with the numerical results 
\fignm{C3.fig:TBEAM_sp}. Since the in-situ observations the superposed field, it is discussed at 
first. The simulated total field compares relatively poor at all locations  
\fignmlp[C3.fig:TBEAM_sp]{Panel a, pink and red arrows}. This is mainly emphasized by higher 
southward orientation at all stations. But after removing all superposing waves and leaving only 
the planar beam component, i.e. waves traveling in NW quadrant, the comparison is more 
favorable along the first line of observations. In actuality, at location of A1 the simulated beam 
coincides with the mean energy flux at A1. And further, \citep{waterhouse2018observations} have 
used a plane-wave fit producing estimate of $150^{\circ}$ and $3.4~kW/m$. Here result is of 
$152^{\circ}$ and $3~kW/m$. This leads to conclusion that the simulated total field has different 
characteristics of interference than it was observed. This point is emphasized in comparison with 
experiment which is the closest to the satelite altimetry observations. The beam here is oriented 
by $140^{\circ}$. Here the total field (pink arrows) corresponds to the observations much better. 
Nevertheless, at stations F6-F9 the result is still bad. This is not surprising because of 
superposition with the waves emerging from South Tasman Rise. If they are removed, the comparison 
is better (not shown). Hence, in the real observations are less influence by internal tidal swell. 
Most probably because of its much faster decay than in the numerical experiments. Probable reason 
is interaction with mesoscale which is especially true to South Tasman Rise waves that are 
generated in region with vigorous currents of ACC.\\
As it was said before spatial variation of the signal in this region is set by interference with 
the waves reflected from Tasmania and ETP. This part of the wavefield is given by green arrows on 
\fignm{C3.fig:TBEAM_sp}. Comparison between the field period average and '2013' experiments 
demonstrates how reflected wave can be variable in magnitude and position. One of consequences of 
this variability is shift in position and strength of nodes and anti-nodes. Considering mean 
distribution of HKE and APE \fignmlp[C3.fig:TBEAM_sp]{Panel c} it is clearly seen that the first 
line of observations in the located in the region between node and anti-node as seen by 
interchanging distribution. While the second line is in much more complex region where no clear 
pattern is seen. It was noted before that this region is a result of multiple wave interference. 
This suggests that such high correspondence of the field observations to non obstructed beam is 
result of good sampling location. At the same time, this can provide additional information on 
reasons for variability in the time-series.\\
At first the observed ratio between HKE and APE at A1 location is given on 
\fignml[C3.fig:TBEAM_KE2PE]{Panel a}. Even though range of variability of simulated is much less 
than in the observations, it additionally emphasized how this parameter in the wavefield can vary 
due to interference. Additionally, spatial distibution shows this. It is found that variability in 
the reflected signal modulaties position of the nodes and anti-nodes which is then translated into 
the ratio. A1 location in the numerical experiments is highly controlled by position of the near by 
anti-node. As it comes closer to the location the ratio is becoming close to 0. While when the 
whole pattern shifts southward, the ratio is increasing towards a theoretical value. Such close 
proximity to a region of high APE leads to more stationary Potential energy, i.e. variation in 
potential energy is less than one in kinetic energy percentage-wise. This is just because any 
change in relatively low kinetic energy will produce be more noticeable. Even though in the 
numerical experiments such variability is produced by variation in the reflected signal orientation 
and magnitude, in the simplest case of partial standing wave energy variability in anti-node due to 
changes in magnitude ratio can be found to be,
\begin{align}
	E_{KE} = (1 - r)^2 + 4r \sin^2{k x},~E_{PE} = (1 - r)^2 + 4r \cos^2{k x}\\
	\frac{\delta E_{KE}}{E_{KE}}\big|_{kx = 0} = \delta r \frac{2}{1 - r},~\frac{\delta 
	E_{PE}}{E_{PE}}\big|_{kx = 0} = \delta r \frac{2}{1 + r}
\end{align}
where $\delta r$ - a small perturbation in the amplitude ratio of two interfering waves. Now at 
anti-node position ($kx = 0$) any changes in ratio between two waves will be amplified in 
kinetic energy. The amplification factor is increasing as $r \rightarrow 1$.\\
During TBEAM observation Potential energy was more stable than kinetic energy. So that previous 
speculation can be applied. And this further emphasize that most probably A1 mooring was indeed 
located close to anti-node and variability in the energy flux was caused as the anti-node position 
was moving in response to changes in the incidence of the beam or reflected signal.\\
These inferences can be further tested if the signal at A1 could be decomposed. Since it is a 
point-wise observation the used here method could not be directly applied. But if it is assumed 
that interfering waves do not have any phase lags, so that cross-terms in energy expressions are 
zero, then non-deterministic decomposition is viable. Here non-deterministic states that the 
actual signals cannot be back-synthesized and only some characteristics could be deduced. Such 
assumption is fulfilled in position between node and anti-node where two interfering waves are in 
quadrature.\\
The details of such decomposition are given in Appendix A.2 and resultant example is shown on 
\fignm{C3.fig:ex_spectra}. Clearly, since the fit is performed for 5 Fourier components, there is 
large aliasing. In actuality, waves directionality separated by $180^{\circ}$ can be 
trustworthy separated. Under other conditions this might lead to some wrongly identified direction 
such is shown by comparison between full spatial analysis and pointwise for the numerical 
experiment. Here direction information was wrongly identified. To further asses applicability an 
investigation was carried when in a half wavelength window surrounding A1 location numerical grid 
points were separated by ratio in HKE and APE. Then the point-wise decomposition was applied and 
results were compared to the spatial analysis \fignmp{C3.fig:test_spectra}. It is clear that 
depending on position importance of phase lags and hence, cross-terms either increases or 
decreases. Surprisingly, the best result is found for locations close to node when the ratio is 
less than 1. Here energy of the incident wave was identified correctly. The reflected wave on the 
other hand is underestimated. The wave bearing of incident signal as well was found to be 
satisfactory.\\
The result of decomposition are presented by \fignm{fig_22_A1_comp.png}. For comparison it is also 
given time variation of the simulated wave. The amount of incident energy in the mooring quite 
closely follows total. This both true for the numerical simulations and TBEAM observations. In the 
numerical simulations as it was pointed out before this happens because of variation in generation 
level. Surprising to see that in the field observations there is some degree of resemblance 
\fignmlp[C3.fig:dcmp_A1]{Panels a, c}. So 
that variable levels of generation can be considered as a reason for change in magnitude of flux 
observed in TBEAM. Secondary, the bearing of the incident wave also varied. This probably was 
produced as a result of refraction by an anticyclonic eddy found near ETP region. The ROMS 
model had a moderate skill in simulating eddy dynamics as it was located a half degree 
south $\sim 70~km$ in comparison either to AVISO or HYCOM. So that eddy's center was located over 
$A1$ and $F5$, while from the field observations the eddy was situated over $F3$. Nevertheless, the 
timing was simulated correctly as most active interaction took place in February 2015. This 
produced in the numerical experiment a southward shift of the beam and consequently, the energy 
flux was turning southward. Similar pattern of change is found in the numerically decomposed signal 
(dots), total flux in TBEAM (gray line) and point-wise decomposition (black line) 
\fignmlp[C3.fig:dcmp_A1]{Panel d}. Yet if the total wavefield is considered 
\fignmlp[C3.fig:dcmp_A1]{Panel b}, then resemblance is weak. It means that reflected signal in 
reality is weaker and modulations due to interference are less pronounced. This can be envisaged 
from \fignml[C3.fig:dcmp_A1]{Panel e} presenting reflected wave. The pointwise decomposition 
clearly overestimates the magnitude as it was found before. Ad-hoc coefficient is about 2-3 
times. Then the reflected wave was at maximum $0.5~kW/m$ during TBEAM observations. This is 
much smaller in comparison to the numerical simulations. The reason for this mismatch comes from 
scrutiny of the reflected wave. It consists from 3 contributions: reflected wave from Tasmania, 
reflected waves from ETP and Cascade Seamount and local generation at the seamount. Here the latter 
constitutes the major part. Dynamics of this process is similar to discussed before the 
Macquarie Ridge, phasing of the incident beam relative to local barotropic tide will affect 
either production of the internal tide or destruction. This is highly dynamical region both in 
terms of internal tide dynamics as many components of the wavefield interact with the seamount as 
well as local conditions of the mesoscale. And it will not be surprising if the generation was not 
predicted correctly. Nevertheless, the pointwise decomposition seems to capture pattern of change 
quite correctly. More precisely, during the first period of TBEAM observations (January 2015) there 
was a considerable fraction of the reflected wave present. This is both seen the total flux bearing 
leaning towards West and ratio KE/PE being below 1. This seems to be also true for the burst of 
energy during 30-35 day of year as the ratio dipped as well as magnitude of the total flux. In the 
later part of the observational period reflected wave is only miniscular part of the signal and 
hence, KE/PE tends towards the plane wave value.\\
This discussion draws conclusion that information provided by A1 mooring and LADCP stations 
were well descriptive of the magnitude and beam position. Nevertheless, LADCP stations seem to 
largely overestimate fluxes because of aliasing with near-inertial internal waves. The time 
variability observed at A1 appears to be a result of at least 3 processes including long-period 
adjustments in baroclinic tide production at the Macquarie Ridge, refraction by mesoscale eddies 
and reflection at Tasmania and Cascade Seamount. All of these stipulate high nonstationarity in the 
observations, so that flux magnitudes and direction can deviate on time scales compared to 
semidiurnal period. Energy transfer between mesoscale and tidal field was not discussed here, but 
seems to be of negligible importance at least during the considered here period. For the purposes 
of providing boundary conditions for the reflection processes in Tasmania, the presented here 
nonstationarity, variability will be inevitably further exaggerated, so that reflection process 
should be considered with great detail.

\section{Conclusions}
Overall, combined discussion of the numerical experiments and field observations suggests that the 
far field signal is affected both by generation levels at the Macquarie Ridge and mesoscale 
conditions. The former process mainly factors magnitude of the incident beam, while mesoscale can 
largely refract beam. The numerical experiments given here were capable to predict gross, 
time-averaged characteristics of the beam and degree of temporal variability. 

Nevertheless,  This interaction was mainly taken place in February of 
2015 which corresponds to reorientation 
Even though the given here result is 
probably not correct, but from comparison of the simulated wave field 

Another reason for variation is change in the reflected wave. So that increase 
leads to additional 
modulation. Even though results are not that trustful in the reflected part, they simple 
retranslate change in the ratio. So that the first period before day 40 when strong reflected 
signal was leading to developing strong interference pattern and large deviation from the plane 
wave value. At the end of that period, decrease in reflected wave was producing more plane wave 
value. Such variations could be related to changes in beam incidence and reflection at East Tasman 
Plateau and Tasmania. There is a similar pattern that follows from changes. So more when incidence 
happens strongly in westward direction, amount of reflected wave is less. And for northward there 
is more relfective component. The total energy flux orientation in the numerical model does not 
stand close to the observed.

%provide ideas on the Tasman tidal beam behavior. The former dataset comprises highly averaged both 
%in time and space baroclinic tide dynamics. \fignm

WE intend to compare with two datasets available. Satelite altimetry emphasizes highly temporal and 
spatially smoothed observations. But field observations highly hetergoneous point-wise 
observations.\\
At first, satelite altimetry shows more northward located beam compared to the numerical 
experiments. 
%Further, the phase speed deviations were superposed with the position of the beam (Fig). 
%The previous result on \fignm{C3.fig:beam_dcmp_cm} can now be discussed further. Now it is shown 
%distribution of semi major axis of variance ellipse \fignm{C3.fig:fig_12_beam_var_cp}. Here it is 
%clearly seen that primary variations are concentrated in regions downstream that are affected by 
%mesoscale. Additional region is next to the Macquarie Ridge due to generation. But further it is 
%impossible to propagation quite rapidly becomes affected by variable ocean conditions. There are 
%three regions one next to the ridge, than in the basin. And also by the ETP there is strong 
%interaction causing beam to deflect.\\
%
%
%
%This rein
%In the current 
%These processes are the most important. And 
%refraction seems to be the major player. Refraction is a well-known phenomena in surface gravity 
%waves. In respective theoretical works the directional spectrum plays a pivotal role. Here these 
%approaches are used to diagnose the spectrum. This is carried out by computing two spectral 
%moments,
%\begin{align}
%\label{C3:eq.m.spectra}
%\bar{\theta} = \arctan \frac{\int_{\pi/2}^{\pi} d \theta S \sin \theta}{\int_{\pi/2}^{\pi} 
%d \theta S \cos \theta }\\
%\sigma_\theta = 2\bigg[ 1 - \sqrt{\Big( \frac{\int_{\pi/2}^{\pi} d \theta S \sin 
%\theta}{\int_{\pi/2}^{\pi} d 
%\theta S} \Big)^2 + \Big( \frac{\int_{\pi/2}^{\pi} d \theta S \cos \theta 
%}{\int_{\pi/2}^{\pi} d \theta S}\Big)^2}  \bigg]
%\end{align}
%where $S=S(\theta)$ - directional spectrum at each grid point obtained previously. These moments 
%simply show mean direction in the spectrum and the spreading of waves around the mean. Clearly, if 
%there is any refraction than the mean direction will undergo changes. But the spectrum spreading 
%will also change as close-to-each wavenumbers will experience slightly different adjustments 
%leading either to more spreading as wave group propagates travels into a region with 
%higher phase speeds than surrounding or focusing in the opposite case.\\
%
%Over the field period simulation there was a continuous change in the beam as mesoscale field was 
%evolving. Over the first period, Januaries 10-15, 2015 (Fig), the beam underwent the most striking 
%adjustment. Clearly, some of the waves were refracted equatorward, this portion then interfered 
%with the rest producing a horizontal reorientation of the fluxes. Similar phenomena was observed 
%in 
%numerical experiments of \citep{dunphy2014focusing}. The region is characterized by northward 
%orientation of spectra and high spreading. The waves cross a complex mesoscale field consisting of 
%two closely located eddies. They have signature in the increased eigenspeed and strong currents. 
%Vorticity effects were smaller. Overall, the eigenspeed was locally increased by $~10\%$. This 
%diverted further north already more northward oriented beam.\\
%The further development are shown on Fig. where the fields are given for mid-February, 10th - 
%15th. 
%The previous discussed eddies have merged producing a feature at (155, -46.5). 
%This time it has no effect on the tidal beam. On part because of weak signature in the eigenspeed, 
%but also since the strongest currents are southward and hence, leading to advection in poleward 
%which is seen in the spectrum orientation. Nevertheless, there are two regions that cause some 
%effects. One eddy is located at (158, 48) where again northward refraction happens. Since this a 
%dipole of two oppositely oriented rotating eddies, there are strong gradients in eigenspeed. This 
%produces high spreading in the spectrum. Fluxes again show a localized horizontal reorientataion. 
%Nevertheless, due to small lengthscales this does not have a substantial effect on the beam. 
%Oppositely to the region by ETP (152, -44) here a region with large cross-beam currents, strong 
%imprint in eigenspeed. This leading to diverting portion of the beam northward.\\
%Thence, two major effects were found one when currents are in cross-beam leading to diverting the 
%beam, scattering by mesoscale eddies and refraction. 
%half-month in  
%later
%But later developments 
%lead to further 
%Over the next we 
%Combined two effects lead 
%Here it is presented two cases that shows how complex such interactions can be (Fig.). In the 
%first 
%case due to a region of abnormal 
%
%The spatial-temporal characteristics complicates direct interpretation of diagnosed energy 
%quantities in describing the tidal beam. Its properties are obscured due to 
%interference and hence, needs additional inferences. This is related to variability. It is unclear 
%what can produce beam's energy levels and its orientation/position. Two reasons could be named as 
%accumulating interaction with mesoscale field and generation producing.\\
%Now comparison of the decomposed planar beam can be made 
%with factual observations made by \citep{waterhouse2018observations, zhao2018satellite}. \\

\begin{table}
	\caption{Internal tidal energy flux properties in the far field}
	\begin{tabular}{ |p{7cm}||p{4cm}|p{4cm}|  }
		\hline
		Source & Flux magnitude [$kW/m$] & Heading [$^\circ$] \\
		\hline
		This study - all experiments & ~ & No mesoscale \\
		This study - only field period & ~ & No mesoscale \\
		Altimetry observations \citep{zhao2018satellite} & $3.9 \pm 2.2$ & $141 \pm 2$ \\
		Field observations \citep{waterhouse2018observations} &   $3.4 \pm 1.4$ & $149 \pm 3$ \\
		\hline
	\end{tabular}
	\label{ch2:table_exp}
\end{table}
\section{Conclusions}

%\newpage
%\iftoggle{only_Chapter} {
%	\appendix
%}
%
%\nottoggle{only_Chapter} {
%	\addcontentsline{toc}{section}{Appendices}
%}
%
%\section*{Appendices}

%\renewcommand{\thesubsection}{\Alph{subsection}}
%\setcounter{subsection}{0}
%\subsection{Analytical model for knife edge}
%Let consider a simplified problem of the internal tide generation at a three dimensional ridge. 
%That is generating topography has extent in along x-axis, $a$, but infinitely small width. We will 
%not pursue full solution of the problem and will not seek actual amplitudes of baroclinic modes, 
%but rather concentrate on defining spatial pattern of the generated waves (might move up to 
%opening). Under such problem statement the actual height is of no importance. And hence, in frame 
%reference traveling with barotropic current, the topography becomes a piston-alike wavemaker that 
%sends out the lowest mode internal tide. Its behavior in flat bottom ocean is well represented by 
%Laplace tidal equations (cf \cite{kelly2012cascade}):
%\begin{align}
%\vec{u}_t + 2 \Omega \vec{k} \times \vec{u} = - \frac{1}{\rho_0} \nabla \cdot p\\
%\nabla \cdot \vec{u} = -(\frac{1}{N^2 - \omega^2}p_z)_{tz}
%\end{align}
%with boundary conditions on the ridge,
%\begin{equation}
%\vec{u}\cdot \vec{n}|_{ridge} = \vec{u}_{bt} \cdot \vec{n} |_{ridge}
%\end{equation}
%And boundary condition in the infinity implying outgoing waves. Note that barotropic current, in 
%general, is given as a current ellipse written in complex form as $\vec{u}_{bt} = u_{bt} + i 
%v_{bt}$. This means that the boundary condition does not have a simple harmonic form. But the 
%barotropic current can be decomposed to clockwise (CW) and counterclockwise components,
%\begin{equation}
%\vec{u}_{bt} = W_{CW} e^{-(i \omega t - \phi_{CW})} + W_{CCW} e^{(i \omega t + \phi_{CCW})}
%\end{equation}
%and the generation of internal tide can be solved separately for two oppositely rotating currents. 
%In further discussion the only one component (CCW) is taken care of since a solution will have 
%similar form with differences arising in sign in front of tidal frequency.\\
%Considering rigid lid approximation and impermeable bottom equations for eigen modes is solved 
%such that wavelength of corresponding mode is introduced. Than the dynamical equations can be 
%reformulated as Helmholtz equation with harmonic temporal dependence implied and appropriate 
%polarization relations,
%\begin{align}
%\nabla^2 p + k_n^2 p = 0\\
%u = \frac{-i \omega p_x + f p_y}{\omega^2 - f^2},~v = \frac{-i \omega p_y - f p_x}{\omega^2 - f^2}
%\end{align}
%Since the problem is now formulated in terms of pressure only, boundary condition takes the 
%following form (\cite{greenspan1968theory}),
%\begin{equation}
%\Big[ \frac{-i \omega p_y - f p_x}{\omega^2 - f^2} \Big]_{ridge} = \vec{u}_{BT}|_{ridge}
%\end{equation}
%The above equations (5-8) state generation of internal tides by vibrating strip of width $a$. Such 
%problem was solved for radiation of acoustic waves by \cite{morse1946methods}. Since in Cartesian 
%coordinates the strip boundary condition does not allow separation of variables, one can employ 
%elliptic coordinate system,
%\begin{equation*}
%x = \frac{a}{2} \cosh \mu \cos \theta,~y = \frac{a}{2} \sinh \mu \sin \theta
%\end{equation*}
%where edges of the strip will be focii of ellipse, boundary condition will take simpler form,
%\begin{align}
%\Big[ \frac{-i \omega p_{\mu} + f p_{\theta}}{\omega^2 - f^2} \Big]_{\mu = 0} = \frac{a}{2} \sin 
%\theta (\vec{u}_{BT}|_{\mu = 0})
%\end{align}
%The factor $\frac{a}{2} \sin \theta$ arises via conversion from Cartesian to elliptical 
%derivatives. Note change of sign for $p_{theta}$ because opposite growth of $\theta$ argument to 
%$x$ argument. Laplace operator in the elliptical coordinates has eigensolutions (''sloshing 
%modes") 
%in form of Mathieu functions (\cite{stratton2007electromagnetic}) that solution for the emitted 
%waves will have form of,
%\begin{equation}
%p \sim \sum_{j} [Se_j, So_j](h, \theta) [Je_j, Jo_j, Ye_j, Yo_j](h, \mu)
%\end{equation}
%with $Se_j,~So_j$ - angular Mathieu functions of order $j$ corresponding to $\cos$ and $\sin$, 
%$Je_j,~Jo_j$ and $Ye_j,~Yo_j$ - radial Mathieu functions of the first and second kind of order $j$ 
%corresponding to Bessel functions. The physical parameter that sets Mathieu functions behavior is 
%$h = {a k_n}{4} $ so that in large distance limit, $h \mu \gg 1,~rk_n \gg a$, they will represent 
%circular harmonics.\\
%Now RHS of boundary condition (9) can be expressed in series of Mathieu functions,
%\begin{equation}
%\sin \theta = \sum_{j = 0}^{\infty} C_{2j + 1} So_{2j + 1} (\theta)
%\end{equation}
%with coefficients $C_{2j + 1}$ defined by normalization constants $N_{2j + 1} = \int_{0}^{2\pi} 
%So_{2j + 1}^2(\theta) d \theta$ and the first Fourier coefficient of $So_{2j + 1}$. The above 
%series for $\sin$ and form of the boundary condition suggest solution in the form,
%\begin{equation}
%p(\mu, \theta) = \sum_{j = 0}^{\infty} \big( A_{2j+1} So_{2j + 1}(\theta) + B_{2j+1} Se_{2j + 
%1}(\theta) \big) Ho^{1}_{2j + 1}(\mu)
%\end{equation}
%Here $Ho^{1}_{2j + 1}(\mu) = Jo_{2j + 1} + i Yo_{2j + 1}$ is a Hankel-Mathieu function. For CCW 
%component to ensure condition for outgoing radiation, the sign should be different, so $Ho^{2}_{2j 
%+ 1}(\mu) = Jo_{2j + 1} - i Yo_{2j + 1}$.\\
%Now substituting (11, 12) into (9) the unknown coefficients $A,~B$ are obtained,
%\begin{align*}
%(-i\omega (\sum A So + B Se) Ho^{\prime} + f (\sum A So^{\prime} + B Se^{\prime}) Ho) = \sum 
%(\omega^2 - f^2) C So
%\end{align*}
%At first, multiplying above equation by $So_{2m + 1}$ and taking integral from 0 to $2 \pi$, so 
%that $\int_{0}^{2\pi} So_{2m + 1} Se_{2j + 1} d \theta = \int_0^{2 \pi} So_{2j + 1}^{\prime} 
%So_{2m 
%+ 1} d \theta = 0,~\int_0^{2 \pi} So_{2m + 1} So_{2m+1} d \theta = No_{2m + 1},~\int_0^{2 \pi} 
%So_{2m + 1} Se_{2j + 1}^{\prime} d \theta = -{N^{\prime}}^{2m+1}_{2j + 1}$. In the last statement 
%orthogonality between $So$ and $Se^{\prime}$ is not satisfied. Equation on each order is obtained
%\begin{align}
%(-i \omega A_{2m + 1} No_{2m + 1}^o Ho_{2m + 1}^{\prime} + f \sum_{j} B_{2j + 1} 
%{Neo^{\prime}}^{2m + 1}_{2j + 1} Ho_{2j + 1}) = C No_{2m + 1}
%\end{align}
%And at second, carrying out the same procedure but with $Se_{2j + 1}$,
%\begin{align}
%(-i \omega B_{2m + 1} Ne_{2m + 1} Ho_{2m + 1}^{\prime} + f \sum_j A_{2j + 1} {Noe^{\prime}}^{2m + 
%1}_{2j + 1} Ho_{2j + 1}) = 0
%\end{align}
%Note that $\int_0^{2\pi} Se^{\prime}_{2j + 1} So_{2m + 1} d \theta = -\int_0^{2\pi} 
%So^{\prime}_{2j + 1} Se_{2m + 1} d \theta$, i.e. ${Noe^{\prime}}^{2m + 1}_{2j + 1} = 
%-({Neo^{\prime}}^{2m + 1}_{2j + 1})^T$.\\
%Equations (13) and (14) form a linear system to find coefficients for different component of the 
%total field. These equations are solved numerically with $j_{max} = 5$ due to rapid convergence of 
%the involved series.
%
%\subsection{Inverse model}
%The model closely follows ideas used in ref-to-Luc, 2010 and -Jody-2016. The internal tide 
%generating ridge is given by point sources each emitting following
%\begin{equation} \label{invm_eq:1}
%p = p_{0} \frac{2}{\pi k d} \cdot e^{i  k  d}
%\end{equation}
%where $k$ - wavenumber associated with eigen mode-1, i.e. $k = \sqrt{\omega ^ 2 - f ^ 2}{c_{eigen} 
%^ 2}$, $d$ - distance between a point source and an observation point. By observation points here 
%and after is meant points in which observations are inverted. The given solution is a solution of 
%pressure distrubance propagation for two dimensional wave equation (p. 22, Frisk) and describes 
%outgoing cylindrical wave. This is a far field approximation ($kd \ll 1$), in the near source zone 
%the solution is substituded by Hankel functions. Here representation is simplified and observation 
%points on the distance less than wavelength are omitted. Though introduction of Hankel function 
%into the inverse model does not involve any additional complexity. By pressure here is thought 
%mode-1 pressure amplitude that can be connected to sea level disturbance or isopycnal 
%displacements.\\
%To describe energy fluxes in the observational points polarization relations for cylindrical 
%Poincare wave are invoked,
%\begin{align} \label{invm_eq:2}
%u = \frac{p_{0}}{\rho_{const}} * \frac{-i \omega \cos(\theta) + f \sin(\theta)}{\omega ^ 2 - f ^ 
%2} \cdot p_{\vec{d}}\\
%v = \frac{p_{0}}{\rho_{const}} * \frac{-i \omega \sin(\theta) - f \cos(\theta)}{\omega ^ 2 - f ^ 
%2} \cdot p_{\vec{d}}
%\end{align}
%where $p_{\vec{d}}$ is a derivative along radius-vector $\vec{d}$,
%\begin{equation}
%p_{\vec{d}} = (i \cdot k - \frac{1}{2 d}) p
%\end{equation}
%In further description of the inverse model it is used following notation, indices $i,~k$ define 
%$i,~k$-th point sources, while $j$ - $j$-th observation point.\\
%The tidally and depth averaged energy fluxes will be given as an interference of pointwise fields 
%from all sources,
%\begin{align}
%F_{j}^x = \frac{1}{2} \sum_k u_{kj}^{\star} \sum_i p_{ij} \int_H^0 \psi_1(z)^2 dz\\
%F_{j}^y = \frac{1}{2} \sum_k v_{kj}^{\star} \sum_i p_{ij} \int_H^0 \psi_1(z)^2 dz
%\end{align}
%Note different indexes for u/v and p meaning that cross multiplication is involved which leads to 
%complex interference pattern. In energy flux formulation normalization coefficient associated with 
%eigenmode structure function are introduced by corresponding mode-1 structure function, 
%$\psi_1(z)$. Coefficient $1/2$ is used for convenience to convert actual time averaging involved 
%to 
%multiplication of complex numbers. In further description the constant coefficients are omitted 
%due 
%to their irreleveance. The previous relations can be expressed in matrix form (it is not fully 
%correct for fluxes, multiplication is done term by term per point),
%\begin{align}
%p_j = B^p_{ji}{p_i},~u_j = B^u_{ji}{p_i},~v_j = B^v_{ji}{p_i}\\
%F^x_j = (B^u_{jk}{p_k})^{\star} B^p_{ji}{p_i},~F^y_j = (B^v_{jk}{p_k})^{\star} B^p_{ji}{p_i} 
%\label{invm_eq:4}
%\end{align}
%where tensor notation is used, i.e. summation is done over same indices. Matrices 
%$B^p_{ij},~B^u_{ij},~B^v_{ij}$ are short notation for generaion model and polarization relations, 
%for example,
%\begin{equation}
%B^p_{ji} = p_i \frac{2}{\pi k d_j} \cdot e^{i  k  d_j}
%\end{equation}
%These can be thought as disretization of operators transforming distribution of sources into 
%interference pattern in pressure and velocity fields.\\
%Apparently, the energy flux relations are non-linear. To deal with this it is proposed an 
%iterative technique. Let at $m$-th iteration there is a known distribution of wave amplitude at 
%sources, $p^m_i$, the total energy flux field can be reconstruced by (\ref{invm_eq:4}). Than it is 
%desired to find a small adjustment $\delta p^m_i$ (``nudge factor") such that residual between 
%observed field and analytical description will be decreased. One can write,
%\begin{align}
%F_{j}^x = (B^u_{jk}(p^m_k + \delta p^m_k))^{\star} B^p_{ji}(p^m_i + \delta p^m_i) = \nonumber\\
%(B^u_{jk} p^m_k)^{\star} B^p_{ji}{p^m_i} + (B^u_{jk} \delta p^m_k)^{\star} B^p_{ji} p^m_i + 
%(B^u_{jk} p^m_k)^{\star} B^p_{ji} \delta p^m_i + (B^u_{jk} \delta p^m_k)^{\star} B^p_{ji} \delta 
%p^m_i\nonumber\\
%F_{j}^x - (B^u_{jk} p^m_k)^{\star} B^p_{ji}{p^m_i} = (B^u_{jk} \delta p^m_k)^{\star} B^p_{ji} 
%p^m_i + (B^u_{jk} p^m_k)^{\star} B^p_{ji} \delta p^m_i + (B^u_{jk} \delta p^m_k)^{\star} B^p_{ji} 
%\delta p^m_i \label{invm_eq:3}
%\end{align}
%The left hand side of (\ref{invm_eq:3}) represents the residual, the right hand side sets a 
%controlling equation to obtain adjustment neceassary to decrease the residual. The last term of 
%RHS 
%shows a non-linear nature of the problem. This is omitted since the purpose of conseqeunt 
%iterative 
%technique is to find the final source distribution such that the model equations (\ref{invm_eq:4}) 
%are satisfied in least square sense. Than the ``nudge-factor" can be found as inverse of 
%\begin{equation}
%F_{j}^x - (B^u_{jk} p^m_k)^{\star} B^p_{ji}{p^m_i} = R_j^x = \Big[ (B^u_{jk} )^{\star} B^p_{ji} 
%p^m_i + (B^u_{jk} p^m_k)^{\star} B^p_{ji} \Big] \delta p^m_i \label{invm_eq:5}
%\end{equation}
%(these equations are not in matrix form, but obsevation point by observation point).
%Hence, the aim of inverse model is to decrease error in representation of energy fluxes. The 
%equation (\ref{invm_eq:5}) can be solved separately for zonal and meridional fluxes and also 
%simultaneously for both directions. That is at each iteration step the nudge-factor is found first 
%for zonal, than for meridional direction and finally, for both simultaneously. At the end pressure 
%distribution is changed by average from all three substeps.\\
%Note the inverse model equation (\ref{invm_eq:5}) is supported by additional condition stating 
%that 
%amplitude is nonnegative, $p_i^m + \delta p_i^m \geq 0$. All of this numerically is solved by 
%linear programming routine \text{lsei} (least square with inequality) provided by LINPACK 
%package.\\
%Here it will be presented a test convergence and number tests on robustness on proposed iterative 
%inverse model. The initial flux field is given by Fig. \ref{invm_fig:1} where by crosses are shown 
%observational points. This define prescribed $F_{j}^x$ or $F_{j}^y$. Note that the prescribed 
%field 
%aims to describe midbasin energy flux field with Tasman shelf ommitted due to presence of 
%reflection and complex bathymetry. The point sources distribution are given by green dots and at 
%the first iteration step are set to $p^0_i = 100 Pa$. The distribution of points sources is 
%representative to distribution of steep bathymetry which is belived to be an internal tide 
%generator. In the inverse model there are only two parameters that describe characteristic of the 
%internal tide, wavenumber and normalization coefficients used in energy flux. Both are found from 
%solving eigenvalue for randomly picked stratification profile. This result in wavelength of 
%$180~km$ which is a representative value for Tasman Sea conditions. In the same way eigenfunctions 
%are obtained and normalization coefficients are found.\\
%Hence, the inverse model does not account for
%\begin{enumerate}
%\item Bathymetry variation
%\item Stratification variability
%\item Variation of barotropic tide along ridges
%\end{enumerate}
%The first two points are thought to have minor effect on internal tidal beam structure. While the 
%third is omitted to preserve simplicity of generation model. Additional tests were done with 
%variation in barotropic tide phase along ridges, but they did not bring any substantial changes in 
%foregoing results.\\
%To show convergence of the inverse model it is given change in pressure amplitude with each 
%iteration. Here convergence is defined by
%\begin{equation*}
%Conv = \sum_i \frac{(p_i^m - p_i^{m-1})^2}{(0.5 \cdot (p_i^m + p_i^{m-1}))^2}
%\end{equation*}
%The iterative solver is stopped when convergence is reaching tolerance. Here it is set to 0.01. 
%From Figure (2a) it is seen that by 17th iteration there is no appreciable change in the inverse 
%solution. This means that influence of non-linear terms in (\ref{invm_eq:3}) became negligible and 
%the distribution of amplitude along the source region is the best in least square sense. The error 
%of such description is given on subsequent panels of Fig. 2, where root-mean-square-error for 
%different energy flux parameters is defined for example zonal component as
%\begin{equation}
%E_{x} = \sqrt{\frac{\sum_{obs} (F_i^x - \hat{F}_i^x)^2}{N_{obs}}}
%\end{equation}
%As it is seen the error is approaching stability for all used components much faster than 
%convergence in amplitude. Note that the error is larger in zonal fluxes. The inverse solution can 
%not predict far field behavior which is believed due to interaction with East Tasman Plateau. The 
%obtained solution is given by Fig. 3. Here it is found that the inverse solution can not well 
%represent the beam close to East Tasman Plateau. The following reasons can be named: interaction 
%with topography and inadequacy of cylindrical wave model in the far-far field. It is believed that 
%the second reason is the main. In general, the inverse solution picks up the central beam pretty 
%well, outlines its boundary and the major region is satisfying manner. As well note that the 
%northern and southern beams are also found in the solution.

\newpage
\section{Figures}

\begin{figure}
	\centering
	\includegraphics[scale = 
	0.5]{/home/dmitry/Work/Research/thesis/FINALE/P3_ITS_GENERATION/figures/fig_1_BT_tide.png}
	\caption{Comparison of $M_2$ sea level oscillations simulated by ROMS (left panel) with 
	TPXO-model (right panel).}
	\label{C3.fig:BT}
\end{figure}

\begin{figure}
	\centering
	\includegraphics[scale = 
	0.5]{/home/dmitry/Work/Research/thesis/FINALE/P3_ITS_GENERATION/figures/fig_2_uni_flux.png}
	\caption{Beam of Tasman Sea with major internal tide production sites identified by superposed 
	heat map.}
	\label{C3.fig:beam}
\end{figure}

\begin{figure}
	\centering
	\includegraphics[scale = 
	0.5]{/home/dmitry/Work/Research/thesis/FINALE/P3_ITS_GENERATION/figures/fig_3_gen.png}
	\caption{Generation of internal tides at Macquarie Ridge. (a) The heatmap illustrates 
	criticality in the region of major generation. And ellipses are representative of barotropic 
	current. (b) The diagnosed conversion rates for 'uniform' experiment. The boxes outline regions 
	used in further analysis. (c) Variability of conversion rates in the three regions identified 
	on the previous panel.}
	\label{C3.fig:gen}
\end{figure}

\begin{figure}
	\centering
	\includegraphics[scale = 
0.5]{/home/dmitry/Work/Research/thesis/FINALE/P3_ITS_GENERATION/figures/fig_4_stand_wave_2by2.png}
	\caption{Standing wave between Macquarie Ridge and Auckland Escarpments. (a) Distribution of 
	horizonal kinetic energy in 'uniform' simulation with overlaid energy flux of total mode-1 
	field. (b) Energy flux map of the westward component and the eastward component (c). Variation 
	in energy characteristics of mode-1 field between Macquarie Ridge and Aucklands Escarpment. The 
	integrated flux was obtained for a 	dashed transect in (c) and is given by arrows. Dots show 
	conversion rates spatially integrated over boxes given on (\fignm{C3.fig:gen}, b).}
	\label{C3.fig:stand_wave}
\end{figure}

\begin{figure}
	\centering
	\includegraphics[scale = 
	0.5]{/home/dmitry/Work/Research/thesis/FINALE/P3_ITS_GENERATION/figures/fig_5_2d_section_sill.png}
	\caption{Distribution of baroclinic fields during ebb and slack tide along transect crossing 
	Macquarie Ridge on \fignm{C3.fig:gen}, (b). (a-d) Baroclinic pressure anomaly with superposed 
	isopycnal displacements given by contour lines. Positive color is assigned to lifted interfaces 
	and negative - for the opposite. Stick lines show currents. And white line is distribution 
	of barotropic velocity. The tideward and leeward sides of Macquarie Ridge are identified by 
	vertical solid and dashed lines. On (e) it is shown time progression of 
	baroptropic-to-baroclinic conversion for both experiments and ridge sides. While (f) gives 
	distribution of period averaged conversion rate along the transect. The same is for (g) where 
	progression of mode-1 baroclinic pressure is shown by phases of total field and elemental 
	components. The phases are referenced to flood current across the sill.}
	\label{C3.fig:gen_2d}
\end{figure}

\begin{figure}
	\centering
	\includegraphics[scale = 
	0.5]{/home/dmitry/Work/Research/thesis/FINALE/P3_ITS_GENERATION/figures/fig_6_scatter_generation.png}
	\caption{Variation of the tideward conversion rates in relation to mode-1 energy transmitted 
	across the sill (a), to travel time across Solanders Trough (b), to knife-edge barrier 
	representative of Macquarie Ridge seeing from the deep ocean.}
	\label{C3.fig:gen_regr}
\end{figure}

\begin{figure}
	\centering
	\includegraphics[scale = 
	0.75]{/home/dmitry/Work/Research/thesis/FINALE/P3_ITS_GENERATION/figures/fig_8_along_beam.png}
	\caption{Basin variation of cross-beam averaged energy characteristics. For all panels - solid 
	black line is ensemble-mean value and thin gray lines for each particular realization. Red 
	lines show one standard deviation in realizations of the spatial averages. The dashed line is 
	produced by regression of the ensemble-means with a second degree polynomial. Panel (a) gives 
	spatially averaged direction of the energy fluxes counted counterclockwise from east. (b) 
	Cross-beam integrated flux. Purple line shows same integrated flux but with no account for 
	relative angle with respect to across beam section. (c) Ratio between spatially averaged 
	horizontal kinetic and available potential energies. Purple line is the ratio for a progressive 
	plane wave.}
	\label{C3.fig:beam_prms}
\end{figure}

\begin{figure}
	\centering
	\includegraphics[scale = 
	0.75]{/home/dmitry/Work/Research/thesis/FINALE/P3_ITS_GENERATION/figures/fig_9_beam_superposition.png}
	\caption{Interference in the mode-1 tidal field represented by ensemble-mean fields 4 circular 	
	quadrants (a-d) organized by counterclockwise order. Panel (a) represents northwesterly 
	traveling components of the angular spectra. The wave field is comprised both by spatial 
	variations in Available Potential Energy (color scale) and energy fluxes (arrows). The inset 
	gives respective variation of across-beam averaged HKE/APE ratio (solid black) and the in-situ 
	group speed (solid purple). The respective dashed lines show predictions following the  
	plane-wave theory. Panels (b-c) represent the wave field in the respective quadrants by the 
	flux vectors. And cumulative result of the superposition in the clockwise order starting from 
	(a) is shown by APE. The insets show effect of the superposition on the spatial averages. Note 
	that here the vector magnitudes are 10 times smaller than on panel (a).}
	\label{C3.fig:beam_dcmp}
\end{figure}

\begin{figure}
	\centering
	\includegraphics[scale = 
	0.75]{/home/dmitry/Work/Research/thesis/FINALE/P3_ITS_GENERATION/figures/fig_10_decomp_cumsum.png}
	\caption{Variability of mode-1 internal tide in the Tasman Sea. The same wave field 
	decomposition as in \fignm{C3.fig:beam_dcmp} was used here. Panel (a) shows variation in the 
	Northwesterly traveling waves. The arrows and color shading are plotted for the ensemble-mean. 
	Between experiments is given by respective variance ellipses. And standard deviation in the 
	spatially averaged characteristics is summarized on insets in top right corners.}
	\label{C3.fig:beam_dcmp_cm}
\end{figure}

\begin{figure}
	\centering
	\includegraphics[scale = 
	0.75]{/home/dmitry/Work/Research/thesis/FINALE/P3_ITS_GENERATION/figures/fig_11_anlt_result_comparison.png}
	\caption{Internal tide generated by a knife-edge ridge in the numerical model (upper row). The 
	field solved by the given analytical model in Appendix A (lower row). Three different regimes 
	are explored with no rotation present ($f = 0$, the left column), with Coriolis force 
	representative of the Macquarie Ridge ($f = 10^{-4}$, the middle column) and strong Coriolis 
	force ($f = 0.8 \omega_{M_2}$, the right column).}
	\label{C3.fig:anlt_sol}
\end{figure}

\begin{figure}
	\centering
	\includegraphics[scale = 
	0.75]{/home/dmitry/Work/Research/thesis/FINALE/P3_ITS_GENERATION/figures/fig_12_beam_models.png}
	\caption{(a) The flux map generated from sources having magnitude scaled by size of points. (b) 
	The field generated only by the central Macquarie Ridge section.}
	\label{C3.fig:beam_inv}
\end{figure}

\begin{figure}
	\centering
	\includegraphics[scale = 
	0.75]{/home/dmitry/Work/Research/thesis/FINALE/P3_ITS_GENERATION/figures/fig_13_generation_variability.png}
	\caption{(a) Variation of conversion rates along the Macquarie ridge. (b) Comparison of the 
	beam's center in the near field. (c) Change in total conversion rates and beam's strength. The 
	realizations in the field experiment are color coded throughout the figure (see panel (a)).}
	\label{C3.fig:gen_var_beam}
\end{figure}

\begin{figure}
	\centering
	\includegraphics[scale = 
	0.75]{/home/dmitry/Work/Research/thesis/FINALE/P3_ITS_GENERATION/figures/fig_14_example_beam_mesoscale.png}
	\caption{(a) Averaged distribution of sea surface height and vorticity in the Tasman Sea in 
	period between 10th - 15th January. Each contour line discerns 0.1 Rossby number. And 	
	distribution of total flux vectors. (b) Phase speed anomaly for the same dates and the planar 
	beam.}
	\label{C3.fig:meso_examp}
\end{figure}

\begin{figure}
	\centering
	\includegraphics[scale = 
	0.75]{/home/dmitry/Work/Research/thesis/FINALE/P3_ITS_GENERATION/figures/fig_15_mesoscale_variability.png}
	\caption{(a) Semi-major axis of variance ellipses. (b) StDev in phase speed anomaly during the 	
	field experiment. And variation in beam's center color coded same as on 
	\fignm{C3.fig:gen_var_beam}}
	\label{C3.fig:var_meso}
\end{figure}

\begin{figure}
	\centering
	\includegraphics[scale = 
	0.75]{/home/dmitry/Work/Research/thesis/FINALE/P3_ITS_GENERATION/figures/fig_16_comparison_sat.png}
	\caption{Comparison of internal tidal beam obtained through numerical simulations and from 
	satelite altrimetry observations. Arrows show their position and magnitude in red and black 
	respectively. Lines with dots show position of the beam center.}
	\label{C3.fig:cmp_sat_nexp}
\end{figure}

\begin{figure}
\centering
\includegraphics[width=0.7\linewidth]{../figures/fig_17_KE_basin_beam}
\caption{Variation in kinetic energy in the central Tasman Sea. White dots show the energy in HYCOM 
simulations, red dots - in the experiments performed here. The purple show variation of the beam 
position in ROMS-experiments relative to satelite altimetry. }
\label{C3.fig:KE_beam}
\end{figure}

\begin{figure}
	\centering
	\includegraphics[width=0.7\linewidth]{../figures/fig_18_spatial_TBEAM.png}
	\caption{Comparison of the energy flux obtained during the field program of TBEAM. On all 
	panels red pink colored arrows illustrate total field, blue - NW quadrant representative of the 
	planar beam, green - eastward traveling waves whose amplitude was tripled to be of comparable 
	size. Black arrows show results of the TBEAM observations. Panel (a) comparison of the field 
	program averaged numerical experiment. The arrows at observational sites were scaled 3 times 
	above background arrows which magnitude is given on the legend. (b) Comparison with '2013' 
	experiment. (c) Average over all experiments. Here additionally distribution of HKE and APE are 
	plotted by shading and lines of red and blue respectively.}
	\label{C3.fig:TBEAM_sp}
\end{figure}

\begin{figure}
	\centering
	\includegraphics[width=0.7\linewidth]{../figures/fig_19_KE2PE_TBEAM.png}
	\caption{(a) Ratio of HKE to APE observed during TBEAM field program (black line) and simulated 
	(red dots). (b-c) Distribution of HKE and APE for two cases of maximum and minimum.}
	\label{C3.fig:TBEAM_KE2PE}
\end{figure}

\begin{figure}
	\centering
	\includegraphics[width=0.7\linewidth]{../figures/fig_20_spectra_exmp.png}
	\caption{Example of spectral decomposition made from point-wise observations. Green line shows 
	result from spatial analysis of antenna. Orange line - result from point-wise decomposition for 
	the same numerical experiment. Blue line is a result for TBEAM.}
	\label{C3.fig:ex_spectra}
\end{figure}

\begin{figure}
	\centering
	\includegraphics[width=0.7\linewidth]{../figures/fig_21_test_decomp.png}
	\caption{Example of spectral decomposition made from point-wise observations. Green line shows 
		result from spatial analysis of antenna. Orange line - result from point-wise decomposition 
		for 
		the same numerical experiment. Blue line is a result for TBEAM.}
	\label{C3.fig:test_spectra}
\end{figure}


\begin{figure}
	\centering
	\includegraphics[width=0.7\linewidth]{../figures/fig_22_A1_comp.png}
	\caption{Result of pointwise decomposition at A1 location. }
	\label{C3.fig:dcmp_A1}
\end{figure}


%\begin{figure}
%\centering
%\includegraphics[scale = 
%0.75]{/home/dmitry/Work/Research/thesis/FINALE/P3_ITS_GENERATION/figures/fig_11_example_beam_mesoscale}
%\caption{Example of refraction. }
%\label{C3.fig:fig_11_beam_mesoscale}
%\end{figure}
%
%\begin{figure}
%\centering
%\includegraphics[scale=0.75]{/home/dmitry/Work/Research/thesis/FINALE/P3_ITS_GENERATION/figures/fig_12_beam_var_cp}
%\caption{}
%\label{C3.fig:fig_12_beam_var_cp}
%\end{figure}

%\begin{figure}
%	\centering
%	\includegraphics[scale = 
%	
%0.5]{/home/dmitry/Work/Research/thesis/FINALE/P3_ITS_GENERATION/figures/fig_11_along_beam_inv.png}
%	\caption{Variation of beam parameters.}
%	\label{C3.fig:beam_alng_var}
%\end{figure}

\bibliographystyle{apa}
\bibliography{/home/dmitry/Bibtex_lib/my_first_lib}

\end{document}