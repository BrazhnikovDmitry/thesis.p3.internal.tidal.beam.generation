\documentclass[12pt]{article}
\input{/home/dmitry/Work/Research/thesis/FINALE/settings.tex}
\newcommand{\SCALEO}{2}
\newcommand{\SCALET}{1.5}
%\doublespacing
%\graphicspath{{/home/dmitry/Work/Research/thesis/FINALE/P3_ITS_GENERATION/figures/}}

%\documentclass[PhD_Thesis.tex]{subfiles}

\begin{document}
	\iftoggle{only_Chapter} {
		\title{Variable internal-tide generation at the Macquarie Ridge.}
		\maketitle
	}

\section*{Abstract}
Variability in the conversion of barotropic tidal energy into internal tidal waves at the Macquarie 
Ridge is addressed by carrying out numerical experiments with variable hydrographic conditions. The 
main goal of the simulations is to compliment a field program, the Tasman TIdal Dissipation 
Experiment (TTIDE), which was designed to study the fate of radiated internal tides as 
they impinge Tasman continental slope located 1000 km to the west. The Macquarie Ridge 
and the nearby Campbell Plateau form a double-ridge system found here to be conducive to strong 
interference and complicated wave mechanics. The waves, emitted from the plateau couple 
with internal wave generation occurring at the ridge, and consequently drive amplification or 
reduction in the local conversion. 
The numerical experiments show that at the Macquarie Ridge energy transfer to mode-1 changes by 
$\pm 
10\%$ in magnitude and exhibits spatial variability. These modulations are inherently linked to the 
coupling with the nearby emitted waves. The remote wave properties are comprehensively examined by 
a 
novel technique for directional decomposition of the interfering wave fields. 
The 
result emphasizes non-plane wave propagation and the importance of multiple reflections on the 
wave amplitudes. It is shown that temporal variability stems from water mass redistribution 
attributed to dynamics of the local front. Additionally, 
local stratification modulates the reflectivity/transmission and hence, the degree of 
the amplification in the double-ridge system. Variable strength of the conversion leads to 
basin-scale changes to the wave 
field that can produce irregular signal in far-field observations and variable amount 
of energy arriving at the Tasmanian slope.\\

\newpage
\section{Introduction}
Baroclinic tides originate from heaving of isopycnal surfaces driven by a strong 
barotropic flow across topographic relief. This process converts barotropic tidal 
energy into baroclinic motions \citep{hendershott1981long}, constitutes a 
third of global budget for lunar semidiurnal constituent \citep{egbert2000significant, 
munk1997once} and contributes significantly to internal wave climate \citep{wunsch1975deep} and 
interior ocean mixing \citep{wunsch2004vertical}. At most conversion sites because of highly 
inclined slopes, the radiated 
internal waves of tidal period (internal tides) have vertical wavenumber spectrum dominated by 
highly stable low modes \citep{st2002role} able to transmit tidal energy over ocean basin-scale 
distances \citep{zhao2016global} with as yet unknown fate. While generation sites have been 
identified 
\citep{morozov1995semidiurnal, simmons2004internal, arbic2010concurrent} and some have been 
studied in detail, e.g. Hawaii Islands \citep{rudnick2003tides}, Luzon Strait 
\citep{alford2015formation}, Mendocino Escarpment \citep{althaus2003internal}, 
the Bay of Biscal 
\citep{gerkema2004internal}, and analytical models have been developed \citep{garrett2007internal}, 
there is still large uncertainty about variability in the conversion and its dynamics.\\

The analytical models of internal wave generation emphasize a ratio between angle of internal wave 
characteristics to bathymetric slope, known as slope's criticality, \citep{sutherland2010internal, 
garrett2007internal} along with a height of topography as primary quantities in setting conversion 
levels \citep{llewellyn2003tidal, petrelis2006tidal}. For tall, steeply inclined submarine ridges 
the energy transfer approaches an upper theoretical limit \citep{petrelis2006tidal, 
st2003generation} making them ``hotspots" of barotropic tide scattering and internal tide 
production \citep{morozov1995semidiurnal, egbert2000significant}. Amount of the conversed energy is 
quantified as correlation between the resultant baroclinic bottom pressure and the 
vertical component of the barotropic tidal current \citep{kurapov2003m}. So that, the expression 
represents work done by the 
barotropic tidal current in displacing the isopycnal interfaces. Clearly, buoyancy frequency 
directly 
impacts the energy transfers \citep{holloway1999internal} with variability especially 
pronounced when change of water properties takes place near supercritical bottom gradients 
\citep{gerkema2004internal}.\\

Nevertheless, larger temporal variability arises when nonlocal, baroclinic tidal signal couples  
with local process of generation \citep{Kelly2010a, zilberman2011incoherent, 
pickering2015structure}. The interaction affects not only the pressure term, but also timing 
of the baroclinic motions relative to the surface tide forcing. Moreover, the phase lag determines 
the sign of the energy transfer. For example, the conversion rate is negative if downward 
barotropic current, coincides with heaving of isopycnals, and hence, the baroclinic wave loses 
energy by performing work against the external force imposed by the barotropic tide. In the 
opposite case of the in-phase coupling, an amplified energy transfer from the surface tide takes 
place. Examples of the coupling can be found in cases when opposite slopes of single ridge 
affect each other \citep{zilberman2011incoherent, echeverri2010internal} or there is spatially 
inhomogeneous distribution of internal tide generation \citep{osborne2011spatial, 
ponte2013coastal}, or when neighboring topographic features drastically alter internal tide 
dynamics \citep{xing1998three, buijsman2012modeling, klymak2013parameterizing, buijsman2014three}. 
The present investigation considers similar processes at the Macquarie Ridge, south of New Zealand 
\fignmp{C3:fig:geo.map}, with emphasizes on temporal and spatial variability of internal 
tide generation.\\

The Macquarie Ridge scatters barotropic energy into internal tidal waves, in the Tasman basin the 
waves constructively interfere \citep{rainville2010interference} and form a baroclinic tidal beam 
\citep{zhao2016global}. Its 
fate is a subject of a recent field program, TTIDE \citep{pinkel2015breaking}, and one of the goals 
of this study is to compliment the observations with numerical simulations. The beam generation is 
complicated because of the neighboring Campbell Plateau. It is possible that similarly to Luzon 
Strait, the in-phase coupling of the radiated waves with local production can substantially amplify 
conversion of barotropic tide \citep{buijsman2012double, echeverri2010internal}. In turn, 
this can affect magnitude of the beam and far-field observations. Oceanographic conditions present 
additional difficulties as the Macquarie Ridge is located in region of interaction between 
subtropical (STW) and subpolar (SAW) water masses \citep{chiswell2015physical}. 
The zone is bounded by two fronts delimited with surface isohalines of $35.1$ and $34.7$ 
\citep{belkin1996southern, hamilton2006structure} \fignmp{C3:fig:geo.map}. The Southern 
Substropical Front 
(S-STF) defines the poleward extent of STW and has strong signature in the upper 100-200 m. The 
associated geostrophic current is weak because of strong density compensation 
\citep{graham2013dynamical}. But topographic 
steering by the ridge produce meanders of S-STF that can result in eddy shedding 
\citep{smith2013interaction}. Mesoscale variability is also induced by atmospheric cyclones and 
local winds that drive front filamentation \citep{james2002summer}. On synoptic scales there is a 
possible 
migration of up to $1^{\circ}$ meridionally \citep{smith2017variability} in response to a seasonal 
cycle of 
insolation and strength of the westerlies.\\

The primary goal of this study is to describe effect of the complex internal wave field on the 
generation of the tidal beam and how the dynamics evolves in response to the changing oceanographic 
conditions. These problems were investigated by means of regional numerical simulations that 
combine baroclinic tide dynamics and variable, realistic hydrographic conditions (Section 
\ref{C3.sec:model}). General 
characteristics of the generation are described in Section \ref{C3.sec:main_res}. Primary 
mechanism of 
amplification is identified and its connection to the remote internal waves is established 
(Section \ref{C3.sec:amp_mech}) by application of a novel decomposition technique (Appendix 
\ref{C3.app:A1}). 
This allows for 
complete description of how generation is governed by three-dimensional wave field, dependent upon 
the ocean state (Section \ref{C3.sec:3d_var}). To conclude, effect of the discussed dynamical 
factors are quantitatively related to the conversion levels and intensity of the tidal beam 
(Section \ref{C3.sec:disc}).\\
 
\newpage

\section{Numerical experiments and analysis}
\subsection{Numerical experiments}
\label{C3.sec:model}
Numerical simulations were performed with Regional Ocean Modeling System 
\citep{shchepetkin2005regional} to study dynamics of internal tide generation around New Zealand, 
propagation in Tasman Sea and reflection by Tasmania. The domain covered the southern Tasman Sea 
from $60^{\circ}$ S to $35^{\circ}$ S, the zonal extent stretched from $142^{\circ}$ to 
$172^{\circ}$ E. The horizontal grid spacing was $1/32^{\circ}$ on average corresponding to 
discretization of 3 km 
in zonal direction and 2.5 km in meridional. The nonuniformly separated, vertical 50 $s$-levels 
were placed to smoothly follow subsurface terrain.\\
Such discretization of vertical momentum equation can induce artificial, horizontal  
along-slope flows \citep{haidvogel1999numerical} due to errors in reproducing of pressure 
gradient force. Especially severe errors are made by steep terrain. The misbehavior is usually 
resolved by artificial smoothing of topography. This procedure additionally increases numerical 
stability, but has an adverse effect on dynamics of internal tide \citep{di2006numerical} since 
its generation sites are collocated with large topographic gradients. To test the numerical 
setup, a sensitivity study was carried out by performing additional experiments with grid size of   
$1/8^{\circ},~1/16^{\circ},~1/64^{\circ}$. Essential for this study 
internal tide behavior manifested at $1/16^{\circ}$ and converged for $1/32^{\circ}$ and 
$1/64^{\circ}$ cases with no marked differences observed.\\

This work addresses the gravest baroclinic mode dynamics in the deep ocean. Spatial 
extent of 
waves is large compared to associated vertical displacements. This ensures linear regime of 
propagation without dispersive and nonhydrostatic effects taken place such as fission into 
solitons. A hydrostatic solver used in ROMS seems to be a proper choice for the simulations. Such 
simplification in wave dynamics was also assumed in previous studies \citep{carter2008energetics, 
merrifield2001generation,  merrifield2002model, kerry2013effects}. In more dynamically 
accurate   
simulations of \citep{kang2012energetics, zhang2011three} the nonhydrostatic effects were found to 
be important only for internal tide propagation in shallow water, while the gravest mode generation 
mostly followed linear dynamics with vertical accelerations to have negligible contribution.\\

The horizontal boundary conditions were imposed to be open for depth-averaged, barotropic flows 
following recommendations of \cite{marchesiello2001open}. The baroclinic fields were  
nudged to zero by linearly increasing lateral viscosity and diffusivity over sponge layers. Through 
the same outer boundaries numerical simulations were forced with barotropic tide. The $M_2$ tidal 
currents and sea level were derived from TPXO atlas, version 7.2 \citep{egbert2002efficient} and 
prescibed as volume transports linearly interpolated onto the model's grid. This study considered 
only $M_2$ constituent since amplitude ratio with $S_2$ is 4-to-1 suggestive of 
slight open-ocean 
spring-neap modulation. The diurnal species are weak in the region except shoals east of New 
Zealand \citep{walters2001ocean}.\\

Several ocean states were prescribed and analyzed separately to investigate variations of 
baroclinic tide dynamics. In the \sqm{uniform}-case setting lateral gradients in water properties 
were 
absent, 
while buoyancy frequency was set to be representative of the Tasman Basin. The second type of 
simulations was performed to investigate interannual and interseasonal variability (Table 1). The 
third calculation was intended to cover whole period of TTIDE/TBEAM/Tshelf field programs 
\citep{pinkel2015breaking}.
\begin{table}
	\caption{Carried out numerical experiments}
	\begin{tabular}{ |p{3cm}||p{5cm}|p{5cm}|  }
		\hline
		\multicolumn{3}{|c|}{Numerical experiments used in this study} \\
		\hline
		Experiment abbreviation & Simulation period & Comments \\
		\hline
		Uniform & ~ & No mesoscale \\
		2012 &   Jan 1st - Jan 15th, 2012 & Interannual \\
		2013 &   Jan 1st - Jan 15th, 2013 & Interannual \\
		2014 &   Jan 1st - Jan 15th, 2014 & Interannual \\
		2013\_Oct &   Oct 1st - Oct 15th, 2013 & Interseaonal \\
		2015\_Mar &   Mat 1st - Mar 15th, 2015 & Interseaonal \\
		2015\_TTIDE$^{\ast}$ &   Jan 1st - Mar 1st, 2015 & Field period \\
		\hline
		\multicolumn{3}{|l|}{\footnotesize$^{\ast}$ the results are named as respective day of 
		year over which post-analysis was performed, e.g. $d20-25$ }\\
		\hline
	\end{tabular}
	\label{ch2:table_exp}
\end{table}
The simulations with variable conditions were at first initialized with HYCOM hindcasts
\footnote{(NAVGEM;	downloaded from hycom.org)} for respective start dates. Then during 
integration, 
along with barotropic tidal flow, time-variable, subtidal fields 
of horizontal currents, temperature and salinity were imposed along the 
boundaries. Air-sea interaction was simulated by utilizing MERRA-reanalysis 
\citep{rienecker2011merra} to prescribe realistic fluxes of heat, moisture and momentum.

\subsection{Internal tide analysis}
\label{C3.sec:anlsys}
Most of the simulations proceeded for 15 days \tblnm{ch2:table_exp}. In all 
cases the first 10 days were regarded as spin up period that allowed a single 
passage of mode-1 internal tide from New Zealand to Tasmania that takes roughly 7 days. After the 
spin-up period three 
dimensional fields of velocity, temperature and salinity were sampled hourly over 5 days. And in 
the TTIDE-experiment the quantities were subsampled with 5 day window, hence, naming convention 
\tblnm{ch2:table_exp}. Then the time series high pass filtered with Butterworth filter 
of 
order $6$ with cut-off frequency of $1/36~cph$ to remove subtidal motions. The filtered signal was 
further fit in a least square sense to $M2$ harmonic. Then the three dimensional field of 
horizontal currents underwent a separation into barotropic and baroclinic signals using following 
decomposition \citep{cummins1997simulation, kunze2002internal, carter2008energetics}
\begin{equation}
\label{ch2:bt_bc_vel}
\vec{u}_{bt}(x,y) = \frac{1}{H} \int_{-H}^{0} \vec{u}(x,y,z)  dz,~\vec{u}_{bc}(x,y,z) =  
\vec{u}(x,y,z) - \vec{u}_{bt}(x,y)
\end{equation}
Pressure associated with internal tidal motions was found from density field via the  
hydrostatic approximation and the same decomposition \citep{kunze2002internal, kelly2010topo},
\begin{equation}
\label{ch2:bt_bc_pres}
p(x,y,z) = \int_{-z}^{0} \rho(x,y,z) g dz,~p_{bc}(x,y) = p(x,y,z) - \frac{1}{H} \int_{-H(x,y)}^{0} 
p(x,y,z) dz
\end{equation}
In the both expressions rigid-lid approximation is used. This is a valid assumption unless 
vertical accelerations are greater than acceleration due to gravity \citep{kelly2010topo}.\\
Each variable was then decomposed into vertical normal modes. The eigenfunctions were obtained by 
solving Sturm-Liouville problem under the hydrostatic approximation,
\begin{equation}
\frac{d}{dz}\Big( \frac{1}{N^2(x,y)}  \frac{d \psi(z)}{dz}\Big) + c^{-2}_n \psi(z) 
= 0
\end{equation}
where $c_n$ is the mode phase speed in non-rotating ocean, $N^2(x, y)$ - local Brunt-Vaisala 
frequency profiles found from time-averaged density fields.\\
This work focuses on energy characteristics of the low mode internal tidal wave field. 
Period-averaged rates of conversion from barotropic tide to 
baroclinic \citep{simmons2004internal, kurapov2003m}, depth-averaged energy flux, horizontal 
kinetic energy (HKE) and available potential energy (APE) are found as
\begin{align}
\label{C3.eq:conv}
C_{bt\to 1} = -\frac{1}{2}(\cj{\vec{u}_{bt}} \cdot \nabla H) p_{1}|_{z = -H(x,y)}\\
\vec{F} = \frac{1}{2} \frac{1}{H} \cj{\vec{u}_1} p_1 \int_{-H}^{0} \psi_1(z) \psi_1(z) dz\\
HKE = \frac{1}{4} \rho_0 \cj{\vec{u}_1} \vec{u}_1 \int_{-H}^{0} \psi_1(z) \psi_1(z) dz\\
APE = \frac{1}{4c_1^2} \rho_0 \cj{p_1} {p}_1 \int_{-H}^{0} \psi_1(z) \psi_1(z) dz
\end{align}
Here the subscript $1$ identifies the first eigenmode, all variables are given in complex number 
notation with the complex conjugate defined by $\cj{}$. The coefficient $\frac{1}{2}$ appears due 
to 
time 
averaging.\\

The mode-1 internal tide fields were additionally subject to directional analysis in order to 
extract elemental signals usually hindered by interference. The details are presented in appendix 
\ref{C3.app:A1} and summarized as follows. Dynamical fields of pressure and currents sampled over 
area-limited footprint can be described by directional 
spectra. Its circular Fourier 
transform was then obtained as solution by damped least squares of an ill-posed inverse problem. 
The 
spectral, directional analysis differs from the approach of \cite{zhao2010long} for two main 
reasons. The used here technique seeks for continuous distribution of 
energy over all compass directions, rather than a finite number of plane waves. This makes a 
difference in regions 
where wave diffraction prevails. For instance, wave field near sites of internal tide generation or 
scattering will be comprised by angular distribution of energy confined to diffractive lobes 
\citep[e.g.,][]{munroe2005topographic, johnston2003internal}. On 
the other hand, because of finite-window sampling artificial lobes are introduced as in classical 
Fourier analysis. The analysis artifacts lead to unambiguous interpretation, so judgment should be 
used when wave fields are back-synthesized over sampling angular sectors. Secondary, 
velocity measurements are utilized along with pressure observations \eqref{C1:uv.eq}. The 
formulation provides additional constrain on the inverse model. Indeed, the velocity equations tend 
to lessen Gibbs phenomena. Furthermore, the directional spectra technique can be extended 
point-wise observations where currents and pressure are measured simultaneously  \footnote{such as 
a stationary mooring}.\\

\textbf{Get rid of it? - Define what you mean by average and stdev}\\
All of the above calculations produced sets of dynamical variables for barotropic, baroclinic 
fields and directional representation of the gravest mode for each experiment. To study 
variability in the generation, mean energy characteristics were defined by simple arithmetic mean 
(ensemble-mean) and variability - by standard deviation. Such definitions have a caveat. 
While ensemble-mean allows a straightforward interpretation of mode-1 dynamics, it 
incorporates  
a portion of variable (non-stationary) signal since any energy characteristic is a nonlinear 
quantity \citep{zaron2014time}. On contrary, energy flux, for instance, found from ensemble-mean 
pressure and currents exclusively provides a stationary part. However, if a mode-decomposed signal 
is considered, finding the means entails averaging of vertical basis functions. As a 
consequence, their orthogonality is not preserved leading to uncertainty in the dynamics. 
Performed comparison (not shown) between an ensemble-mean of flux and a flux of ensemble-means did 
not reveal significant differences in spatial structure, though the latter results were smaller.

\section{Results}
\subsection{General outlook on internal tide generation at the Macqurie Ridge}
\label{C3.sec:main_res}
The semidiurnal barotropic tide propagates into the Southern Tasman Sea from the North 
\fignmp{C3.fig:BT}. Its phase increases in counterclockwise manner with maximum amplitude of sea 
level located along New Zealand's coast, consistent with Kelvin wave dynamics 
\citep{walters2001ocean}. The simulated sea surface tidal oscillation closely followed TPXO7.2 with 
gross 
features well captured, but the sea level amplitude and cotidal (phase) lines exhibited wave-like, 
small scale perturbations. This is a typical manifestation of low-mode baroclinic waves 
\citep{niwa2004three}. The baroclinic tidal field \fignmp{C3.fig:beam} appeared as a complex 
pattern produced by 
multiple generation sites determined by steep topographic relief. The primary sites were located  
just south of New Zealand as flow of the barotropic Kelvin wave interacted with relief of the 
Macquarie Ridge. As a result, several low-mode beams \citep{rainville2010interference} were emitted 
into the Tasman 
sea. Henceforth, analysis concerns the most energetic beam originated from a site centered at 
$49.5^{\circ}S$ (inset on \fignm{C3.fig:geo.map}), hereafter referred to as the Macquarie 
Ridge.\\

The major beam was mainly generated on highly supercritical slopes where kinetic energy of the 
barotropic tidal currents was conversed into available potential energy of the internal tidal 
waves. Geographically, the steepest slopes are found at two seamounts whereas a gap separating the 
two has more gradual relief. Such seabed terrain steered the barotropic 
current and forced rectilinear flow in the gap \citep{holloway1999internal}, but the current 
amplitude did not significantly increase. The described conditions suggest that the conversion 
ought to occur primarily 
at the 
seamounts. However, simulated distribution of the conversion rates \fignmlp[C3.fig:gen]{B} and 
spatial averages over the topographic features \fignmlp[C3.fig:gen]{C} revealed different 
perspective. The gap contribution was significant and sometimes the largest. All averaging 
regions   
exhibited 
unusually high variability in magnitude. As well, the spatial pattern was rearranging between 
the experiments, location of the conversion hot spot was moving between southern and northern 
parts of the ridge.\\

The Macquarie ridge transfers $1.7 \pm 0.2~GW$ of the surface tide into mode-1 baroclinic tide. 
This 
is a half amount at Kaena Ridge, Hawaii \citep{carter2008energetics} and tenfold smaller 
than Luzon Strait \citep{alford2011energy, kerry2014impact}. But similar to 
Luzon Strait the Macquarie ridge features complex internal tide dynamics that was manifested as 
regions of negative conversion (green shading, \fignml[C3.fig:gen]{B}). Such reverse energy 
transfer and destruction of internal tide is driven by coupling with non-local baroclinic waves 
\citep{Kelly2010a}. In 
fact, adjacent slopes of the Campbell Plateau 
(Auckland Escarpment) provided a source of such remote signal. Further superposition with the 
ridge-emitted waves created a standing wave \fignmlp[C3.fig:stand_wave]{A}. The wave presence was 
evident in the distribution of HKE marked by a node in the middle of Solander trough. Additionally, 
around the node energy flux formed a circular pattern \fignmp{C3.fig:stand_wave} with 
counterclockwise progression since the Coriolis parameter is negative in the Southern Hemisphere.\\

The standing wave was separated into elemental east-west traveling components 
\fignmlp[C3.fig:stand_wave]{B, C}. On the eastward side of the Macquarie Ridge generation took 
place at the same seamounts. On slopes of the escarpment the conversion had richer 
distribution of positive and negative sign manifesting complicated structure of energy transfers. 
Part of the reason is in more complicated sea bottom relief and the barotropic forcing 
\fignmlp[C3.fig:BT]{A, B}. Dynamically though, the impinging, ridge-emitted waves were 
back-reflected into the trough, and at the same time they were interfering with the local 
generation, 
hence, producing the negative conversion.\\

Respective energy transfer across the trough was quantified and compared with spatially integrated 
conversion rates \fignmlp[C3.fig:stand_wave]{D}. On average, each elemental wave field 
was almost equally divided amongst waves reflected from the supercritical slopes and wave energy 
converted from the barotropic field. Though the system exhibited large departures from the average 
state due to variable conditions of the coupling on the trough's slopes.\\

It is apparent that the coupling represents key mechanism to describe variability of the 
conversion \fignmlp[C3.fig:gen]{C}. The associated wave mechanics is examined further for a 
stand-out result 
of \sqm{2014} when conversion in the gap was the weakest. Alongside with \sqm{2014}, experiment   
\sqm{d10-15} is also considered to characterize what appears to be a normal situation at the 
Macquarie Ridge.
%The obtained internal tide description is similar to Luzon strait where near-resonance conditions 
%exist 
%between two almost parallel ridges \citep{buijsman2012double, buijsman2014three} separated by a 
%distance of 0.6 mode-1 wavelength. In the case of the Macquarie Ridge and the Campbell Plateau the 
%separation distance is about the same. But resonance should be weaker since the former has an 
%orientation of $15^{\circ}$ relative to the 
%latter.\\
%
%The geometrical characteristics suggest relevance of wave's travel-time to the observed 
%variability in baroclinic tide 
%production levels and its spatial pattern. To examine this the internal tide dynamics is further 
%studied along a two-dimensional transect crossing the gap (horizontal line on 
%\fignml[C3.fig:gen]{B}) in simulations with diametrically different generation regimes (\sqm{2014} 
%and \sqm{d10-15, 2015}, \fignml[C3.fig:gen]{C}, \fignmlp[C3.fig:stand_wave]{D}). Consequent 
%results are similar to findings of \cite{echeverri2010internal, buijsman2012double} with 
%exception that here the decomposition method augments variations in the mode-1 wave field.\\

%Variation in the coupling process was responsible for the reported changes in the conversion. 

\subsection{Coupling of local generation with remote waves}
\label{C3.sec:amp_mech}
\textbf{NEW:}
\newpage


The experiments, \sqm{2014} and \sqm{d10-d15} serve to illustrate different pathways that the 
energy transfer can take. \fignml[C3.fig:gen_2d]{A-D} 
depicts respective wave field developments 
throughout the tidal stages along a section crossing the gap; \fignml[C3.fig:gen_2d]{E-F} 
shows time series of the 
conversion in two locations on the ridge. At ebb, on the ridge's west side 
\fignmlp[C3.fig:gen_2d]{A, C}, the downward barotropic current forced isopycnals 
to 
dip near bottom. 
At that moment the energy was transferred from the barotropic tide into the baroclinic waves as the 
conversion was found to be positive in 
both experiments \fignmlp[C3.fig:gen_2d]{E}. However, in \sqm{2014}, at about hour 7, the transfer 
became 
negative as, 
contrary to the forcing, isopycnals began to rise and fully formed by slack stage 
\fignmlp[C3.fig:gen_2d]{B}. At that moment the barotropic current reversed direction and the 
conversion changed sign again. Here, the process of energy transfer was distorted and resulted 
in weak internal tide generation 
\fignmlp[C3.fig:gen_2d]{E}. 
The example of \sqm{d10-15} revealed other regime. Throughout the ebb flow the conversion rate was  
large and positive as the isopycnals were permanently depressed signifying larger loss of the 
barotropic energy.\\

The striking difference originated from dynamics of the remote internal waves. During the ebb flow 
in \sqm{d10-15} a mode-1 wave crossed the ridge, evident in the 
baroclinic pressure by a characteristic vertical dipole pattern. The wave travel was accompanied by 
downward isopycnal motion along both slopes of the ridge. On the west side the interaction was 
constructive because the mode-1 propagation was in phase with the barotropic current, this 
was driving the strongly positive energy transfer. On the east side the forcing was reversed and 
the transfer was negative \fignmlp[C3.fig:gen_2d]{G, E}. In contrast, the mode-1 wave 
was not observed on the ridge slopes in \sqm{2014}. The east side exhibited positive conversion 
that led to radiation  from the ridge crest of an internal wave ray \fignmlp[C3.fig:gen_2d]{A, 
dashed line}. Apparently, the mentioned previously lifting of the isopycnals in the middle of the 
ebb flow on the west side in \sqm{2014} resulted from the ray propagation across the ridge. This 
illustrated destructive interference of the remote wave with the local generation.\\

It should be noted that the remote mode-1 wave was present in the Solanders trough 
($162.25^{\circ}$) in both experiments, but the wave travel paths were different. This conclusion 
was supported by the directional decomposition and respective 
wave phase distribution \fignmp{C3.fig:2d_phase}. The trough was occupied 
by a standing wave pattern seen as almost constant phase and hence, no propagation of the total 
signal. 
Near the Campbell plateau ($162.25^{\circ}$) the decomposition revealed almost equal contribution 
of the east and west-traveling waves. However, at the ridge in \sqm{d10-15} the total signal was 
seen to propagate across the ridge with the westward component to be dominant. In \sqm{2014} 
the eastward wave was stronger and the phase tended to $-180^{\circ}$ (relative to the west side's 
surface tide) implying in-phase relation with the local barotropic tide and local generation. In 
both experiments the westward wave reached the ridge at similar phase of the barotropic tide. 
The further interaction was of similar nature and thence, can not explain the differences 
in the mode-1 travel and the energy transfer pathways. Therefore, it is necessary to elaborate the 
westward wave characteristics.\\


\subsection{Spatial structure of the wave fields}
\label{C3.sec:3d_var}
\fignmp{C3.fig:wv_fld_dist} illustrates the west and eastward wave components in the considered 
experiments. Here, a notable feature bring light on the difference 
The westward wave in 
\sqm{2014} was small as generation at the Campbell Plateau was diminished (see also 
\fignml[C3.fig:gen]{C}). And since lines of constant phase (isochrons) were slanted relative to the 
ridge, such 
largely oblique incidence even more reduced across topography propagation. While in all other 
experiments the cross-ridge component was much stronger as the incidence was more direct. 
Furthermore, the westward wave characteristics were not permanent which is illustrated 
by 
transition happened over 3 weeks in the TTIDE-simulation \fignmlp[C3.fig:wv_fld_dist]{B-C}. The 
overall 
production at the Macquarie Ridge increased from 1.6 to 1.9 GW \fignmlp[C3.fig:gen]{D} as the 
westward 
wave started to arrive later. The resulted smaller phase lag with the local production led to the 
observed growth.\\

The westward wave field was shaped by dynamics occurring on the slopes of the Campbell 
Plateau where similar to the Macquarie Ridge reflection coupled with local generation. In the 
numerical experiments the coupling process exhibited highly variable spatial pattern of 
generation/destruction. For \sqm{2014} there was a strong destruction evident near $(166.5E, 
48.75S)$, \fignmlp[C3.fig:wv_fld_dist]{A} with the other cases showing insignificant energy loss. 
This 
location corresponded to impingement of the eastward-traveling wave produced at the northern 
seamount \fignmlp[C3.fig:wv_fld_dist]{bottom row}. And as before its phase lag to the surface  
tide 
was a primary factor. That is, earlier arrival such as in \sqm{d10-15} and \sqm{d35-40} 
corresponded to weak energy losses and this also explained higher production rates at the 
Campbell 
Plateau in the later period \fignmlp[C3.fig:gen]{C}.\\

It was straightforward to establish a relation between the phase of the eastward waves and mode-1 
eigenspeed under extreme conditions. For example, the generation in \sqm{2014} was   
exacerbated by negative phase speed anomaly of $15\%$, i.e. $\sim 20^{\circ}$ or 
$40$ minute later arrival. Opposite situation was seen for \sqm{uniform}-experiment where higher 
phase speed led to much earlier arrivals and very strong amplification (compare 
\fignml[C3.fig:gen]{B} with \fignml[C3.fig:wv_fld_dist]{A-C}). But in the 
cases of modest departure, such interpretation was complicated with initial phase of the 
eastward waves. The field emitted by the northern seamount was about at $0^{\circ}$ in 
\sqm{d10-15}, while in \sqm{d35-40} - $\sim 50^{\circ}$. In turn, the difference 
arose from incidence and arrival of the westward waves.\\

The westward wave characteristic heavily depends on complicated pattern of multiple site 
generation/reflection at the Campbell Plateau rather than anomalies in the phase speed alone. This 
is contrary to the eastward waves. This field is produced by simpler generation pattern. And 
since the isochrons closely repeated the emitting seamount's topography, the plane wave description 
is fairly appropriate and hence, facilitates linear phase dependence on inverse of the phase 
speed.\\

Accordingly, phase speed and stratification change should be thought as a triggering mechanism for 
adjustments in the generation patterns. To illustrate this, respective hydrographic conditions are 
presented on \fignm{C3.fig:ocean_cond}. For the region sea surface salinity draws distinction 
between warmer 
subtropical waters (STW is seen by brighter colors on the figure) and fresher subantarctic waters 
(SAW - colder color) and is well representative of the upper water column density structure. As 
consequence, there was a strong correspondence with phase speed anomalies 
\fignmlp[C3.fig:wv_fld_dist]{lower panels}. This is exemplified for two locations. In the 
first case it was observed development of a frontal filament following passage of an atmospheric 
cyclone \fignmlp[C3.fig:ocean_cond]{B, C}. The advected STW forced stronger pycnocline in the 
upper 
200 m (\fignm{C3.fig:ocean_cond} lower panels designated by '1', blue and green lines) and the 
positive anomaly. It is notable that in \sqm{2014} (red line) a mixture of SAW and STW led to 
weaker stratification as there was pronounced density compensation. The second position was chosen 
to display a warm-core eddy that created large, localized deviations from the mean 
\fignmlp[C3.fig:wv_fld_dist]{D}. The weak vertical density gradient was present again and 
Brunt-Vaisala frequency was exceptionally small in the upper 100 m. Despite this the eigenspeed was 
high. This was caused by deeper pycnocline ($\sim 1000~m$) that separates fresher Antarctica 
Intermediate Water from the upper ocean \citep{chiswell2015physical}. In this experiment the eddy 
had strong saline signature that reinforced the secondary pycnocline and it exerted higher 
influence on the eigenfunctions. The examples lead to conclusion that presence of STW produce 
positive anomalies, while opposite is true for SAW. And the converging water masses has direct 
effect on the internal tide kinematics.

\section{Discussion}
\label{C3.sec:disc}

\subsection{Conversion}
\label{C3.sec:disc_conv}
The presented evidences state importance of the wave field in Solander trough for the internal 
tidal beam. Dynamics on the trough's slopes is closely linked to the coupling processes and is 
affected by the ocean state. It also seems that stationary regime can be hardly observed. Changes 
in stratification and phase speeds control respective lags in the arrival times to affect energy 
transfers and strength of the westward waves that mediate the beam generation.\\

This conclusion is evaluated by the energy characteristics plotted on 
\fignml[C3.fig:ener_var]{A, B}. It is seen that variability of generation at the gap explains  
$75\%$ of variability in the overall production at the Macquarie ridge. While the generation at the 
gap is mainly related to strength of the incident, westward wave. The amplitude was 
quantified as integral of energy flux across a red line on \fignml[C3.fig:wv_fld_dist]{F} and 
given by black line on \fignml[C3.fig:ener_var]{B}. Correlation coefficient between the 
generation and the flux is $R^2 = 50\%$, but after omitting \sqm{uniform} experiment, $R^2$ rises 
to $80\%$. The \sqm{uniform} experiment is an outlier because the prescribed stratification,  
typical of the deep Tasman basin, led to unrealistically high phase speeds in the 
generation region. The westward wave arrived in-phase with the local barotropic tide on the 
lee side of the ridge and partly out-phase on the western side seen as destruction on 
\fignml[C3.fig:gen]{B} at the northern seamount. This regime 
is clearly opposite to the other experiments \fignmlp[C3.fig:gen]{C}.\\

Direct effect of variable stratification on the generation has also to be investigated because of 
the highly dynamical ocean environment. To account for this, an analytical model of 
\cite{st2003generation} was applied. The ridge was thought to be a two-dimensional knife-edge 
topography whose height was found from the topography's mean depth in WKB-scaled vertical 
coordinate \citep{althaus2003internal} relative to the abyssal depth. The knife-edge barrier was  
transversely stretched for 240 km and was subject to barotropic current of $0.03~m/s$ 
representative of the ROMS simulations. Generally, under conditions of strongly stratified ocean 
the knife-edge ridge is taller and sheds more baroclinic energy. But the calculation resulted in 
converse dependence \fignmlp[C3.fig:ener_var]{C}. The inconsistency was resolved by 
considering effect of stratification on energy transmission across the knife-edge barrier. To 
estimate this the ridge's height ratio was obtained relative to Solander Trough. Than closed-form 
solutions of \citep{larsen1969internal},
\begin{equation}
\label{C3.eq:ke_transm}
R_{knife} = \sin^2 \frac{1}{2} \pi h - \frac{1}{2} P_2( \cos \pi h ) + (\frac{1}{2} - \sin^2 
\frac{1}{2} \pi h) P_1( \cos \pi h ),~
T_{knife} = (1 - R_{knife})^2
\end{equation}
where $P_1,~P_2$ - the Legendre polynomials and $h$ - ratio of depth occupied by the barrier to 
the ocean depth, was used. The values obtained for each simulation are plotted on 
\fignmlp[C3.fig:ener_var]{A, empty diamonds} and then they were used to scale the overall 
incident flux (integrated across blue line on \fignml[C3.fig:stand_wave]{C}). 
\fignml[C3.fig:ener_var]{C} suggests the change in stratification affects ridge's transmission 
and taller ridge allows less across-ridge transport. The result does not exclude the implied role 
of stratification, but rather emphasizes the primary mechanism of amplified generation due to the 
remote wave.\\

It is apparent that the calculation based on the knife-edge theory is an underestimate. For 
instance, using \eqref{C3.eq:ke_transm} transmission across the gap alone is about $15\%$ 
\fignmlp[C3.fig:ener_var]{B}, but this contradicts largely subcritical slope and only small  
change in the crossing mode-1 \fignmlp[C3.fig:gen_2d]{C-D}. Using the decomposition method 
different estimates were obtained by taking ratio between the trough's incident to the westward 
fluxes on the 
ridge \fignmlp[C3.fig:gen_dynamics]{A-B, purple filled dots}. The 
estimates seem to be more plausible as the gap alone transmits $47 \pm 8~\%$ and the Macquarie 
ridge - $35 \pm 4~\%$. Nevertheless, these values should also 
be taken with a grain of salt as the wave field near the ridge's top is not homogeneous breaking the 
underlying assumption of the method. In actuality, if the above estimates were used in the same way 
as the knife-edge calculations, i.e. to produce \fignmlp[C3.fig:ener_var]{D}, no correlation 
would be found. Though pertinent outcome of the analysis is in similar character of variation for 
the differently estimated transmission coefficients. Therefore, dynamical effect of stratification 
is to control across ridge passage of waves.\\

Spatially different transmission or reflection along the Macquarie Ridge also explains spatial 
distribution of conversion and its variation \fignmlp[C3.fig:gen]{C}. This was quantified by 
differentiating \eqref{C3.eq:conv} with respect to involved physical parameters and than replacing the 
differentials with variances \citep{kerry2014impact}. The obtained expressions detail how much of 
change in the conversion is caused by particular parameter \fignmp{C3.fig:var_sp}. It is found that 
total variation amounts to 0.9 GW, while previously it was reported at 0.2 GW (Section 
\ref{C3.sec:main_res}). The disparity arises as the parameters are not statistically independent. 
For example, change in vertical velocity $w_{bt}$ also affects bottom baroclinic pressure producing 
non-zero covariance between the two. Nevertheless, the obtained distribution shows a pattern that 
corroborates previous observations. The variation due to phase lags are the strongest in the gap 
and smaller over the seamounts, while opposite holds for magnitude of the bottom pressure. 
Therefore, in the gap because of the weak local generation and comparable magnitude of the remote 
signal, the westward wave smears both amplitude and phase of the resultant bottom pressure. On the 
steeper flanks of the seamounts the local generation is strong and the superposition mainly  
modulates the amplitude. Such point is easily demonstrated by plotting vector diagrams. The variability analysis demonstrates that on more gradual slopes remote waves perturb conversion stronger than on steep slopes. 
Such as at the Auckland Escarpment, all of variability is dictated by phase of the arriving eastward 
waves supporting discussion in Section \ref{C3.sec:3d_var}.\\
Additionally, there is about $10\%$ variations associated with the barotropic forcing which is mostly found in regions of the highly supercritical seamounts. Variable  steering of the barotropic tide could be named as a reason  \citep{holloway1999internal}. More likely though, distortions were introduced into vertically averaged component by strong shoaling of internal tide.\\

\subsection{Standing wave}
\label{C3.sec:disc_sw}
Additional effect of variable transmission is to change energy balance of the standing wave. To 
investigate this analogy of Solander trough to Fabry-Perot interferometer can be 
invoked\footnote{see another application of such approach to internal wave problem by 
\cite{mathur2010internal}} as follows. Let conversion rates on the slopes be 
$C_M$ and $C_C$ and reflection coefficients - $r_M,~r_C$ with indexes denoting respectively the leeward 
side of the Macquarie Ridge and the Campbell Plateau. Strength of the west component initially is 
just $C_C$, after the first bounce from the ridge, then from the plateau and adding a portion of 
energy that is emitted from the ridge and reflected by the plateau, the flux becomes $r_C r_M C_C 
+ r_C C_M$, the second crossing brings $r_C r_M (r_C r_M) C_C + r_C r_M (r_C) C_M$, the $n$-th 
- $(r_C r_M)^n C_C + (r_C)^n (r_M)^{n-1} C_M$. At some moment, the total flux is $F_W = 
\sum_{i = 1}^N (r_C r_M)^i C_C + (r_C)^i (r_M)^{i-1} C_M$. Now considering equilibrium state, $N 
\rightarrow \infty$, so transition to geometrical series is made, one arrives to
\begin{equation}
\label{C3.eq:fl_fabry}
F_W = \frac{1}{1 - r_C r_M} (C_C + r_C C_M)
\end{equation}
The equation can be further solved for the reflection coefficient. Consider averaged result 
from the numerical experiments \fignmlp[C3.fig:stand_wave]{D} with $C_M \simeq C_C = 0.45~GW$ and 
$F_W = 1.2~GW$ and assuming $r_C = r_M$, it is found that the reflection is at $63\%$ which in 
close agreement with the transmission estimates derived previously. In reality energy losses are 
present during conversion, so letting the wave field to acquire $2/3 \cdot 0.45$ 
from the barotropic tide, the coefficient becomes $75\%$. Regardless, the point is 
clear that such parametrization of the standing wave is appropriate.\\

The Fabry-Perot description stresses amplification of the wave field by a factor in 
\eqref{C3.eq:fl_fabry}. If the multiple reflection were absent, strength of the westward field would be    
close to conversion at Auckland Escarpment and consequently, magnitude of the beam would be smaller. In fact, 
the test experiment of resolution $1/8^{\circ}$ the Macquarie Ridge was heavily unresolved and 
smoothed having very weak reflectance. As a result, the standing wave was absent and the simulated 
beam was weak and narrow. The amplification factor as well reveals coupling between the opposite 
slopes. As their respective reflectance changes, the consequent effect is multiplicative and nonlinear.\\

The description by \eqref{C3.eq:fl_fabry} also augments asymmetrical response in the 
East-West fluxes to variations in the emitting's slope reflectivity. The results on 
\fignml[C3.fig:stand_wave]{D} indicated more 
consistency of the eastward component in the TTIDE-simulation. This implies less variation of the 
ridge reflection than at the plateau that is apparent from \fignm{C3.fig:wv_fld_dist}. Worth noting 
is equality of the averaged E-W components that raises question about its physical nature or 
being an artifact of small number of experiments and/or of the chosen ocean conditions.\\

\subsection{Comparison with Luzon Strait}
\label{C3.sec:disc_luzon}
The same arguments were previously identified by \cite{buijsman2012double} and 
\cite{klymak2013parameterizing} in work related to Luzon strait. These workers (authors?) either numerically integrated or solved mode-matching equations for a problem of two-dimensional internal wave enclosed by ridges. In the later  realistic, three-dimensional simulation \citep{buijsman2014three} reported the amplification factor of about 
3.5\footnote{\citep{buijsman2014three} presented in how many times increase was larger compare to 
when interference is absent (non-resonant), i.e. their definition is 
$\Psi_C = (C_{res} - C_{non-res})/C_{non-res}$.}. The expression 
\eqref{C3.eq:fl_fabry} rather states one-dimensional description of the multiple reflection 
with all phenomenology hidden in separately diagnosed energy terms. Nevertheless, the lower bound can be obtained as ratio of the flux to the conversion, $F/C = 2.6$. This estimate does not account for increase in conversion in comparison to non-resonant condition. However, larger amplification at Luzon Strait also arises as the ridges are near-parallel which promotes more wide-spread amplification. In Solander Trough slanting incidence of the waves \fignmp{C3.fig:wv_fld_dist} lead to spatial variability of the enhancement.\\

It is also worth comparing the documented here variability with results of \cite{kerry2014impact} 
who reported $10~\%$ change in the conversion rates for Luzon strait. This is the same as at the 
Macquarie Ridge while one 
expects, as a first order guess, linear scaling of variability with the amplification factor. 
Apparently, the similar levels are because of different mechanisms. The study of 
\cite{kerry2014impact} 
emphasized role of far-field remote waves radiated by the Mariana arc as primary reason for the 
changes. In this work driving mechanism of the variability was found to be in 
adjustment of the standing wave via the slopes' reflection and stratification. Question arises as 
why Luzon Strait 
does not host similar process and exhibit large deviations being located near vigorous Kuroshio? 
Reason can be 
found in large 
disparity 
between the ridges' height as \cite{klymak2013parameterizing} concludes that the eastern ridge ``casts 
a shadow" over the western. Their respective WKB-heights are 0.2 and 0.6. Following the previous 
discussion on the asymmetry any changes in the dynamical properties of the higher eastern ridge 
influence contribution of the western ridge which is smaller. Its depth also 
inhibits strong effect of the upper ocean conditions. Hence, the eastern 
ridge energy input should be steady if direct influence of stratification on the generation is 
neglected. Though such inequality can produce noticeable shifts in the standing wave, i.e. in the phase of baroclinic pressure and the node's position that can provide explanation of the field observations of \cite{buijsman2014three}.

\section{Summary and Conclusions}
The numerical experiments were conducted to investigate dynamics of the Tasman Sea internal tidal 
beam in realistic and variable oceanographic conditions. This chapter was concerned with origin of 
the beam. The generation takes place along the Macquarie Ridge as the barotropic tide sloshes across 
the steep relief. But the process was enhanced as the strong remote signal constructively interfered with the local production of the baroclinic tide. The largest amplification was localized at the topographic gap separating two highly supercritical 
seamounts. The resulting spatial pattern exemplified role of topographic inhomogeneity in shaping 
conversion ``hotspots".\\

Such dynamical regime of the Macquarie Ridge is analogous to Luzon strait. Both represent case of 
the coupling between lineal internal-tide producing topographies. In the considered 
case the 
remote wave had origin on the slopes of the Campbell Plateau. These waves superposed with the 
ridge-emitted one to form the standing wave that masked magnitude of the elemental, east-west 
oriented components and hindered energy budget analysis. But with the directional decomposition 
technique it was possible to quantify reflection/transmission and compare it with conversion of 
the barotropic tide. These estimates augmented importance of the multiple reflection from the 
opposing slopes. This observation led to application of the Fabry-Perot theory. The obtained simple 
relation was able to describe asymmetries in the east-west signals. As well, the system exhibited 
nonlinear amplification factor that depended on reflection/transmission properties of the involved 
topographies.\\

Degree of the amplification varied with changing hydrographic conditions. And 
since the region is located where two distinct water masses converge into the frontal zone, 
it was presented strong evidence that redistribution of the water types triggers swings in the 
conversion 
rates. Such process was explained by employing two mechanisms. In terms of kinematics the upper column 
stratification directly altered internal tide phase speed and consequently, phase lags with the 
barotropic forcing. While dynamical effect involved variation of the reflection/transmission 
coefficients. It is worth emphasizing that both processes were not fully separated, since the 
estimated coefficients were themselves dependent on the phase of the incident waves. At the same 
time, this suggests existence of a feedback mechanism that forces system of the two coupled 
generators to achieve equilibrium it negative. Such feedback loop could potentially explain the 
symmetry in the ensemble-averaged properties even though reflection at the Campbell Plateau was 
highly 
variable.\\

The variability in energy levels of the remote waves were translated into the tidal beam magnitude. 
However, some disparity between the beam's strength and conversion by the Macquarie Ridge were 
observed. Its nature was not investigated, but likely due to three-dimensionality of the problem 
and dissipative processes. Nevertheless, on average, the beam acquired $87 \pm 7\%$ of the 
converted 
energy. 
Such efficiency is higher than $74\%$ found in studies concerning generation of the beam radiated 
from the Hawaiian islands \citep{carter2008energetics}, since unstable higher modes were not 
accounted for in the reported here budgets. The modeled here losses are hypothesized to be because 
of numerical dissipation \citep{di2006numerical}. Discussion of the dissipation was not considered 
in any depth since the coarse resolution did not 
allow direct simulation of wave breaking and no subgrid parametrization was utilized. Additionally, 
the analysis was not intended to provide robust estimates and accounting of energy losses 
was left out. Though the intriguing dynamics of the generation invites investigation on the 
dissipative processes and nonlinear interactions.\\

The presented dynamical picture of generation at the Macquarie Ridge only further emphasized role 
of remote waves in producing time-variable records through nonlinear coupling. The numerical 
experiments 
exposed role of 
variable ocean environment in shaping those variations. And by virtue of the detailed mechanisms 
response cascaded into the variable strength of the tidal beam. It is unclear over 
what distances away from the generating ridge signature of such processes could be still discerned 
from interactions with mesoscale field of the open ocean. And then what level of 
intermittency should be expected at a distant observational point due to the generation alone.

\section*{Appendices}

\renewcommand{\thesubsection}{\Alph{subsection}}
\setcounter{subsection}{0}
\subsection{Application of directional spectra to internal tidal fields}
\label{C3.app:A1}
Similar approaches were used previously in internal tide field programs by 
\cite{hendry1977observations, lozovatsky2003spatial}\footnote{They were based on array 
beamforming method and stationarity of the field} or satellite altimetry by \cite{dushaw2002mapping}, 
or in surface wave studies \citep{longuet1961observations, munk1963directional, long1986inverse}. 
Let pressure of complicated wave field to be described by an angular spectrum
\begin{equation}
\label{C1:eq.spectrum}
p(\vec{r}, t) = \int_0^{2\pi}  d\theta_k S(\theta_k) e^{i \vec{k}(\theta_k) \cdot \vec{r} + 
	\phi(\theta_k) - i \omega t}
\end{equation}
Here each monochromatic wave of wavenumber $k$ travels in direction $\theta$ with 
energy $S(\theta)^2 d\theta$ and phase $\phi(\theta)$. Now time dependence is dropped and all 
physical quantities are given by corresponding complex amplitudes. The statement can be 
reformulated in terms of Fourier transform \citep{munk1963directional}. Recall Jacobi-Anger 
expansion,
\begin{equation}
p(r, \theta) = e^{i \vec{k}(\theta) \cdot \vec{r}} = \sum_{m = -\infty}^{m = \infty} i^{m} J_{m}(k 
r) e^{im(\theta - \theta_k)}
\end{equation}
that shows a field at point $(r, \theta)$ produced by plane wave can be expanded in series of 
Bessel functions and circular functions. Then substitution produces
\begin{equation}
\label{C1:p.eq}
p(r, \theta) = \sum_{m=-\infty}^{m=\infty} J_m(kr) e^{im(\theta + \pi/2)} \Big\{ \int_0^{2\pi}  
d\theta_k 
S(\theta_k) 
e^{i\phi(\theta_k)} e^{-im\theta_k} \Big\}
\end{equation}
Term in brackets represent convolution integrals defining circular Fourier coefficients of 
order $m,~A_m - i B_m$. Series \eqref{C1:p.eq} state an inverse problem that seeks the 
unknown coefficients from the known, measured pressure field that is sampled at a set of points 
$(r_i, \theta_i)$ and if infinite series is truncated at some order $N$. Real and imaginary parts 
constitute two separate problems.\\
The same steps are undertaken for velocities as plane wave polarization relations 
\citep[e.g.,][]{muller2000scattering} are utilized. The following is found
\begin{align}
\label{C1:uv.eq}
\begin{Bmatrix}
u \\ v
\end{Bmatrix}
= \frac{1}{2} \sum_{m = -N}^{m = N} J_{m} (kr) e^{im(\theta + \pi/2)}
\begin{Bmatrix}
(\omega - f) A_{m + 1} + (\omega + f) A_{m - 1} - i [(\omega - f) B_{m + 1} + (\omega + f) B_{m - 
	1}] \\ 
(\omega - f) B_{m + 1} - (\omega + f) B_{m - 1} + i [ (\omega - f) A_{m + 1} - (\omega + f) A_{m - 
	1}]
\end{Bmatrix}
\end{align}
The dependence of currents on wave bearing causes splitting of Fourier coefficients. To describe 
velocity field higher circular harmonics have to be used. Or physically, currents have higher 
spatial wavenumber. As well \eqref{C1:uv.eq} emphasizes an asymmetry between clockwise 
and counterclockwise motions due to the Coriolis effect. Now formally the inverse problem can be stated 
\begin{equation}
y = K x
\end{equation}
where $y$ is comprised of measured dynamical variables, $K$ combines bracketed terms in 
\eqref{C1:p.eq} and \eqref{C1:uv.eq} and $x$ are the unknown Fourier coefficients. Generally, the 
problem is over-determined (i.e., number of unknowns is smaller than number of 
measurements) and unstable to small errors in data, so the model yields physically inconsistent 
results. This is circumvented by seeking a damped least square solution \citep{munk2009ocean} where 
a minimization function is 
\begin{equation}
\label{C1:Tikh_prob}
J = ||K x - y||^2_2 + \alpha ||x||^2_2
\end{equation}
The unknown regularization parameter $\alpha$ acts as a high-pass filter in a singular value 
decomposition of $K$ \citep{bennett1992inverse}. In field studies this is usually set by a 
signal-to-noise ratio \citep{munk2009ocean}, since the parameter scales noise variance (residue) 
to actual signal's strength. To obtain $\alpha$ in data-driven way a straightforward approach is 
adapted that based on 
trade-off curve method \citep{hansen1993use}. In \eqref{C1:Tikh_prob} amount of allowed error 
competes with solution's variance. An optimal parameter is said to balance these factors. This is 
seen as a rapid change in behavior of a curve that relate residue with model's norm as amount of regularization varies. In most cases the curve has a sharp corner connecting aforementioned limits, hence, the 
method's name is a L-curve \citep{hansen1999curve}. And the corner is thought to occur for an 
optimal regularization parameter.\\
In the presented here analysis the equations \eqref{C1:p.eq} and \eqref{C1:uv.eq} are sampled at locations along concentric 
arrays placed at $\lambda,~0.5\lambda,~0.25\lambda$ where $\lambda$ is mode-1 wavelength found in the central point. This clearly assumes uniformity of the wave field over sampling window. The footprint could be changed to accommodate some spatial variability, but only in expanse of resolving power. Further, at each location $u,~v,~p$ from the numerical simulations are used as input data and for the central point of the array Fourier coefficients are found. The results are then used in reconstructions over an angular sector to obtain elemental components in the respective grid point.

\subsection{Derivation of conversion from the first principles}
\label{C3.app:B}
While \eqref{C3.eq:conv} provides a convenient way to quantify energy transformations, it does 
not have much room for physical interpretation. Such as in complicated situation when along with 
local baroclinic tide production a remote signal is present. The resultant perturbation of bottom 
pressure might lead to rather ambiguous result of internal tide destruction. It is said that such 
regime is to occur when a phase difference between $w_{bt}$ and $p_{1}$ is in range of 
$(\pi/2,~3\pi/2)$. This is understood as an internal tide performing work against barotropic 
forcing. Unfortunately, this statement does not directly follow from \eqref{C3.eq:conv}. Hence, to 
provide cleaner physical picture, let derive the expression for conversion rate from the first 
principles. Body force of \citep{baines1982internal}, $F_B$ is performing work by displacing 
isopycnal 
surfaces throughout a water column,
\begin{equation}
\label{C3:eq.convd1}
C(z) = \frac{dW(z)}{dt} = F_{B} w_{bc} = \frac{N^2 (-\vec{u} \cdot \nabla h) z}{i \omega h} 
\frac{d 
\xi}{dt} = \frac{w_{bt}}{i \omega h} \frac{z d(-b)}{dt} = -\frac{1}{h} w_{bt} zb
\end{equation}
where isopycnal displacements $\xi$ were changed to buoyancy, $b = -N^2 \xi$ and temporal 
variation 
was assumed to be harmonic, $\sim e^{i \omega t}$. At the final step, integration by parts can be 
employed as,
\begin{equation}
\label{C3:eq.convd2}
\int_{-h}^{0} z b dz = \int_{-h}^{0} z d \big( \int^0_{z} b dz \big) = \big( z \int^0_{z} b dz 
\big)\big|_{-h}^0 - \int_{-h}^{0} dz \int_{z}^{0} b dz^{\prime} = h (\int_{-h}^{0}b dz - 
\frac{1}{h} \int_{-h}^{0} dz \int_{z}^{0} b dz^{\prime})
\end{equation}
The last expression is a bottom pressure perturbation. Noteworthy, the baroclinicity condition was 
not employed. Combining \eqref{C3:eq.convd1} and \eqref{C3:eq.convd2}, familiar result 
for conversion rate is obtained. Note that similar approach to \eqref{C3:eq.convd1} was used 
by \citep{nash2006structure} to estimate an upper limit on emitted energy.\\

\newpage
\section{Figures}

\begin{figure}
	\centering
	\includegraphics[scale = 
	\SCALET]{/home/dmitry/Work/Research/thesis/FINALE/P3_ITS_GENERATION/figures/fig_1_bathy_geo.png}
	\caption{Domain of numerical simulations with geographical locations used throughout the text. 
	By pink line it is given mean position of two fronts in HYCOM simulations.}
	\label{C3:fig:geo.map}
\end{figure}

\begin{figure}
	\centering
	\includegraphics[scale = 
	\SCALET]{/home/dmitry/Work/Research/thesis/FINALE/P3_ITS_GENERATION/figures/fig_2_BT_tide.png}
	\caption{Comparison of $M_2$ sea level oscillations simulated by ROMS (A) with 
	TPXO-model (B). Shading shows distribution of the tidal amplitude. Progression of the wave is 
	illustrated by black lines of constant phase (in degrees).}
	\label{C3.fig:BT}
\end{figure}

\begin{figure}
	\centering
	\includegraphics[scale = 
	\SCALET]{/home/dmitry/Work/Research/thesis/FINALE/P3_ITS_GENERATION/figures/fig_3_uni_flux.png}
	\caption{The internal tidal beam in Tasman Sea is showing by mode-1 energy fluxes. Major 
	generation sites identified by underlying shading.}
	\label{C3.fig:beam}
\end{figure}

\begin{figure}
	\centering
	\includegraphics[scale = 
	\SCALEO]{/home/dmitry/Work/Research/thesis/FINALE/P3_ITS_GENERATION/figures/fig_4_gen.png}
	\caption{Generation of internal tides at the Macquarie Ridge. (A) Slope's criticality and 
	barotropic forcing given by current ellipses. (B) The diagnosed conversion rates in the 
	'uniform' experiment. The boxes outline regions used in (C) and further calculations. (C) 
	Variability of conversion rates in the three regions. Geographical change (abscissa) is given 
	by weighted-mean latitude of the conversion in corresponding averaging box. Results for 
	experiments discussed in the text are connected by lines. Grey ellipses mark 1 standard 
	deviation.}
	\label{C3.fig:gen}
\end{figure}

\begin{figure}
	\centering
	\includegraphics[scale = 
\SCALEO]{/home/dmitry/Work/Research/thesis/FINALE/P3_ITS_GENERATION/figures/fig_5_stand_wave_2by2.png}
	\caption{Standing wave between the Macquarie Ridge and the Campbell Plateau. (A) Distribution 
	of HKE in 'uniform' simulation with superposed energy flux in the total mode-1 field. (B-C) 
	Energy flux map of the westward component and the eastward component, respectively. (D) 
	Variation in energy characteristics of mode-1 field. The integrated flux was obtained for a 	
	blue transect in (C) and is given by arrows. Dots are for spatially integrated conversion rates 
	over boxes given on \fignml[C3.fig:gen]{B}. In the upward direction it is given values for 
	westward integrated flux and Auckland Escarpment and downward - Eastward flux and leeward 
	(eastern) side generation by the Macquarie Ridge. The continuous line connects results for the 
	experiment covering TTIDE field period.}
	\label{C3.fig:stand_wave}
\end{figure}

\begin{figure}
	\centering
	\includegraphics[scale = 
	1.25]{/home/dmitry/Work/Research/thesis/FINALE/P3_ITS_GENERATION/figures/fig_6_2d_section_sill.png}
	\caption{(A-D) Distribution of baroclinic pressure anomaly with superposed isopycnal 
	displacements given by contour lines during ebb and slack tidal stages in the two experiments. 
	Positive (red) color is assigned to lifted interfaces and negative (blue) - for the opposite. 
	Insets between (A) and (C), (B) and (D) provide vertical component of the barotropic current 
	during the considered time moments in the experiments (blue - \sqm{2014} and red - 
	\sqm{d10-15}). The west and east sides of the Macquarie Ridge are 
	designated by vertical solid lines used throughout this figure. (E) Evolution of the conversion 
	for both experiments at ridge sides. (F) Along-section distribution of the period-averaged 
	conversion rate along. The line color is the same as for the insets with the current.}
	\label{C3.fig:gen_2d}
\end{figure}

\begin{figure}
	\centering
	\includegraphics[scale = 
	1.5]{/home/dmitry/Work/Research/thesis/FINALE/P3_ITS_GENERATION/figures/fig_7_phase_distr.png}
	\caption{Phase of mode-1 baroclinic pressure at the ocean bottom along the section previously 
	considered on \fignm{C3.fig:gen_2d}. Solid line gives the total wave signal. Dashed and dotted 
	lines provide the result of the directional decomposition, i.e. the westward and eastward 
	traveling waves	respectively. All phases are referenced relative to the barotropic current on 
	the west side of the Macquarie Ridge. Horizontal dashed lines represent -180, -90, 0 and 
	90$^\circ$ phase lags. The east and west sides of the Macquarie Ridge are identified by 
	vertical solid lines.}
\label{C3.fig:2d_phase}
\end{figure}

\begin{figure}
	\centering
	\includegraphics[scale = 
	\SCALEO]{/home/dmitry/Work/Research/thesis/FINALE/P3_ITS_GENERATION/figures/fig_7_flux_examp.png}
	\caption{Energy and phase distribution in the decomposed wave field. The upper row shows the 
	westward component and bottom row - eastward component. Respective energy fluxes are shown by 
	black arrows. And distribution of phase is given by blue lines separated by 
	$50^{\circ},~1.75~h$. To emphasize differences isochrons for $90^{\circ}$ and $180^{\circ}$ are 
	designated by red lines. Note that the components are referenced relative to phase of the 
	barotropic tide at arrival topography, i.e. the westward component  - relative to the 
	Macquarie Ridge and the eastward component - to the Auckland Escarpment. On the upper row 
	background shading shows distribution of conversion rates and on the lower row - mode-1 
	eigenspeed anomaly.}
	\label{C3.fig:wv_fld_dist}
\end{figure}

\begin{figure}
	\centering
	\includegraphics[scale = 
	\SCALEO]{/home/dmitry/Work/Research/thesis/FINALE/P3_ITS_GENERATION/figures/fig_8_ocean_conds.png}
	\caption{Sea surface salinity in the considered experiments (A-C). Lower panels show 
	vertical water column Brunt-Vaisala frequency, temperature and salinity at locations 1 and 2 
	marked on (A-C). The corresponding eigenspeeds of mode-1 are given.}
	\label{C3.fig:ocean_cond}
\end{figure}

\begin{figure}
	\centering
	\includegraphics[scale = 
	\SCALET]{/home/dmitry/Work/Research/thesis/FINALE/P3_ITS_GENERATION/figures/fig_9_ener_relations.png}
	\caption{Panels A and B show temporal variability in the conversion (filled green), the beam 
	intensity 	(empty green diamonds), the westward incident flux (black dots) for the all 
	Macquarie Ridge and the gap respectively. Variability of transmission coefficient is superposed 
	with purple shaded panel. The variability is given for the analytically found coefficients 
	(empty diamonds) and the decomposition analysis (filled dots). On the panel B it is also 
	plotted change in the phase of the wave incident on the gap in the red shaded panel. Results 
	for the 
	TTIDE-simulation are connected by solid lines and the others by dashed lines. Panels C and D 
	illustrate dependence of the ROMS calculated conversion for the Macquarie Ridge (y-axis) on 
	either the semi-analytical calculation for the knife-edge barrier or the estimate for 
	energy transmitted across the ridge along (x-axis). On both panels colors show change in 
	environmental parameter (WKB-height) and respective values are given on the legends.}
	\label{C3.fig:ener_var}
\end{figure}

\begin{figure}
	\centering
	\includegraphics[scale = 
	\SCALEO]{/home/dmitry/Work/Research/thesis/FINALE/P3_ITS_GENERATION/figures/fig_10_var_gen.png}
	\caption{Deviation in the conversion due to variations in barotroptic vertical velocity (A), 
	baroclinic mode-1 pressure (B) and phase lag term (C). The spatially integrated values over the 
	gap only and all ridge are also presented.}
	\label{C3.fig:var_sp}
\end{figure}

\bibliographystyle{apa}
\bibliography{/home/dmitry/Bibtex_lib/my_first_lib}

\end{document}