\documentclass[12pt]{article}
\input{/home/dmitry/Work/Research/thesis/FINALE/settings.tex}
%\doublespacing
%\graphicspath{{/home/dmitry/Work/Research/thesis/FINALE/P3_ITS_GENERATION/figures/}}

%\documentclass[PhD_Thesis.tex]{subfiles}

\begin{document}
	\iftoggle{only_Chapter} {
		\title{Generation of internal tidal beam in the Tasman Sea}
		\maketitle
	}

\section*{Abstract}
The Macquarie Ridge south of New Zealand is a moderate generator of internal tidal waves (ITs) 
forming 
a tidal beam. The beam was subject to study of TTIDE/TBEAM field program. Here, beam's generation 
and characteristics are investigated by means of numerical experiments with prescribed different 
mesoscale conditions. At the strongest site, conversion of barotropic tidal produce on average 
1.6 GW of baroclinic mode-1 and with a range of $30~\%$. This variation is mainly associated with 
amplitude of remote ITs originating on slopes of Campbell Plateau. The two almost parallel 
conversion sites form a system similar to a semi-enclosed resonator. Its efficiency of energy 
extraction is shown to depend on local stratification by (a) changing WKB-scaled topography and (b) 
phase lag between the sites. The produced baroclinic waves are radiated into deep ocean as a 
spatially tight beam. Obtained here far-field energy characteristics ($152^{\circ}$ and 4 kW/m ) 
are in a good agreement with previously reported altimeter and in-situ observations. Though, the 
numerical experiments points to spatial variation. These modulations arise from interference with 
intricate wave field setting in due to reflection from Tasmania and multiple generators found 
throughout Tasman Basin. Among experiments variability arises both due to spatial changes of 
production hot spots along Macquarie Ridge and interaction with mesoscale eddies found near 
Tasmania. Effect of the former process is estimated by a semi-analytical model. Besides a direct 
dependence of baroclinic tide amplitude on conversion magnitude, its spatial distribution can 
modulate beam heading by $\pm 3^{\circ}$ which results in far-field position shifts. (No ending, no 
strong statement so far).
Macquarie Ridge, Tasman Basin where the, where not.\\
%This might lead to large variation of beam's magnitude. In the far-field beam's can also be 
%refractive by eddy field. All of these processes are discussed in terms of the field observations. 

\section{Introduction}
Baroclinic semidiurnal tides originate as a strong barotropic flow along topography forces heaving 
of isopynal surfaces. This process renders scattering of barotropic tidal energy into baroclinic 
motions \citep{hendershott1981long}. The so dispersed energy constitutes a third of global budget 
for lunar semidiurnal constituent \citep{egbert2000significant, munk1997once} and contributes 
significantly to internal wave climate (ref?). At most conversion sites because of highly inclined 
slopes, internal waves of tidal period (internal tides) are radiated as low baroclinic mode. Due to 
their large length scales decay rates are subtle. This makes low mode tide efficient in carrying 
baroclinic energy (kinetic?) over distances comparable to size of ocean basins. While generation 
sites were identified \citep{morozov1995semidiurnal, simmons2004internal, arbic2010concurrent, 
zhao2016global}, some 
were studied in detail \citep{rudnick2003tides, klymak2011breaking, althaus2003internal} and 
analytical models have been developed \citep{garrett2007internal}, little is known on how fast and 
where internal tidal energy is dissipated (deposited). A lot of uncertainty arises because of close 
relation between the waves and the dynamical oceanic medium, so the wave field is subject to 
continuous change.\\
Water column stratification directly impacts internal wave dispersion. In fact, analytical models 
of generation emphasize a ratio between angle of internal wave characteristics to bathymetric slope 
\citep{garrett2007internal} along with a height of topography as primary quantities in setting 
conversion levels \citep{llewellyn2003tidal, petrelis2006tidal}. For tall, steeply inclined 
submarine ridges the energy transfer approaches an upper theoretical limit 
\citep{petrelis2006tidal, st2003generation} making them to be ``oases" of barotropic tide 
scattering 
and internal tide production \citep{morozov1995semidiurnal, egbert2000significant}. Clearly, the 
energy rates can be modulated by changing buoyancy frequency \citep{holloway1999internal}, 
especially when seasonal transformation of water properties happens at same depths as the steepest 
bottom gradients 
\citep{gerkema2004internal}. Nevertheless, in later field studies it was realized that presence of 
external baroclinic tidal signal leads to even larger temporal variability \citep{Kelly2010a, 
zilberman2011incoherent, pickering2015structure}. This might occur as opposite ridge slopes 
affect each other \citep{nash2004internal, zilberman2011incoherent, echeverri2010internal} or due 
to spatially inhomogeneous distribution of production hotspots \citep{osborne2011spatial, 
ponte2013coastal}\footnote{there should a verb here, needs to be restructured}, or as separate 
topographic features mediate each others generation energy levels \citep{xing1998three, 
buijsman2012modeling, buijsman2014three}. This study addresses temporal and spatial variability of 
tide production happening at the Macquarie Ridge, south from New Zealand. Quick recourse to a map 
of 
the 
Tasman Sea \fignmp{C3:fig:geo.map} suggests that location of major sea bottom features leads to 
complex internal tide regime representative both of Kaena Ridge and Luzon Strait.\\
the Macquarie Ridge emits energy forming a spatially confined beam \citep{simmons2004internal, 
zhao2016global}. This is an ubiquitous characteristic of low mode internal tide propagation in the 
deep ocean that is thought to be a result of multiple source interference 
\citep{rainville2010interference}. The Tasman beam carries away most of the 
conversed energy and partly deposits it on the Tasmanian continental slope found $\sim 1000~km$ 
from 
the ridge. To detail contributing (concurrent) physical processes several field experiments 
(TBEAM/TTIDE/Tshelf) were conducted \citep{pinkel2015breaking} along with an investigation of 
satellite altimetry observations \citep{zhao2018satellite}. The latter results were favorably 
compared to averaged in-situ measurements \citep{waterhouse2018observations} corroborating 
(existence of?) northwesterly propagating low mode beam of small decay rate. Nevertheless, the 
observed temporal variability of the beam's heading and amplitude needs an interpretation to 
restrain boundary conditions for a problem of shoaling (scattering and reflection?) internal tide 
on three dimensional topography and consequent energy dissipation \citep{klymak2016reflection}.\\
The non-stationary behavior in propagation of the baroclinic tidal waves results from interaction 
with varying oceanographic conditions \citep[e.g.,][]{mooers1975several}. Depending on involved 
length scales and magnitudes, different regimes can be realized \citep{buhler2014waves}\footnote{I 
didn't read Buhler, 2014 book, just he discusses 
in great detail the topic}. On the first order, when the (wave-flow) scales are largely detached as 
in  
geometric optics limit, the oceanic conditions simply change mode-1 phase speed and cause wave 
front refraction \citep{rainville2006propagation, zaron2014time, kelly2016internal}. The phenomena 
is augmented in presence of considerable mean flows with vertical structure 
\citep{park2006internal, buijsman2017eq}. This can further produce non negligible Doppler 
shifting 
\citep{chavanne2010surface} and shifts of apparent wave frequency in strong vortical flows 
\citep{kunze1985near}. In the higher orders, nonlinear interactions lead to scattering into 
high modes \citep{dunphy2014focusing}, directional spreading \citep{wagner2017asymptotic, 
dunphy2017low} and nonintuitive energy transfers via resonant triad interactions with geostrophic 
turbulence \citep{ward2010scattering}. Still in typical oceanographic setting the first order 
mechanisms are the most widespread \citep{kelly2016internal, zaron2014time}. The latter 
work as well investigated role of generation in the producing time variable far-field. This was  
 hypothesized by \citep{wunsch1975internal} who suggested that ``energetic beams will be moved 
comparatively large distances by small changes in angle and may be missed by isolated instruments" 
\footnote{this is a direct quote, marks are ok?}.\\
In the setting of the Tasman Sea both phenomena are plausible reasons to produce the documented 
variation 
in incidence of the low mode tidal beam on the Tasman continental slope. To identify 
cause-and-effect 
relationship numerical experiments are carried out with different conditions of the oceanic medium 
(Section 2). Variable levels of internal tide production are examined in Section 3a and 
resultant beam's characteristics are quantified in its traverse of the Tasman Basin (Section 3b). 
These results are brought together to be studied in terms of a semi-analytical generation 
model and action of mesoscale (Section 4a). This helps to provide context for the field 
observations (Section 4b). This follows by conclusions. And in the Appendices mathematical nuances 
are described in greater detail.

\newpage

\section{Numerical experiments and analysis}
\subsection{Numerical experiments}
To study variability of internal tide generation around New Zealand and its propagation numerical 
simulations were performed with Regional Ocean Modeling System \citep{shchepetkin2005regional}. 
The numerical domain covered the southern Tasman Sea from subantractic waters of $60^{\circ}$ S 
to subtropics in $35^{\circ}$ S. And the zonal extent stretched from $142^{\circ}$ to $172^{\circ}$ 
E. This ensued correct representation of reach regional oceanographic conditions. The horizontal 
grid spacing was taken to be of $1/32^{\circ}$ corresponding on average to discretization of 3 km 
in zonal direction and 2.5 km in meridional. The nonuniformly separated, vertical 50 $s$-levels 
were placed to smoothly follow subsurface terrain.\\
\begin{figure}
	\centering
	\includegraphics[scale = 0.35]{../figures/fig_1.png}
	\caption{Domain of numerical simulations with geographical locations used in the text}
	\label{C3:fig:geo.map}
\end{figure}
Such discretization of vertical momentum equation tends to induce artificial, horizontal  
along-slope flows \citep{haidvogel1999numerical} due to errors in reproducing of pressure 
gradient force. Especially severe errors are made by steep terrain. The misbehavior is usually 
resolved by artificial smoothing of topography. This procedure additionally increases numerical 
stability, but has an adverse effect on internal tide generation \citep{di2006numerical} since 
primary production sites are collocated with large topographic gradients. To test the numerical 
setup, a sensitivity study was carried out with simulations of 
$1/8^{\circ},~1/16^{\circ},~1/64^{\circ}$ horizontal resolution. The essential for this study 
internal tide behavior manifested at $1/16^{\circ}$ and converged for $1/32^{\circ}$ and 
$1/64^{\circ}$ cases. There no marked differences were observed, except a substantial increase in 
high mode content which is in line with \footnote{previous investigations} 
\citep{di2006numerical}.\\
(bad transition)This work addresses the gravest baroclinic mode dynamics in the deep ocean. Spatial 
extent of 
waves is large compared to associated vertical displacements. This ensures linear regime of 
propagation without dispersive and nonhydrostatic effects taken place such as fission into 
solitons. A hydrostatic solver used in ROMS seems to be a proper choice for the simulations. Such 
simplification in wave dynamics was assumed in previous studies \citep{carter2008energetics, 
merrifield2001generation,  merrifield2002model, kerry2013effects}. In more dynamically accurate   
simulations of \citep{kang2012energetics, zhang2011three} the nonhydrostatic effects are found to 
be important only for internal tides in shallow waters, while for main part generation follows 
linear dynamics with vertical accelerations to have a negligible contribution.\\
The horizontal boundary conditions were imposed to be open for depth-averaged, barotropic flows 
following recommendations of \citep{marchesiello2001open}. The baroclinic fields are 
nudged to zero by linearly increased lateral viscosity and diffusivity over sponge layers. Through 
the same outer boundaries numerical simulations were forced with barotropic tide. The tidal 
currents and sea level are derived from TPXO atlas, version 7.2 \citep{egbert2002efficient} and 
prescibed as linearly interpolated volume transports. It was used only the largest semidiurnal 
constituent $M_2$. Amplitude ratio between the principal lunar and solar components are 4-to-1 
suggestive of slight open-ocean spring-neap modulation. The diurnal species are weak in the region 
except shoals east of New Zealand \citep{walters2001ocean}.\\
To investigate variations of baroclinic tide dynamics several ocean states were prescribed and 
analyzed separately. In the simplest setting lateral gradients in water properties were absent, 
while buoyancy frequency was set to be representative of the Tasman Basin. The second set of 
simulations 
was comprised to investigate interannual and interseasonal variability (Table 1). And the third 
calculation was intended to cover period of TTIDE/TBEAM/Tshelf field programs 
\citep{pinkel2015breaking}, a single experiment once initialized was left to proceed for three 
months.
\begin{table}
	\caption{Carried out numerical experiments}
	\begin{tabular}{ |p{3cm}||p{5cm}|p{5cm}|  }
		\hline
		\multicolumn{3}{|c|}{Numerical experiments used in this study} \\
		\hline
		Experiment abbreviation & Simulation period & Comments (reason?) \\
		\hline
		Uniform & ~ & No mesoscale \\
		2012 &   Jan 1st - Jan 15th, 2012 & Interannual \\
		2013 &   Jan 1st - Jan 15th, 2013 & Interannual \\
		2014 &   Jan 1st - Jan 15th, 2014 & Interannual \\
		2013\_Oct &   Oct 1st - Oct 15th, 2013 & Interseaonal \\
		2015\_Mar &   Mat 1st - Mar 15th, 2015 & Interseaonal \\
		2015\_TTIDE$^{\ast}$ &   Jan 1st - Mar 1st, 2015 & Field period \\
		\hline
		\multicolumn{3}{|l|}{\footnotesize$^{\ast}$ the results are named as respective day of 
		year over which post-analysis was performed, e.g. $d20-25$ }\\
		\hline
	\end{tabular}
	\label{ch2:table_exp}
\end{table}
The simulations with variable conditions were at first initialized with HYCOM hindcasts 
\footnote{(NAVGEM;	downloaded from hycom.org)} for respective start dates. Then during 
integration, 
along with barotropic tidal flow, time-variable, subtidal two dimensional fields 
\footnote{(vertical coordinate and along boundary coordinate)} of horizontal currents, temperature 
and salinity were imposed onto the numerical ocean on the boundaries. Conditions of air-sea 
interface obtained from MERRA-reanalysis \citep{rienecker2011merra} were as well utilized to 
prescribe realistic insolation, air temperature, EP rates and most importantly, wind stresses.

\subsection{Internal tide analysis}
As seen in \tblnm{ch2:table_exp} the simulations were carried out for 15 days or longer. The 
first 10 days were regarded as spin up of baroclinic tide generation and propagation. Roughly, it 
takes about 7 days for the mode-1 signal to travel from New Zealanda to Tasmania. After the spin-up 
period, three 
dimensional fields of velocity, temperature and salinity were sampled hourly. These 
were subject to high pass filtering with Butterworth filter of order $6$ with cut off time of $36$ 
hours to remove subtidal motions. Left-out signal was further fit in a least square sense to the 
principle semidiurnal harmonic. Then the three dimensional fields 
underwent a separation into barotropic and baroclinic signals \citep{cummins1997simulation, 
kunze2002internal, carter2008energetics}. A depth-averaged current is thought to represent a pure 
barotropic signal and any vertical deviation is attributed to a baroclinic wave,
\begin{equation}
\label{ch2:bt_bc_vel}
\vec{u}_{bt}(x,y) = \frac{1}{H} \int_{-H}^{0} \vec{u}(x,y,z)  dz,~\vec{u}_{bc}(x,y,z) =  
\vec{u}(x,y,z) - \vec{u}_{bt}(x,y)
\end{equation}
To obtain distribution of pressure, at first, a linear equation of state and respective 
\textit{TS}-fields are used to determine density perturbations. At second, via the hydrostatic 
approximation consequent vertical integration leads to total pressure field. This 
is then subject to baroclinicity condition, stating that baroclinic pressure anomaly associated 
with internal tides is taken to be a deviation from the depth-averaged,
\begin{equation}
\label{ch2:bt_bc_pres}
p(x,y,z) = \int_{-z}^{0} \rho(x,y,z) dz,~p_{bc}(x,y) = p(x,y,z) - \frac{1}{H} \int_{-H(x,y)}^{0} 
\rho(x,y,z) dz
\end{equation}
In the both expressions rigid-lid approximation is used. This is a valid assumption unless flow 
vertical accelerations are greater than acceleration due to gravity \citep{kelly2010}.\\
Vertical distribution of each dynamical variable was then decomposed into vertical modes. The 
structure functions were 
obtained from local Brunt-Vaisala frequency profiles found from time-averaged density fields. 
These were used in Sturm-Liouville problem under the hydrostatic approximation,
\begin{equation}
\frac{d}{dz}\Big( \frac{1}{N^2}  \frac{d \psi(z)}{dz}\Big) + c^{-2}_n \psi(z) 
= 0
\end{equation}
where $c_n$ is the mode phase speed in non-rotating ocean. The first 3 eigenmode structural 
function obtained per each numerical grid point were used to describe respective vertical 
distribution in the dynamical fields. And further presented results focus only on mode-1 energetics 
of its primary importance.\\
Here the low mode is characterized by rates of conversion from barotropic tide to baroclinic 
\citep{simmons2004internal, kurapov2003m} and depth-averaged energy flux that were found as
\begin{align}
\label{C3:eq.conv}
C_{bt\to 1} = -\frac{1}{2}(\cj{\vec{u}_{bt}} \cdot \nabla H) p_{1}|_{z = -H(x,y)}\\
\vec{F} = \frac{1}{2} \frac{1}{H} \cj{\vec{u}_1} p_1 \int_{-H}^{0} \psi_1(z) \psi_1(z) dz
\end{align}
The fractional coefficient appears because complex amplitudes are used in the expressions.\\
Additionally, the mode-1 internal tide field was subject to directional analysis in order to 
extract primary signals associated with the tidal beam. The analysis's details are presented in 
appendix A and summarized as follows. Dynamical fields of pressure and currents sampled over 
finite number of grid points are described here in terms of directional spectra. Its circular 
Fourier transform is then obtained as solution by weighted least squares of an ill-posed inverse 
problem. The directional analysis is 
different from widely accepted approach of \cite{zhao2010long} by two main reasons. The underlying 
model equations i.e., \eqref{C1:p.eq}, \eqref{C1:uv.eq} produce simultaneous fit to all compass 
directions, rather than a finite number of plane waves. This makes a difference in regions 
where wave diffraction prevails. Such as wave field near sites of internal tide generation or 
scattering will be comprised by angular distribution of energy confined to some diffractive lobe 
rather than a single plane wave. On the other hand, because of finite-window sampling artificial 
lobes are introduced as in Fourier analysis. The analysis artifacts lead to unambiguous 
interpretation, so judgment should be used when wave fields are back-synthesized. Secondary, 
velocity measurements are utilized along with pressure observations. The formulation provides 
additional constrain. Indeed, the velocity equations tend to lessen Gibbs phenomena. But 
more importantly, fitting of directional spectra to current and pressure observations can 
be rewritten for point-wise measurements (appendix A.2) as in a case of stationary mooring.\\

These calculations had produced a set of dynamical variables of barotropic and baroclinic 
fields in each experiment. The obtained values per experimenter hereafter will be referred as a 
realization. For instance, the longest experiment, 2015\_TTIDE had 10 realizations. To study 
variability of the system, mean values were defined as arithmetic mean,
\begin{equation}
<\bullet> = \frac{1}{N} \sum_i \bullet_i
\end{equation}
where $\bullet$ a field being averaged and $N$ is a number of experiments used. The variation 
between realizations is studied by mean deviation,
\begin{equation}
\Delta \bullet = \frac{1}{N} \sum_i (\bullet_i - <\bullet>)
\end{equation}

\subsection{Discrete Fourier Decomposition by inverse modeling} \footnote{I will move it to 
Appendix and will leave just a paragraph or two}
 Similar methods 
were used previously in internal tide field programs \citep{hendry1977observations, 
lozovatsky2003spatial} \footnote{that were based on array beamforming method and stationarity of 
the field} or 
satellite altimetry \citep{dushaw2002mapping} or in surface wave studies 
\citep{longuet1961observations, munk1963directional, long1986inverse}. Let 
mode-1 pressure in complicated seas to be described by an angular spectrum
\begin{equation}
\label{C1:eq.spectrum}
p(\vec{r}, t) = \int_0^{2\pi}  d\theta_k S(\theta_k) e^{i \vec{k}(\theta_k) \cdot \vec{r} + 
\phi(\theta_k) - i \omega t}
\end{equation}
Here each elementary (monochromatic) sine wave of wavenumber $k$ travels in direction $\theta$ with 
energy $S(\theta)^2 d\theta$ and temporal (spatial) lag of $\phi(\theta)$. The statement can be 
reformulated in terms of Fourier coefficients \citep{munk1963directional} by application of 
Jacobi-Anger expansion,
\begin{equation}
p(r, \theta) = e^{i \vec{k}(\theta) \cdot \vec{r}} = \sum_{m = -\infty}^{m = \infty} i^{m} J_{m}(k 
r) e^{im(\theta - \theta_k)}
\end{equation}
shows that a field at point $(r, \theta)$ produced by plane wave can be expanded in series of 
Bessel functions and circular functions. Then its substitution into \eqref{C1:eq.spectrum} and 
reorganization lead to
\begin{equation}
\label{C1:p.eq}
p(r, \theta) = \sum_{m=-\infty}^{m=\infty} \big[ \int_0^{2\pi}  d\theta_k S(\theta_k) 
e^{i\phi(\theta_k)} e^{-im\theta_k} \big] i^m J_m(kr) e^{im\theta}
\end{equation}
Term in brackets (square brackets) represent convolution integrals defining Fourier coefficients of 
order $m,~A_m - i B_m$. Thence, series \eqref{C1:eq.series} state a model equation to find the 
unknown coefficients from the known, measured pressure field that were sampled at a set of points 
$(r_i, \theta_i)$ and if infinite series is truncated at some order $N$. Real and imaginary parts 
will constitute two separate problems allowing deterministic definition of the spectrum.\\
The same steps are repeated but with current velocities instead. Plane wave  
polarization relations \citep[e.g.,][]{muller2000scattering} are inserted into 
\eqref{C1:eq.spectrum} and the following equations are found,
\begin{align}
\label{C1:uv.eq}
\begin{Bmatrix}
u_i \\ v_i
\end{Bmatrix}
= \frac{1}{2} \sum_{m = -N}^{m = N} J_{m} (kr_i) e^{im(\theta + \pi/2)}
\begin{Bmatrix}
(\omega - f) A_{m + 1} + (\omega + f) A_{m - 1} - i [(\omega - f) B_{m + 1} + (\omega + f) B_{m - 
1}] \\ 
(\omega - f) B_{m + 1} - (\omega + f) B_{m - 1} + i [ (\omega - f) A_{m + 1} - (\omega + f) A_{m - 
1}]
\end{Bmatrix}
\end{align}
The dependence of currents on wave bearing causes splitting of Fourier coefficients and 
asymmetry via Coriolis effect. This results points out that to describe velocity field 
higher circular harmonics have to be used. Physically, velocity field has higher spatial 
wavenumber. But in \eqref{C1:uv.eq} additionally, the asymmetry is observed 
for clockwise and counterclockwise components.\\
The inverse model combines dynamical relations of \eqref{C1:p.eq} and \eqref{C1:uv.eq} into a 
matrix equation
\begin{equation}
y = K x
\end{equation}
Generally, it is unstable to small errors in data and produce physically inconsistent results. This 
can circumvented by seeking a damped least square solution \citep{munk2009ocean} where a 
minimization function is given by
\begin{equation}
\label{C1:Tikh_prob}
J = ||K x - y||^2_2 + \alpha ||x||^2_2
\end{equation}
The unknown regularization parameters $\alpha$ acts as a high-pass filter in a singular value 
decomposition of $K$ \citep{bennett1992inverse}. In field studies this is usually set by a 
signal-to-noise ratio \citep{munk2009ocean}, since the parameter scales noise variance (residue) 
to actual signal's strength. To obtain $\alpha$ in data-driven way a straightforward approach is 
adapted that based on 
trade-off curve method \citep{hansen1993use}. In \eqref{C1:Tikh_prob} amount of allowed error 
is competing with solution's variance. An optimal parameter should balance these factors. This is 
seen as a rapid change in behavior of curve associating residue with model's norm as regularization 
varies. In most cases the curve has a sharp corner connecting aforementioned limits, hence, the 
method's name is a L-curve \citep{hansen1999curve}. And the corner is to occur for an optimal 
regularization parameter.\\
The equations \eqref{C1:p.eq} and \eqref{C1:uv.eq} are sampled at locations in a concentric 
arrays placed at $\lambda,~0.5\lambda,~0.25\lambda$ where $\lambda$ is a local mode-1 wavelength. 
At each location $u,~v,~p$ are used as data and for a region embraced by array Fourier coefficients 
are found. And these then are used in reconstructions.\\
The method used here is different from \citep{zhao2010long} for two main reasons. The model 
equations produce simultaneous fit of all the components, rather than a finite number of a single 
directed plane waves. This can make a difference in regions where diffraction is important such as 
near internal tide generation or scattering regions. And at second, velocity field is utilized 
which provides an additional constrain. Moreover, in synthetic experiments with 
\eqref{C1:Tikh_prob} where instead of $L2$-norm regularization it was used $L1$-norm, the results 
were approaching one of plane wave technique of \citep{zhao2010long}. Additionally, the proposed 
method can be utilized for a single mooring where half-space separation is necessary.

\section{Results}
\subsection{Generation of internal tidal beam}
Surface tide arrives to Southern Tasman Sea from North (\fignm{C3.fig:BT}). Its 
advancement happens in counterclockwise manner with maximum amplitude of sea level located along 
New Zealand's coast. This is a typical Kelvin wave behavior \citep{walters2001ocean}. Also the 
barotropic tide produces strong currents in shallow Bass Strait, but relatively weak anywhere else 
in the basin. The simulated sea surface tidal oscillation closely follows TPXO atlas with gross 
features well captured. Presence of baroclinic field manifests in perturbation of sea level 
magnitude and cotidal lines. In the basin this have a striking wavy character. This corresponds to 
propagation of low mode tidal wave (\fignm{C3.fig:beam}).\\
The baroclinic tidal field represents a complex pattern produced by multiple generation sites 
defined by steep topography. Primary production sites are just south of New Zealand. Here 
barotropic Kelvin wave faces prominent Macquarie Ridge stretched for 2000 km. As barotropic current 
decays away from the coastline conversion lessens as well. Nevertheless, low-mode beams are emitted 
from many locations. Major conversion happens at $49.5^{\circ}$ shedding away the strongest beam. 
This and two nearby beams were identified in altimetric observations \citep{zhao2018satellite}. 
Henceforth, analysis is concentrated on the most energetic, central beam.\\
The central beam is produced by tidal currents impinging on supercritical bathymetry 
(\fignm{C3.fig:gen}). Depth of the highest conversion is between 1000-3000 m and spatially confined 
to two seamounts that are separated with a sill. It has less inclined slopes and plays 
lesser role in production of the tidal beam. On average, this region of Macquarie Ridge converts 
1.6 GW of surface tide. This is half of production of Kaena Ridge, Hawaii 
\citep{carter2008energetics} and much less than Luzon Strait \citep{}. Pattern of conversion 
(\fignm{C3.fig:gen}) also exhibits regions of internal tide destruction produced by complex 
dynamics caused by superposition. In fact, an oppositely located Aucklands Escarpment presents an 
important 
source of 
baroclinic energy. \fignm{C3.fig:beam} clearly illustrates existence of a standing wave in 
Solanders Trough. The total field is characterized 
by a node in horizontal kinetic energy with fluxes revolving in counterclockwise direction 
\fignm{C3.fig:stand_wave} because of Southern Hemisphere. By the method proposed in 
ref-to methods, the standing wave is separated into elemental east-west directed components 
(\fignm{C3.fig:stand_wave}). On leeward side of Macquarie Ridge generation occurs at the same 
seamounts but there is a region of destruction that coincides with incidence of waves emitted by 
the escarpment. There generation has reacher structure both due to more complicated 
topography that is crisscrossed by canyons but also because of Macquarie ridge produced waves. 
Quantifying energy transfer across the trough and comparing with spatially integrated conversion 
rates points out to fact that wave energy is being recirculated by slope's supercritical 
reflection and only partly fed by barotropic field. In overall, such system is similar to Luzon 
strait where resonance conditions exist between two parallel ridges \citep{buijsman2014three}. In 
case of Macquarie Ridge and Campbell Plateau resonance is only partial since the former has a slant 
orientation of $15^{\circ}$. Though at $49.5^{\circ}$ the distance corresponds to 3/4 of mode-1 
wavelength. Such spacing is sensible to phase lags and can either lead to intensification or 
destruction of generation. This is illustrated by comparison of two simulations that presents 
dynamically different regimes of generation.\\
%While \eqref{C3:eq.conv} provides a convenient way to quantify energy transformation, it does 
%not have much room for physical interpretation. Such as in complicated situation when along with 
%local baroclinic tide production a remote signal is present, resultant perturbation of bottom 
%pressure might lead to rather ambiguous result of internal tide destruction. It is said that such 
%regime is to occur when a phase difference between $w_{bt}$ and $p_{1}$ is in range of 
%$(\pi/2,~3\pi/2)$. This is understood as an internal tide performing work against barotropic 
%forcing. Unfortunately, this statement does not directly follow from \eqref{C3:eq.conv}. Hence, to 
%provide cleaner physical picture, let derive expression for conversion rate from the first 
%principles. Body force of \citep{baines1982} is performing work by displacing isopycnal surfaces 
%throughout a water column,
%\begin{equation}
%\label{C3:eq.convd1}
%C(z) = \frac{dW(z)}{dt} = F_{B} w_{bc} = \frac{N^2 (-\vec{u} \cdot \nabla h) z}{i \omega h} 
%\frac{d 
%\xi}{dt} = \frac{w_{bt}}{i \omega h} \frac{z d(-b)}{dt} = -\frac{1}{h} w_{bt} zb
%\end{equation}
%where isopycnal displacements $\xi$ were changed to buoyancy, $b = -N^2 \xi$ and temporal 
%variation 
%was assumed to be harmonic, $\sim e^{i \omega t}$. On a final step, integration by parts can be 
%employed as,
%\begin{equation}
%\label{C3:eq.convd2}
%\int_{-h}^{0} z b dz = \int_{-h}^{0} z d \big( \int^0_{z} b dz \big) = \big( z \int^0_{z} b dz 
%\big)\big|_{-h}^0 - \int_{-h}^{0} dz \int_{z}^{0} b dz^{\prime} = h (\int_{-h}^{0}b dz - 
%\frac{1}{h} \int_{-h}^{0} dz \int_{z}^{0} b dz^{\prime})
%\end{equation}
%The last expression is a bottom pressure perturbation. Noteworthy, a baroclinicity condition was 
%not employed. Combining \eqref{C3:eq.convd1} and \eqref{C3:eq.convd2}, familiar result 
%for conversion rate is obtained. Note that a similar approach to \eqref{C3:eq.convd1} was used 
%by \citep{nash2006structure} to estimate an upper limit on emitted energy.\\
Conversion rate shows how much work was done by baroptropic tide to displace isopycnal interfaces. 
This is understood as work against buoyant forces. But it can happen that barotropic forcing will 
be oppositely directed if somewhere in the water column other forces are present. For case of 
'2014' 
simulation (\fignm{C3.fig:gen_2d}, a-b) during ebb tide on tideward side there is net energy 
conversion to baroclinic field even that along bottom an internal wave ray is developing by upward 
displaced interfaces as a result of previous tidal phase. Hence, at these location a newly 
generated internal wave does work against downward barotropic flow. And this produces negative 
conversion (\fignm{C3.fig:gen_2d}, e) at some moment. As tide turns conversion changes sign 
again. Overall, period averaged transfer is positive, i.e. surface tide losses energy. In 
the other experiment presented (d10-15, \fignm{C3.fig:gen_2d}, c-d), there is an intensification 
due 
to advancement of a mode-1 wave. In actuality, its propagation from Solanders Trough 
($165^{\circ}$) 
and over the sill is the major difference between two simulations. So in '2015'-setting similar 
along slope advancement of an internal wave ray is observed, but now due to shoaling mode-1 
conversion is positive throughout tidal cycle.\\
The contrary situation is found on leeward side ($164.5^{\circ}$) where surface tide current has the
opposite direction. For '2015' it appears that propagating mode-1 is losing energy since it does 
work against barotropical forcing by dipping isopycnals. Though in transition some energy is lost 
from surface tide. In '2014' there is a reflection of an internal wave ray as phase is advancing 
onto the sill. This coincides with upward barotropic flow that in total leads to intensification. 
In total, period averaged conversions have different signs on leeward side. Comparison of mode-1 
wave 
position in the trough at ebb tide (\fignm{C3.fig:gen_2d}, a,c) suggests difference in timing of 
the 
remote wave arrival or its advancement. This is explored on (\fignm{C3.fig:gen_2d}, g). The total 
signal shows a region of low phase change eastward of $165^{\circ}$ that corresponds to 
concentration of kinetic energy (\fignm{C.3:stand_wave}, a). Decomposition of the signal presents 
it roughly as a sum of the ridge generated waves (eastward propagation) and the escarpment 
originated (westward) waves. The actual relation will depend on relative magnitudes 
\citep{martini2007diagnosing}. 
But westward of $165^{\circ}$ there is almost a free propagation in '2015' of the ``escarpment" 
waves as the total signal closely corresponds to them. But in '2014' due to sills generation and  
strong reflection, phase difference with local baroptropic tide falls below 
$90^{\circ}$, so that there is an intensification of generation.\\
The described situation provides a several approaches to narrate variability in conversion rates. 
At first, it is clear that amount of remote energy crossing Macquarie Ridge through the sill will 
shape the overall conversion. Hence, energy incident from leeward side can provide such 
estimate. It is quantified from the numerical simulations by line-integration of energy fluxes 
through leeward side (\fignm{C3.fig:gen}). Additionally, this amount is modulated by reflection of 
the 
sill. Here it is thought as a knife-edge barrier for which reflectivity was analytically found by  
\citep{larsen1969internal} with similar investigations of \citep{klymak2013parameterizing}.  
Sill's depth is obtained as a depth jump from the trough to the sill that were WKB-scaled. The 
resultant calculation is given by (\fignm{C3.fig:gen_regr}, a) where total conversion of tideward 
side 
of Macquarie Ridge is plotted against amount of transmitted mode-1 energy. At second, efficiency 
of generation will depend on phase difference between the remote waves and local forcing. This is 
estimated from mode-1 eigenspeed for Solanders Trough. Taken distance separating Macquarie Ridge 
and Aucklands Escarpment to be about $115~km$ (average mode-1 wavelength $155~km$), the time to 
cross the trough can be found (\fignm{C3.fig:gen_regr}, b). And additional factor is environmental 
changes that are associated with overall efficiency of generation by the ridge. Again by applying 
WKB-scaling variation of ridge's depth relative to the surrounding deep ocean is found and then 
scaled to converted energy by application of theory by \citep{st2003generation}. The 
mean barotropic current of $0.03~m/s$ generation was considered to produce 
(\fignm{C3.fig:gen_regr}, 
c).\\
The linear regression for the first parameters has correlation coefficient $R^2$ slightly higher 
than 0.5, and for linear multiple variable regression it increases to $0.7$. The least dependence 
is found for stratification variability on the tideward side. Though if all three parameters 
combined $R^2$ becomes $80\%$. Hence, most variability is associated with leeward dynamics and not 
local stratification. Additionally, all three environmental parameters do not always change in 
similar fashion suggestive for different mesoscale dynamics occurring in the deep sea, over 
Macquarie Ridge and in Solanders Trough. This is not surprising since the region is affected by 
frontal zone and reach in subtidal dynamics \citep{smith2013interaction}. Further, inclusion of 
transmission coefficient increases correlation, but most variability comes from amount of energy 
traveling from Aucklands Escarpment 
(\fignm{C3.fig:stand_wave}, c). This is much harder problem to estimate since generation in that 
region has much more spotty character (\fignm{C3.fig:gen}) because of complex topography and large 
influence of eastward traveling waves which can also act either to intensify local generation or 
destroy baroclinic tides.
%So in total, generation at Macquarie Ridge is occurring in 
%complex dynamical environment that largely modulated by ocean conditions. 


%Generation of baroclinic tide primarily happnes south of New Zealand. 
%in circular fashion 
%Southern Tasman Sea represents a case of complicated of 
%internal tidal field. Even baroptropic 
%currents are 
%small and topography is not steep, there are a lot of regions surrounding the Southern Tasman sea 
%to produce internal tides (Figure 1). To name a few Southern Tasman Rise, Lord Howe Rise, Bass 
%Strait. Though the primary production site is found at Macquarie Ridge that stretches poleward 
%from 
%New Zealand for almost 2000 km. There several internal tidal beams are emitted along different 
%sections. These were previously identified in satelite altimetry \citep{zhao2016global}. The major 
%beam is produced where most of energy transfer from baroptropic tide to baroclinic field happens 
%at 
%$45^\circ$S.\\
%The steep topography and strong semidiurnal currents produce convergence of barotropic flow. This 
%primarily occurs at two seamounts and along 3000 m isobath. Here the topography is super critical 
%internal waves to be radiated in the low mode signal. The other generation is located by sill 
%connecting two seamounts. This pattern and its spatial distribution is subject to variability. As 
%it is observed there could occur a shift in position.\\
%As it is seen on Figure 1 this region has a complex internal wave field. The changes in generation 
%attributed to interaction of remote tidal waves and locally generated. This is illustrated by two 
%opposite experiments, 2014 and 2015. It is quite obvious to see marked difference in Solander 
%trough. On tidal flood stage barotropic tide is climbing up flanks of Macquarie Ridge causing 
%development of vertical velocity. At the same time, in 2015 there is a propagation of mode-1 
%internal wave which then slams generation. On the opposite, generation at 2014 shows beam pattern, 
%so there is a direct generation of internal wave. This is then emphasized by period-averaged 
%conversion. It shows different signs of conversion.\\
%To study these difference we perform directional decomposition (Figure). The total field is 
%comprised of two waves oppositely directed: one generated on lee side of Macquarie Ridge and the 
%other at Aucklands Escarpment they result in a superposed field. Rather than employing mechanism 
%of 
%resonance it is more apparent that the total field and consequent energy transfer happens 
%depending 
%on amplitudes of the mentioned waves and due to different slanting angles. In the simplest case of 
%equal ratio there is no energy transport and portion of it is directed towards.\\

\newpage
\subsection{Characteristics of the Tasman tidal beam}
Widespread energy conversion at the Macquarie Ridge produce a clearly defined internal tidal beam 
(\fignm{C3.fig:beam}). Here its energy characteristics are investigated in terms of spatial 
averages and ensemble means. The spatial averaging takes place in across beam direction 
illustrated on \fignm{C3.fig:beam}, the ensemble mean is defined as an arithmetic mean of energy 
quantities among all the experiments. The results are given on \fignm{C3.fig:beam_prms} for 
spatially-averaged orientation of the energy fluxes to represent beam's heading. Then beam's 
strength is expressed in terms of the across-beam integrated flux. And to diagnose underlying 
dynamics the final panel shows ratio of spatially-averaged horizontal kinetic energy (HKE) to 
available potential energy (APE). All 
these quantities exhibit high degree of spatial and temporal variability as regions of 
highs and lows interchange throughout the beam propagation. At first, this complexity is left 
behind by fitting a second-degree polynomial to the ensemble-means (\fignm{C3.fig:beam_prms}, 
dashed lines). The fits now emphasize large-scale, gross characteristics of the beam propagation. 
Such as by the Macquarie Ridge the beam is mainly oriented towards west, but then markedly turns by 
$25^\circ$ towards north. This transition occurs over the first $200~km$. And further, there is 
only a slight tendency to the equator. Away from the generation site, this open-ocean behavior is 
in accord with previous studies \citep{cummins2001north, rainville2006propagation, 
zhao2018satellite}. They recognized meridional decrease of Coriolis parameter as a cause for 
decline in phase speed and equatorward refraction of the internal tides. Thereby, near Tasmania the 
beam is directed northwesterly by angle of $150^\circ$.\\

The other important feature is a decay in the beam's strength. Initially, it carries 1.4 GW 
(\fignm{C3.fig:beam_prms}, (b)), but then the energy flux vectors are slowly diminishing. This 
decline is induced by two main factors. At first, geometrical spreading lessens wave energy 
crossing the constant integration transects. At second, dynamical processes such as friction, 
energy transfers to higher modes and interaction with ocean circulation introduce irreversible 
losses into the wave field. To account only for the physical mechanisms, the geometrical effects 
are removed by calculating energy flux divergence. This produced a mean estimate of $35~\%$ energy 
loss along the basin transverse and the beam brings $0.9~GW$ of mode-1 energy to Tasmania.\\

On scales comparable to mode-1 wavelength ($\sim 180~km$) the depicted gross characteristics are 
perturbed by  
notable deviations. For instance, energy flux vectors divert by $\pm 30^\circ$ following an 
undulating pattern (\fignm{C3.fig:beam_prms}, (a)). Similar spatial variability is observed in the 
other characteristics. In fact, there is a mutual relation between the alterations. Each turning of 
the 
flux vectors aligns with either growth or decline in magnitude and hence, in the integrated flux. 
At these local extrema the mechanical energy is partitioned similarly to a theoretical plane wave 
relation (\fignm{C3.fig:beam_prms}, (c)). On contrary, where horizontal kinetic energy overtakes 
available potential energy, the mean flux orientation aligns with the large-scale propagation 
pattern and the flux magnitudes are representative of the gross beam's intensity. This situation as 
well 
holds when $HKE < APE$. The regions where one form of the mechanical 
energy prevails correspond to nodes and antinodes. \footnote{In this study such terminology is used 
in more 
general sense since no pure standing waves were found.} Respectively, there baroclinic 
currents are at 
its strongest or pressure amplitudes are at maximum. Therefore, energy travels in a winding 
fashion going round a system of nodes and antinodes. This aspect of propagation is indicative of 
interference \citep{martini2007diagnosing, zhao2010long} between the internal tidal beam and 
distant baroclinic tides.\\

For further investigation the superposed wave field is decomposed by the same method of 
directional spectra (Appendix A) as before. However, this time multi-directional components are 
\footnote{back-}synthesized via integration over 4 circular quadrants. The so-obtained wave fields 
are then subject to the same procedure of spatial and temporal averaging. The results are shown on 
\fignm{C3.fig:beam_dcmp}. Here, not surprisingly, the northwestern quadrant 
\fignmlp[C3.fig:beam_dcmp]{a} contains the largest wave-field components and highlights plane-wave 
advancement of the tidal beam. Further on, this extracted component will be referred to as a 
planar. This major property is emphasized in a ratio between HKE and APE that closely follows the 
theoretical prediction. \footnote{Notable that the gross characteristics given on 
\fignm{C3.fig:beam_prms} are quite representative of the planar beam.} As well 
in-situ\footnote{apparent} group speed defined as a ratio of flux magnitude to total mechanical 
energy is in close agreement with the mode-1 dispersion relation\footnote{Minor oscillations are 
artifacts of analysis owing to finite size of fitting windows. }. Its relevance is additionally 
augmented by slight meridional increase in the group speed for the same reason as the meridional 
internal tide refraction. On average, it takes 6.8 days with the speed of 1.7 m/s to transmit 
energy 
from the 
Macquarie Ridge to Tasmania. This estimate is close to one obtained during the TBEAM study when 
``age" of the tide, i.e. lag period between observations of spring-neap modulation in barotropic 
and baroclinic fields, was found to be 6.7 days \citep{waterhouse2018observations}. Nevertheless, 
the field observations have shown spatial variations of the in-situ group speed and large 
deviations from the plane wave value. This point is discussed further as the planar beam becomes 
obscured by interference.\\

Distant baroclinic waves are radiated from multiple bathymetric features scattered around the 
Tasman Sea. The Lord Howe Rise and Gilbert Seamount \fignmp{C3:fig:geo.map} produce southwesterly 
wave 
components shown by the flux vectors on \fignml[C3.fig:beam_dcmp]{b}. They intersect the planar 
beam in the middle of the Tasman Basin. The resultant superposition is noticeable in APE field 
\fignmlp[C3.fig:beam_dcmp]{b} as undulating spatial pattern emerge. Length scale of the modulations 
will be given as wave-vector difference of the two approximately perpendicular fields, $2 
\pi/\Delta = \frac{1}{2} (\vec{k}_{beam} - \vec{k}_{SW}) = \frac{1}{2} \frac{2 \pi}{180~km} \cos 
\frac{\pi}{4} \rightarrow  \Delta = 255~km$. This is the distance between two consequent nodes or 
antinodes. On the other hand, the HKE-to-APE ratio is set by distance between a node and an 
anti-node and hence, oscillates twice faster, i.e. $\sim 130~km$. This roughly corresponds to the 
observed spatial variation of $180~km$, the estimate can be improved by considering more realistic 
angle between the two interfering wave fields. Further, as the southeastern quadrant 
\fignmlp[C3.fig:beam_dcmp]{c} is taken into account, the modulations show $\sim 90~km$ length scale 
of variation. This is a result of interference with oppositely traveling waves produced as the 
planar beam reflects from Tasmania. The partly standing wave is especially noticeable near the East 
Tasman Plateau where regions of high APE are alternating with low-level bands. The pattern sharpens 
further as baroclinic tides emitted at the South Tasman Rise are added 
\fignmlp[C3.fig:beam_dcmp]{d}. 
Now the overall field has only weak resemblance to the planar beam as small-scale spatial structure 
prevails. It is evident in antiphase relation between the given energy characteristics. Notable 
that the ratio is oscillating around the theoretical value, while the in-situ group speed becomes 
smaller its planar value as the complexity arises. In the total field information travels with 
$1.4~m/s$ resulting in 1.5 day later arrival of the spring-neap cycle to the far-field. This result 
contradicts the TBEAM observations of the age of the tide.\\

\footnote{This paragraph will undergo large changes}Overall, it was established that in the complex 
field interference has not affected 
actual wave group propagation, but on contrary the in-situ group speed can largely deviate from the 
plane-wave dispersion relation. But principles of wave mechanics \citep[e.g.,][]{Lighthill2001} or 
\citep[e.g.,][]{leblond1978preface} equate the in-situ group speed with the energy propagation 
velocity, hence, there is an inconsistency that deserves explanation. Contrary to the in-situ group 
speed, definition of the group speed by $c_g = \pder[\omega]{k}$ is based on dispersion in wave 
packet consisted of closely 
spaced harmonics such as the principle lunar and solar tidal species. Their respective wave fields 
can be largely distinct so that wave group propagation takes absolutely different pathway. This can 
be realized in multiple source wave field where not all internal tide production sites have 
pronounced spring-neap cycle. In the setting of the Tasman Sea $S_2$ barotropic wave has strong 
enough currents at the Macquarie Ridge to create far-flung internal waves. At other locations 
cross-isobath currents are weaker and so internal tide generation is less expected. Unfortunately 
satellite observations of the principle solar mode-1 tide \citep{zhao2017global} do not provide any 
conclusive arguments. So it is speculated that $S_2$ internal tide is not impacted by interference 
and the beat signal from the Macquarie Ridge is transmitted with the planar value even that 
baroclinic $M_2$ field has complicated spatial structure. Hence, the apparent group speed is 
reasonable to estimate arrival of the spring-neap beat if and only if the tidal constituents 
experience similar spatial modulation due to superposition \citep[e.g.,][]{holloway2003spring}.\\

The previous description of the ensemble-mean wave fields suggests possibility of interference to 
control some of temporal variability. This is investigated by calculation of standard deviations in 
the spatial averages. These results are presented on 
\fignm{C3.fig:beam_prms} by red (standard deviation) and thin gray lines (each realization). 
Clearly, previously outlined characteristics are preserved and stationary. Such that large 
differences in the energy partitioning between the experiments are collocated with position of 
nodes and antinodes. And as the flux vectors turn around these points, vector orientation can 
substantially vary. Nevertheless, the integrated flux alters less and differences are more uniform 
through the course of the beam propagation. These facts suggest that variance in magnitude of the 
interfering waves produce the shifts. To detail such processes the temporal variability is 
investigated in the previously obtained reconstructions \fignmp{C3.fig:beam_dcmp_cm}. Here the 
along-beam distribution of standard deviations and energy flux variance ellipses are documented as 
complexity arises by adding the components in counterclockwise manner. 
\fignml[C3.fig:beam_dcmp_cm]{a} depicts planar beam variation. It is mainly seen in magnitude 
of the wave field as a result of changes in conversion levels at the Macquarie Ridge. As the beam 
travels away variability also becomes apparent in the energy flux direction and hence, heading of 
the beam. This is observed as variance ellipses are obliquely oriented to the mean vectors. 
Overall, there is a high degree of spatial uniformity in the planar beam variability, so that 
standard deviations in the spatial averages are relatively constant 
\fignmlp[C3.fig:beam_dcmp_cm]{a, inset}. Nevertheless, deviation in the beam heading increases as 
the beam travels across the basin. These features allude to changes in generation or large-scale 
refraction by ocean media as main reasons and are discussed further in the text.\\
The planar beam and its simple temporal variability start to disappear as the distant waves are 
superposed. Inclusion of the internal tidal swell (Southwesterly\footnote{ or Northeasterly} 
waves, \fignml[C3.fig:beam_dcmp_cm]{b}) leads to appearance of nodes-antinodes and consequently,  
concentration of the energy density. This is translated into temporal variability. As an example, 
deviations in HKE are not monotonic anymore and experience jumps in magnitude over 180 km distance. 
The same holds for the variance of the energy fluxes. Tilt of the ellipses from the mean 
follows an alternating pattern of increase and decrease. This coincides with changes in the 
ensemble-mean flux magnitude. Additionally, the interference increases variation of the spatially 
averaged flux orientation. Further, the wave field and its changes 
become more elaborate as the reflected waves are added \fignmlp[C3.fig:beam_dcmp_cm]{c}. While 
previously the ellipses were still rectilinear and partly aligned with the planar beam heading, 
they now take oval shape and hence, represent more isotropic distribution. This is understood 
from emergence of the partial standing wave that reorients the flux vectors transversely to 
interfering components. Such alignment itself and any changes result in isotropic 
variability. And similar to the mean field, the temporal changes tend to concentrate 
following a strict spatial pattern highlighted by half mode-1 wavelength modulations in the 
spatially averaged 
characteristics. This property is preserved in the total field when Northeasterly swell from the 
South 
Tasman Rise is included. There is a slight increase in variability next to the East Tasman Plateau 
where the swell intersects with the main wave field \fignmlp[C3.fig:beam_dcmp]{d}. As well 
orientation of the flux vectors and the variance ellipses tend to slightly turn clockwise.\\

\footnote{The following part will also go through rewrite}Overall, 
there is a large spatial-temporal variability produced both by changes in the planar beam and in 
the remote waves. Nevertheless, the total field is characterized by the energy fluxes persistently 
directed in NW direction (except maybe the southern beam edge) and have magnitude representative of 
the planar beam. The spatial variations can lead to some ambiguity, but do not seem to obscure the 
gross characteristics of the tidal beam. On the other hand, in the given results it is difficult to 
identify reasons for temporal variability if point-wise measurements of the total field are taken. 
In the following discussion the found planar-beam variations are scrutinized in terms of generation 
and mesoscale effects and obtained ideas are then applied to the TBEAM mooring observations. 

%\footnote{It is worth noting that while 
%ensemble-mean allows a straightforward interpretation of mode-1 dynamics, it will incorporate a
%portion of variable (non-stationary) signal since any energy characteristic is a nonlinear 
%quantity 
%\citep{zaron2014time}. On contrary, for instance, a flux found from ensemble-mean pressure and 
%currents will exclusively provide a stationary part. Yet if a mode-decomposed signal is 
%considered, 
%finding the means will entail averaging of vertical basis functions. As a consequence, their 
%orthogonality will not be preserved leading to ambiguity in dynamics. Henceforth, an ordinary mean 
%over realizations of beam's energetics is considered. Performed comparison (not shown) between an 
%ensemble-mean of flux and a flux of ensemble-mean did not reveal significant differences in 
%spatial 
%structure, though magnitude of latter was roughly half smaller.}

\newpage
\section{Discussion}
The given results detailed variability in the internal tide production at the Macquarie ridge 
and in the deep-ocean characteristics of the gravest mode tidal beam. The latter inevitably 
affects internal tide reflection from the Tasmania continental slope. Hence, a special attention 
should be given for causes of such baroclinic tide variability. Additionally, this will provide a 
valuable setting for the satelite altimetry and the field observations. So this discussion is 
focused on the field period experiment. At first, the tidal beam properties are related to 
magnitude and heterogeneity of its generation. Then, importance of oceanographic conditions in the 
Tasman Sea is discriminated. And at final, the numerical results are compared with the observations.

\subsection{Analytical model for generation at a knife-edge ridge}
Effect of generation variability on the beam is assessed by an analytical model (appendix B) 
which is an extension of \citep{rainville2010interference} but with no ad-hoc parameters. 
The model solves for a wave field radiated into rotating ocean of mode-1 equivalent depth by a 
vibrating strip \citep{morse1946methods}. To evaluate its validity a complimentary numerical 
study was carried out with MITgcm model. There a knife-edge topography was placed against a 
semidiurnal barotropic tide. The forcing was prescribed as $U(t) = 0.01 \cos 
\omega_{M_2} t$ and $V(t) = 0.01 \frac{f}{\omega_{M_2}} \sin \omega_{M_2} t$ along a western  
boundary. All boundaries were of open type, but with sponges imposed to damp baroclinic fields. 
Physical parameters of the complimentary experiments are given in \tblnm{C3:tab.gen.prms}. The 
emitted baroclinic field was then diagnosed by the same methods as described before. The comparison 
between 
the numerical experiments and analytical solution are presented on \fignm{C3.fig:anlt_sol}. 
Overall, the analytical solution is well representative of spatial pattern of the radiated waves. 
Its 
salient feature is a curvature of wavefront near the ridge. The model of 
\citep{rainville2010interference} represents it by convex shape that is a reminiscence of 
assumed a single pole generation offset by an ad-hoc parameter. It is shown here that near the 
ridge wavefronts at first perturbed by diffraction and are concave, then they flatten and 
over only 
several wavelengths become akin to the single pole model. The last transition happens as an extent 
of generating ridge cannot be discerned or $kr \gg 1$. Similar effect is seen in semi-analytical 
calculations of \citep{zhang2014modeling} who summed contributions from point sources placed along 
topography.\\
The previous investigations have not incorporated rotational effect. But in presence of Coriolis 
force along-ridge transports as well become subject to no-flow boundary condition \eqref{C3.a2:bc}. 
This spatially redistributes emitted radiation leading to asymmetry \fignmlp[C3.fig:anlt_sol]{Panel 
b, c}. Such that the tideward and leeward wave fields are shifted one to another. The resulting 
conversion and flow across the ridge for the gravest mode are spatially non-uniform, then in order 
to balance the forcing barotropic flow, higher modes will be modulated as well. Hence, any 
calculation based on point-wise approach \citep[e.g.][]{st2003generation} is misleading in presence 
of rotation. As a result, wave breaking will also engrave some spatial pattern 
\citep{buhler2007instability}. These inferences have not been explored as the analytical solution 
does not incorporate inter-modal coupling, but the numerical results support non-uniform 
distribution of the high-mode energy with each mode having a distinct pattern.\\
Another effect associated with Coriolis force is observed as energy flux vectors appear to be 
slanted near the ridge \fignmlp[C3.fig:anlt_sol]{Panel c}. The oblique radiation and asymmetry are 
highly dependent on squared number of wavelengths aligned along the ridge or as a non-dimensional 
parameter $q = (k L_{ridge}/4)^2$. For the Macquarie Ridge $q \simeq (2\pi/180km \cdot 200km/4) = 
2.5$, so the third experiment was carried out to explore radiation pattern in such setting. The 
similar poleward slanting is seen in the Tasman tidal beam that was illustrated by marked turning 
in the near-field \fignmp{C3.fig:beam_prms} and was diagnosed by the angular decomposition as 
Southwestern waves \fignmp{C3.fig:beam_dcmp}. This phenomena was originally identified by 
\citep{zhao2018satellite} in the satellite-derived observations whose explanation was based 
on phase lags in forcing barotropic wave. Though such argument can not be refuted, it seems to be 
of the secondary importance since $M_2$-wave has phase difference of $\sim 10^{\circ}$ 
\fignmp{C3.fig:BT} at ends of the Macquarie Ridge.\\
Here presented explanation and analytical approach is of fundamental nature as its characteristic 
of spiraling outward waves and spatial non-uniformity was also reported in numerical experiments 
\citep{munroe2005topographic} and an analytical solution for circular seamount 
\citep{baines2007internal}. Further to highlight ubiquity of the physical mechanism the given 
analytical solution is taken to extreme of the critical latitude \fignmlp[C3.fig:anlt_sol]{Panel 
d}. In this case the constraint on the along-ridge flow is comparable to normal transport and wave 
becomes trapped by the ridge. The similar process and pattern was observed in the numerical studies 
of 
baroclinic wave trapping for diurnal tide at Oahu \citep{smith2017coastal} and for semidiurnal tide 
at a sill located in a channel at latitude slightly above the critical \citep{hughes2018tidal} 
representative of the Canadian Archipelago.\\

The analytical solution is now used to compose the Tasman tidal beam following the superposition 
principle \citep{rainville2010interference}. The Macquarie Ridge comprises multiple sources 
that constructively interfere to produce the beam \citep{klymak2016reflection}. The principle will 
be applied inversely, i.e. from the distribution of energy in the beam let obtain 
magnitude of sources each represented by a vibrating strip. So the ridge is discretized by $30~km$ 
strips that placed tangential to $3000$ m isobath. Their respective strengths are 
found by minimizing mismatch in flux vectors between synthesized analytical solution and the ROMS 
experiments. This is a nonlinear problem that is solved iteratively (computational details are 
given in Appendix A.2). For 'uniform' experiment the result is presented on \fignm{C3.fig:beam_inv} 
and can be compared with \fignm{C3.fig:beam}. The synthesized far-field structure agrees well when 
radiation from remote sources is omitted \fignmlp[C3.fig:beam_inv]{e.g., a}.  And the 
inversely-found magnitude of sources and their position have strong correspondence to the actual 
internal tide production sites. This model is then applied to study effect of variable generation. 
But it will be only considered major production site at the Macquarie Ridge 
\fignmlp[C3.fig:gen]{a}. And the analytical strips are adjusted in extent and orientation to 
represent the previously considered topographic features \fignmlp[C3.fig:gen]{c}. So that the 
spatially-averaged conversion rates can be directly substituted as $magnitude \sim 
Conversion^{1/2}$ into the analytical model. An example is given on \fignml[C3.fig:beam_inv]{b}. 
The beam's spatial finiteness is lost and the emitted field is similar to point-source generation, 
but the simplified formulation directly relates non-stationary generation to the beam 
characteristic.\\

It is considered the field-period simulation where conversion rates were growing overtime in the 
northern part of the ridge \fignmlp[C3.fig:gen_var_beam]{a}. Resulting effect on the beam will 
be diagnosed by identifying its ''center of mass",
\begin{equation}
L_{C} = \frac{\int_0^{L} dr |F(r)|r}{\int_0^{L} dr |F(r)|}
\end{equation}
where $L_{C}$ is a position of the center of mass in the cross beam direction, $L$ - length 
of an across-beam transect, $|F(r)|$ - magnitude of energy flux along the transect. This procedure 
is carried out for the Northwest quadrant spectral component and the analytical solutions. 
Comparison between two confirms that the beam has shifted its position by $15~km$ northward 
\fignmlp[C3.fig:gen_var_beam]{b} following the shift in generation pattern. An opposing example of 
southerly concentrated production was observed \fignmlp[C3.fig:gen]{b} in 'uniform' experiment and 
the resulting beam was occupying the southernmost location of all experiments. Hence, spatial 
pattern in generation shapes near-field beam properties. These effects can be only discerned over 
1-2 wavelengths - a distance prompted by the analytical model where the solutions converge to 
uniform point-wise model. In contrary, the numerical simulations become irrespective to generation 
much faster as effects of interaction with mesoscale are building up.\\
Another consequence of variable levels in generation is alterations in the beam intensity. Over the 
considered period mode-1 production increased by $30\%$ \fignmlp[C3.fig:gen_var_beam]{c}. Unlike to 
previous, this characteristic seems to be consistent throughout the whole Tasman basin. This is 
shown by considering temporal variability in the integrated flux in the mid-field and far-field 
locations. Note that the far-field series was shifted in color to accommodate a time lag necessary 
for energy propagation ($\sim 6.7~days$). The time series demonstrate similar modulation 
pattern that follows the ridge conversion. Magnitude of changes declines with distance slower 
than $1/r$ as predicted by the analytical model, though the inverse solution 
\fignmlp[C3.fig:beam_inv]{a} demonstrates similar spreading law. Hence, because of beam's linearity 
to account for the spreading mismatch the analytical results are scaled by mean intensity in the 
numerical experiments at respective locations. The result is depicted by diamonds on 
\fignml[C3.fig:gen_var_beam]{c}. Strong dependence reinforces the previous observation, while any 
departures are thought to be because of interaction with variable mesoscale circulation.

\begin{table}
	\caption{Parameters of numerical experiments for generation at knife-edge topography}
	\begin{tabular}{ |p{7cm}||p{7cm}|  }
		\hline
		Parameter & Value \\
		\hline
		$\Delta x \times \Delta y \times \Delta z$ & $8km \times 8km \times 80m$ \\
		$L \times W \times H$ & $1800km \times 1800km \times 2000m$ \\
		Knife-edge height & $800m$ \\
		$N^2,~\lambda_1$ & $1.5\cdot 10^{-5}rad/sec,~110km$ \\
		\hline
		\multicolumn{2}{|l|}{Parameter space} \\
		\hline
		Knife-edge width, $L_{ridge}$ & $240km,~240km,~160km,~240km$ \\
		Coriolis parameter, $f$ & $0,~\minus 10^{\minus4},~\minus 10^{\minus4},~\minus 0.99 
		\omega_{M_2}(\minus1.39\cdot10^{\minus4})$ \\
		\hline
	\end{tabular}
	\label{C3:tab.gen.prms}
\end{table}

\subsection{Variability of the Tasman Sea beam}
The Tasman beam is inevitably affected by mesoscale circulation in the region. 
\fignml[C3.fig:meso_examp]{a} gives an example of oceanograhpic conditions. The 
deep ocean of the Tasman Sea is intersected by a frontal zone where subtropical and subantarctic water masses collide. Their interaction lead to eddy formation. These features are relatively weak and do not posses substantial kinetic energy in comparison to Tasmania region where extension of East Australian Current might shed stronger eddies. The basin vortices have Rossby number of $Ro \sim O(0.1)$ and have an extent less than $100km$. In such setting nonlinear terms representative of tide-mesoscale interaction and energy exchange will scale as $Ro\frac{f}{\omega_{M_2}} max(1, \frac{L_{eddy}}{L_{wave}}) \sim 0.05$ following \cite{dunphy2017low}. Thus, the major phenomena will restrain to refraction with negligible weak scattering and energy transfers. Further, effects of refraction can be diagnosed as changes in phase speed in presence of background circulation \citep{zaron2014time} that follows
\begin{align}
\omega_{obs} = f_e^2 + \frac{\omega_{obs}}{\omega} gD |k|^2\\
c_{obs}^2 = \big( \frac{\omega_{obs}}{|k|} \big)^2 = \frac{\omega_{obs}}{\omega} \frac{gD}{1 - 
\frac{f_e^2}{\omega^2}}
\end{align}
where $\omega_{obs} = \omega + \vec{k}\cdot \vec{U}$ - observed, Eulerian frequency which is 
Doppler shifted intrinsic frequency, $f_e = f(1 + \frac{1}{2} \frac{\zeta}{f})$ - effective 
Coriolis frequency adjusted by rotation in vortical flow of $\zeta$, $c_0^2 = gD$ - phase speed of 
internal tide in non-rotating ocean with effective depth of $D$. The latter term will encompass 
variations in stratification. The dispersion relation was then used to obtain time-variable phase 
speeds. To do so, three dimensional currents from the numerical experiments were temporally and 
vertically averaged. And since Doppler shift depends on alignment between wave propagation and 
currents, a bulk beam orientation of $150^\circ$ was used in calculations.\\
Results \fignmlp[C3.fig:meso_examp]{b} show that major regions of variations are confined to the 
meanders. The deviations due to stratification alone on average describe 50\% of anomaly, due to 
background currents 30\% and due to vorticity - 20\%. Overall phase speed deviates by $0.2~m/s$ or 
$\sim5\%$ from the mean. Even though these changes appear to be small they can substantially divert 
beam's propagation. This is easily estimated using shallow-water gravity theory of refraction 
(citation) as
\begin{align}
\frac{\Delta \theta}{\Delta s} = \frac{1}{c} \frac{\delta c}{\Delta n}
\end{align}
where $\Delta s$ and $\Delta n$ are along-beam and cross-beam coordinates. Setting them 
equal and $\delta c/c = 10\%$ (gradient between positive and negative anomaly) produce 
reorientation of the beam by $\Delta \theta 5^{\circ}$. \footnote{These calculations were supported by ray 
tracing which results are not conclusive and not given.} This corresponds to the results presented previously (Section). Now as part of the beam refracts, it interferes with non-interacted portion leading to 
obscure patterns as illustrated on \fignml[C3.fig:meso_examp]{b}. The beam seems to be dissected in 
two parts with a portion actually driven away, while another being focused, but reoriented strongly southward.\\
Clearly, the phase speed deviations force variations in the beam propagation. These were further studied by variance 
ellipses \fignmlp[C3.fig:beam_dcmp_cm]{a}. Since the respective semi-minor axis are small, it is worth investigating only semi-major axis \fignmlp[C3.fig:var_meso]{a}. The obtained distribution is noted to be mainly associated with directional 
variance. The first strong region is located by the Macquarie Ridge in accordance with the previous 
discussion on generation. Consequent regions come in slanted manner representative of beam interference with itself such as on \fignml[C3.fig:meso_examp]{a}. It can be also interpreted as focusing of energy (dunphy). It happens after internal tide passage over areas of high phase speed variance or active mesoscale circulation \fignmlp[C3.fig:beam_dcmp_cm]{b}. The figure illustrates relation to the beam center of mass. As the beam approaches Tasmania, its location can undergo drastic changes. During the field period simulation the beam was constantly southward within $30~km$ from 'uniform'-experiment even though generation was shown to be of opposite influence. This was mainly a result of refraction by warm-core eddy \fignmlp[C3.fig:beam_dcmp_cm]{b}. Additionally, over all experiments the deviations could be much larger that position shifted in meridional direction by $\pm 30~km$.\\
The numerical experiments presented large scale variability in the beam incidence on Tasman 
continental slope as a result of interaction with mesoscale circulation. It seems that changes can 
appear on different time scales. The analysis has not shown presence of any seasonality in the beam's characteristics, but rather chaotic behavior as refractive effects accumulate as the beam transverse the Tasman basin. These 
interactions seem not to redistribute energy and only divert it since mesoscale currents are 
relatively weak and driven mainly by small frontal zone instabilities.

\subsection{Comparison with altimetry observations and TBEAM observations}
Results of the presented investigations are placed together with satellite altimeter measurements  
\citep{zhao2018satellite} and TBEAM's in-situ observations \citep{waterhouse2018observations} to 
provide comprehensive picture of the beam parameters and their variability on diverse 
spatiotemporal scales. The satellite investigations of \citep{zhao2018satellite} provide an average 
viewpoint that is obtained from multiple satellite missions covering the recent 20 years. Results 
from the numerical experiments compare favorably with observed from space internal-tide 
characteristics as seen in a summary \ref{C3:tab.prms}. Further, plan-view comparison 
\fignmp{C3.fig:cmp_sat_nexp} provides more details. Here over the first half of the Tasman Basin 
both beams are similar in position, 
travel direction and magnitude. This is primarily because of consistency in geography of internal 
tide production. This was tested by application of the inverse approach (section) to satellite-derived beam 
\fignmlp[C3.fig:cmp_sat_nexp]{Scaled by magnitude dots along the Macquarie Ridge}. The same part of 
the ridge was identified as the primary energy emitter. Yet it seems that conversion at the 
seamounts was undervalued in the numerical experiments because of performed topography smoothing. 
This led to 200-300 m deeper seamounts compare to actual bathymetry and hence, production of  
internal wave took place in less stratified waters as the seamounts were located deeper away from 
thermocline. This also suggests that variability of stratification might have higher influence on 
the internal tide conversion than reported here. On the other hand, the primary variability 
mechanism (Section 3.1) associated with passing of non-local baroclinic tides is realistic as 
Campbell Plateau emitted waves can be clearly identified in the altimeter observations.\\

Major differences are found away from the generating ridge, close to Tasmania. As previously discussed (section)
the internal tide appearance in the far-field is produced by refraction on inhomogeneities in 
oceanic media. Then any biases in the simulated dynamical state of the Tasman sea will lead to 
biases in the beam characteristics. As well small number and short temporal coverage of the 
performed experiments are by no means sufficient to derive a statistically adequate mean. These 
points are further illustrated by calculating kinetic energy of subinertial, barotropic currents in 
the Tasman Basin in the ROMS experiments and HYCOM simulations (which were used as an initial 
state estimate and boundary forcing in the ROMS experiment) \fignmp{C3.fig:KE_beam}. Overall, 
ROMS shows a good skill in reproducing prescribed ocean states. Nevertheless, salient 
feature of the time series is in a growth of the energy level happened after February 2013. Nature 
of the increase is likely related to numerical parameters of HYCOM experiments. Additional 
comparison between AVISO satellite product and the ROMS experiments led to conclusion that 
important mesoscale features were similar, but amplified and accompanied with higher 
horizontal shear. Inevitably, different strength of mesoscale currents will drive varying amount of 
refraction. On \fignm{C3.fig:KE_beam} this is depicted by diamonds representing the beam heading. Over the first period with weaker currents, the beam was only slightly diverted and hence, closer to the satellite altimeter. But after 2013 intensification the beam was refracted much higher. \fignml[C3.fig:var_meso]{b} additionally illustrates that the performed numerical experiments are biased towards poleward location. Regardless of the average position, during the field period simulation \fignmlp[C3.fig:KE_beam]{inset} it is seen a growth of kinetic energy caused by 
spinning up of an anticyclonic eddy near East Tasman Plateau. This process is consistent with 
satellite observed changes in sea surface anomaly and geostrophic currents. And hence, the 
numerical simulations can provide a valuable perspective on the TBEAM measurements.\\

\begin{table}
	\caption{Comparison of internal tidal beam characteristics}
	\begin{tabular}{ |p{4cm}||p{1.75cm}|p{1.75cm}||p{1.75cm}|p{1.75cm}||p{1.75cm}|p{1.75cm}| }
		\toprule
		\multirow{2}{*}[-3pt]{Results} & \multicolumn{2}{c}{The Macquarie ridge} & 		
		\multicolumn{2}{c}{Tasmania} & \multicolumn{2}{c}{Plane wave}\\

		\cmidrule{2-7}
		 & \scriptsize{Integrated flux [$GW$]} & \scriptsize{Mean Heading [$^\circ$]}  & 
		 \scriptsize{Integrated flux [$GW$]} &  \scriptsize{Mean Heading [$^\circ$]} & 
		 \scriptsize{Magnitude [$kW/m$]} & \scriptsize{Heading [$^\circ$]}\\
		\midrule
		Numerical simulations & $1.6 \pm 0.2$ & $163 \pm 3$ & $0.95 \pm 0.1$ & $151 \pm 4.5$ & 
		$2.9 \pm 1.1$ & $156 \pm 9$ \\
		Field period only & & & & & $2.5 \pm 0.6$ & $162 \pm 8$ \\ 
%		Field period sim. & asas & asas & asas &as \\
		Altimetry &  $1.2$ & $161$ & $0.9$ & $142$ & $3.9 \pm 2.2$ & $141 \pm 2$ \\
		TBEAM &   \textit{~} & \textit{N/A} & \textit{N/A} & \textit{N/A} & $3.4 \pm 1.4$ & $149 
		\pm 3$ \\
		\bottomrule
	\end{tabular}
	\label{C3:tab.prms}
\end{table}


The field period simulation demonstrated southward beam position with $\pm 15~km$ shifts 
and baroclinic tide intensification \fignmlp[C3.fig:gen_var_beam]. These prepositions will be 
assessed against the in-situ observations \citep{waterhouse2018observations}. The TBEAM program 
comprised a stationary, moored measurements over period of 49 days (A1 mooring) 
and short-term spatial sampling across the beam by yoyoing CTD rosette with mounted upward/downward 
looking ADCPs (F2 - F9, LADCP stations)  (see \fignm{C3.fig:cmp_sat_nexp} and 
\fignm{C3.fig:TBEAM_sp}). The 
A1 mooring revealed a convoluted, highly variable history of the baroclinic tide at a single 
location, and the LADCP stations gave a glimpse into spatial extent and structure of the beam. 
But the latter observations have to be considered with caution due to contamination by 
energetic near-inertial currents followed major storm events. More on data processing, 
uncertainties and analysis is to be found in \citep{waterhouse2018observations}. Part of their   
investigation included fitting a plane wave and an estimate on the bulk beam parameters 
\tblnm{C3:tab.prms}. All three approaches, altimetry, field measurements and numerical modeling, 
compare well and combined yield the averaged internal tidal wave of $\sim 3.4 kW/m$ incident on 
Tasmania from the direction of $150^{\circ}$. Disparities in the results are further evaluated to 
gain additional insights.\\
The total field for the gravest mode \fignml[C3.fig:TBEAM_sp]{Panel A} reveals a large contrast 
between the in-situ observations and averaged the field period experiment 
\fignmlp[C3.fig:TBEAM_sp]{Panel a, black and red arrows}. At the stations closest to ETP (F2-F5 and 
A1) the simulated energy fluxes are oriented towards south. Such spatial variability is induced by 
interference between the beam component (blue arrows) and oppositely traveling reflected waves 
(green arrows). If spectral components of Northwestern quadrant \fignmlp[C3.fig:beam_dcmp]{a} are considered alone, then 
the contrast is less pronounced. In fact, at location 
of A1 mooring site the model and measurements show excellent agreement, but it is believed to be fortuitous. Overall, the simulations have different characteristics of the interfering fields than it was 
observed in-situ. This fact is emphasized by examining a numerical experiment where the beam was 
closest to the satellite-derived one \fignmlp[C3.fig:TBEAM_sp]{Panel b}. Now the total field at 
F2-F5, A1 corresponds to the observations much better. Nevertheless, at other stations mismatch 
is still large. This is not surprising due to superposition with the waves emerging from South 
Tasman Rise. If that signal is removed, the comparison improves (not shown). These facts indicate 
the TBEAM observations were also less influenced by internal tidal swell and mainly was unidirectional. \\
\fignml[C3.fig:TBEAM_sp]{Panel a and b} emphasize variability of the total wave field due to 
alterations in incidence and reflection. This is further augmented in a partial standing wave and 
chiefly in spatial distributions of HKE and APE. On average \fignmlp[C3.fig:TBEAM_KE2PE]{Panel 
a} locations of the TBEAM stations are characterized both by an excess of kinetic 
or potential energy. Similar pattern is observed in the in-situ observations, but direct 
comparison for LADCP stations is not feasible because of contamination by near-inertial wave 
energy. But the observations at A1 furnish a time-series of the ratio HKE to APE 
\fignmlp[C3.fig:TBEAM_KE2PE]{Panel d}. Over the first period APE was much larger 
than kinetic part which is consistent with an anti-node located nearby. In the later period the 
ratio was tending towards a plane wave value, and hence, the anti-node magnitude was lessen either 
because of change in position or diminish of interference phenomena. Whilst the numerical 
simulation 
demonstrates less magnitude of variability and different temporal evolution. Nevertheless, it is 
used to exemplify standing wave behavior \fignmlp[C3.fig:TBEAM_KE2PE]{Panel b, c}. Two regimes are 
presented when the 
anti-node is relatively weak and located away from A1, and the opposite situation when presence of 
interference is stronger. Such perturbation are primarily driven by proportion and spatial 
distribution of the reflected energy. It is surprising to note that even variability involves 
mainly the anti-node strength and position, potential energy is more stationary, i.e. its relative 
change is smaller than one in kinetic part. This is simply because any change in relatively low 
levels of kinetic energy is more noticeable and further can be illustrated by a one-dimensional 
example \citep{martini2007diagnosing} where 
\begin{align}
E_{KE} \sim (1 - r)^2 + 4r \sin^2{k x},~E_{PE} \sim (1 - r)^2 + 4r \cos^2{k x}
\end{align}
Application of differential yields
\begin{align}
\frac{\delta E_{KE}}{E_{KE}}\big|_{kx = 0} = \delta r \frac{2}{1 - r},~\frac{\delta 
	E_{PE}}{E_{PE}}\big|_{kx = 0} = \delta r \frac{2}{1 + r}
\end{align}
where $\delta r$ - a small perturbation in the amplitude ratio of two interfering waves. Then at 
the anti-node locations ($kx = 0$) any changes in wave content will be amplified in the kinetic 
part. The amplification factor becomes infinitely large as $r \rightarrow 1$ and a pure standing 
wave develops. In the numerical experiments location A1 exhibited higher variability in kinetic 
energy since $\sigma_{APE}/<E_{APE}> = 33\%$ and $\sigma_{HKE}/<E_{HKE}> = 81\%$.\\
During the TBEAM program measurements of only $9\%$ of potential energy was non-stationary, whereas 
kinetic energy - $33\%$. This lends additional confidence that the A1 mooring was located 
close to the anti-node and variability in tidal energetics was caused by adjustments in 
interference pattern. Further these inferences are tested by spectral decomposition. Since the 
moored measurement provides only a point-wise observation, the used previously method (Appendix A.1) can 
not be directly applied. But if it is assumed that interfering waves do not have any phase lags or 
that cross-terms in energy quantities can be omitted, then a non-deterministic decomposition is 
applicable. It implies that the actual signals can not be back-synthesized and only magnitude or 
energy parameters could be obtained. The assumption is theoretically fulfilled in position between 
a node and an anti-node where two interfering waves are in quadrature. So A1 location makes it 
possible to deduce parameters of wave components taken part in interference.\\
The details of the non-deterministic decomposition are expressed in Appendix A.2 and an example 
is demonstrated on \fignm{C3.fig:ex_spectra}. The fit seeks only 5 circular Fourier components 
which leads to low angular resolution, only opposite traveling waves can be separated. Under more 
complicated conditions the method produces wrong estimate of direction such as in the example for the numerical 
experiment. Main energy lobe is off by $\sim 20^{\circ}$ in comparison to decomposition based on 
antenna 
(Appendix A.1). To further assess applicability, numerical simulation results from a wavelength 
window centered at A1 were divided following the ratio in HKE and APE. Then the point-wise 
decomposition was applied and compared to the antenna analysis \fignmp{C3.fig:test_spectra}. The 
results highlight importance of interference cross-terms in relation to a position in the standing 
wave. The worst case is when measurements are made near a node, so that phase lags lead to large 
bias. But unlike the theoretical implications, the best performance is found near locations close to 
an anti-node or when the ratio is less than 1. The contradiction arises due to Poincare wave 
dynamics that involve appearance of phase lags between pressure and currents via Coriolis force. 
Hence, in the current formulation the point-wise decomposition considers interference of 
non-rotating waves. Regardless this caveat, energy of the incident wave was estimated 
correctly as well as its bearing \fignmlp[C3.fig:test_spectra]{a, b}. The reflected portion on the 
other hand was underestimated, and its travel direction is never correctly diagnosed. Overall, the 
point-wise decomposition provides a conservative estimate of the incident wave field and a 
qualitative picture on the reflected wave.\\
Here the described point-wise method is used to document temporal variability over a sliding 5 day \fignmp{C3.fig:dcmp_A1}. To aid comparison the numerical simulation are also plotted. In the model and measurements amount of incident energy is similar to the total signal. And both show a growth intermittent with a short period of decline (doy 40-50). As for the numerical experiments the growth was driven in part by increase in the baroclinic-tide production at the Macquarie Ridge. But primarily variation arose from the time-variable refraction causing spatial energy redistribution seen as movement of the beam and/or its focusing (section). To estimate these contributions consider the beam to be of a Gaussian-bell shape in transect, $F(x_n) = A e^{-\frac{(x_n - a)^2}{2w^2}}$, where $x_n$ - 
across beam coordinate, $a$ - location of the beam's maximum, $w$ - width defined where intensity 
falls to 2/3 of its maximum. From the numerical experiments an average width is $70~km$ and mean 
distance to A1 is $d_{A1} = x_n - a = 90~km$. Further taking differential produces,
\begin{align}
\frac{\delta F(x_n)}{F(x_n)} = \frac{\delta A}{A} + \frac{\delta d_{A1}}{w} \frac{d_{A1}}{w} + 
\frac{\delta w}{w} \frac{d_{A1}^2}{w^2}
\end{align}
The terms represent variation in point-wise observation due to change in beam's overall magnitude, 
in position relative to the mooring, and due to width of the beam. Using values from the numerical 
experiments, average contributions are estimated as $\sim 10\%$, $\sim 50\%$ and $\sim 40\%$. \fignml[C3.fig:gen_var_beam]{d} presents how the actual parameter varies and their 
contributions in each particular case. It is apparent that overall increase is mainly associated 
with beam becoming closer to the observational point \fignmlp[C3.fig:gen_var_beam]{d, black line}. 
And then this is additionally modulated by width of the beam \fignmlp[C3.fig:gen_var_beam]{d, blue 
line}. For instance, two extreme values at the start and the end of the field-period simulation (day 10 and 55) correspond to cases of focused beam. In the first case, position and overall weak magnitude led to smaller 
flux ($1.3~kW/m$), whereas around day 55 closer position and stronger intensity produced higher value 
($4.4~kW/m$). In other instances (between 20 and 30 day of year) the different contributions are of opposing effect leading to only small alterations in the flux magnitude. Further, let 
consider an incidence angle \fignmlp[C3.fig:gen_var_beam]{d}. Its variations are closely related to 
the changes in beam's bulk parameters. This is emphasized by any southward (angle is increasing) 
deviation corresponding to beam location further away from the mooring and smaller flux magnitude. 
And opposite holds for the northward orientation.\\

Comparison of the numerical experiment with the TBEAM observations suggest an explanation for the measured evolution  \fignmlp[C3.fig:dcmp_A1]{Panels a, c}. Using the previous inferences the A1 mooring recorded shifts in position of the beam that served as a primary reason for flux magnitude decrease or increase, as well as variability in the incident angle. Mechanism is provided by refraction by an anticyclonic eddy that was present during the field period. The ROMS model had a moderate skill in simulating this mesoscale dynamics and so the point-wise obtained parameters are in close correspondence to the measurements. Though the eddy in the numerical 
experiments was situated a half degree south $\sim 70~km$ in comparison either to AVISO or HYCOM, 
its center was over $A1$ and $F5$, while from the satellite altimetry - over $F3$. Nevertheless, 
the timing was simulated correctly as most active interaction has taken place in February 2015. 
This produced in the numerical experiment a southward shift of the beam and consequently, the 
energy flux turned southward \fignmlp[C3.fig:dcmp_A1]{Panel c, dots} and later shift back by 
$10^\circ$ northward.\\

The point-wise decomposition also emphasizes that the reflected portion of the energy was weaker in the 
TBEAM measurements \fignmlp[C3.fig:dcmp_A1]{Panel b}. Following assessment of the method and result 
for the numerical simulations the reflected part in the TBEAM is 3-4 times smaller than it was 
calculated. Thence, the reflected wave was at maximum $0.5~kW/m$. It 
was mostly present during January and ended with a short burst over the first week of February.  
Later in the month the reflected wave was mostly absent. Such temporal history is strongly 
supported by changes in the KE-PE ratio that was mostly below 1 during the first period 
elucidatory of strong interference. As well presence of the reflected wave explains southward 
orientation of the flux in the TBEAM time-series between 10-20 January and small overall flux. The 
same observations hold for the burst in the first week of February. Note that the ratio 
between the incident and reflected wave will largely shape the total field appearance. Meaning that 
the variability of the incident signal is superposed on the reflected signal to produce convoluted 
history of the total signal characterized by rapid temporal changes. Further, the ratio between 
reflected and incident waves highlights the mismatch in the total field seen on 
\fignmlp[C3.fig:TBEAM_sp]{a}. Adjusting for the bias in the reflected wave by calculating linear 
regression from the assessment study \fignmlp[C3.fig:test_spectra]{a} it is obtained that during 
the TBEAM study the reflected wave constituted on average $10\%$ with maximum of $30\%$ and in case 
of the numerical experiments - $40(25)\%$ with upper limit - $60(47)\%$. In the parenthesis it is 
given estimates calculated from the decomposition based on antenna. Hence, the point-wise 
decomposition overestimates this ratio, mainly because of a small bias in the incident signal for 
which adjustment was not undertaken. Thence, the numerical experiments have several times larger 
reflected energy present at A1. Origin of this radiation located at several locations, i.e. 
Tasmania continental slope, reflected waves from the ETP and local generation at the 
Cascade seamount. The latter source constitutes the major portion. Dynamics of barotropic energy 
conversion at this location is similar to the Macquarie Ridge, i.e. phasing of the incident beam 
relative to local barotropic tide will lead to either production of the internal tide or 
destruction. However, dynamically the region is much more complex both in terms of internal tide 
dynamics as many wave field components interact with the seamount, but also due to stronger 
mesoscale currents. For example, in all the experiments with mesoscale currents there is an overall 
generation at the seamount, but in the 'uniform' experiment the internal tides are mainly 
destroyed with the reflected energy being minuscule at A1. So it seems likely that the ROMS 
experiments were not able to predict the reflected portion of the internal tide dynamics adequately at TBEAM study sites.\\

This discussion draws conclusion that information provided by the A1 mooring and LADCP stations 
were well descriptive of the beam characteristics. Nevertheless, LADCP stations seem to 
largely overestimate fluxes because of aliasing with near-inertial internal waves. The time 
variability observed at the A1 mooring appears to be a result of at least 3 processes including 
long-period adjustments in baroclinic tide production at the Macquarie Ridge, refraction by 
mesoscale eddies and variability of reflection at Tasmania and the Cascade Seamount. All of these highlight 
non-stationarity in the observations, so that energy characteristics can largely vary on 
time scales compared to the semidiurnal period. As well change in position of the beam relative to 
the mooring can further complicate interpretation of the signal. So that extrapolation from a 
single-moored observation does not necessarily produce a satisfactory estimate on the beam bulk 
parameters such as intensity. Nevertheless, performed here comparison between the numerical 
experiments and the factual observations conclusively shows that during the TBEAM observational 
period the beam was located and oriented southward then the time-averaged picture provided 
by satellite altimetry. This was emphasized by \tblnm{C3:tab.prms}. While the numerical experiment 
might overestimate importance of refractive phenomena due to stronger mesoscale feature in the 
region, it is also likely that the beam was shifting its location which is documented by energy 
flux variability. These observations are pertinent to the beam reflection from Tasman continental 
slope as they provide a necessary boundary condition for further investigations.

\section{Conclusions}
%Overall, combined discussion of the numerical experiments and field observations suggests that the 
%far field signal is affected both by generation levels at the Macquarie Ridge and mesoscale 
%conditions. The former process mainly factors magnitude of the incident beam, while mesoscale can 
%largely refract beam. The numerical experiments given here were capable to predict gross, 
%time-averaged characteristics of the beam and degree of temporal variability. 
%
%Nevertheless,  This interaction was mainly taken place in February of 
%2015 which corresponds to reorientation 
%Even though the given here result is 
%probably not correct, but from comparison of the simulated wave field 
%
%Another reason for variation is change in the reflected wave. So that increase 
%leads to additional 
%modulation. Even though results are not that trustful in the reflected part, they simple 
%retranslate change in the ratio. So that the first period before day 40 when strong reflected 
%signal was leading to developing strong interference pattern and large deviation from the plane 
%wave value. At the end of that period, decrease in reflected wave was producing more plane wave 
%value. Such variations could be related to changes in beam incidence and reflection at East Tasman 
%Plateau and Tasmania. There is a similar pattern that follows from changes. So more when incidence 
%happens strongly in westward direction, amount of reflected wave is less. And for northward there 
%is more relfective component. The total energy flux orientation in the numerical model does not 
%stand close to the observed.
%
%%provide ideas on the Tasman tidal beam behavior. The former dataset comprises highly averaged 
%%both 
%%in time and space baroclinic tide dynamics. \fignm
%
%WE intend to compare with two datasets available. Satelite altimetry emphasizes highly temporal 
%and 
%spatially smoothed observations. But field observations highly hetergoneous point-wise 
%observations.\\
%At first, satelite altimetry shows more northward located beam compared to the numerical 
%experiments. 
%Further, the phase speed deviations were superposed with the position of the beam (Fig). 
%The previous result on \fignm{C3.fig:beam_dcmp_cm} can now be discussed further. Now it is shown 
%distribution of semi major axis of variance ellipse \fignm{C3.fig:fig_12_beam_var_cp}. Here it is 
%clearly seen that primary variations are concentrated in regions downstream that are affected by 
%mesoscale. Additional region is next to the Macquarie Ridge due to generation. But further it is 
%impossible to propagation quite rapidly becomes affected by variable ocean conditions. There are 
%three regions one next to the ridge, than in the basin. And also by the ETP there is strong 
%interaction causing beam to deflect.\\
%
%
%
%This rein
%In the current 
%These processes are the most important. And 
%refraction seems to be the major player. Refraction is a well-known phenomena in surface gravity 
%waves. In respective theoretical works the directional spectrum plays a pivotal role. Here these 
%approaches are used to diagnose the spectrum. This is carried out by computing two spectral 
%moments,
%\begin{align}
%\label{C3:eq.m.spectra}
%\bar{\theta} = \arctan \frac{\int_{\pi/2}^{\pi} d \theta S \sin \theta}{\int_{\pi/2}^{\pi} 
%d \theta S \cos \theta }\\
%\sigma_\theta = 2\bigg[ 1 - \sqrt{\Big( \frac{\int_{\pi/2}^{\pi} d \theta S \sin 
%\theta}{\int_{\pi/2}^{\pi} d 
%\theta S} \Big)^2 + \Big( \frac{\int_{\pi/2}^{\pi} d \theta S \cos \theta 
%}{\int_{\pi/2}^{\pi} d \theta S}\Big)^2}  \bigg]
%\end{align}
%where $S=S(\theta)$ - directional spectrum at each grid point obtained previously. These moments 
%simply show mean direction in the spectrum and the spreading of waves around the mean. Clearly, if 
%there is any refraction than the mean direction will undergo changes. But the spectrum spreading 
%will also change as close-to-each wavenumbers will experience slightly different adjustments 
%leading either to more spreading as wave group propagates travels into a region with 
%higher phase speeds than surrounding or focusing in the opposite case.\\
%
%Over the field period simulation there was a continuous change in the beam as mesoscale field was 
%evolving. Over the first period, Januaries 10-15, 2015 (Fig), the beam underwent the most striking 
%adjustment. Clearly, some of the waves were refracted equatorward, this portion then interfered 
%with the rest producing a horizontal reorientation of the fluxes. Similar phenomena was observed 
%in 
%numerical experiments of \citep{dunphy2014focusing}. The region is characterized by northward 
%orientation of spectra and high spreading. The waves cross a complex mesoscale field consisting of 
%two closely located eddies. They have signature in the increased eigenspeed and strong currents. 
%Vorticity effects were smaller. Overall, the eigenspeed was locally increased by $~10\%$. This 
%diverted further north already more northward oriented beam.\\
%The further development are shown on Fig. where the fields are given for mid-February, 10th - 
%15th. 
%The previous discussed eddies have merged producing a feature at (155, -46.5). 
%This time it has no effect on the tidal beam. On part because of weak signature in the eigenspeed, 
%but also since the strongest currents are southward and hence, leading to advection in poleward 
%which is seen in the spectrum orientation. Nevertheless, there are two regions that cause some 
%effects. One eddy is located at (158, 48) where again northward refraction happens. Since this a 
%dipole of two oppositely oriented rotating eddies, there are strong gradients in eigenspeed. This 
%produces high spreading in the spectrum. Fluxes again show a localized horizontal reorientataion. 
%Nevertheless, due to small lengthscales this does not have a substantial effect on the beam. 
%Oppositely to the region by ETP (152, -44) here a region with large cross-beam currents, strong 
%imprint in eigenspeed. This leading to diverting portion of the beam northward.\\
%Thence, two major effects were found one when currents are in cross-beam leading to diverting the 
%beam, scattering by mesoscale eddies and refraction. 
%half-month in  
%later
%But later developments 
%lead to further 
%Over the next we 
%Combined two effects lead 
%Here it is presented two cases that shows how complex such interactions can be (Fig.). In the 
%first 
%case due to a region of abnormal 
%
%The spatial-temporal characteristics complicates direct interpretation of diagnosed energy 
%quantities in describing the tidal beam. Its properties are obscured due to 
%interference and hence, needs additional inferences. This is related to variability. It is unclear 
%what can produce beam's energy levels and its orientation/position. Two reasons could be named as 
%accumulating interaction with mesoscale field and generation producing.\\
%Now comparison of the decomposed planar beam can be made 
%with factual observations made by \citep{waterhouse2018observations, zhao2018satellite}. \\

%\newpage
%\iftoggle{only_Chapter} {
%	\appendix
%}
%
%\nottoggle{only_Chapter} {
%	\addcontentsline{toc}{section}{Appendices}
%}
%
\section*{Appendices}

\renewcommand{\thesubsection}{\Alph{subsection}}
\setcounter{subsection}{0}
\subsection{Analytical solution for generation at knife-edge ridge}
Let consider a simplified problem of the internal tide generation at a ridge that has an 
extent $a$ in zonal direction, but infinitely small width. Full solution will not be pursued, but 
rather focus is made on outlining spatial pattern of emitted waves. Under such problem statement 
ridge height is of no importance, and the topography thence is similar to a piston-like 
wavemaker that radiates the internal tide. Its dynamics in flat-bottom ocean is expressed by 
Laplace tidal equations that upon substitution of tidal temporal dependence yield,
\begin{align}
\nabla^2 p + k_n^2 p = 0\\
u = \frac{-i \omega p_x + f p_y}{\omega^2 - f^2},~v = \frac{-i \omega p_y - f p_x}{\omega^2 - f^2}
\label{C3.a2:eq.laplace}
\end{align}
here $k_n$ is a wavenumber of the emitted waves. Two boundary conditions are imposed. 
On the ridge there should be no mass transport, 
\begin{equation}
\vec{u}\cdot \vec{n}|_{ridge} = \vec{u}_{bt} \cdot \vec{n} |_{ridge}
\end{equation}
and the emitted waves should be outgoing. Since the problem is formulated in terms of pressure 
only, the boundary condition takes the following form \citep{greenspan1968theory},
\begin{equation}
\Big[ \frac{-i \omega p_y - f p_x}{\omega^2 - f^2} \Big]_{ridge} = v_{BT}|_{ridge}
\label{C3.a2:eq.bc0}
\end{equation}
The ridge boundary is described by a strip $-a/2 < x < a/2,~y=0$. The equations 
\eqref{C3.a2:eq.laplace}, \eqref{C3.a2:eq.bc0} state the simplified problem for generation of 
internal tides by the knife-edge ridge of width $a$. If 
rotational effects are omitted, the problem is identical to 
radiation of acoustic waves by a vibrating strip \citep{morse1946methods}. That solution is now 
extended to include Coriolis force. Since in Cartesian coordinates the strip boundary condition 
does not allow separation of variables, this is circumvented by employing the elliptic coordinate 
system,
\begin{align*}
x = \frac{a}{2} \cosh \mu \cos \theta,~y = \frac{a}{2} \sinh \mu \sin \theta,~\Delta = \frac{a}{2} 
(\sinh^2 \mu + \sin^2 \theta)\\
\partial_x = 
\frac{1}{\Delta} (\partial_{\mu} \sinh \mu \cos \theta - \partial_{\theta} \cosh \mu 
\sin \theta),~\partial_y = \frac{1}{\Delta} (\partial_{\mu} \cosh \mu \sin \theta + 
\partial_{\theta} \sinh \mu \cos \theta)
\end{align*}
where $\mu = const$ graphs ellipses expanding from a reference center and $\theta$ is constant 
along hyperbolas. The strip is equivalent to a degenerated ellipse with semimajor axis equal to 
$a/2$ and semiminor - to $0$. The boundary condition now will take much simpler form,
\begin{align}
\Big[ \frac{-i \omega p_{\mu} + f p_{\theta}}{\omega^2 - f^2} \Big]_{\mu = 0} = \frac{a}{2} \sin 
\theta (v_{BT})
\label{C3.a2:eq.bc}
\end{align}
Laplace operator in the elliptical coordinates has eigensolutions (''sloshing modes") in form of 
Mathieu functions \citep{gutierrez2003mathieu, mclachlan1951theory}. Then solution for the emitted 
waves is sought in such form where dependence along ellipses $\mu = const$ and along hyperbola 
$\theta = const$ is separated,
\begin{equation}
p \sim \sum_{j} [Se_j, So_j](h, \theta) \times [Je_j, Jo_j, Ye_j, Yo_j](h, \mu)
\end{equation}
with $Se_j,~So_j$ - even and odd angular Mathieu functions of order $j$ corresponding to $\cos$ and 
$\sin$ in a polar coordinate system, $Je_j,~Jo_j$ and $Ye_j,~Yo_j$ - radial Mathieu functions of 
the first and second kind of order $j$ corresponding to Bessel functions. The physical parameter 
that sets Mathieu functions behavior is $h = \frac{a k_n}{4} $. For clarity of the further 
analysis it is necessary to represent Mathieu functions by their corresponding Fourier series 
\citep{morse1946methods},
\begin{align}
Se_j(h, \theta) = \sum_n Be_n^{j}(h) \cos( n\theta),~So_j(h, \theta) = \sum_n Bo_n^{j}(h) \sin( 
n\theta)
\end{align}
The Fourier coefficients are usually found by solving recurrence relations 
\citep[e.g.,][]{stamnes1995new} and here software of \citep{erricolo2013algorithm} was employed for 
the necessary calculations. The same coefficients are stand in expansion of the radial functions, 
but Bessel functions are substituted for trigonometric functions.\\
To proceed with the solution, let RHS of the boundary condition \eqref{C3.a2:eq.bc} be expressed as 
series of odd Mathieu functions,
\begin{align}
\sin \theta = \sum_{j = 0}^{\infty} C_{2j + 1} So_{2j + 1} (\theta) \\
C_{2j + 1} = \frac{Bo^{2j + 1}_1}{Noo_{2j + 1}}
\end{align}
where normalization constant $Noo_{2j + 1} = \int_{0}^{2\pi} So_{2j + 1}^2(\theta) d \theta = 
\sum_n \pi (Bo_{2n + 1}^{2j + 1})^2$. The 
decomposition of $\sin \theta$ and structure of the ridge boundary condition suggest solution in 
the form,
\begin{equation}
p(\mu, \theta) = \sum_{j = 0}^{\infty} A_{2j+1} So_{2j + 1}(\theta) Ho_{2j + 1}(\mu) + B_{2j+1} 
Se_{2j + 1}(\theta) He_{2j + 1}(\mu)
\label{C3.a2:eq.form}
\end{equation}
Here $H\{o/e\}_{2j + 1}(\mu) = J\{o/e\}_{2j + 1}(\mu) + i Y\{o/e\}_{2j + 1}(\mu)$ are 
Hankel-Mathieu 
functions to ensure boundary condition for radiation to be outgoing. Substituting the proposed 
solution 
into \eqref{C3.a2:eq.bc},
\begin{align*}
-i\omega \sum (A So Ho^{\prime} + B Se He^{\prime}) + f \sum (A So^{\prime} Ho + B Se^{\prime} He) 
= \sum (\omega^2 - f^2) C So
\end{align*}
where prime denotes derivative with respect to appropriate variable and all radial functions are 
evaluated at $\mu = 0$. Indexes were also dropped for simplicity. At first, after multiplying the 
above equation by $So_{2m + 1}$, integrating from 0 to $2 \pi$, and after orthogonality relations 
are made use of one obtains
\begin{align}
-i \omega A_{2m + 1} Noo_{2m + 1} Ho_{2m + 1}^{\prime} + f \sum_{j} B_{2j + 1} 
Neo^{2m + 1}_{2j + 1} He_{2j + 1} = (\omega^2 - f^2) C_{2m + 1} Noo_{2m + 1}
\label{C3.a2:eq1}
\end{align}
Note the second term on LHS that appears only because of Coriolis force and tangential component of 
the flow in \eqref{C3.a2:eq.bc}, and represent ``spatial coupling" between the odd and even basis 
functions. Their inter-dependence is determined by $Neo_{2j+1}^{2m + 1} = -\pi \sum_n 
(2n + 1) Be_{2n + 1}^{2j + 1} Bo_{2n + 1}^{2m + 1}$. At second, carrying out the same steps but 
with $Se_{2m + 1}$, the following matrix equation is found
\begin{align}
-i \omega B_{2m + 1} Nee_{2m + 1} He_{2m + 1}^{\prime} - f \sum_j A_{2j + 1} {Neo}^{2m + 
1}_{2j + 1} Ho_{2j + 1} = 0
\label{C3.a2:eq2}
\end{align}
Equations \eqref{C3.a2:eq1} and \eqref{C3.a2:eq2} form a linear system to find the unknown 
coefficients. These equations are solved numerically only for the first $j_{max} = 5$ due to rapid 
convergence of the involved series.

\subsection{Inverse model}
The model closely follows ideas used in ref-to-Luc, 2010 and -Jody-2016. The internal tide 
generating ridge is given by point sources each emitting following
\begin{equation} \label{invm_eq:1}
p = p_{0} \frac{2}{\pi k d} \cdot e^{i  k  d}
\end{equation}
where $k$ - wavenumber associated with eigen mode-1, i.e. $k = \sqrt{\omega ^ 2 - f ^ 2}{c_{eigen} 
^ 2}$, $d$ - distance between a point source and an observation point. By observation points here 
and after is meant points in which observations are inverted. The given solution is a solution of 
pressure distrubance propagation for two dimensional wave equation (p. 22, Frisk) and describes 
outgoing cylindrical wave. This is a far field approximation ($kd \ll 1$), in the near source zone 
the solution is substituded by Hankel functions. Here representation is simplified and observation 
points on the distance less than wavelength are omitted. Though introduction of Hankel function 
into the inverse model does not involve any additional complexity. By pressure here is thought 
mode-1 pressure amplitude that can be connected to sea level disturbance or isopycnal 
displacements.\\
To describe energy fluxes in the observational points polarization relations for cylindrical 
Poincare wave are invoked,
\begin{align} \label{invm_eq:2}
u = \frac{p_{0}}{\rho_{const}} * \frac{-i \omega \cos(\theta) + f \sin(\theta)}{\omega ^ 2 - f ^ 
2} \cdot p_{\vec{d}}\\
v = \frac{p_{0}}{\rho_{const}} * \frac{-i \omega \sin(\theta) - f \cos(\theta)}{\omega ^ 2 - f ^ 
2} \cdot p_{\vec{d}}
\end{align}
where $p_{\vec{d}}$ is a derivative along radius-vector $\vec{d}$,
\begin{equation}
p_{\vec{d}} = (i \cdot k - \frac{1}{2 d}) p
\end{equation}
In further description of the inverse model it is used following notation, indices $i,~k$ define 
$i,~k$-th point sources, while $j$ - $j$-th observation point.\\
The tidally and depth averaged energy fluxes will be given as an interference of pointwise fields 
from all sources,
\begin{align}
F_{j}^x = \frac{1}{2} \sum_k u_{kj}^{\star} \sum_i p_{ij} \int_H^0 \psi_1(z)^2 dz\\
F_{j}^y = \frac{1}{2} \sum_k v_{kj}^{\star} \sum_i p_{ij} \int_H^0 \psi_1(z)^2 dz
\end{align}
Note different indexes for u/v and p meaning that cross multiplication is involved which leads to 
complex interference pattern. In energy flux formulation normalization coefficient associated with 
eigenmode structure function are introduced by corresponding mode-1 structure function, 
$\psi_1(z)$. Coefficient $1/2$ is used for convenience to convert actual time averaging involved 
to 
multiplication of complex numbers. In further description the constant coefficients are omitted 
due 
to their irreleveance. The previous relations can be expressed in matrix form (it is not fully 
correct for fluxes, multiplication is done term by term per point),
\begin{align}
p_j = B^p_{ji}{p_i},~u_j = B^u_{ji}{p_i},~v_j = B^v_{ji}{p_i}\\
F^x_j = (B^u_{jk}{p_k})^{\star} B^p_{ji}{p_i},~F^y_j = (B^v_{jk}{p_k})^{\star} B^p_{ji}{p_i} 
\label{invm_eq:4}
\end{align}
where tensor notation is used, i.e. summation is done over same indices. Matrices 
$B^p_{ij},~B^u_{ij},~B^v_{ij}$ are short notation for generaion model and polarization relations, 
for example,
\begin{equation}
B^p_{ji} = p_i \frac{2}{\pi k d_j} \cdot e^{i  k  d_j}
\end{equation}
These can be thought as disretization of operators transforming distribution of sources into 
interference pattern in pressure and velocity fields.\\
Apparently, the energy flux relations are non-linear. To deal with this it is proposed an 
iterative technique. Let at $m$-th iteration there is a known distribution of wave amplitude at 
sources, $p^m_i$, the total energy flux field can be reconstruced by (\ref{invm_eq:4}). Than it is 
desired to find a small adjustment $\delta p^m_i$ (``nudge factor") such that residual between 
observed field and analytical description will be decreased. One can write,
\begin{align}
F_{j}^x = (B^u_{jk}(p^m_k + \delta p^m_k))^{\star} B^p_{ji}(p^m_i + \delta p^m_i) = \nonumber\\
(B^u_{jk} p^m_k)^{\star} B^p_{ji}{p^m_i} + (B^u_{jk} \delta p^m_k)^{\star} B^p_{ji} p^m_i + 
(B^u_{jk} p^m_k)^{\star} B^p_{ji} \delta p^m_i + (B^u_{jk} \delta p^m_k)^{\star} B^p_{ji} \delta 
p^m_i\nonumber\\
F_{j}^x - (B^u_{jk} p^m_k)^{\star} B^p_{ji}{p^m_i} = (B^u_{jk} \delta p^m_k)^{\star} B^p_{ji} 
p^m_i + (B^u_{jk} p^m_k)^{\star} B^p_{ji} \delta p^m_i + (B^u_{jk} \delta p^m_k)^{\star} B^p_{ji} 
\delta p^m_i \label{invm_eq:3}
\end{align}
The left hand side of (\ref{invm_eq:3}) represents the residual, the right hand side sets a 
controlling equation to obtain adjustment neceassary to decrease the residual. The last term of 
RHS 
shows a non-linear nature of the problem. This is omitted since the purpose of conseqeunt 
iterative 
technique is to find the final source distribution such that the model equations (\ref{invm_eq:4}) 
are satisfied in least square sense. Than the ``nudge-factor" can be found as inverse of 
\begin{equation}
F_{j}^x - (B^u_{jk} p^m_k)^{\star} B^p_{ji}{p^m_i} = R_j^x = \Big[ (B^u_{jk} )^{\star} B^p_{ji} 
p^m_i + (B^u_{jk} p^m_k)^{\star} B^p_{ji} \Big] \delta p^m_i \label{invm_eq:5}
\end{equation}
(these equations are not in matrix form, but obsevation point by observation point).
Hence, the aim of inverse model is to decrease error in representation of energy fluxes. The 
equation (\ref{invm_eq:5}) can be solved separately for zonal and meridional fluxes and also 
simultaneously for both directions. That is at each iteration step the nudge-factor is found first 
for zonal, than for meridional direction and finally, for both simultaneously. At the end pressure 
distribution is changed by average from all three substeps.\\
Note the inverse model equation (\ref{invm_eq:5}) is supported by additional condition stating 
that 
amplitude is nonnegative, $p_i^m + \delta p_i^m \geq 0$. All of this numerically is solved by 
linear programming routine \text{lsei} (least square with inequality) provided by LINPACK 
package.\\
Here it will be presented a test convergence and number tests on robustness on proposed iterative 
inverse model. The initial flux field is given by Fig. \ref{invm_fig:1} where by crosses are shown 
observational points. This define prescribed $F_{j}^x$ or $F_{j}^y$. Note that the prescribed 
field 
aims to describe midbasin energy flux field with Tasman shelf ommitted due to presence of 
reflection and complex bathymetry. The point sources distribution are given by green dots and at 
the first iteration step are set to $p^0_i = 100 Pa$. The distribution of points sources is 
representative to distribution of steep bathymetry which is belived to be an internal tide 
generator. In the inverse model there are only two parameters that describe characteristic of the 
internal tide, wavenumber and normalization coefficients used in energy flux. Both are found from 
solving eigenvalue for randomly picked stratification profile. This result in wavelength of 
$180~km$ which is a representative value for Tasman Sea conditions. In the same way eigenfunctions 
are obtained and normalization coefficients are found.\\
Hence, the inverse model does not account for
\begin{enumerate}
\item Bathymetry variation
\item Stratification variability
\item Variation of barotropic tide along ridges
\end{enumerate}
The first two points are thought to have minor effect on internal tidal beam structure. While the 
third is omitted to preserve simplicity of generation model. Additional tests were done with 
variation in barotropic tide phase along ridges, but they did not bring any substantial changes in 
foregoing results.\\
To show convergence of the inverse model it is given change in pressure amplitude with each 
iteration. Here convergence is defined by
\begin{equation*}
Conv = \sum_i \frac{(p_i^m - p_i^{m-1})^2}{(0.5 \cdot (p_i^m + p_i^{m-1}))^2}
\end{equation*}
The iterative solver is stopped when convergence is reaching tolerance. Here it is set to 0.01. 
From Figure (2a) it is seen that by 17th iteration there is no appreciable change in the inverse 
solution. This means that influence of non-linear terms in (\ref{invm_eq:3}) became negligible and 
the distribution of amplitude along the source region is the best in least square sense. The error 
of such description is given on subsequent panels of Fig. 2, where root-mean-square-error for 
different energy flux parameters is defined for example zonal component as
\begin{equation}
E_{x} = \sqrt{\frac{\sum_{obs} (F_i^x - \hat{F}_i^x)^2}{N_{obs}}}
\end{equation}
As it is seen the error is approaching stability for all used components much faster than 
convergence in amplitude. Note that the error is larger in zonal fluxes. The inverse solution can 
not predict far field behavior which is believed due to interaction with East Tasman Plateau. The 
obtained solution is given by Fig. 3. Here it is found that the inverse solution can not well 
represent the beam close to East Tasman Plateau. The following reasons can be named: interaction 
with topography and inadequacy of cylindrical wave model in the far-far field. It is believed that 
the second reason is the main. In general, the inverse solution picks up the central beam pretty 
well, outlines its boundary and the major region is satisfying manner. As well note that the 
northern and southern beams are also found in the solution.

\newpage
\section{Figures}

\begin{figure}
	\centering
	\includegraphics[scale = 
	0.5]{/home/dmitry/Work/Research/thesis/FINALE/P3_ITS_GENERATION/figures/fig_1_BT_tide.png}
	\caption{Comparison of $M_2$ sea level oscillations simulated by ROMS (left panel) with 
	TPXO-model (right panel).}
	\label{C3.fig:BT}
\end{figure}

\begin{figure}
	\centering
	\includegraphics[scale = 
	0.5]{/home/dmitry/Work/Research/thesis/FINALE/P3_ITS_GENERATION/figures/fig_2_uni_flux.png}
	\caption{Beam of Tasman Sea with major internal tide production sites identified by superposed 
	heat map.}
	\label{C3.fig:beam}
\end{figure}

\begin{figure}
	\centering
	\includegraphics[scale = 
	0.5]{/home/dmitry/Work/Research/thesis/FINALE/P3_ITS_GENERATION/figures/fig_3_gen.png}
	\caption{Generation of internal tides at Macquarie Ridge. (a) The heatmap illustrates 
	criticality in the region of major generation. And ellipses are representative of barotropic 
	current. (b) The diagnosed conversion rates for 'uniform' experiment. The boxes outline regions 
	used in further analysis. (c) Variability of conversion rates in the three regions identified 
	on the previous panel.}
	\label{C3.fig:gen}
\end{figure}

\begin{figure}
	\centering
	\includegraphics[scale = 
0.5]{/home/dmitry/Work/Research/thesis/FINALE/P3_ITS_GENERATION/figures/fig_4_stand_wave_2by2.png}
	\caption{Standing wave between Macquarie Ridge and Auckland Escarpments. (a) Distribution of 
	horizonal kinetic energy in 'uniform' simulation with overlaid energy flux of total mode-1 
	field. (b) Energy flux map of the westward component and the eastward component (c). Variation 
	in energy characteristics of mode-1 field between Macquarie Ridge and Aucklands Escarpment. The 
	integrated flux was obtained for a 	dashed transect in (c) and is given by arrows. Dots show 
	conversion rates spatially integrated over boxes given on (\fignm{C3.fig:gen}, b).}
	\label{C3.fig:stand_wave}
\end{figure}

\begin{figure}
	\centering
	\includegraphics[scale = 
	0.5]{/home/dmitry/Work/Research/thesis/FINALE/P3_ITS_GENERATION/figures/fig_5_2d_section_sill.png}
	\caption{Distribution of baroclinic fields during ebb and slack tide along transect crossing 
	Macquarie Ridge on \fignm{C3.fig:gen}, (b). (a-d) Baroclinic pressure anomaly with superposed 
	isopycnal displacements given by contour lines. Positive color is assigned to lifted interfaces 
	and negative - for the opposite. Stick lines show currents. And white line is distribution 
	of barotropic velocity. The tideward and leeward sides of Macquarie Ridge are identified by 
	vertical solid and dashed lines. On (e) it is shown time progression of 
	baroptropic-to-baroclinic conversion for both experiments and ridge sides. While (f) gives 
	distribution of period averaged conversion rate along the transect. The same is for (g) where 
	progression of mode-1 baroclinic pressure is shown by phases of total field and elemental 
	components. The phases are referenced to flood current across the sill.}
	\label{C3.fig:gen_2d}
\end{figure}

\begin{figure}
	\centering
	\includegraphics[scale = 
	0.5]{/home/dmitry/Work/Research/thesis/FINALE/P3_ITS_GENERATION/figures/fig_6_scatter_generation.png}
	\caption{Variation of the tideward conversion rates in relation to mode-1 energy transmitted 
	across the sill (a), to travel time across Solanders Trough (b), to knife-edge barrier 
	representative of Macquarie Ridge seeing from the deep ocean.}
	\label{C3.fig:gen_regr}
\end{figure}

\begin{figure}
	\centering
	\includegraphics[scale = 
	0.75]{/home/dmitry/Work/Research/thesis/FINALE/P3_ITS_GENERATION/figures/fig_8_along_beam.png}
	\caption{Basin variation of cross-beam averaged energy characteristics. For all panels - solid 
	black line is ensemble-mean value and thin gray lines for each particular realization. Red 
	lines show one standard deviation in realizations of the spatial averages. The dashed line is 
	produced by regression of the ensemble-means with a second degree polynomial. Panel (a) gives 
	spatially averaged direction of the energy fluxes counted counterclockwise from east. (b) 
	Cross-beam integrated flux. Purple line shows same integrated flux but with no account for 
	relative angle with respect to across beam section. (c) Ratio between spatially averaged 
	horizontal kinetic and available potential energies. Purple line is the ratio for a progressive 
	plane wave.}
	\label{C3.fig:beam_prms}
\end{figure}

\begin{figure}
	\centering
	\includegraphics[scale = 
	0.75]{/home/dmitry/Work/Research/thesis/FINALE/P3_ITS_GENERATION/figures/fig_9_beam_superposition.png}
	\caption{Interference in the mode-1 tidal field represented by ensemble-mean fields 4 circular 	
	quadrants (a-d) organized by counterclockwise order. Panel (a) represents northwesterly 
	traveling components of the angular spectra. The wave field is comprised both by spatial 
	variations in Available Potential Energy (color scale) and energy fluxes (arrows). The inset 
	gives respective variation of across-beam averaged HKE/APE ratio (solid black) and the in-situ 
	group speed (solid purple). The respective dashed lines show predictions following the  
	plane-wave theory. Panels (b-c) represent the wave field in the respective quadrants by the 
	flux vectors. And cumulative result of the superposition in the clockwise order starting from 
	(a) is shown by APE. The insets show effect of the superposition on the spatial averages. Note 
	that here the vector magnitudes are 10 times smaller than on panel (a).}
	\label{C3.fig:beam_dcmp}
\end{figure}

\begin{figure}
	\centering
	\includegraphics[scale = 
	0.75]{/home/dmitry/Work/Research/thesis/FINALE/P3_ITS_GENERATION/figures/fig_10_decomp_cumsum.png}
	\caption{Variability of mode-1 internal tide in the Tasman Sea. The same wave field 
	decomposition as in \fignm{C3.fig:beam_dcmp} was used here. Panel (a) shows variation in the 
	Northwesterly traveling waves. The arrows and color shading are plotted for the ensemble-mean. 
	Between experiments is given by respective variance ellipses. And standard deviation in the 
	spatially averaged characteristics is summarized on insets in top right corners.}
	\label{C3.fig:beam_dcmp_cm}
\end{figure}

\begin{figure}
	\centering
	\includegraphics[scale = 
	0.75]{/home/dmitry/Work/Research/thesis/FINALE/P3_ITS_GENERATION/figures/fig_11_anlt_result_comparison.png}
	\caption{Internal tide generated by a knife-edge ridge in the numerical model (upper row). The 
	field solved by the given analytical model in Appendix A (lower row). Three different regimes 
	are explored with no rotation present ($f = 0$, the left column), with Coriolis force 
	representative of the Macquarie Ridge ($f = 10^{-4}$, the middle column) and strong Coriolis 
	force ($f = 0.8 \omega_{M_2}$, the right column).}
	\label{C3.fig:anlt_sol}
\end{figure}

\begin{figure}
	\centering
	\includegraphics[scale = 
	0.75]{/home/dmitry/Work/Research/thesis/FINALE/P3_ITS_GENERATION/figures/fig_12_beam_models.png}
	\caption{(a) The flux map generated from sources having magnitude scaled by size of points. (b) 
	The field generated only by the central Macquarie Ridge section.}
	\label{C3.fig:beam_inv}
\end{figure}

\begin{figure}
	\centering
	\includegraphics[scale = 
	0.75]{/home/dmitry/Work/Research/thesis/FINALE/P3_ITS_GENERATION/figures/fig_13_generation_variability.png}
	\caption{(a) Variation of conversion rates along the Macquarie ridge. (b) Comparison of the 
	beam's center in the near field. (c) Change in total conversion rates and beam's strength. The 
	realizations in the field experiment are color coded throughout the figure (see panel (a)).}
	\label{C3.fig:gen_var_beam}
\end{figure}

\begin{figure}
	\centering
	\includegraphics[scale = 
	0.75]{/home/dmitry/Work/Research/thesis/FINALE/P3_ITS_GENERATION/figures/fig_14_example_beam_mesoscale.png}
	\caption{(a) Averaged distribution of sea surface height and vorticity in the Tasman Sea in 
	period between 10th - 15th January. Each contour line discerns 0.1 Rossby number. And 	
	distribution of total flux vectors. (b) Phase speed anomaly for the same dates and the planar 
	beam.}
	\label{C3.fig:meso_examp}
\end{figure}

\begin{figure}
	\centering
	\includegraphics[scale = 
	0.75]{/home/dmitry/Work/Research/thesis/FINALE/P3_ITS_GENERATION/figures/fig_15_mesoscale_variability.png}
	\caption{(a) Semi-major axis of variance ellipses. (b) StDev in phase speed anomaly during the 	
	field experiment. And variation in beam's center color coded same as on 
	\fignm{C3.fig:gen_var_beam}}
	\label{C3.fig:var_meso}
\end{figure}

\begin{figure}
	\centering
	\includegraphics[scale = 
	0.75]{/home/dmitry/Work/Research/thesis/FINALE/P3_ITS_GENERATION/figures/fig_16_comparison_sat.png}
	\caption{Comparison of internal tidal beam obtained through numerical simulations and from 
	satelite altrimetry observations. Arrows show their position and magnitude in red and black 
	respectively. Lines with dots show position of the beam center.}
	\label{C3.fig:cmp_sat_nexp}
\end{figure}

\begin{figure}
\centering
\includegraphics[width=0.7\linewidth]{../figures/fig_17_KE_basin_beam}
\caption{Variation in kinetic energy in the central Tasman Sea. White dots show the energy in HYCOM 
simulations, red dots - in the experiments performed here. The purple show variation of the beam 
position in ROMS-experiments relative to satelite altimetry. }
\label{C3.fig:KE_beam}
\end{figure}

\begin{figure}
	\centering
	\includegraphics[width=0.7\linewidth]{../figures/fig_18_spatial_TBEAM.png}
	\caption{Comparison of the energy flux obtained during the field program of TBEAM. On all 
	panels red pink colored arrows illustrate total field, blue - NW quadrant representative of the 
	planar beam, green - eastward traveling waves whose amplitude was tripled to be of comparable 
	size. Black arrows show results of the TBEAM observations. Panel (a) comparison of the field 
	program averaged numerical experiment. The arrows at observational sites were scaled 3 times 
	above background arrows which magnitude is given on the legend. (b) Comparison with '2013' 
	experiment. (c) Average over all experiments. Here additionally distribution of HKE and APE are 
	plotted by shading and lines of red and blue respectively.}
	\label{C3.fig:TBEAM_sp}
\end{figure}

\begin{figure}
	\centering
	\includegraphics[width=0.7\linewidth]{../figures/fig_19_KE2PE_TBEAM.png}
	\caption{(a) Ratio of HKE to APE observed during TBEAM field program (black line) and simulated 
	(red dots). (b-c) Distribution of HKE and APE for two cases of maximum and minimum.}
	\label{C3.fig:TBEAM_KE2PE}
\end{figure}

\begin{figure}
	\centering
	\includegraphics[width=0.7\linewidth]{../figures/fig_20_spectra_exmp.png}
	\caption{Example of spectral decomposition made from point-wise observations. Green line shows 
	result from spatial analysis of antenna. Orange line - result from point-wise decomposition for 
	the same numerical experiment. Blue line is a result for TBEAM.}
	\label{C3.fig:ex_spectra}
\end{figure}

\begin{figure}
	\centering
	\includegraphics[width=0.7\linewidth]{../figures/fig_21_test_decomp.png}
	\caption{Example of spectral decomposition made from point-wise observations. Green line shows 
		result from spatial analysis of antenna. Orange line - result from point-wise decomposition 
		for 
		the same numerical experiment. Blue line is a result for TBEAM.}
	\label{C3.fig:test_spectra}
\end{figure}


\begin{figure}
	\centering
	\includegraphics[width=0.7\linewidth]{../figures/fig_22_A1_comp.png}
	\caption{Result of pointwise decomposition at A1 location. }
	\label{C3.fig:dcmp_A1}
\end{figure}


%\begin{figure}
%\centering
%\includegraphics[scale = 
%0.75]{/home/dmitry/Work/Research/thesis/FINALE/P3_ITS_GENERATION/figures/fig_11_example_beam_mesoscale}
%\caption{Example of refraction. }
%\label{C3.fig:fig_11_beam_mesoscale}
%\end{figure}
%
%\begin{figure}
%\centering
%\includegraphics[scale=0.75]{/home/dmitry/Work/Research/thesis/FINALE/P3_ITS_GENERATION/figures/fig_12_beam_var_cp}
%\caption{}
%\label{C3.fig:fig_12_beam_var_cp}
%\end{figure}

%\begin{figure}
%	\centering
%	\includegraphics[scale = 
%	
%0.5]{/home/dmitry/Work/Research/thesis/FINALE/P3_ITS_GENERATION/figures/fig_11_along_beam_inv.png}
%	\caption{Variation of beam parameters.}
%	\label{C3.fig:beam_alng_var}
%\end{figure}

\bibliographystyle{apa}
\bibliography{/home/dmitry/Bibtex_lib/my_first_lib}

\end{document}